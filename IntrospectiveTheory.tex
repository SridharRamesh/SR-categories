\section{Introspective theories and GL-categories}

% Non-evil definition
\subsection{Definition}
\begin{definition}
A \defined{pre-introspective theory} is a lexcategory $T$, a $T$-indexed lexcategory $C$, and a lexfunctor $F$ from the self-indexing of $T$ to $C$, like so:

\[\begin{tikzcd}
	{\op{T}} && {\LexCat}
	\arrow["{T/-}"{name=0}, from=1-1, to=1-3, shift left=2]
	\arrow["{C}"{name=1, swap}, from=1-1, to=1-3, shift right=2]
	\arrow[Rightarrow, "{F}"', from=0, to=1]
\end{tikzcd}\]

If $C$ is small or locally small, then we say this is an \defined{introspective theory} or \defined{locally introspective theory}, respectively.
\end{definition}

We write $\langle T, C, F \rangle$ to refer to a pre-introspective theory when we wish to be fully explicit about its structure. But in typical abuse of language, we also often refer to it simply by the name of its underlying lexcategory $T$, when this would not cause confusion.

\begin{construction}
If $\langle T, C, F \rangle$ is a pre-introspective theory, and any lexfunctor $G : C \to D$ is given for some other $T$-indexed lexcategory $D$, then $\langle T, D, G \circ F \rangle$ is itself a pre-introspective theory, like so: 

\[\begin{tikzcd}
	{\op{T}} && {\LexCat}
	\arrow["{T/-}"{name=0}, from=1-1, to=1-3, shift left=5]
	\arrow["{C}"{name=1, description}, from=1-1, to=1-3]
	\arrow["{D}"{name=2, swap}, from=1-1, to=1-3, shift right=5]
	\arrow[Rightarrow, "{F}"', from=0, to=1]
	\arrow[Rightarrow, "{G}"', from=1, to=2]
\end{tikzcd}\]

Of course, this yields an introspective or locally introspective theory just in case $D$ is small or locally small, respectively.
\end{construction}

\begin{construction}
If $\langle T, C, F \rangle$ is a pre-introspective theory, and $t$ is any object in $T$, then the slice category $T/t$ can be equipped in a natural way as a pre-introspective theory as well. If we start from an introspective or locally introspective theory, then so respectively will be the result of this construction.
\end{construction}
\begin{proof}
Consider the forgetful functor $\Sigma : T/t \to T$. Composition and whiskering with this gives us the triple $\langle T/t, \Sigma C, \Sigma F \rangle$, like so:

\[\begin{tikzcd}
	{\op{\left(T/t\right)}} & {\op{T}} && {\LexCat}
	\arrow["{T/-}"{name=0}, from=1-2, to=1-4, shift left=2]
	\arrow["{C}"{name=1, swap}, from=1-2, to=1-4, shift right=2]
	\arrow["{\Sigma}", from=1-1, to=1-2]
	\arrow[Rightarrow, "{F}"', from=0, to=1]
\end{tikzcd}\]

This is a pre-introspective theory as the top composite is the same as the slice category presheaf $(T/t)/-$ on $T/t$ (that is, slice categories within slice categories are obtained by first applying the forgetful functor into the ambient category and then taking the ordinary slice category).

When $C$ is small or locally small, then so respectively is $\Sigma C$. The representing objects for $\Sigma C$ will simply be those for $C$, mapped into $T/t$ via $\Sigma$'s right adjoint (i.e., pulled back along the unique morphism from $t$ to $1$).
\end{proof}

When we abuse language and speak of $T/t$ as an introspective theory, the above construction is what we mean. That said, there is another natural way to equip $T/t$ as an introspective theory as well.

\begin{TODOblock}
There is another way to equip $T/t$ as an introspective theory, in which we use as our internal category the slice category $C/t$ so to speak. Construction 1.3 is actually equivalent to doing this and then applying Construction 1.2 to the internal forgetful lexfunctor from $C/t$ to $C$. This equivalence is easier to see once we see how the natural transformation taking $t$ to $\Hom_C(1, t)$ works.
\end{TODOblock}

\begin{construction}[Obtaining S and N]
Given a pre-introspective theory $\langle T, C, F \rangle$, we obtain immediately a lexfunctor from $T$ to $C(1)$, given by the component of $F$ at the terminal object $1$. Let us call this lexfunctor $S$.

We shall also obtain, naturally in $x : T$, a map from $x$ to $\Hom_C(1, F(x))$. Let us call this $N$. The specific construction of $N$ is like so: 

\begin{comment}
Note that in a general pre-introspective theory, $\Hom_C(1, F(x))$ amounts to a $T$-indexed set, i.e. a presheaf over $T$, which may or may not be representable. However, if we are dealing with an introspective theory, the presheaf corresponding to $\Hom_C(1, F(x))$ will be representable (as it is given by pulling back a global object of $\Ob(C)^2$ along the domain and codomain projections from $\Mor(C)$, with each of these being representable). By the Yoneda lemma, we can in any case speak of all of this equivalently as a function from $\Hom_T(t, x)$ to $\Hom_{C(t)}(1, F(t)(x))$, natural in $t$ and $x$ both from $T$
\end{comment}

Given any $T$-indexed lexfunctor $F$ between any $T$-indexed categories $D$ and $C$, the global action of $F$ gives us a $T$-indexed map from $\Hom_{D(1)}(1, x)$ to $\Hom_{C(1)}(1, F(x))$ natural in $x : D(1)$, where the domain and codomain of this map are $T$-indexed sets; i.e., presheaves on $T$. In our particular case, we are taking $D$ as the self-indexing of $T$, and so it remains only to understand which presheaf over $T$ amounts to $\Hom_{D(1)}(1, x)$ for an object $x$ in $T = D(1)$. In general, the self-indexing is such that $\Hom_{D(1)}(y, x)$ will be the presheaf $\Hom_T(y \times -, x)$. But when $y = 1$, this is simply the presheaf represented by $x$ itself, exactly as we desired.
\end{construction}

\begin{TODOblock}
Check that the above makes sense.
\end{TODOblock}

It turns out, we can recover $F$ from arbitrary such $N$ and $S$ too. That is, this last construction has an inverse. \TODO