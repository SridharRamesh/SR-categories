\section{Introspective theories}

% Non-evil definition
\subsection{Definition}
\begin{definition} \label{DefnPreIntrospIndexed}
A \defined{pre-introspective theory} is a lexcategory $T$, a $T$-indexed lexcategory $C$, and a lexfunctor $\introF$ from the self-indexing of $T$ to $C$, like so:

\[\begin{tikzcd}
	{\op{T}} && {\LexCat}
	\arrow["{T/-}"{name=0}, from=1-1, to=1-3, shift left=2]
	\arrow["{C}"{name=1, swap}, from=1-1, to=1-3, shift right=2]
	\arrow[Rightarrow, "{\introF}"', from=0, to=1]
\end{tikzcd}\]
\end{definition}

We write out the triple $\langle T, C, \introF \rangle$ to refer to a pre-introspective theory when we wish to be fully explicit about its structure. But in typical abuse of language, we also often refer to it simply by the name of its underlying lexcategory $T$ or of the pair $\langle T, C \rangle$, when this would not cause confusion. We will frequently use the same name $\introF$ as though it applies to all introspective theories simultaneously, in the same way that notation like $+$ or $\times$ is overloaded as applying over all rings simultaneously.

\begin{definition} \label{DefnIntrospIndexed}
An \defined{introspective theory} is a pre-introspective theory $\langle T, C \rangle$ in which $C$ is small.
\end{definition}

The concept of an introspective theory is the fundamental object of our interest and study in these notes. We shall show that this simple definition already suffices to exhibit and capture all the fundamental phenomena of \Goedel\ codes, diagonalization, the \Goedel\ incompleteness theorems, and \Loeb's theorem. And we shall show that all the traditional instances of \Goedel's incompleteness phenomena arise from special cases of this purely algebraic structure. We will also demonstrate some previously unrealized functorial fixed point results for this structure, and show some interesting applications of these.

We shall also introduce some further generalizations of this concept, in order to be able to state results along the way in their fullest generality or point out connections to related work or interesting structures that are not quite introspective theories per se but are closely related. But throughout these notes, if at any time the abstractions seem daunting or distracting, remember that the concrete concept which matters most is the concept of an introspective theory as defined above.

We shall now make an observation about an alternative but equivalent way to specify this data.

\begin{theorem}\label{SNCorrespondence}
Given a lexcategory $T$ and a $T$-indexed lexcategory $C$, specifying a $T$-indexed lexfunctor from the self-indexing $T/-$ to $C$ is equivalent to specifying a (non-indexed) lexfunctor $\introS$ from $T$ to the global aspect of $C$, as well as specifying a natural transformation from $t$ in $T$ to $\Hom_C(1, \introS(t))$.
\end{theorem}

As preparation for the proof of \cref{SNCorrespondence}, we will need two lemmas.

\begin{lemma}\label{Lemma1}
If $T$ is a category with terminal object, then a map from $D$ to $C$ as $T$-indexed Xes is the same as a map $\introS$ from $D(1)$ to $C(1)$ along with a map from $D$ to $C$ as $T$-indexed (Xes under\footnote{In the sense of co-slices.} $D(1)$).
\end{lemma}
\begin{proof}
This is completely straightforward and short. Note that the map $\introS$ will be used to treat $C$ as an X under $D(1)$. \TODO
\end{proof}

In order to state the next lemma, some terminology: If $T$ is a lexcategory, then for each object of $t$, we can construct the free lexcategory under $T$ augmented with a global element of $t$. Call this $T[1 \to t]$. Also, for any $f : s \to t$ in $T$, we can get a map from $T[1 \to t]$ to $T[1 \to s]$ by sending the generic global element of $t$ in $T[1 \to t]$ to the result of applying $f$ to the generic global element of $s$ in $T[1 \to s]$. This action is clearly functorial. Thus, $T[1 \to -]$ comprises a $T$-indexed lexcategory under $T$. \TODOinline{Move this and \cref{Lemma2} itself to Preliminaries. It should be mentioned at the same time that we grant $T/-$ the name \quote{self-indexing} to begin with.}

\begin{lemma}\label{Lemma2}
$T[1 \to -]$ is $T/-$, as a $T$-indexed lexcategory under $T$.
\end{lemma}
\begin{proof}
This is a standard observation, and also simple enough to show. \TODO
\end{proof}

Now, we shall prove \cref{SNCorrespondence}.

\begin{proof}
Let $T$ be a lexcategory, and let $C$ be some $T$-indexed lexcategory. By \cref{Lemma1}, a map from the self-indexing $T/-$ to $C$ as $T$-indexed lexcategories is the same as a lexfunctor $\introS$ from $T$ to the global aspect of $C$, along with a map from $T/-$ to $C$ as $T$-indexed (lexcategories under $T$). The map $\introS$ will be used to treat $C$ as a lexcategory under $T$.

Next we apply \cref{Lemma2}. The map from $T/-$ to $C$ as $T$-indexed (lexcategories under $T$) is the same as choosing, in a natural way over all $t$ in $T$, some $t$-defined value in $\Hom_C(1, \introS(t))$. That is, a natural transformation from $t$ to $\Hom_C(1, \introS(t))$.
\end{proof}

\begin{remark}\label{IntrospGeneralDoctrine}
It wasn't fundamentally important that we were dealing with lexcategories here. \Cref{Lemma1} only required a terminal object. And for any algebraish notion extending categories-with-global-elements, there is some free construction $T[1 \to -]$. (Even the role terminality plays here is to some degree eliminable). In particular, for categories-with-finite-products, there is also a simple independent account of the free structure $T[1 \to t]$, amounting to the full subcategory of $T/t$ on just those slices which are projections from some $t \times s$ to $t$ (the so-called \quote{simple self-indexing}). So we get the same result if we replace throughout \quote{lexcategory} and \quote{lexfunctor} by finite-product structure, and replace $T/-$ by this simple self-indexing. The same also holds for for any notion extending the notion of lexcategories which is automatically inherited by slice categories and preserved by pullback, or any notion extending the notion of finite-product-categories which is automatically inherited by these reduced slice categories and preserved by their pullbacks. For example, toposes or locally cartesian closed categories or cartesian closed categories or categories with finite products and finite coproducts over which they distribute or etc. Even infinitary notions could be used; categories with all limits of cardinality up to some regular cardinal, say.
\end{remark}

As a result of \cref{SNCorrespondence}, we can give another definition essentially equivalent to \cref{DefnPreIntrospIndexed}.

\begin{definition}\label{DefnPreIntrospSN}
A \defined{pre-introspective theory} is a lexcategory $T$, a $T$-indexed lexcategory $C$, a lexfunctor $\introS$ from $T$ to the global aspect of $C$, and a map $\introN$ from $t$ to $\Hom_C(1, S(t))$, natural in $t$ in $T$.
\end{definition}

As before, we may write out $\langle T, C, \introS, \introN \rangle$ to be fully explicit, but in typical abuse of language, may refer to a pre-introspective theory by simply naming $T$ or the pair $\langle T, C\rangle$. We will frequently use the same names $\introS$ and $\introN$ as though they apply simultaneously to all introspective theories (in the same way that notation like $+$ and $\times$ is overloaded as applicable to separate rings simultaneously).

\begin{lemma}[$\introS$ Matches $\introN$]\label{SMatchesN}
Within a pre-introspective theory $\langle T, C \rangle$, let $t$ be some object of $T$ and let $x$ be some globally defined element of $t$. Then $\introS(x)$ and $\introN_t(x)$ yield equal global elements of $S(t)$ within the global aspect of $C$.
\end{lemma}
\begin{proof}
Apply the naturality square for $\introN$ to the morphism from $1$ to $t$ representing $x$. One path around the square will be the morphism in $T$ from $1$ to $\Hom_C(1, \introS(t))$ representing $\introN_t(x)$. The other path around the square will be the unique identity morphism in $T$ from $1$ to $\Hom_C(1, 1)$ representing the identity morphism on $1$, followed by the morphism from $\Hom_C(1, 1)$ to $\Hom_C(1, \introS(t))$ representing composition in $C$ with the morphism in $C$ representing $S(x)$ as a global element of $S(t)$. As composition with identity leaves values unchanged, this is also the morphism from $1$ to $\Hom_C(1, \introS(t))$ representing $S(x)$.
\end{proof}

As the name \quote{pre-introspective theory} suggests, there are some further definitions to build atop this.

\begin{definition}\label{DefnIntrosp}
If $\langle T, C \rangle$ is a pre-introspective theory and $C$ is small or locally small, then we say this is an \defined{introspective theory} or \defined{locally introspective theory}, respectively.
\end{definition}

While it is sometimes easier to prove theorems about (pre-)introspective theories by using \cref{DefnPreIntrospIndexed}, it will usually be easier to show structures actually are (pre-)introspective theories by using \cref{DefnPreIntrospSN}. But this is not the only benefit of \cref{DefnPreIntrospSN}. The value of this new definition is that there is much less data around. In particular, when we wish to turn this into a lex definition in section \TODO, we will find the coherence conditions much easier to manage. It will also be easier to define the appropriate notion of homomorphisms between (pre-)introspective theories by thinking about \cref{DefnPreIntrospSN}.

\Cref{DefnPreIntrospSN} also allows us to quickly appreciate the significance of introspective theories from a functorial semantics point of view. An introspective theory is precisely an essentially algebraic theory (this is the role of $T$) extending the theory of lexcategories (this is the role of $C$), such that every model of the theory is equipped with a designated homomorphism (this is the role of $\introN$) into an internal model of the same theory (this is the role of $\introS$).

Recall that a small indexed lexcategory is one which is equivalent to some small indexed strict lexcategory (with possibly multiple non-isomorphic such choices available). It is occasionally of use to imagine some particular such choice has been pinned down, leading to the following definition.

\begin{definition}
By an \defined{inner-strict introspective theory}, we mean a pre-introspective theory $\langle T, C\rangle$ along with a specific choice for how to construe $C$ as a small indexed strict lexcategory.
\end{definition}

Thus, an introspective theory can always be construed as some inner-strict introspective theory, though multiple non-isomorphic such choices may be available.

We call this \quote{inner strict} to emphasize that we've taken every choice concerning the representation of $C$ which was allowed to vary over non-isomorphic objects or non-equal parallel morphisms in $T$, and fixed some particular such choice for it, but we've not imposed strict structure on $T$ itself.

\subsection{Some useful constructions}
\begin{construction}\label{IntrospInternalMap}
If $\langle T, C, \introF \rangle$ is a pre-introspective theory, and any lexfunctor $G : C \to D$ is given for some other $T$-indexed lexcategory $D$, then $\langle T, D, G \circ \introF \rangle$ is itself a pre-introspective theory, like so: 

\[\begin{tikzcd}
	{\op{T}} && {\LexCat}
	\arrow["{T/-}"{name=0}, from=1-1, to=1-3, shift left=5]
	\arrow["{C}"{name=1, description}, from=1-1, to=1-3]
	\arrow["{D}"{name=2, swap}, from=1-1, to=1-3, shift right=5]
	\arrow[Rightarrow, "{\introF}"', from=0, to=1]
	\arrow[Rightarrow, "{G}"', from=1, to=2]
\end{tikzcd}\]

Of course, this yields an introspective or locally introspective theory just in case $D$ is small or locally small, respectively.
\end{construction}

\begin{construction}\label{IntrospSlice}
If $\langle T, C, \introF \rangle$ is a pre-introspective theory, and $t$ is any object in $T$, then the slice category $T/t$ can be equipped in a natural way as a pre-introspective theory as well. If we start from an introspective or locally introspective theory, then so respectively will be the result of this construction.
\end{construction}
\begin{proof}[Details]
Consider the forgetful functor $\Sigma : T/t \to T$. Composition and whiskering with this gives us the triple $\langle T/t, \Sigma C, \Sigma \introF \rangle$, like so:

\[\begin{tikzcd}
	{\op{\left(T/t\right)}} & {\op{T}} && {\LexCat}
	\arrow["{T/-}"{name=0}, from=1-2, to=1-4, shift left=2]
	\arrow["{C}"{name=1, swap}, from=1-2, to=1-4, shift right=2]
	\arrow["{\Sigma}", from=1-1, to=1-2]
	\arrow[Rightarrow, "{\introF}"', from=0, to=1]
\end{tikzcd}\]

This is a pre-introspective theory as the top composite is the same as the self-indexing $(T/t)/-$ of $T/t$ (that is, slice categories within slice categories are obtained by first applying the forgetful functor into the ambient category and then taking the ordinary slice category).

When $C$ is small or locally small, then so respectively is $\Sigma C$. The representing objects for $\Sigma C$ will simply be those for $C$, mapped into $T/t$ via $\Sigma$'s right adjoint (i.e., pulled back along the unique morphism from $t$ to $1$).
\end{proof}

When we abuse language and speak of $T/t$ as an introspective theory, the above construction is what we mean. That said, there is another natural way to equip $T/t$ as an introspective theory as well.

\begin{TODOblock}
There is another way to equip $T/t$ as an introspective theory, in which we use as our internal category the slice category $C/t$ so to speak. \Cref{IntrospSlice} is actually equivalent to doing this and then applying \cref{IntrospInternalMap} to the internal forgetful lexfunctor from $C/t$ to $C$. This equivalence is easier to see once we see how the natural transformation taking $t$ to $\Hom_C(1, t)$ works. (Which we now have sketched at \cref{SNCorrespondence}). Perhaps it's also true that in the other direction, we apply \cref{IntrospInternalMap} to the constant lexfunctor from $C$ to $C/t$? Huh, and in the first case, I might have thought the relevant lexfunctor from $C/t$ to $C$ was pullback along the generic global element of $t$? Hm. Anyway, let us think about this more later.
\end{TODOblock}

\begin{TODOblock}
Make the useful observation that, as the theory of introspective theories is essentially lex, we can take products of introspective theories in the straightforward way.
\end{TODOblock}