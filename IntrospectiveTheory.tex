\section{Introspective theories}

% Non-evil definition
\subsection{Definition}

\begin{definition} \label{DefnPreIntrospIndexed}
A \defined{pre-introspective theory} is a lexcategory $T$, a $T$-indexed lexcategory $C$, and a lexfunctor $\introF$ from the self-indexing of $T$ to $C$, like so:

\[\begin{tikzcd}
	{\op{T}} && {\LexCat}
	\arrow["{T/-}"{name=0}, from=1-1, to=1-3, shift left=2]
	\arrow["{C}"{name=1, swap}, from=1-1, to=1-3, shift right=2]
	\arrow[Rightarrow, "{\introF}"', from=0, to=1]
\end{tikzcd}\]
\end{definition}

We write out the triple $\langle T, C, \introF \rangle$ to refer to a pre-introspective theory when we wish to be fully explicit about its structure. But in typical abuse of language, we also often refer to it simply by the name of its underlying lexcategory $T$ or of the pair $\langle T, C \rangle$, when this would not cause confusion. We will frequently use the same name $\introF$ as though it applies to all introspective theories simultaneously, in the same way that notation like $+$ or $\times$ is overloaded as applying over all rings simultaneously.

\begin{definition} \label{DefnIntrospIndexed}
An \defined{introspective theory} is a pre-introspective theory $\langle T, C \rangle$ in which $C$ is small.
\end{definition}

The concept of an introspective theory is the fundamental object of our interest and study in these notes. We shall show that this simple definition already suffices to exhibit and capture all the fundamental phenomena of \Goedel\ codes, diagonalization, the \Goedel\ incompleteness theorems, and \Loeb's theorem. And we shall show that all the traditional instances of \Goedel's incompleteness phenomena arise from special cases of this purely algebraic structure. We will also demonstrate functorial fixed point results for this structure, and show some interesting applications of these.

We shall also introduce some further generalizations of this concept, in order to be able to state results along the way in their fullest generality or point out connections to related work or interesting structures that are not quite introspective theories per se but are closely related. But throughout these notes, if at any time the abstractions seem daunting or distracting, remember that the concrete concept which matters most is the concept of an introspective theory as defined above.

We shall now make an observation about an alternative but equivalent way to specify this data.

\begin{theorem}\label{SNCorrespondence}
Given a lexcategory $T$ and a $T$-indexed lexcategory $C$, specifying a $T$-indexed lexfunctor from the self-indexing $T/-$ to $C$ is equivalent to specifying a (non-indexed) lexfunctor $\introS$ from $T$ to the global aspect of $C$, as well as specifying a natural transformation from $t$ in $T$ to $\Hom_C(1, \introS(t))$.
\end{theorem}

As preparation for the proof of \cref{SNCorrespondence}, we will need two lemmas. Both of these lemmas should be understood as obvious, folklore, etc, but we will write out proofs of them for self-containedness.

\begin{lemma}\label{Lemma1}
If $Y$ is a category with initial object $0$ and $X$ is a (2-)category, then to any functor $f : Y \to X$, we can associate a corresponding functor $f'$ from $Y$ to the slice category $f(0)/X$.

Furthermore, if $D$ and $C$ are parallel functors from $Y$ to $X$, then a natural transformation from $D$ to $C$ amounts to the same thing as a map $\introS$ from $D(0)$ to $C(0)$ along with a natural transformation from $D'$ to $\introS^{*} \circ C'$, where $\introS^{*} : C(0)/X \to D(0)/X$ is the functor between these slice categories given by composition with $\introS$.
\end{lemma}
\begin{proof}
The first half of the lemma is thoroughly straightforward. $f : Y \to X$ induces a functor between the arrow category of $Y$ and the arrow category of $X$, and since $Y$ sits inside its arrow category as the slice category $0/Y$, this gives us a functor from $Y$ to the arrow category of $X$, which when followed by the domain projection back to $X$ is constantly $f(0)$. This is our $f'$.

The second half is also straightforward to mechanically verify when $X$ is a 1-category. This lemma should be understood as a triviality. But we will take some care to write out in detail an abstract demonstration that works just as well when $X$ is a 2-category (or indeed, when all categories involved are of whatever higher dimension), so that (in keeping with our linguistic convention) the functors involved are pseudofunctors, the natural transformations are pseudonatural transformations, etc, without having to get our hands dirty manually fussing about higher-dimensional coherence data. Throughout the remainder of this proof, all references to \quote{category}, \quote{functor}, etc, are in the sense of whatever dimension of higher-categories encapsulates both $Y$ and $X$.

Let $Z$ be the category obtained by augmenting $Y$ with a new object $0_Z$ and unique maps from $0_Z$ to each object of $Y$. We have an inclusion functor $i : Y \to Z$, and this inclusion is fully faithful, in the sense that the induced map $\Hom_Y(y_1, y_2) \to \Hom_Z(i(y_1), i(y_2))$ is an equivalence for all $y_1, y_2 \in \Ob(Y)$.

The unique maps from $0_Z$ to each object in the range of $i$ constitute a diagram of this form:

\[\begin{tikzcd}
	& 1 \\
	Y && Z
	\arrow["\unique", from=2-1, to=1-2]
	\arrow["{0_Z}", from=1-2, to=2-3]
	\arrow[""{name=0, anchor=center}, "i"', from=2-1, to=2-3]
	\arrow[Rightarrow, from=1-2, to=0]
\end{tikzcd}\]

What's more, because of how $Z$ was constructed by freely augmenting $Y$ with a new object and co-cone from it to the inclusion of $Y$, this diagram satisfies the universal property that for any other similar diagram
\[\begin{tikzcd}
	& 1 \\
	Y && Z'
	\arrow["\unique", from=2-1, to=1-2]
	\arrow[from=1-2, to=2-3]
	\arrow[""{name=0, anchor=center}, from=2-1, to=2-3]
	\arrow[Rightarrow, from=1-2, to=0]
\end{tikzcd}\]
there is a unique functor from $Z$ to $Z'$ commutatively relating the two diagrams. In jargon, this universal property is summarized by saying $Z$ (along with the data of $0_Z$ and $i$) is the co-comma of the unique functor from $Y$ to $1$ and the identity functor from $Y$ to $Y$.

Now, observe that $i$ has a left adjoint, the functor $q : Z \to Y$ such that $q \circ i$ is the identity on $Y$ and such that $q$ of the initiality co-cone for $0_Z$ in $Z$ is the initiality co-cone for $0$ in $Y$. That is, $q$ is the functor obtained by the co-comma property for $Z$ as applied to this diagram expressing the initiality co-cone of $0$ in $Y$:

\[\begin{tikzcd}
	& 1 \\
	Y && Y
	\arrow["\unique", from=2-1, to=1-2]
	\arrow["{0}", from=1-2, to=2-3]
	\arrow[""{name=0, anchor=center}, "\id"', from=2-1, to=2-3]
	\arrow[Rightarrow, from=1-2, to=0]
\end{tikzcd}\]

It is straightforward to verify that this $q$ is indeed left adjoint to $i$, as any data in $Z$ is either from the fully faithful inclusion of $Y$ or from the initiality co-cone for $0_Z$, and $\Hom_Y(q(i(y_1)), y_2) \iso \Hom_Y(y_1, y_2) \iso \Hom_Z(i(y_1), i(y_2))$ naturally in $y_1, y_2$ from $Y$, and $\Hom_Y(q(0_Z), y) = \Hom_Y(0, y) \iso 1 \iso \Hom_Z(0_Z, i(y))$ naturally in $y$ from $Y$.

Now consider any two parallel functors $D, C : Y \to X$. Because $q \circ i$ is the identity on $Y$, we have that $\Nat(D, C) \iso \Nat(D \circ q \circ i, C)$, where $\Nat$ denotes the space of natural transformations between these functors. But because $q \dashv i$, we in turn have that $\Nat(D \circ q \circ i, C) \iso \Nat(D \circ q, C \circ q)$.

Finally, let us consider what a natural transformation between $D \circ q$ and $C \circ q$ amounts to. This is the same as a functor from $Z$ to the arrow category of $X$ whose domain and codomain projections to $X$ yield $D \circ q$ and $C \circ q$. But by the co-comma property of $Z$, this functor out of $Z$ corresponds to data of the following form:

\[\begin{tikzcd}
	& 1 \\
	Y && {\arrowcat{X}}
	\arrow["\unique", from=2-1, to=1-2]
	\arrow[from=1-2, to=2-3]
	\arrow[""{name=0, anchor=center}, from=2-1, to=2-3]
	\arrow[Rightarrow, from=1-2, to=0]
\end{tikzcd}\]

such that the rightmost arrow of this diagram corresponds to some arrow $\introS$ in $X$ whose domain is $(D \circ q)(0_Z) = D(0)$ and whose codomain is $(C \circ q)(0_Z) = C(0)$, and such that the bottom arrow of this diagram corresponds to a natural transformation from $D \circ q \circ i \iso D$ to $C \circ q \circ i \iso C$. The 2-cell in the above diagram then corresponds to the remaining data necessary for us to construe this natural transformation from $D$ to $C$ as simply the codomain projection of a natural transformation between $D'$ and $\introS^{*} \circ C'$, the functors from $Y$ to $D(0)/X$ as mentioned in the statement of this lemma.
\end{proof}

\TODOinline{Phew! That made a mountain out of a molehill. But perhaps people sometimes appreciate such written-out detail.}

In order to state the next lemma, some terminology: If $T$ is a lexcategory, then for each object of $t$, we can construct the free lexcategory extending $T$ with a global element of $t$. Call this $T[1 \to t]$. Also, for any $f : s \to t$ in $T$, we can get a map from $T[1 \to t]$ to $T[1 \to s]$ by sending the generic global element of $t$ in $T[1 \to t]$ to the result of applying $f$ to the generic global element of $s$ in $T[1 \to s]$. This action is clearly functorial. Thus, $T[1 \to -]$ comprises a $T$-indexed lexcategory-under-$T$. \TODOinline{Move this and \cref{Lemma2} itself to Preliminaries. It should be mentioned at the same time that we grant $T/-$ the name \quote{self-indexing} to begin with. And we should do the same for finite product categories and the simple self-indexing.}

\begin{lemma}\label{Lemma2}
$T[1 \to -]$ is $T/-$, as a $T$-indexed lexcategory-under-$T$.
\end{lemma}
\begin{proof}
This is a standard observation \TODOinline{Cite 1.10.15 of Bart Jacobs' \quote{Categorical logic and type theory}, although this claims it without proof}, and also simple enough to show. \TODOinline{This whole proof is not really necessary, once the cite is given}

To start off, we construe $T/-$ as not merely a $T$-indexed lexcategory, but furthermore a $T$-indexed lexcategory-under-$T$ (equivalently, $T/1$) by appeal to the first half of \cref{Lemma1} as applied to $T/- : \op{T} \to \LexCat$. Note that, for any particular object $t$ of $T$, this identifies $T/1$ inside $T/t$ via pullback along the unique map from $t$ to $1$; thus, $T$ is identified as the subcategory of \quote{constant} data within $T/t$. We may explicitly refer to this inclusion lexfunctor as $i_t : T \to T/t$. (Beware that in general, $i_t$ is faithful but not full!)

Now, what we want to show that given any fixed diagram in $\LexCat$ of this form
\[\begin{tikzcd}
	& T \\
	{T/t} && V
	\arrow["{i_t}"', from=1-2, to=2-1]
	\arrow["f", from=1-2, to=2-3]
\end{tikzcd}\]
there is a correspondence between commutative triangles extending this diagram with a lexfunctor from $T/t$ to $V$, and elements of $\Hom_V(1, f(t))$.

Well, suppose given $m : 1 \to f(t)$ in $V$. Consider now how the action of $f$ on arrows induces a functor $f' : T/t \to V/f(t)$. Because $f$ preserves finite limits, and finite limits in a category determine the finite limits in its slice categories, this functor $f'$ also preserves finite limits. Furthermore, by pullback along $m$, we get a lexfunctor $m^* : V/f(t) \to V$. Thus, altogether, we get a lexfunctor $m^* \circ f' : T/t \to V$. What's more, the following diagram commutes:

\[\begin{tikzcd}
	& T \\
	{T/t} && V \\
	& {V/f(t)}
	\arrow["{i_t}"', from=1-2, to=2-1]
	\arrow["f", from=1-2, to=2-3]
	\arrow["{f'}"', from=2-1, to=3-2]
	\arrow["{m^*}"', from=3-2, to=2-3]
\end{tikzcd}\]

To see that this diagram commutes, observe that it can be decomposed like so:

\[\begin{tikzcd}
	& T \\
	{T/t} & V & V \\
	& {V/f(t)}
	\arrow["{i_t}"', from=1-2, to=2-1]
	\arrow["f", from=1-2, to=2-3]
	\arrow["{f'}"', from=2-1, to=3-2]
	\arrow["{m^*}"', from=3-2, to=2-3]
	\arrow["f"', from=1-2, to=2-2]
	\arrow["{i_{f(t)}}"', from=2-2, to=3-2]
\end{tikzcd}\]

Here, $i_{f(t)}$ is the inclusion lexfunctor from $V$ into $V/f(t)$ in the analogous way to $i_t : T \to T/t$. The left half of this diagram commutes because, $f$ being a lexfunctor, the operations \quote{Pull back along the map from $t$ to $1$ and then apply $f$ to the resulting slice} and \quote{Apply $f$ and then pull back along the map from $f(t)$ to $1$} are the same. The right half of this diagram commutes because $m^* \circ i_{f(t)}$ is identity, because this amounts to pulling back along two morphisms in a row, ultimately pulling back along a path from $1$ to $1$, and any morphism from $1$ to $1$ is the identity. \TODOinline{Phrase this better?}

Thus, every element of $\Hom_V(1, f(t))$ induces a corresponding commutative triangle of lexfunctors from $i_t : T \to T/t$ to $f : T \to V$.

Conversely, any such commutative triangle induces an element of $\Hom_V(1, f(t))$. To see this, we just need to see that there is some global element of $i_t(t)$ already in $T/t$.

Note that $i_t(t)$ is one of the projections from $t \times t$ to $t$. Furthermore, the terminal object in $T/t$ is the identity slice from $t$ to $t$. So the following commutative triangle serves as a global element of $i_t(t)$ within $T/t$:

\[\begin{tikzcd}
	t && {t \times t} \\
	& t
	\arrow["id"', from=1-1, to=2-2]
	\arrow["{i_t(t)}", from=1-3, to=2-2]
	\arrow["{\langle \id, \id \rangle}", from=1-1, to=1-3]
\end{tikzcd}\]

Let us use the name $g_t$ for this global element of $i_t(t)$ within $T/t$.

Now, we need to show that these two processes are inverse. \TODO

Finally, we must show that the re-indexings within $T$
\end{proof}

Now, we shall prove \cref{SNCorrespondence}.

\begin{proof}
Let $T$ be a lexcategory, and let $C$ be some $T$-indexed lexcategory. By \cref{Lemma1} (keeping in mind the contravariance of the functors defining indexed structures), a map from the self-indexing $T/-$ to $C$ as $T$-indexed lexcategories is the same as a lexfunctor $\introS$ from $T$ to the global aspect of $C$, along with a map from $T/-$ to $C$ as $T$-indexed (lexcategories under $T$). The map $\introS$ will be used to treat $C$ as a lexcategory under $T$.

Next we apply \cref{Lemma2}. The map from $T/-$ to $C$ as $T$-indexed (lexcategories under $T$) is the same as choosing, in a natural way over all $t$ in $T$, some $t$-defined value in $\Hom_C(1, \introS(t))$. That is, a natural transformation from $t$ to $\Hom_C(1, \introS(t))$.
\end{proof}

\begin{remark}\label{IntrospGeneralDoctrine}
It wasn't fundamentally important that we were dealing with lexcategories here. \Cref{Lemma1} only required a terminal object. And for any algebraish notion extending categories-with-global-elements, there is some free construction $T[1 \to -]$. (Even the role terminality plays here is to some degree eliminable, though we have no interest for now in eliminating it). In particular, for categories-with-finite-products, there is also a simple independent account of the free structure $T[1 \to t]$, amounting to the full subcategory $T//t$ of $T/t$ on just those slices which are projections from some $t \times s$ to $t$ (the so-called \quote{simple self-indexing}). So we get the same result if we replace throughout \quote{lexcategory} and \quote{lexfunctor} by finite-product structure, and replace $T/-$ by this simple self-indexing. The same also holds for for any notion extending the notion of lexcategories which is automatically inherited by slice categories and preserved by pullback, or any notion extending the notion of finite-product-categories which is automatically inherited by these reduced slice categories and preserved by their pullbacks. For example, toposes or locally cartesian closed categories or cartesian closed categories or categories with finite products and finite coproducts over which they distribute or etc. Even infinitary notions could be used; categories with all limits of cardinality up to some regular cardinal, say.
\end{remark}

As a result of \cref{SNCorrespondence}, we can give another definition essentially equivalent to \cref{DefnPreIntrospIndexed}.

\begin{definition}\label{DefnPreIntrospSN}
A \defined{pre-introspective theory} is a lexcategory $T$, a $T$-indexed lexcategory $C$, a lexfunctor $\introS$ from $T$ to the global aspect of $C$, and a map $\introN$ from $t$ to $\Hom_C(1, S(t))$, natural in $t$ in $T$.
\end{definition}

As before, we may write out $\langle T, C, \introS, \introN \rangle$ to be fully explicit, but in typical abuse of language, may refer to a pre-introspective theory by simply naming $T$ or the pair $\langle T, C\rangle$. We will frequently use the same names $\introS$ and $\introN$ as though they apply simultaneously to all introspective theories (in the same way that notation like $+$ and $\times$ is overloaded as applicable to separate rings simultaneously).

\beginNamedManualIndexSort{lemma}{$\introS$ Matches $\introN$}{S Matches N}\label{SMatchesN}
Within a pre-introspective theory $\langle T, C \rangle$, let $t$ be some object of $T$ and let $x$ be some globally defined element of $t$. Then $\introS(x)$ and $\introN_t(x)$ yield equal global elements of $S(t)$ within the global aspect of $C$.
\closeNamed{lemma}
\begin{proof}
Apply the naturality square for $\introN$ to the morphism from $1$ to $t$ representing $x$. One path around the square will be the morphism in $T$ from $1$ to $\Hom_C(1, \introS(t))$ representing $\introN_t(x)$. The other path around the square will be the unique identity morphism in $T$ from $1$ to $\Hom_C(1, 1)$ representing the identity morphism on $1$, followed by the morphism from $\Hom_C(1, 1)$ to $\Hom_C(1, \introS(t))$ representing composition in $C$ with the morphism in $C$ representing $S(x)$ as a global element of $S(t)$. As composition with identity leaves values unchanged, this is also the morphism from $1$ to $\Hom_C(1, \introS(t))$ representing $S(x)$.
\end{proof}

As the name \quote{pre-introspective theory} suggests, there are some further definitions to build atop this.

\begin{definition}\label{DefnIntrosp}
If $\langle T, C \rangle$ is a pre-introspective theory and $C$ is small or locally small, then we say this is an \defined{introspective theory} or \defined{locally introspective theory}, respectively.
\end{definition}

While it is sometimes easier to prove theorems about (pre-)introspective theories by using \cref{DefnPreIntrospIndexed}, it will usually be easier to show structures actually are (pre-)introspective theories by using \cref{DefnPreIntrospSN}. But this is not the only benefit of \cref{DefnPreIntrospSN}. The value of this new definition is that there is much less data around to explicitly fuss about. In particular, when we wish to turn this into a lex definition in section \TODO, we will find the coherence conditions much easier to manage. It will also be easier to define the appropriate notion of homomorphisms between (pre-)introspective theories by thinking about \cref{DefnPreIntrospSN}.

\Cref{DefnPreIntrospSN} also allows us to quickly appreciate the significance of introspective theories from a functorial semantics point of view. An introspective theory is precisely an essentially algebraic theory (this is the role of $T$) extending the theory of lexcategories (this is the role of $C$), such that every model of the theory is equipped with a designated homomorphism (this is the role of $\introN$) into an internal model of the same theory (this is the role of $\introS$).

Recall that a small indexed lexcategory is one which is equivalent to some small indexed strict lexcategory (with possibly multiple non-isomorphic such choices available). It is occasionally of use to imagine some particular such choice has been pinned down, leading to the following definition.

\begin{definition}
By an \defined{inner-strict introspective theory}, we mean a pre-introspective theory $\langle T, C\rangle$ along with a specific choice for how to construe $C$ as an internal lexcategory; that is, as a small indexed strict lexcategory.
\end{definition}

Thus, an introspective theory can always be construed as some inner-strict introspective theory, though multiple non-isomorphic such choices may be available. \TODOinline{Cite how every small indexed non-strict lexcategory can become a small indexed strict lexcategory}

We call this \quote{inner strict} to emphasize that we've taken every choice concerning the representation of $C$ which was allowed to vary over non-isomorphic objects or non-equal parallel morphisms in $T$, and fixed some particular such choice for it, but we've not imposed strict structure on $T$ itself.

\subsection{Some useful constructions}
\begin{construction}\label{IntrospInternalMap}
If $\langle T, C, \introF \rangle$ is a pre-introspective theory, and any lexfunctor $G : C \to D$ is given for some other $T$-indexed lexcategory $D$, then $\langle T, D, G \circ \introF \rangle$ is itself a pre-introspective theory, like so: 

\[\begin{tikzcd}
	{\op{T}} && {\LexCat}
	\arrow["{T/-}"{name=0}, from=1-1, to=1-3, shift left=5]
	\arrow["{C}"{name=1, description}, from=1-1, to=1-3]
	\arrow["{D}"{name=2, swap}, from=1-1, to=1-3, shift right=5]
	\arrow[Rightarrow, "{\introF}"', from=0, to=1]
	\arrow[Rightarrow, "{G}"', from=1, to=2]
\end{tikzcd}\]

Of course, this yields an introspective or locally introspective theory just in case $D$ is small or locally small, respectively.
\end{construction}

\begin{construction}\label{IntrospSlice}
If $\langle T, C, \introF \rangle$ is a pre-introspective theory, and $t$ is any object in $T$, then the slice category $T/t$ can be equipped in a natural way as a pre-introspective theory as well. If we start from an introspective or locally introspective theory, then so respectively will be the result of this construction.
\end{construction}
\begin{proof}[Details]
Consider the forgetful functor $\Sigma : T/t \to T$. Composition and whiskering with this gives us the triple $\langle T/t, \Sigma C, \Sigma \introF \rangle$, like so:

\[\begin{tikzcd}
	{\op{\left(T/t\right)}} & {\op{T}} && {\LexCat}
	\arrow["{T/-}"{name=0}, from=1-2, to=1-4, shift left=2]
	\arrow["{C}"{name=1, swap}, from=1-2, to=1-4, shift right=2]
	\arrow["{\Sigma}", from=1-1, to=1-2]
	\arrow[Rightarrow, "{\introF}"', from=0, to=1]
\end{tikzcd}\]

This is a pre-introspective theory as the top composite is the same as the self-indexing $(T/t)/-$ of $T/t$ (that is, slice categories within slice categories are obtained by first applying the forgetful functor into the ambient category and then taking the ordinary slice category).

When $C$ is small or locally small, then so respectively is $\Sigma C$. The representing objects for $\Sigma C$ will simply be those for $C$, mapped into $T/t$ via $\Sigma$'s right adjoint (i.e., pulled back along the unique morphism from $t$ to $1$).
\end{proof}

When we abuse language and speak of $T/t$ as an introspective theory, the above construction is what we mean. That said, there is another natural way to equip $T/t$ as an introspective theory as well.

\begin{TODOblock}
There is another way to equip $T/t$ as an introspective theory, in which we use as our internal category the slice category $C/t$ so to speak. \Cref{IntrospSlice} is actually equivalent to doing this and then applying \cref{IntrospInternalMap} to the internal forgetful lexfunctor from $C/t$ to $C$. This equivalence is easier to see once we see how the natural transformation taking $t$ to $\Hom_C(1, t)$ works. (Which we now have sketched at \cref{SNCorrespondence}). Perhaps it's also true that in the other direction, we apply \cref{IntrospInternalMap} to the constant lexfunctor from $C$ to $C/t$? Huh, and in the first case, I might have thought the relevant lexfunctor from $C/t$ to $C$ was pullback along the generic global element of $t$? Hm. Anyway, let us think about this more later.

Having thought about it more, I think none of this is really worth mentioning. The point is that we can use \cref{IntrospInternalMap} in conjunction with any of the following functors: The forgetful (i.e., summation) functor from $C/t$ to $C$, the "instantiation" pullback from $C/t$ to $C$ along the generic global element of $t$ within $T/t$, or the "delta"/"constant" functor from $C$ to $C/t$. And going from $C$ to $C/t$ and then back to $C$ via delta and instantiation is the same as identity. I think that's all any of this amounts to, but it's not of interest. The interesting thing for introspective theories is just the category noted above.
\end{TODOblock}

\begin{TODOblock}
Make the useful observations that the theory of introspective theories is essentially lex, and that we can take therefore take products of introspective theories in the straightforward way.
\end{TODOblock}

\subsection{The box operator}
\TODOinline{Think and write about this for pre-introspective theories too}

The following notation will be very convenient for us going forward. It is also suggestive of the connections with modal logic we will explore in a later chapter.

Let $\langle T, C \rangle$ be a locally introspective theory.

We say a presheaf on $C$ is locally $T$-small if the map from its category of elements to $C$ has $T$-small fibers. In other words, $P(c)$ is represented by an object of $T/t$ for each $t$-definable object $c$ of $C$.

Note that the category of locally $T$-small presheaves on $C$ is itself a $T$-indexed lexcategory. We will refer to this as $\Psh{C}$.

Thus, we have three $T$-indexed lexcategories of note: $T$ itself (considered as a $T$-indexed category through the self-indexing $T/-$), $C$, and $\Psh{C}$.

Between these, we also have a cycle of $T$-indexed lexfunctors, like so:

\[\begin{tikzcd}
	&& {T/-} \\
	\\
	C &&&& {\Psh{C}}
	\arrow["\introF"', from=1-3, to=3-1]
	\arrow["{c \; \mapsto \Hom_{C}(-, c)}"', from=3-1, to=3-5]
	\arrow["{P \mapsto P(1)}"', from=3-5, to=1-3]
\end{tikzcd}\]

Here, the bottom arrow is the Yoneda embedding, sending each object of $C$ to the corresponding representable presheaf. The right arrow takes a presheaf to its global elements. The left arrow is the $\introF$ which is part of the structure of an introspective theory.

In general, we will write $\Box$ for a roundtrip around this diagram, starting from any of its three nodes.

Thus, we will write $\Box$ for the $T$-indexed lexfunctor from $T$ to itself given by $t \mapsto \Hom_C(1, \introF(t))$.

We will ALSO write $\Box$ for the $T$-indexed lexfunctor from $C$ to itself given by $c \mapsto \introF(\Hom_C(1, c))$.

And we will ALSO write $\Box$ for the $T$-indexed lexfunctor from $\Psh{C}$ to $\Psh{C}$, which sends the presheaf $P$ to the presheaf represented by $\introF(P(1))$.

As the Yoneda embedding is naturally thought of as the inclusion of a full subcategory, identifying $C$ with the corresponding representable presheaves within $\Psh{C}$, we may also think of this last instance of $\Box$ as a $T$-indexed lexfunctor from $\Psh{C}$ to $C$.

Note that, having set up these various notions of $\Box$, we find that $\Box$ and each of the maps in the diagram \quote{commutes} in the appropriate sense; that is, they can be seen as preserving each other.

For example, $\introF$ preserves $\Box$, in that both $\introF(\Box(-))$ and $\Box(\introF(-))$ yield the same $T$-indexed functor from the self-indexing of $T$ to $C$. This is readily seen by unwinding their definitions: These are both $\introF(\Hom_C(1, \introF(-)))$.

We also have that taking $\Hom_C(1, -)$ preserves $\Box$ in the same way. $\Hom_C(1, \Box(-)) = \Box(\Hom_C(1, -)) = \Hom_C(1, \introF(\Hom_C(1, -)))$. More generally, consider the map $G$ which assigns to every locally $T$-small presheaf $P$ upon $C$ its object of global elements $P(1)$. Then $G$ preserves $\Box$ in the same way: $G(\Box P) = \Box(G(P)) = \Hom_C(1, \introF(P(1)))$, where we make sense of $G(\Box P)$ by identifying $\Box P$ within $c$ with the presheaf it represents.

Given object $c$ of $C$, and locally $T$-small presheaf $P$ on $C$, we will write $c \implies P$ to indicate the exponential presheaf $P^c$. That is, the presheaf $P(- \times c)$. We may also write $c \implies d$ where $d$ is an object of $C$ as well, to mean $c \implies P$ for the presheaf $P = \Hom_C(-, d)$ represented by $d$. Thus, $c \implies d$ is the presheaf $\Hom_C(- \times c, d)$.

Note that for any locally $T$-small presheaf $P$ on $C$ and object $c$ of $C$, we have that $P(c)$ can be identified with the presheaf $c \implies P$ evaluated at $1$, and thus $\introF(P(c))$ can be identified with $\Box(c \implies P)$. In particular, by considering the case when $P$ is represented by object $d$, we find that $\Box(c \implies d)$ is naturally identifiable with $\introF(\Hom_C(c, d))$.

The above was all discussed for $T$, $C$, and $\Psh{C}$ considered as $T$-indexed lexcategories, but this all descends to corresponding structure on their global aspects as well. Keep in mind, in the global aspect context, $\introF$ is the same as $\introS$, so wherever in the above we discussed $\introF$, a corresponding statement holds as well for $\introS$, when considering just the global aspect.

The choice of this $\Box$ notation for these purposes is meant to convey an analogy with the $\Box$ operator of modal logic, and in particular, with the provability operator of provability logic. We will explore this more in a later chapter. \TODO