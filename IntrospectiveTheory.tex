\filestart

\section{Introspective theories}

\subsection{Preview}
In this chapter, we introduce the central object of our interest, the notion of an \quote{introspective theory}.

An introspective theory is an essentially algebraic theory such that every model of the theory includes a lexcategory with an internal model of the same theory, as well as a homomorphism from the overall model into the global aspect of the internal model.

We will give two formal definitions of an introspective theory, and prove them equivalent. The second formal definition we give will directly correspond to the previous paragraph. The first formal definition we give will be a bit more compact, but framed in the language of indexed categories.

En route to discussing introspective theories, we also discuss some more general notions we call \quote{pre-introspective theories}, \quote{locally introspective theories}, and so on, which will be of some use to us as well.

% Non-evil definition
\subsection{First definition (indexed style)}

\begin{definition} \label{DefnPreIntrospIndexed}
A \defined{pre-introspective theory} is a lexcategory $T$, a $T$-indexed lexcategory $C$, and a lexfunctor $\introF$ from the self-indexing of $T$ to $C$, like so:

\[\begin{tikzcd}
	{\op{T}} && {\LexCat}
	\arrow["{T/-}"{name=0}, from=1-1, to=1-3, shift left=2]
	\arrow["{C}"{name=1, swap}, from=1-1, to=1-3, shift right=2]
	\arrow[Rightarrow, "{\introF}"', from=0, to=1]
\end{tikzcd}\]
\end{definition}

We write out the triple $\langle T, C, \introF \rangle$ to refer to a pre-introspective theory when we wish to be fully explicit about its structure. But in typical abuse of language, we also often refer to it simply by the name of its underlying lexcategory $T$ or of the pair $\langle T, C \rangle$, when this would not cause confusion. We will frequently use the same name $\introF$ as though it applies to all introspective theories simultaneously, in the same way that notation like $+$ or $\times$ is overloaded as applying over all rings simultaneously.

\begin{definition} \label{DefnIntrospIndexed}
An \defined{introspective theory} is a pre-introspective theory $\langle T, C \rangle$ in which $C$ is \repsmall/.
\end{definition}

We shall show in later chapters how this simple concept of an introspective theory already suffices to exhibit and capture all the fundamental phenomena of \Goedel/\ codes, diagonalization, the \Goedel/\ incompleteness theorems, and \Loeb/'s theorem. And we shall show that all the traditional instances of \Goedel/'s incompleteness phenomena arise from special cases of this purely algebraic structure. We will also demonstrate functorial fixed point results for this structure, and show some interesting applications of these.

We shall also introduce some further generalizations of this concept, in order to be able to state results along the way in their natural generality or point out connections to related work or interesting structures that are not quite introspective theories per se but are closely related. But throughout these notes, if at any time the abstractions seem daunting or distracting, remember that the concrete concept which matters most is the concept of an introspective theory as defined above.

The example-oriented reader may immediately demand an example of a pre-introspective theory, to orient themselves. Here is the simplest example (or class of examples) of a pre-introspective theory:

\begin{example} \label{TrivialPreIntrospIndexed}
Let $T$ be one's favorite lexcategory, any lexcategory of one's choice. Then a pre-introspective theory can be made from this where we take $\introF$ to be the identity (with $C$ as the self-indexing of $T$).
\end{example}

Alas, this simple example of a pre-introspective theory is almost never an introspective theory (that is to say, a lexcategory's self-indexing is almost never \repsmall/; indeed, the only case in which this happens is the trivial one where $T$ is the terminal category\footnote{This can be seen as a consequence of our categorical \Loeb/'s theorem established in a later chapter.}).

Here, then, is an equally simple, or perhaps even simpler, example of an introspective theory:

\begin{example}
Let $T$ be one's favorite lexcategory, any lexcategory of one's choice. Then a pre-introspective theory can be made from this where we take $C$ to be the terminal indexed lexcategory, with $\introF$ as the unique map into this.
\end{example}

This is indeed an introspective theory (the terminal lexcategory being \repsmall/ over any indexing category with a terminal object). But alas, although this last example can be as nontrivial as one likes in terms of the structure of $T$, it is of course trivial in all its further structure.

Nontrivial introspective theories do exist and we will give some archetypal examples of them soon enough. But in order to do so, it will be convenient to first develop some further machinery on how (pre-)introspective theories may be presented.

\subsection{Second definition (non-indexed style)}

We shall now make an observation about an alternative but equivalent way to specify the data of a pre-introspective theory.

\begin{theorem}\label{SNCorrespondence}
Given a lexcategory $T$ and a $T$-indexed lexcategory $C$, specifying a pre-introspective theory $\langle T, C, \introF \rangle$ (i.e., specifying a $T$-indexed lexfunctor from the self-indexing $T/-$ to $C$) is equivalent to specifying a (non-indexed) lexfunctor $\introS$ from $T$ to the global aspect of $C$, along with specifying maps from each $t \in \Ob(T)$ to $\Hom_C(1, \introS(t))$, naturally in $t$.
\end{theorem}

\begin{proof}
Let $T$ be a lexcategory, and let $C$ be some $T$-indexed lexcategory. By \magicref{Lemma1} (keeping in mind the contravariance of the functors defining indexed structures), a map from the self-indexing $T/-$ to $C$ as $T$-indexed lexcategories is the same as a lexfunctor $\introS$ from $T$ to the global aspect of $C$, along with a map from $T/-$ to $C$ as $T$-indexed objects of $T/\LexCat$ (where the map $\introS$ is used to treat $C$ as a $T$-indexed object of $T/\LexCat$).

Next we apply \magicref{SelfIndexingIsFree}. The map from $T/-$ to $C$ as $T$-indexed objects of $T/\LexCat$ is the same as choosing, in a natural way over all $t$ in $T$, some $t$-defined value in $\Hom_C(1, \introS(t))$. That is, maps from each $t \in \Ob(T)$ to $\Hom_C(1, \introS(t))$, comprising a natural transformation.
\end{proof}

\begin{remark}\label{IntrospGeneralDoctrine}
It wasn't fundamentally important that we were dealing with lexcategories here. The use of \magicref{Lemma1} as applied to $\op{C}$ only required a terminal object in $C$. And for the invocation of \magicref{SelfIndexingIsFree}, we only needed that there is some free construction of adjoining global elements. (Even the role terminality plays here is to some degree eliminable, though we have no interest for now in eliminating it). In particular, we get a completely analogous result when lexcategories are replaced throughout by any of the structures noted in \magicref{SelfIndexingIsFreeCorollary}, including for categories with finite products using the simple self-indexing.
\end{remark}

As a result of \cref{SNCorrespondence}, we can give an alternative definition equivalent to \cref{DefnPreIntrospIndexed}:

\begin{definition}\label{DefnPreIntrospSN}
A \defined{pre-introspective theory} is a lexcategory $T$, a $T$-indexed lexcategory $C$, a lexfunctor $\introS$ from $T$ to the global aspect of $C$, and a natural transformation $\introN$ from each $t \in \Ob(T)$ to $\Hom_C(1, \introS(t))$.
\end{definition}

Much as before, we may write out $\langle T, C, \introS, \introN \rangle$ to be fully explicit, but in typical abuse of language, will refer to a pre-introspective theory by simply naming $T$ or the pair $\langle T, C\rangle$. We will frequently use the same names $\introS$ and $\introN$ as though they apply simultaneously to all such structures (in the same way that notation like $+$ and $\times$ is overloaded as applicable to separate rings simultaneously).

The definition of an introspective theory remains exactly as before (\cref{DefnIntrospIndexed}) regardless of how one thinks of pre-introspective theories, but for reminder's sake:

\begin{definition} \label{DefnIntrospSN}
An \defined{introspective theory} is a pre-introspective theory $\langle T, C \rangle$ in which $C$ is small.
\end{definition}

While it may sometimes be easier to prove theorems about (pre-)introspective theories by using \cref{DefnPreIntrospIndexed}, it will often be easier to show structures actually are (pre-)introspective theories by using \cref{DefnPreIntrospSN}. But this is not the only benefit of \cref{DefnPreIntrospSN}. The reduction of the full indexed lexfunctor $\introF$ to just its global aspect ($\introS$) and a natural transformation between 1-functors means much less data around to explicitly fuss about. In particular, when we wish to turn this into a lex definition in section \TODO, we will find the appropriate coherence conditions much easier to manage. It will also be easier to define the appropriate notion of homomorphisms between (pre-)introspective theories by thinking about \cref{DefnPreIntrospSN}.

\Cref{DefnPreIntrospSN} also allows us to quickly appreciate the significance of introspective theories from a functorial semantics point of view. An introspective theory is precisely an essentially algebraic theory (this is the role of $T$) extending the theory of lexcategories (this is the role of $C$), such that every model of the theory (which thus has an underlying lexcategory as its interpretation of $C$) is equipped with a designated homomorphism (this is the role of $\introN$) into an internal model of that same theory in its underlying lexcategory (this is the role of $\introS$). In short, every model has a homomorphism into a further internal model.

\subsection{Some concrete examples}

Let us now finally give the example-oriented reader a nontrivial example of an introspective theory by which to orient themselves. (On the other hand, the reader who prefers to consider abstract definitions without immediately diving into a worked out example of a highly concrete flavor may skip this section at this introductory time if they find its details a distraction. To each reader at their own taste!).

\subsubsection{Example based on a traditional logical theory}
In this subsection, we will give two closely related examples. The first example we present is somewhat atypical of general introspective theories, but important nonetheless. It is very similar to the arithmetic universe constructions considered by Joyal in his account of \Godel/'s incompleteness theorem and by others following up on this (Joyal himself never published this work, but a detailed account has been given in \autocite{van2020g}, building off the formalization of the initial arithmetic universe given in \autocite{maietti10a}). 

Although very similar, the category we use in this first example is not exactly the same as the initial arithmetic universe considered in \autocite{van2020g} and \autocite{maietti10a}. The variant and presentation we give is intended to feel natural to an audience of traditional logicians. The connection of this construction to the initial arithmetic universe will be discussed in more detail later at \magicref{Sigma1Model}. 

After having given this first example, we will then tweak it slightly into another introspective theory which provides much better intuition for the general nature of introspective theories.

\newcommand{\Zfin}{\mathrm{Z}}
\newcommand{\ZfinSigma}{\mathrm{Z}_{\Sigma_1}}
\newcommand{\InnerZfin}{\mathrm{Z}'}
\newcommand{\InnerZfinSigma}{\mathrm{Z}_{\Sigma_1}'}
\begin{construction}\label{SigmaModelSimple}
Let us start with the first-order logical theory ZF-Finite: This is the theory ZF but with the axiom of infinity replaced by its negation\footnote{This theory happens to be equivalent to Peano Arithmetic in a suitable sense, but it will be more convenient for us to speak in terms of ZF-Finite so as not to fret about codings of a sort every modern mathematician readily takes for granted in a ZF-style context.}. The universe this theory describes is the hereditarily finite sets $V_{\omega}$. Throughout this construction, whenever we speak of formulae, we mean formulae in the language of ZF-Finite, and whenever we speak of provability, we mean provable within ZF-Finite.

Certain formulae are $\Sigma_1$. These are the formulae which consist of an initial string of unbounded existential quantifiers (ranging over the entire universe), after which all other quantifiers are bounded (ranging only over the elements of some particular definable finite set\footnote{Definable in the sense that it is given by a term in the language of ZF, when this language is taken to include the standard constructors of ZF-Finite (e.g., powerset, pairing, etc) as function symbols. This term may contain parameters given by previously quantified variables. We trust the reader understands the standard notion of $\Sigma_1$ formulae in the ZF-Finite context or equivalent Peano Arithmetic context.}).

Now let us define a category whose objects are the $\Sigma_1$ formulae with one free variable. Such formulae amount to certain definable subsets of the universe $V_{\omega}$; that is, they describe classes of hereditarily finite sets. (Note that the classes these formulae describe may themselves be infinite! For example, the tautologically true formula describes the class of all hereditarily finite sets.)

Given two such objects $\phi(n)$ and $\psi(m)$, we take as morphisms between these any $\Sigma_1$ formula $F(n, m)$ which provably acts as the graph of a function between the corresponding classes. That is, such that both $\forall n, m . F(n, m) \implies (\phi(n) \wedge \psi(m))$ and $\forall n . \phi(n) \implies \exists! m . F(n, m)$ are provable.

Two such morphisms $F(n, m)$ and $G(n, m)$ are considered equal just in case $\forall n, m . F(n, m) \biimplies G(n, m)$ is provable. 

Finally, morphisms compose in the expected way for graphs of functions; that is, the composition of $F(n, p)$ with $G(p, m)$ is given by $(G \circ F)(n, m) = \exists p (F(n, p) \wedge G(p, m))$.

We omit here the straightforward details of verifying that this structure we have just described does indeed satisfy the rules to be a category. Indeed, it is furthermore a regular category (that is, it has finite limits and pullback-stable image factorization; it has finite products because of the definability of ordered pairs in ZF-Finite, and it furthermore has equalizers and image factorization using suitable instances of Separation in ZF-Finite). However, it is not an exact category (that is, not every equivalence relation in this category admits a corresponding quotient). Let $\ZfinSigma$ be its ex/reg completion.

(There is not in general any need for the categories involved in an introspective theory to be exact, or even regular. They need only have finite limits. However, for the particular construction we are outlining now, this ex/reg completion is the $\ZfinSigma$ we need to look at.)

More explicitly, we can describe $\ZfinSigma$ like so:

Its objects are the $\Sigma_1$ binary relations $\phi(n, m)$ which can be proven to be partial equivalence relations (i.e., symmetric and transitive), thus corresponding to certain subquotients of the universe of all hereditarily finite sets.

Given any two such formulae $\phi(n_1, n_2)$ and $\psi(m_1, m_2)$, a morphism in $T$ from $\phi$ to $\psi$ is a $\Sigma_1$ formula $F(n, m)$ which provably corresponds to the graph of a function between the corresponding subquotients of the universe. That is, such that the universal closures of all the following are provable:

$F(n, m) \implies \phi(n, n) \wedge \psi(m, m)$

$\phi(n_1, n_2) \wedge \psi(m_1, m_2) \wedge F(n_1, m_1) \implies F(n_2, m_2)$

$\phi(n, n) \implies \exists m [F(n, m)]$

$F(n, m_1) \wedge F(n, m_2) \implies \psi(m_1, m_2)$.

Two such formulae $F(n, m)$ and $F'(n, m)$ are considered to be equal as morphisms from $\phi$ to $\psi$ if they are provably equivalent (that is, if both $\forall n, m . F(n, m) \implies F'(n, m)$ and $\forall n, m . F'(n, m) \implies F(n, m)$ are provable).

Given morphisms $F : \phi \to \psi$ and $G: \psi \to \chi$ of this sort, we again define their composition in the usual way of composing functions represented as graphs, as $(G \circ F)(n, m) = \exists p [F(n, p) \wedge G(p, m)]$.

This all describes the category $\ZfinSigma$, which one can verify is indeed a category and moreso, an exact category.

Note that our construction of $\ZfinSigma$ is such that the objects of $\ZfinSigma$, the morphisms of $\ZfinSigma$, the equality relation on morphisms of $\ZfinSigma$, the composition structure of $\ZfinSigma$, the finite limit structure of $\ZfinSigma$, etc, are all definable within the language of ZF-Finite; indeed, all definable by $\Sigma_1$ formulae. (In particular, keep in mind that provability in ZF-Finite is itself a $\Sigma_1$ property). Thus, there is a lexcategory $\InnerZfinSigma$ internal to $\ZfinSigma$ which corresponds to this very same construction of $\ZfinSigma$ we have just described. And we have a lexfunctor $\introS$ from $\ZfinSigma$ to the global aspect of $\InnerZfinSigma$ which sends each piece of the construction of $\ZfinSigma$ to the corresponding piece of the construction of $\InnerZfinSigma$. This is all straightforward.

As the last bit of introspective theory structure, we must build a natural transformation $\introN$ from the identity endofunctor on $T$ to the endofunctor $\Hom_{\InnerZfinSigma}(1, \introS(-))$ on $\ZfinSigma$. The core idea behind this $\introN$ is simple. Essentially, to every hereditarily finite set $x$, we can assign it a code $\code{x}$, which is an explicit term in the language of ZF-Finite denoting that set. The easy way to do this is to recursively assign to each set $\{a, b, c, \ldots\}$ the term describing a finite set whose members are explicitly enumerated by the terms assigned to $a, b, c, \ldots$. We thus send a set such as $\{\{\}, \{\{\}\}\}$ to the term in the language of ZF-Finite which might be called \quote{$\{\{\}, \{\{\}\}\}$} within quotation marks, and so on.

This gives us a function $\code{-}$ from hereditarily finite sets to terms in the language of ZF-Finite which describe hereditarily finite sets. This function $\code{-}$ is definable by a $\Sigma_1$ formula and thus gives a morphism in $T$. This serves as the component of $\introN$ at the object of $\ZfinSigma$ describing the collection of ALL hereditarily finite sets. 

(The categorically oriented reader may think of this recursive definition of $\code{-}$ as a catamorphism, where the collection of all hereditarily finite sets is understood as the initial algebra for the covariant finite powerset functor.)

All the other objects of $\ZfinSigma$ are subquotients of that object (and similarly for the objects of $\InnerZfinSigma$), and therefore the components of the natural transformation $\introN$ at these other objects can now be obtained uniquely so long as certain factorizations exist. That is to say, the component of $\introN$ at any object $\phi(n_1, n_2)$ of $T$ will also be given by the action of $\code{-}$, but for this to indeed work to map $t$ into $\Hom_{\InnerZfinSigma}(1, \introS(t))$, we need to know that $\code{-}$ when acting on individuals which are related by the partial equivalence relation $\phi$ produces terms which provably describe individuals related by $\phi$.

This is where the $\Sigma_1$-ness of $\phi$ plays a vital role. We can prove that, for any $\Sigma_1$ property $\phi$, for all $x$, whenever $\phi$ holds of $x$, it furthermore provably holds of $x$ (in the sense that the particular term $\code{x}$, when substituted into the argument of the particular formula defining $\phi$, yields a sentence which is derivable in the formal system ZF-Finite).

Finally, let us observe the naturality of this $\introN$. Consider the general form of its naturality squares:

\[\begin{tikzcd}
	\phi & \psi \\
	{\Hom_{\InnerZfinSigma}(1, \introS(\phi))} & {\Hom_{\InnerZfinSigma}(1, \introS(\psi))}
	\arrow["{\code{-}}"', from=1-1, to=2-1]
	\arrow["m", from=1-1, to=1-2]
	\arrow["{\code{-}}", from=1-2, to=2-2]
	\arrow["{\introS(m) \circ -}"', from=2-1, to=2-2]
\end{tikzcd}\]

This says that, for any definable unary formula $m(x)$, it is provably the case that for every $x$, we have that applying the function $m$ to $x$ and then constructing the term encoding the result ($\code{m(x)}$) is a provably equivalent term to taking the term representing $x$ and substituting it into the argument of the formula defining $m$ (what might be called $m(\code{x})$ or perhaps $\code{m}(\code{x})$ or at any rate $\introS(m)(\code{x})$). To be clear, by the provable equivalence of terms here, we do not mean syntactic identity as symbol-strings; rather, we mean that there is a provable equality sentence whose left and right sides are comprised of these terms. That is, whatever the actual result of the function $m$ on the input $x$ is, we must have that this is also provably the same as applying $m$ to the input $x$. Here, again, the $\Sigma_1$-ness of the formula defining $m$ comes to our rescue, telling us that truth entails provability in the appropriate way.

Thus, we obtain an introspective theory $\langle \ZfinSigma, \InnerZfinSigma, \introS, \introN \rangle$. This concludes our first example of an instrospective theory!
\end{construction}

\bigskip
However, $\langle \ZfinSigma, \InnerZfinSigma, \introS, \introN \rangle$ is not actually the most typical introspective theory! It has special properties which we should not expect of a general introspective theory. Its internal $\InnerZfinSigma$ acts as a perfect mirror image of $\ZfinSigma$, and can thus itself be equipped as an internal introspective theory. The internal $\InnerZfinSigma$ has in some informal sense no further objects (or morphisms, or equations) beyond the range of $\introS$. All of this is not typical for an introspective theory.

\begin{construction}
Let us describe now a more archetypal introspective theory, to guide the reader's intuitions better for how general introspective theories act.

Throughout the construction of $\ZfinSigma$, we have imposed a $\Sigma_1$ constraint on formulae (both on the formulae defining objects and on the formulae defining morphisms). If we drop all such $\Sigma_1$ constraints and allow arbitrary formulae, we get by the same construction an analogous category $\Zfin$. $\ZfinSigma$ sits inside $\Zfin$ as a subcategory (but not a full subcategory! The inclusion from $\ZfinSigma$ into $\Zfin$ is faithful, but not full).

Just as the construction of $\ZfinSigma$ could itself be carried out in ZF-Finite to get an $\InnerZfinSigma$ internal to $\ZfinSigma$, so too can the construction of $\Zfin$ can be carried out in ZF-Finite, to get an $\InnerZfin$ internal to $\ZfinSigma$. Yes, this $\InnerZfin$ is internal to $\ZfinSigma$, not just internal to $\Zfin$! Even though $\Zfin$ includes as its objects and morphisms formulae which are not $\Sigma_1$, the description of $\Zfin$ (as a category whose objects are symbol-strings for which certain other symbol-strings exist, and whose morphisms are symbol-strings for which certain other symbol-strings exist, and so on) is $\Sigma_1$.

Finally, the inclusion of $\ZfinSigma$ into $\Zfin$ yields, analogously, an inclusion from $\InnerZfinSigma$ into $\InnerZfin$, internal to $\ZfinSigma$. This means the functor $\introS$ from $\ZfinSigma$ into the global aspect of $\InnerZfinSigma$ can just as well be thought of as having $\InnerZfin$ for its codomain, and similarly the natural transformation $\introN$ can just be well as thought of in this context. (This way of making one introspective theory from another is an instance of the general construction \magicref{IntrospInternalMap}.)

Summarizing, we get an introspective theory $\langle \ZfinSigma, \InnerZfin, \introS, \introN \rangle$, where $\ZfinSigma$ is the lexcategory of $\Sigma_1$-definable hereditarily finite sets and $\Sigma_1$-definable functions between them up to provable equivalence in ZF-Finite, $\InnerZfin$ is the lexcategory internal to $\ZfinSigma$ of arbitrary definable sets and arbitrary definable functions between them up to provable equivalence in ZF-Finite, $\introS$ assigns to each piece of $\ZfinSigma$ the corresponding (globally defined) piece of $\InnerZfin$, and $\introN$ is the definable function which sends any hereditarily finite set to the canonical term describing it, as well as witnessing the provable entailment from truth to provability for $\Sigma_1$ formulae.
\end{construction}

Phew! What a long walk it was to get to describing that example! All the better, then, that we have formalized introspective theories so abstractly, and can work with them without having to fuss about such concrete details as in that example. But this is indeed the archetypal example it will be best to keep in mind to guide the reader's intuition throughout all further discussion.

\begin{warningenv}
While we have above constructed introspective theories $\langle \ZfinSigma, \InnerZfinSigma \rangle$ and $\langle \ZfinSigma, \InnerZfin \rangle$, the reader should be cautioned that there is no natural introspective theory $\langle \Zfin, \InnerZfin \rangle$ or the like. As a check of their understanding, the reader is encouraged to think about why this is.
\end{warningenv}

\subsubsection{Example based on a presheaf category}
In this subsection, we will give an example of an introspective theory based on presheaf categories.

\TODOinline{Finish writing out the details on this example. Get rid of the "hopefully" words. Write out a naive version first, without size conditions, then show the obstruction, to motivate the size condition.}

\begin{construction}
The following example takes place in the context of a metatheory such as ZF.

In the following, by a ZF-set, I mean a set in the sense of ZF; that is, an element of the cumulative hierarchy, which comes with a well-defined rank (keep in mind that two bijective ZF-sets may have very different ranks!).

We make use both of the fact that ZF is a material rather than structural set theory (so that ZF-sets come with well-defined ranks), and of the fact that ZF's Axiom of Replacement ensures the existence of the cumulative hierarchy at least as far as (indeed much further than) $V_{\omega^2}$. [Note that the following construction could not be carried out as written in a weaker context such as the internal logic of a topos with natural numbers object, where the cumulative hierarchy needn't go further than $V_{\omega \times 2}$.

First of all, consider some sequence $W_1, W_2, W_3, \ldots$, indexed by positive integers, where each $W_i$ is a full sublexcategory of $\Set$, and each $W_j$ can be seen as an internal lexcategory within $W_i$ for $j < i$.

For example, we can take the positive limit stages of the von Neumann hierarchy, with $W_i = V_{\omega \times i}$ for positive integer $i$. Each of these is a full sublexcategory of $\Set$ (closed under finite limits because ranks of finite limits can be taken as at most finitely greater than the supremum of the ranks of all factors), and each $W_j$ can be seen as an internal lexcategory within $W_i$, for $j < i$.

Now, consider the positive integers both as a poset $P$ and as a discrete category $|P|$. Of course, the latter has an inclusion functor into the former.

We can consider presheaves on $P$, $\Psh{P}$, and we can consider presheaves on $|P|$, $\Psh{|P|}$. The inclusion functor $|P| \to P$ gives us a forgetful functor $\Psh{P} \to \Psh{|P|}$. Because limits in presheaf categories are computed componentwise, this forgetful functor is a lexfunctor.

Now let us consider a particular full subcatgory of $\Psh{|P|}$. An object of $\Psh{|P|}$ is a sequence of sets $s_1, s_2, s_3, \ldots$. The full subcategory we are interested in is comprised of those objects where each $s_i$ is one of the sets in $W_i$. This is like imposing a smallness constraint, but the sense of \quote{smallness} is not constant but grows ever looser as we go further into the sequence.

This full subcategory of $\Psh{|P|}$ is equivalently the product of $W_i$ over all positive integers $i$. Because the $W_i$ are closed under finite limits, we have that this is indeed a full sublexcategory of $\Psh{|P|}$.

This full sublexcategory of $\Psh{|P|}$ induces as its preimage under the forgetful lexfunctor $\Psh{P} \to \Psh{|P|}$ also a full sublexcategory of $\Psh{P}$. Let us call this $T$. That is, $T$ comprises those presheaves on the positive integers whose $i$-th set is an element of $W_i$.

This is the first piece of the introspective theory we are building here.

Secondly, let us consider a particular internal lexcategory $C$ within $T$. Specifically, the $i$-th component of $C$ is the product of the lexcategories $W_j$ over all $j < i$. As for the action of the presheaf as we go from higher-indexed components to lower-indexed components, we simply project away the no longer needed factors.

Hopefully it is clear what I mean by this and clear that it does indeed define a lexcategory $C$ internal to $T$. (In particular, all the lexcategorical structure commutes appropriately with the presheaf actions because all the lexcategorical structure is defined componentwise on these products and the presheaf actions just project away components.)

What's more, hopefully it is clear that the global aspect of $C$ is precisely the product of $W_i$ over all positive integers $i$. This is precisely the full sublexcategory of $\Psh{|P|}$ which $T$ maps into under our aforementioned forgetful lexfunctor. Thus, said forgetful functor acts as a lexfunctor $\introS$ from $T$ into the global aspect of $C$.

Under this correspondence, it is hopefully also clear that given an object $c$ in the global aspect of $C$, which can be viewed as a sequence of sets $s_1, s_2, s_3, \ldots$ with $s_i$ in $W_i$, the corresponding object $\Hom_C(1, c)$ in $T$ is such that its $i$-th set is the product of $s_j$ over $j < i$, with presheaf action again given by projecting away unneeded factors.

Thus, the natural transformation $\introN$ that we need to make this an introspective theory amounts to the following: Its component at any object of $T$ given by sets $s_1, s_2, s_3, \ldots$, should have an $i$-th component which is a function from $s_i$ to the product of $s_j$ over all $j < i$. We obtain this function just from the presheaf action for the given object of $T$, and observe that this does indeed satisfy the appropriate properties to comprise a morphism in $T$. For the naturality of this transformation, we need to know that any morphism in $T$ commutes appropriately with these presheaf actions, and indeed it will by the definition of morphisms in $T$.

Thus, $\langle T, C, \introS, \introN \rangle$ comprises an introspective theory.
\end{construction}

This is an important archetypal example. It corresponds closely to the interpretation of GL modal logic using a Kripke frame given by a sequence of worlds indexed by positive integers, each world accessing the lower-indexed worlds. It is hopefully clear how an analogous construction works for any other well-founded Kripke frame (at any rate, we will discuss this more in the Models chapter).

\TODOinline{Discuss the modal operator in this context, in the Modal logic chapter}

\subsection{Constructions building new (pre-)introspective theories}
Now let us discuss some general constructions for building new (pre-)introspective theories from old ones or from other data.

\begin{construction}\label{IntrospInternalMap}
If $\langle T, C, \introF \rangle$ is a pre-introspective theory, and any lexfunctor $G : C \to D$ is given for some other $T$-indexed lexcategory $D$, then $\langle T, D, G \circ \introF \rangle$ is itself a pre-introspective theory, like so: 

\[\begin{tikzcd}
	{\op{T}} && {\LexCat}
	\arrow["{T/-}"{name=0}, from=1-1, to=1-3, shift left=5]
	\arrow["{C}"{name=1, description}, from=1-1, to=1-3]
	\arrow["{D}"{name=2, swap}, from=1-1, to=1-3, shift right=5]
	\arrow[Rightarrow, "{\introF}"', from=0, to=1]
	\arrow[Rightarrow, "{G}"', from=1, to=2]
\end{tikzcd}\]

Of course, this yields an introspective or locally introspective theory just in case $D$ is \repsmall/ or locally \repsmall/, respectively.
\end{construction}

\begin{construction}\label{IntrospPullback}
If $\langle T, C, \introF \rangle$ is a pre-introspective theory, $U$ is any lexcategory, and $\Sigma: U \to T$ is any functor which preserves pullbacks (we do not require $\Sigma$ to preserve the terminal object), then $\langle U, \pullAlong{\Sigma} C \rangle$ can naturally be equipped as an introspective theory, like so:
\end{construction}
\begin{proof}[Details]
% https://q.uiver.app/?q=WzAsMyxbMCwwLCJcXG9we1V9Il0sWzIsMCwiXFxvcHtUfSJdLFs0LDAsIlxcTGV4Q2F0Il0sWzEsMiwiVC8tIiwwLHsib2Zmc2V0IjotMn1dLFsxLDIsIkMiLDIseyJvZmZzZXQiOjJ9XSxbMCwxLCJcXG9we1xcU2lnbWF9Il0sWzAsMiwiVS8tIiwwLHsib2Zmc2V0IjotNSwiY3VydmUiOi0zfV0sWzMsNCwiXFxpbnRyb0YiLDIseyJzaG9ydGVuIjp7InNvdXJjZSI6MjAsInRhcmdldCI6MjB9fV0sWzYsMSwiXFxTaWdtYSIsMCx7InNob3J0ZW4iOnsic291cmNlIjoyMH19XV0=
\[\begin{tikzcd}
	{\op{U}} && {\op{T}} && \LexCat
	\arrow[""{name=0, anchor=center, inner sep=0}, "{T/-}", shift left=2, from=1-3, to=1-5]
	\arrow[""{name=1, anchor=center, inner sep=0}, "C"', shift right=2, from=1-3, to=1-5]
	\arrow["{\op{\Sigma}}", from=1-1, to=1-3]
	\arrow[""{name=2, anchor=center, inner sep=0}, "{U/-}", shift left=5, curve={height=-18pt}, from=1-1, to=1-5]
	\arrow["\introF"', shorten <=1pt, shorten >=1pt, Rightarrow, from=0, to=1]
	\arrow["\Sigma", shorten <=3pt, Rightarrow, from=2, to=1-3]
\end{tikzcd}\]

The 2-cell labelled $\Sigma$ above indicates the action of $\Sigma$ when acting as a lexfunctor from $U/u$ to $T/(\Sigma u)$ for each object $u$ in $U$. (Note that, as finite limits in slice categories are given by pullbacks in the underlying category, and as $\Sigma$ preserves pullbacks, we do indeed have that this functor from $U/u$ to $T/(\Sigma u)$ preserves finite limits.)

By \cref{RepsmallRightAdjoint} or \cref{RepSmallRightAdjointFibers}, if $\Sigma$ has a right adjoint, we can further observe that if $C$ is small or locally small, then so respectively will be $\pullAlong{\Sigma} C$.
\end{proof}

\sTODOinline{Note that functors between lexcategories preserving pullbacks and having right adjoints are commonly studied; this is the same as the notion of a geometric functor into a slice category. So there is a panoply of examples of this construction.}

A particular special case of the above which is often of importance is the following:

\openNamed{construction}{Slice Pre-Introspective Theories}\label{IntrospSlice}
If $\langle T, C, \introF \rangle$ is a pre-introspective theory, and $t$ is any object in $T$, then the slice category $T/t$ can be equipped in a natural way as a pre-introspective theory as well. If we start from an introspective or locally introspective theory, then so respectively will be the result of this construction.
\closeNamed{construction}
\begin{proof}[Details]
By the previous construction (\cref{IntrospPullback}), using the forgetful functor $\Sigma : T/t \to T$, which preserves pullbacks and has a right adjoint.
\end{proof}

When we abuse language and speak of $T/t$ as an introspective theory, the above construction is what we mean.

\sTODOinline{This freely augments $T$ with a single global point of $t$. There should be some generalization of the above that freely augments $T$ with as many global points of as many types as we like; that is, given a lexfunctor $M$ from $T$ to $\Set$ serving as a particular model of $T$, we should be able to freely augment $T$ to $T[M]$ such that all $T[M]$'s models extend $M$. I believe this amounts to Definition 7.14 of "A General Framework for the Semantics of Type
Theory" by Uemura, or section 5.4 of Uemura's PhD thesis ("Abstract and Concrete Type Theories").

Perhaps more generally, whenever we have a lexfunctor from $T$ to $S$, we can left or right Kan extend along this lexfunctor to turn a (pre)(locally)(actual)introspective theory structure on $T$ into such structure on $S$? Using the fact that left Kan extension along lexfunctors is lex, and left Kan extension along arbitrary functors takes representables to representables. Or the fact that right Kan extension along arbitrary functors is Lex. The slice category construction is then also the special case of this where we consider the lex map from $T$ into $T/t$ (which has an adjoint).}

\begin{construction}\label{SubPreIntrosp}
If $\langle T, C, \introF \rangle$ is a pre-introspective theory, and $S$ is a full sub-lexcategory\footnote{That is, a full subcategory whose inclusion functor is a lexfunctor.} of $T$, then $\langle S, C, \introF \rangle$ is a pre-introspective theory, where we now consider $\introF$ as restricted to acting on $S$.
\end{construction}
\begin{proof}[Details]
By \cref{IntrospPullback} again, taking $\Sigma$ to be the inclusion functor.
\end{proof}

\begin{construction}\label{SubCPreIntrosp}
If $\langle T, C, \introF \rangle$ is a pre-introspective theory, and $D$ is a full sub-lexcategory of $C$ containing the range of $\introF$, then $\langle T, D, \introF \rangle$ is a pre-introspective theory (where $\introF$ is now taken to have codomain $D$).
\end{construction}

The last two constructions are often fruitfully combined: Given a pre-introspective theory $\langle T, C, \introF \rangle$, we may first pass from $T$ to a sub-lexcategory $S$ of $T$ and then, after having done so, find that $\introF$ when restricted to $S$ factors through a sub-lexcategory $D$ of $C$.

\begin{TODOblock}
Make the useful observations that the theory of introspective theories is essentially lex, and that we can take therefore take products of introspective theories in the straightforward way. (We could also take limits of strict introspective theories more generally, but that involves talking about object equality, which we don't really want to do.). We can also take sub-introspective theories generated as the hull of subsets of their objects and morphisms, or other such free constructions.
\end{TODOblock}

\begin{construction}\label{TrivialPreIntrosp}
If $T$ is any lexcategory, it can be equipped as a pre-introspective theory, taking $C$ to be given by the self-indexing, and $\introS$ and $\introN$ to be the canonical isomorphisms of the appropriate type. (This is the same construction as discussed at \cref{TrivialPreIntrospIndexed}.)

This will be a locally introspective theory just in case $T$ is locally cartesian closed. However, it will be an introspective theory only in the trivial case that $T$ is the terminal category. This will follow from the categorical \Loeb/'s theorem we develop in the next chapter.

In the same way, if $T$ is any category with finite products, it can be equipped as a pre-introspective finite product theory, taking $C$ to be given by the simple self-indexing, and $\introS$ and $\introN$ to be the canonical isomorphisms of the appropriate type. This will be a locally introspective finite product theory just in case $T$ is cartesian closed.

\TODOinline{The constructions of the last two paragraphs are trivial and in a way a distraction, because our goal is introspective theories in the end, and these do not give us that. However, this way of looking at categories with finite products/CCCs/LCCCs is useful for drawing traditional corollaries of our general results.}
\end{construction}

\begin{observation}\label{CartesianClosedLocallyIntrosp}
The last construction can be much further generalized. Here we do so, using a number of observations:
\TODOinline{Move these wherever they actually belong}

Given a lex endofunctor $\Box$ under identity (that is, equipped with a natural transformation $\introN$ from identity to $\Box$) on a lex category $T$, we can automatically extend $\Box$ to a $T$-indexed lex endofunctor under identity on the self-indexing of $T$, taking the action on objects of $\Box_X$ on a slice $f : Y \to X$ above $X$ to be given by applying $\Box$ to $f$ to achieve a slice $\Box f : \Box Y \to \Box X$, then pulling this back along $\introN_Y : X \to \Box X$ to get another slice above $X$. The action on morphisms and the lexness of this construction are straightforward if tedious to show, as are the relevant functoriality and naturality conditions for this $\Box$ to be a well-defined $T$-indexed endofunctor \TODO.

This construction also works replacing lexness throughout by finite product structure and using the simple self-indexing correspondingly.

Furthermore, if $T$ is a cartesian closed category and $\Box$ is a finite product preserving endofunctor on $T$, we can obtain another cartesian closed category with the same objects as $T$ but in which a morphism from $A$ to $B$ is what had originally been a global section of $\Box (A \implies B)$, with the obvious composition structure.

The end result of these observations is that a pre-introspective cartesian closed theory in which $\introF$ is essentially surjective on objects amounts to essentially the same thing as a cartesian closed category $T$, a finite product preserving endofunctor $\Box$ on $T$, and a natural transformation $\introN$ from identity to $\Box$ on $T$. We take our category $C$ to be given by the construction of the last paragraph (taken as $T$-indexed by performing this relative to each aspect of the simple self-indexing of $T$), and we obtain $\introS$ by noting that a morphism from $A$ to $B$ in (any aspect of) $T$ corresponds to a global section of $A \implies B$, which can then be hit with $\Box$ to obtain a global section of $\Box (A \implies B)$. The result is automatically locally introspective.

Some version of this should apply as well to locally cartesian closed categories, but there's some bother about defining what the objects in the construction from two paragraphs ago should be, when they are given by equalizers in $C$ that don't already exist in $T$.

However, locally introspective (locally) cartesian closed theories cannot be fully introspective, or even just have the \Loeb/\ property, except in a somewhat degenerate sense. $((\Box A) \implies A) \implies A$ will automatically be inhabited, as we have $(\Box A \implies A) \vdash \Box (\Box A \implies A)$ by $\introN$, which in turn entails $\Box A$ by \Loeb/, and thus $(\Box A) \implies A \vdash A$. When $A$ is thought of as falsehood $0$, this amounts to $\neg \neg \Box 0$, asserting the double-negation of the inconsistency of $C$. In a Boolean context, this forces $C$ to be trivial.

\TODOinline{Write the analogue of this for geminal categories}
\end{observation}

\subsection{Variations from our core definition}
\TODOinline{Note that essentially all constructions and results in this chapter generalize to finite product theories; standardize vocabulary around this}

There are some slight variations on the concept of a (pre-)introspective theory which will occasionally be of use to us in future chapters, for technical reasons or for showing the greater generality of some results. In this section, we discuss such variations. But we remind the reader that our main object of interest is introspective theories. If ever this zoo of other named concepts grows intimidating, think always of introspective theories as the North Star.

\begin{definition}\label{DefnPreIntrospSNGeneralized}
Recall that a pre-introspective theory is a lexcategory $T$, a $T$-indexed lexcategory $C$, a lexfunctor $\introS$ from $T$ to the global aspect of $C$, and a natural transformation $\introN$ from $t$ in $T$ to $\Hom_C(1, \introS(t))$. Note how lex structure (that is, finite limit structure) is used throughout this definition.

More generally, for any notion of categorical structure extending the concept of a category-with-terminal-object, we may consider the situation of a category $T$ with such structure, a $T$-indexed category $C$ with such structure, a functor $\introS$ from $T$ to the global aspect of $C$ preserving such structure, and a natural transformation $\introN$ from $t$ in $T$ to $\Hom_C(1, \introS(t))$. In this way, we could speak of \defined{pre-introspective X theories} for various X; pre-introspective finite product theories, pre-introspective terminal object theories, pre-introspective countable limit theories, and so on.

But when we leave the nature of the theory unqualified, we always mean finite limits by default.

Note that whenever every instance of structure X is also an instance of structure Y, we will have that every pre-introspective X theory is also a pre-introspective Y theory.

Finally, on rare occasion it is even worthwhile to observe the extreme generality where we do not demand a terminal object. That is, a category $T$, a $T$-indexed category $C$, a functor $\introS$ from $T$ to the global aspect\footnote{Note that this global aspect is well-defined even if $T$ lacks a terminal object.} of $C$, a designated object $1$ in the global aspect of $C$ (not presumed terminal), and a map $\introN$ from $t$ to $\Hom_C(1, \introS(t))$, natural in $t$ in $T$. We shall call this a \defined{pre-introspective unary theory}.

Note that every pre-introspective terminal object theory can canonically be equipped as a pre-introspective unary theory by taking the designated object in $C$ to be its terminal object.

As ever, we may write out $\langle T, C, \introS, \introN \rangle$ or $\langle T, C, \introS, \introN, 1 \rangle$ to be fully explicit, but in typical abuse of language, will usually say $\langle T, C \rangle$ to describe a pre-introspective X theory.

When $C$ is small, we may call this an \defined{introspective X theory}. (We will only do this in contexts where it makes sense to speak of Xes internal to Xes. The minimal such context is when X is finite limit theories). 

When $C$ is locally small, we may call this a \defined{locally introspective X theory}. The most notable example of this is to speak of \defined{locally introspective finite product theories}. (Indeed, we will only use this terminology in contexts where a $T$-indexed locally small $X$ is taken to an $S$-indexed locally small $X$ by any $X$-homomorphism from $T$ to $S$, and the minimal such context is when X is finite product theories).
\end{definition}

There is one final definition we will need for technical reasons later on. Recall that a small indexed lexcategory is one which is equivalent to some small indexed strict lexcategory (with possibly multiple non-isomorphic such choices available). It is occasionally of use to imagine some particular such choice has been pinned down, leading to the following definition.

\begin{definition}\label{InnerStrictDefn}
By an \defined{inner-strict introspective theory}, we mean an introspective theory $\langle T, C\rangle$ along with a specific choice for how to construe $C$ as an internal lexcategory; that is, a specific choice of $\Ob(C)$ as an object in $T$, as well as specific choices of internal maps equipping $C$ with chosen basic limits.
\end{definition}

Thus, an introspective theory can always be construed as some inner-strict introspective theory, though multiple non-isomorphic such choices may be available. \TODOinline{Cite how every small indexed non-strict lexcategory can become a small indexed strict lexcategory}

We call this \quote{inner strict} to emphasize that we've taken every choice concerning the representation of $C$ which was allowed to vary over non-isomorphic objects or non-equal parallel morphisms in $T$, and fixed some particular such choice for it, but we've not imposed strict structure on $T$ itself.

\subsection{The interaction of \texorpdfstring{$\introS$}{S} and \texorpdfstring{$\introN$}{N}}
We gather here two small but useful lemmas for reasoning about (pre-)introspective theories, concerning the interaction of $\introS$ and $\introN$.

\openNamedManualIndexSort{lemma}{$\introS$ With $\introN$}{S With N}\label{SWithN}
Within a pre-introspective theory $\langle T, C \rangle$, let $F : X \to t$ be a morphism of $T$, and let $x$ be any generalized element of $X$. We have that $\introN_t(F(x)) = \introS(F) \circ_C \introN_X(x)$.
\closeNamed{lemma}
\begin{proof}
This is just the naturality square for $\introN$ with respect to $F$.

% https://q.uiver.app/?q=WzAsNCxbMCwwLCJYIl0sWzAsMSwiXFxIb21fQygxLCBcXGludHJvUyhYKSkiXSxbMSwxLCJcXEhvbV9DKDEsIFxcaW50cm9TKHQpKSJdLFsxLDAsInQiXSxbMCwxLCJcXGludHJvTl97WH0iLDJdLFsxLDIsIlxcaW50cm9TKEYpIFxcY2lyY19DIC0iLDJdLFswLDMsIkYiXSxbMywyLCJcXGludHJvTl97dH0iXV0=
\[\begin{tikzcd}
	X & t \\
	{\Hom_C(1, \introS(X))} & {\Hom_C(1, \introS(t))}
	\arrow["{\introN_{X}}"', from=1-1, to=2-1]
	\arrow["{\introS(F) \circ_C -}"', from=2-1, to=2-2]
	\arrow["F", from=1-1, to=1-2]
	\arrow["{\introN_{t}}", from=1-2, to=2-2]
\end{tikzcd}\]
\end{proof}

\openNamedManualIndexSort{lemma}{$\introS$ Matches $\introN$}{S Matches N}\label{SMatchesN}
Within a pre-introspective theory $\langle T, C \rangle$, let $t$ be some object of $T$ and let $\epsilon : 1 \to t$ in $T$ be taken as defining a global element $e$ of $t$. Then the global element $\introS(\epsilon)$ of $\Hom_C(\introS(1), \introS(t))$ is equal to the the global element $\introN_t(e)$ of $\Hom_C(1, \introS(t))$ under the canonical isomorphism identifying $\Hom_C(\introS(1), \introS(t))$ with $\Hom_C(1, \introS(t))$.

In short, $\introS$ and $\introN$ take global elements in $T$ to equal global elements of $C(1)$.
\closeNamed{lemma}
\begin{proof}
Consider the following commutative diagram  in $\Psh{T}$.

% https://q.uiver.app/?q=WzAsNyxbMiwwLCIxIl0sWzMsMCwidCJdLFszLDEsIlxcSG9tX0MoMSwgXFxpbnRyb1ModCkpIl0sWzIsMSwiXFxIb21fQygxLCBcXGludHJvUygxKSkiXSxbMywyLCJcXEhvbV9DKFxcaW50cm9TKDEpLCBcXGludHJvUyh0KSkiXSxbMiwyLCJcXEhvbV9DKFxcaW50cm9TKDEpLCBcXGludHJvUygxKSkiXSxbMCwyLCIxIl0sWzAsMywiXFxpbnRyb05fMSIsMl0sWzMsMiwiXFxpbnRyb1MoXFxlcHNpbG9uKSBcXGNpcmMgLSJdLFsxLDIsIlxcaW50cm9OX3QiXSxbMCwxLCIqIFxcbWFwc3RvIGUiXSxbMiw0LCItIFxcY2lyYyAhIl0sWzMsNSwiLSBcXGNpcmMgISIsMl0sWzUsNCwiXFxpbnRyb1MoXFxlcHNpbG9uKSBcXGNpcmMgLSJdLFs2LDAsIiIsMSx7ImN1cnZlIjotNSwibGV2ZWwiOjIsInN0eWxlIjp7ImhlYWQiOnsibmFtZSI6Im5vbmUifX19XSxbNiw1LCIqIFxcbWFwc3RvIFxcaWRfe1xcaW50cm9TKDEpfSJdLFs2LDQsIiogXFxtYXBzdG8gXFxpbnRyb1MoXFxlcHNpbG9uKSIsMix7ImN1cnZlIjo1fV1d
\[\begin{tikzcd}
	&& 1 & t \\
	&& {\Hom_C(1, \introS(1))} & {\Hom_C(1, \introS(t))} \\
	1 && {\Hom_C(\introS(1), \introS(1))} & {\Hom_C(\introS(1), \introS(t))}
	\arrow["{\introN_1}"', from=1-3, to=2-3]
	\arrow["{\introS(\epsilon) \circ -}", from=2-3, to=2-4]
	\arrow["{\introN_t}", from=1-4, to=2-4]
	\arrow["{* \mapsto e}", from=1-3, to=1-4]
	\arrow["{- \circ !}", from=2-4, to=3-4]
	\arrow["{- \circ !}"', from=2-3, to=3-3]
	\arrow["{\introS(\epsilon) \circ -}", from=3-3, to=3-4]
	\arrow[curve={height=-30pt}, Rightarrow, no head, from=3-1, to=1-3]
	\arrow["{* \mapsto \id_{\introS(1)}}", from=3-1, to=3-3]
	\arrow["{* \mapsto \introS(\epsilon)}"', curve={height=30pt}, from=3-1, to=3-4]
\end{tikzcd}\]

The top arrow is $\epsilon : 1 \to t$, thought of as sending the unique element of $1$ to $e$. The top rectangle is the naturality square for $\introN$ with respect to $\epsilon$.

The bottom rectangle is the associativity square for composition in $C$ (specifically, on one side composing with $\introS(\epsilon) : \introS(1) \to \introS(t)$ and on the other side composing with the unique morphism $! : \introS(1) \to 1$). Note that the right arrow of this associativity rectangle is the canonical isomorphism given by $\introS(1)$ being a terminal object of $C$.

The bottom wedge is the identity law for composition in $C$ (specifically, composing after the identity on $\introS(1)$).

Finally, the left wedge commutes because, as $\introS(1)$ is a terminal object of $C$, we have that $\Hom_C(\introS(1), \introS(1))$ is a terminal object of $\Psh{T}$; thus, any two parallel maps into it are equal. (Indeed, all arrows in the left wedge are unique isomorphisms between terminal objects.)

Now consider the composites around this commutative diagram along the two outermost paths. Along the bottom, the unique element of $1$ is sent to $\introS(\epsilon)$. Along the top and right, it is sent to $\introN_t(e)$ and then along the canonical isomorphism. This completes the proof.

(We would not ordinarily bother to distinguish between $1$ and $\introS(1)$ or in general explicitly write out the coherence isomorphisms for a product preserving functor, but in this one example it may be illuminating to see these distinctions and isomorphisms explicitly.)
\end{proof}

\TODOinline{Note that in our Box notation, the above identifies $\introN$ and $\Box$, so far as their action on global elements of $T$ goes}

\subsection{Recap}
We have defined the central notion of our interest, the concept of an introspective theory. We have proven that two different definitions of this concept are equivalent. We have also discussed a number of slight variations on this concept. We have seen how a canonical example of an introspective theory can be constructed by considering $\Sigma_1$ formulae in familiar theories such as ZF-Finite. Finally, we have discussed a number of other constructions which generate new introspective theories or pre-introspective theories from existing ones or from other categorical data (such as generating pre-introspective theories from cartesian closed categories).

\fileend