\section{Introspective theories and GL-categories}

% Non-evil definition
\subsection{Definition}
\begin{definition} \label{IntrospectiveTheory}
A \defined{pre-introspective theory} is a lexcategory $T$, a $T$-indexed lexcategory $C$, and a lexfunctor $F$ from the self-indexing of $T$ to $C$, like so:

\[\begin{tikzcd}
	{\op{T}} && {\LexCat}
	\arrow["{T/-}"{name=0}, from=1-1, to=1-3, shift left=2]
	\arrow["{C}"{name=1, swap}, from=1-1, to=1-3, shift right=2]
	\arrow[Rightarrow, "{F}"', from=0, to=1]
\end{tikzcd}\]

If $C$ is small or locally small, then we say this is an \defined{introspective theory} or \defined{locally introspective theory}, respectively.
\end{definition}

We write $\langle T, C, F \rangle$ to refer to a pre-introspective theory when we wish to be fully explicit about its structure. But in typical abuse of language, we also often refer to it simply by the name of its underlying lexcategory $T$, when this would not cause confusion.

\begin{construction}
If $\langle T, C, F \rangle$ is a pre-introspective theory, and any lexfunctor $G : C \to D$ is given for some other $T$-indexed lexcategory $D$, then $\langle T, D, G \circ F \rangle$ is itself a pre-introspective theory, like so: 

\[\begin{tikzcd}
	{\op{T}} && {\LexCat}
	\arrow["{T/-}"{name=0}, from=1-1, to=1-3, shift left=5]
	\arrow["{C}"{name=1, description}, from=1-1, to=1-3]
	\arrow["{D}"{name=2, swap}, from=1-1, to=1-3, shift right=5]
	\arrow[Rightarrow, "{F}"', from=0, to=1]
	\arrow[Rightarrow, "{G}"', from=1, to=2]
\end{tikzcd}\]

Of course, this yields an introspective or locally introspective theory just in case $D$ is small or locally small, respectively.
\end{construction}

\begin{construction}
If $\langle T, C, F \rangle$ is a pre-introspective theory, and $t$ is any object in $T$, then the slice category $T/t$ can be equipped in a natural way as a pre-introspective theory as well. If we start from an introspective or locally introspective theory, then so respectively will be the result of this construction.
\end{construction}
\begin{proof}[Details]
Consider the forgetful functor $\Sigma : T/t \to T$. Composition and whiskering with this gives us the triple $\langle T/t, \Sigma C, \Sigma F \rangle$, like so:

\[\begin{tikzcd}
	{\op{\left(T/t\right)}} & {\op{T}} && {\LexCat}
	\arrow["{T/-}"{name=0}, from=1-2, to=1-4, shift left=2]
	\arrow["{C}"{name=1, swap}, from=1-2, to=1-4, shift right=2]
	\arrow["{\Sigma}", from=1-1, to=1-2]
	\arrow[Rightarrow, "{F}"', from=0, to=1]
\end{tikzcd}\]

This is a pre-introspective theory as the top composite is the same as the self-indexing $(T/t)/-$ of $T/t$ (that is, slice categories within slice categories are obtained by first applying the forgetful functor into the ambient category and then taking the ordinary slice category).

When $C$ is small or locally small, then so respectively is $\Sigma C$. The representing objects for $\Sigma C$ will simply be those for $C$, mapped into $T/t$ via $\Sigma$'s right adjoint (i.e., pulled back along the unique morphism from $t$ to $1$).
\end{proof}

When we abuse language and speak of $T/t$ as an introspective theory, the above construction is what we mean. That said, there is another natural way to equip $T/t$ as an introspective theory as well.

\begin{TODOblock}
There is another way to equip $T/t$ as an introspective theory, in which we use as our internal category the slice category $C/t$ so to speak. Construction 1.3 is actually equivalent to doing this and then applying Construction 1.2 to the internal forgetful lexfunctor from $C/t$ to $C$. This equivalence is easier to see once we see how the natural transformation taking $t$ to $\Hom_C(1, t)$ works.
\end{TODOblock}

\begin{construction}\label{SAndN}
Given a pre-introspective theory $\langle T, C, F \rangle$, we obtain immediately a lexfunctor from $T$ to $C(1)$, given by the component of $F$ at the terminal object $1$ of $T$. Let us call this lexfunctor $S$.

We shall also obtain, naturally in $x : T$, a map from $x$ to $\Hom_C(1, F(x))$. Let us call this $N$. The specific construction of $N$ is like so:
\end{construction}
\begin{proof}[Details]
Given any functor $F$ between any $T$-indexed categories $D$ and $C$, we obtain a function from $\Hom_D(y, x)$ to $\Hom_C(F(y), F(x))$ natural in $x$ and $y$. For $F$ a lexfunctor, we can specialize this to a function from $\Hom_D(1, x)$ to $\Hom_C(1, F(x))$ natural in $x$. This is a $T$-indexed collection of functions, applicable to objects $x$ of $D$ defined over any object in $T$ and natural with respect to all morphisms defined over any object in $T$, but in particular this is all applicable to the globally defined objects and morphisms of $D$.

If we now take $D$ to be the self-indexing of $T$, its globally defined objects and morphisms are $T$ itself, and for each such object $x$, we may identify $\Hom_D(1, x)$ with $x$ itself. Thus, we obtain a natural transformation from $x$ to $\Hom_C(1, F(x))$ as desired.
\end{proof}

It turns out, we can recover $F$ from arbitrary such $N$ and $S$ too. That is, \cref{SAndN} has an inverse. \TODO

Thus, an equivalent definition to \cref{IntrospectiveTheory} is like so:

\begin{definition}\label{IntrospectiveTheoryNS}
A \defined{pre-introspective theory} is a lexcategory $T$, a $T$-indexed lexcategory $C$, a lexfunctor $S$ from $T$ to the global aspect of $C$, and a map $N$ from $t$ to $\Hom_C(1, S(t))$, natural in $t$ in $T$.
\end{definition}

As before, introspective or locally introspective theories are those where $C$ is small or locally small. The value of this new definition is that there is much less data around. In particular, when we wish to turn this into a lex definition \TODO, we will find the coherence conditions much easier to manage. It will also be easier to define the appropriate notion of homomorphisms between pre-introspective theories by thinking about this definition.