\documentclass[12pt]{article}
% The above really belongs in the prelude, but Overleaf gets a little weird about that not being in the main document.
% The document class really belongs in the prelude, but Overleaf gets a little weird about that not being in the main document.

% Just some made up sizing for now. Can tinker with this later.
% \usepackage[a4paper, margin=1in, headsep=5pt]{geometry}

% Copied from Reid
\usepackage{amssymb,amsmath, amsthm}
\usepackage[utf8]{inputenc} % allow utf-8 input
\usepackage[T1]{fontenc}    % use 8-bit T1 fonts
\usepackage{url}            % simple URL typesetting
\usepackage{booktabs}       % professional-quality tables
\usepackage{amsfonts}       % blackboard math symbols
\usepackage{nicefrac}       % compact symbols for 1/2, etc.
\usepackage{microtype}      % microtypography

% Put in by me, Sridhar
\usepackage{color}          % Color definitions (used for `blue`, `red`)
\usepackage{newpxtext}      % Use a nice text font
\usepackage{enumitem}       % Bulleted lists
  \setenumerate{parsep=10pt} % Futz with the exact value later
\usepackage{comment}        % Multiline comments
\raggedbottom               % No more underfull vbox errors
\usepackage{quiver}         % Commutative diagrams
\usepackage{indentfirst}    % Indent first paragraph in a section
\usepackage{csquotes}       % Quotation marks without needing `` and ''
\usepackage[style=alphabetic]{biblatex}
\bibliography{references}   % Uses the file `references.bib` for bibliography information.
\usepackage{etoolbox}       % Generic LaTeX tools, including nice conditionals. Maybe unnecessary.
\usepackage{underscore}     % Underscore subscripts in text mode. Just useful because I do this in a hurry in TODO notes often.

\renewcommand\qedsymbol{$\blacksquare$} % Change QED symbol to black square so it is not confused with the modal box

% The following are packages that might need to be loaded before hyperref
\usepackage{imakeidx}       % Indexing
  \makeindex[intoc]         % Include index in table of contents
  
% And now, hyperref!
\usepackage[colorlinks=true, linkcolor=blue, citecolor=teal]
  {hyperref}                % Hyper-linked references (note that certain packages need to be loaded before this instead of after this, such as indexing)
  
% And now... Packages that might need to be loaded after hyperref.
\usepackage[conf={restate}]
  {proof-at-the-end}        % Allows for moving proofs of theorems to later in the document, with auto-linking, restating, etc
\usepackage[capitalize, nameinlink]
  {cleveref}                % References named by type and number
\newcommand*{\parensref}[1]{\hyperref[{#1}]{\nameref*{#1} (\cref*{#1})}}              % References named by explicit name and parenthesized type and number
\newcommand*{\reverseparensref}[1]{\hyperref[{#1}]{\cref*{#1} (\nameref*{#1})}}   

% It's not ideal that I have to manually keep track of when to use \cref vs. when to use \nameref or \parensref.
% The following solution is based on https://tex.stackexchange.com/a/66096/231784.
%\newcommand{\magicref}[1]{%
%  \if\vcenter\getrefbykeydefault{#1}{name}{}\vcenter
%    \cref{#1}%
%  \else
%    \nameref{#1}%
%  \fi
%}
\newcommand{\magicref}[1]{%
  \if\vcenter\getrefbykeydefault{#1}{name}{}\vcenter
    \cref{#1}%
  \else
    \parensref{#1}%
  \fi
}
\newcommand{\magicreverseparensref}[1]{%
  \if\vcenter\getrefbykeydefault{#1}{name}{}\vcenter
    \cref{#1}%
  \else
    \reverseparensref{#1}%
  \fi
}

% Listing unused references in the compilation logs
% \usepackage[norefs, nocites]{refcheck}

% We have to use the workaround from https://tex.stackexchange.com/questions/87610/making-refcheck-work-with-cleveref:

%%% Infrastructure    
\makeatletter
\newcommand{\refcheckize}[1]{%
  \expandafter\let\csname @@\string#1\endcsname#1%
  \expandafter\DeclareRobustCommand\csname relax\string#1\endcsname[1]{%
    \csname @@\string#1\endcsname{##1}\wrtusdrf{##1}}%
  \expandafter\let\expandafter#1\csname relax\string#1\endcsname
}
\makeatother
%%%

%%% Now we add the reference commands we want refcheck to be aware of
% \refcheckize{\magicref}

% For subfiles, in a toggle-able way
\newif\ifsubfiles

% For TODO counters, in a toggle-able way
\newif\ifDisplayTODOCount

\newcommand{\filestart}{
    \ifsubfiles
    %\documentclass[./main.tex]{subfiles}
    % Commented out while we don't use subfiles as this appearance of the documentclass command is making Overleaf think of this prelude as an individually compilable file. See https://www.overleaf.com/learn/how-to/Set_Main_Document
    \begin{document}
    \fi
}

\newcommand{\fileend}{
    \ifsubfiles
    \end{document}
    \fi
}
\newcommand{\fileinclude}[1]{
    \setcounter{TODOcounter}{0}
    \ifsubfiles
    \newpage \subfile{#1}
    \else
    \include{#1}
    \fi
    \newcounter{#1Counter}
    \setcounter{#1Counter}{\value{TODOcounter}}
}

% Commands defined by Sridhar for this document

% Remember: \newtheorem{command name}{displayed name}[parent counter] makes a theorem whose counter is subordinate to counter, but \newtheorem{command name}[shared counter]{displayed name} makes a theorem which shares the counter. Also remember that there is automatically a counter available named "section".

% Playing around
\newtheoremstyle{envstyle}
  {\topsep}
  {\topsep}
  {}
  {}
  {\scshape \color{purple}}
  {}
  {.5em}
  {}

%\theoremstyle{plain}
\theoremstyle{envstyle}
\newtheorem{theorem}{Theorem}[section]
\newtheorem{lemma}[theorem]{Lemma}
\newtheorem{construction}[theorem]{Construction}
\newtheorem{example}[theorem]{Example}
\newtheorem{observation}[theorem]{Observation}
\newtheorem{corollary}[theorem]{Corollary}
\newtheorem{corollarytoproof}[theorem]{Corollary (to proof)}
\newtheorem{conjecture}[theorem]{Conjecture}

%\theoremstyle{definition}
\theoremstyle{envstyle}
\newtheorem{definition}[theorem]{Definition}
\newtheorem{convention}[theorem]{Convention}

\newcommand{\defined}[1]{\textbf{#1}\index{#1}} % For indicating the defined term in a definition
\newcommand{\definedManualIndexSort}[2]{\textbf{#1}\index{#2@#1}}

\newcommand{\openNamed}[2]{\begin{#1}[#2]\index{#2}}
\newcommand{\openNamedManualIndexSort}[3]{\begin{#1}[#2]\index{#3@#2}}
\newcommand{\closeNamed}[1]{\end{#1}}

%\theoremstyle{remark}
\theoremstyle{envstyle}
\newtheorem*{remark}{Remark}

% This doesn't quite work yet. The word "Proof" is bolded?
% \renewenvironment{proof}{\paragraph{\scshape \color{blue} Proof:}}{\hfill$\square$}

\newtheoremstyle{redstyle}
  {\topsep}
  {\topsep}
  {\color{red}}
  {}
  {\scshape \color{red}}
  {}
  {.5em}
  {}
  
\theoremstyle{redstyle}
\newtheorem*{TODOblock}{TO DO}

% Warning! LaTeX will swallow the following space stupidly in all of these commands. You may have to place a backslash-space after them instead.
\newcommand{\TODO}{{\color{red} TO DO}}
\newcommand{\TODOinline}[1]{\TODO\ {\color{red} #1}}
\newcommand{\Goedel}{G\"odel}
\newcommand{\Godel}{\Goedel}
\newcommand{\Loeb}{L\"ob}
\newcommand{\Lob}{\Loeb}

% Using \operatorname for constants is maybe a little wrong, as it suggests a closer semantic association to the right. For now, we just use \mathrm.
\newcommand{\const}[1]{\mathrm{#1}} % Typesetting named constants
\newcommand{\constcat}[1]{\const{#1}} % Typesetting named constant categories
\newcommand{\arrowcat}[1]{\operatorname{Arrow}(#1)} % Typesetting arrow categories
\newcommand{\Hom}{\operatorname{Hom}}
\newcommand{\Iso}{\operatorname{Iso}}
\newcommand{\Nat}{\operatorname{Nat}}
\newcommand{\dom}{\operatorname{dom}}
\newcommand{\cod}{\operatorname{cod}}
\newcommand{\op}[1]{#1^{\mathrm{op}}} % If we include the `physics` package, then this needs to be renewcommand, as \op is defined for \outerproduct by the `physics` package.
\newcommand{\Ob}{\operatorname{Ob}}
\newcommand{\Mor}{\operatorname{Mor}}
\newcommand{\Set}{\constcat{Set}}
\newcommand{\Cat}{\constcat{Cat}}
\newcommand{\LexCat}{\constcat{LexCat}}
\newcommand{\id}{\const{id}}
\newcommand{\unique}{!}
\newcommand{\iso}{\simeq}
\newcommand{\Psh}[1]{\operatorname{Psh}(#1)}
\newcommand{\Indexed}[1]{\operatorname{Indexed}(#1)}

\newcommand{\introF}{\mathcal{F}}
\newcommand{\introS}{\mathcal{S}}
\newcommand{\introN}{\mathcal{N}}
\newcommand{\glQuote}{\mathcal{J}}

\newcommand{\GLCatTheory}{\mathrm{Th}(\mathrm{GL})}

\newcommand{\outertheory}{X}
\newcommand{\innertheory}{Y}

\renewcommand{\implies}{\Rightarrow}
\newcommand{\biimplies}{\iff}

\newcommand{\code}[1]{\lceil #1 \rceil}

\newcommand{\pullAlong}[1]{#1^*}

% Should maybe use the word "representable" for this?
% Hack to define a command name whose name has to end with a 
% slash, thus preventing spacing problems. (Otherwise, space 
% automatically gets swallowed after invoking the command.)
% TODO: Use this same hack for Goedel and Loeb.
\def\repsmall/{repsmall}

\newcommand{\adjointTo}{\dashv}

\title{Abstract Provability Structures}

\author{
  Sridhar Ramesh\\
  \texttt{sramesh@berkeley.edu}
}

\begin{document}
\maketitle

\begin{abstract}
The aim of these notes is to identify and draw attention to a certain simple and categorically natural kind of mathematical structure which both serves as an abstract environment for the reasoning used in establishing \Loeb's theorem in its traditional instances, and furthermore allows this and the associated theorems and fixed-point results of the \Goedel-\Loeb\ modal logic to be vastly generalized.

Several such \Loeb-style fixed point results have been explored in the literature before, but these notes aim to highlight a particularly minimal, simple, general abstraction that covers several threads of work in the literature, abstracting both the work on the \Goedel-\Loeb\ incompleteness theorems via arithmetic universes a la Joyal, and the work on \Loeb's theorem as a guarded fixed point combinator and on guarded (co)inductive types.

Notably, as differentiated from much other work in the literature on guarded fixed points, we do not take the existence of a guarded fixed points as a presumption of our framework by fiat, but rather achieve them as derived from the framework.

The core idea is the identification of those essentially algebraic theories satisfying the property that every model of these theories contains also, as part of its structure, a homomorphism into an internal model of the same theory.

This structure turns out to be viewable as a categorification of the Hilbert-Bernays derivability conditions of provability logic. In particular, it induces a bifunctor on each model of such a theory which is formally similar to the $\Box(A \implies B)$ operator of the \Goedel-\Loeb\ modal logic. Furthermore, as mentioned, various fixed-point theorems within these categories will be demonstrated, both at the level of terms and at the level of types. From these, the traditional instances of \Loeb's theorem in the context of provability of propositions falls out as a special case.

Because of the level of abstraction of our framework, we are also able to discuss free constructions fitting the framework, and we explore the nature of these free constructions in some detail.

Along the way, the relationship between \Loeb's theorem and presheaves is also highlighted, and the distinction and yet also relationship between categorical models of provability logic which do and don't validate $X \vdash \Box X$ in general is formally clarified.

It should be noted that this work was originally developed during my time in graduate school from 2006 to 2013, but I did not write it up at that time. Since that time, there has been much independent development of related ideas by others. My hope is that there remains some valuable original contribution in the ideas here, at least in the particularly minimal framework and abstract presentation.
\end{abstract}

\tableofcontents

% \section{Introduction}
TODO: Write introduction.

\subsection{Outline}

% \section{Multiply internal structures}

\subsection{What we're about to do}
I presume for now that the reader is already familiar with the concepts of essentially algebraic theories (on any number of sorts), the functorial semantics correspondence between these and categories with finite limits (aka, lex categories), and the concept of an internal category this gives rise to (as the theory of categories is itself essentially algebraic and thus can be interpreted within a category with finite limits), and in the same vein, an internal category with finite (chosen) limits.

There are two more concepts we'll need, which can be explained in many different ways. I'll spend a bit of time explaining them and the phenomena around them in lots of ways, to let them sink in. The first new concept is that of ``multiply internal'' structures. The second new concept is that of maps from ordinary structures to multiply internal structures.

\subsection{Multiply Internal Structures}
We will often use the terminology of describing an object of a lex category as a set internal to that category. Similarly, we describe a morphism of a lex category as a function internal to that category.

\begin{definition}An {\bf internal set} with {\bf parent} $C$ is an object in $C$. An {\bf internal function} with {\bf parent} $C$ is a morphism in $C$.
\end{definition}

An ordinary structure is made of sets and functions between those sets (or finite limits of those sets). In the same way, we can have a structure internal to a lex category, which simply means having internal sets (i.e., objects) and internal functions (i.e., morphisms) play the role of sets and functions. In this way, we can speak of not only sets internal to a lex category but also, e.g., groups internal to a lex category, or lex categories internal to a lex category, or models of any essentially algebraic theory internal to a lex category. 
[Note that the constants of a structure can be thought of as 0-ary functions, i.e. functions with domain $1$, and thus we can make sense of these within internal structures as well, as morphisms with domain the terminal object $1$.].

The fact that the theory of lex categories is itself essentially algebraic, so that we can speak of lex categories internal to lex categories, lends itself to a productive iteration, like so:

An internal set [or internal function] is a lex category and a designated object [or morphism] within that category. The theory of a lex category and a designated object within that category is an essentially algebraic theory, and thus can itself be interpreted internal to a lex category. A model of this theory within a lex category $C_0$ is an internal lex category $C_1$ in $C_0$ (thus, objects $\Ob(C_1)$ and $\Mor(C_1)$ in $C_0$, and some further suitable morphisms between limits of these, satisfying certain equations) along with a designated object [or morphism] of $C_1$ given by a morphism $: 1 \to \Ob(C_1)$ [or $: 1 \to \Mor(C_1)$] in $C_0$.

We call this setup (of a set-[or function]-internal-to-$C_1$ internal to $C_0$) a set [or function] doubly internal to $C_0$. And in the same way, we can speak of doubly internal structures made of doubly internal sets and doubly internal functions between finite limits of them; e.g., doubly internal groups, doubly internal categories, and so on.

And we can continue to iterate in the same way: The theories of doubly internal sets and of doubly internal functions can themselves be interpreted internal to any lex category, giving rise to the concepts of triply internal sets and triply internal functions. And so on, ad infinitum:

\begin{definition}
An {\bf $n$-internal} lex category is made of $n$-internal sets and $n$-internal functions in the place of usual sets and functions, and an {\bf $(n + 1)$-internal set} or {\bf $(n + 1)$-internal function} is a globally defined object or morphism of an $n$-internal lex category.

(The base case of these recursive definitions is that a $0$-internal structure is the ordinary sense of that structure. It would also be natural to consider the category $\Set$ of ordinary sets and ordinary functions to thus be a (-1)-internal lex category.)

When we are not concerned about the specific value of $n$, we say {\bf multiply internal} to encompass $n$-internal for all $n$.
\end{definition}

Put another way:
\begin{definition}
For any essentially algebraic theory $T$, there is an essentially algebraic theory $\rm{Internal}(T)$ of lex categories with an internal model of $T$. We refer to models of $\rm{Internal}^n(T)$ as {\bf $n$-internal} models of $T$.
\end{definition}

In this way, we can speak of $n$-internal models of sets, of functions, of elements of sets, of lex categories, and so on.

Note that each $(n + 1)$-internal structure sits within an $n$-internal lex category, as the theory $\rm{Internal}^{n + 1}(T)$ extends the theory $\rm{Internal}^{n + 1}(\emptyset) = \rm{Internal}^{n}(\rm{LexCat})$, where $\emptyset$ is the initial essentially algebraic theory postulating no sorts, operations, or equations, and $\rm{LexCat} = \rm{Internal}(\emptyset)$ is the theory of lex categories.

\begin{definition}
In the preceding relationship, we refer to the $n$-internal lex category as the {\bf parent} of the $(n + 1)$-internal structure.
\end{definition}

In this way, we get a tree whose root node is the category $\Set$ of actual sets, with the children of this being all lex categories, the children of these being their internal lex categories, the children of those being doubly internal lex categories, and so on. We can also consider as part of this tree all $n$-internal sets for each $n$, as nodes under the appropriate parent lex category they are internal to.

Note that throughout this tree, there are many different multiply internal sets naturally named ``$1$'', one such child for each different multiply internal lex category. It usually will cause no confusion to simply call all these by the same name ``$1$'', as context will make clear which is the appropriate object being discussed (or will be such that any different choices of interpretation lead to equivalent statements). If ever necessary for disambiguation, we write $1_C$ to mean the terminal object within the category $C$.

Another important point to observe is as to the concept of $n$-internal ``elements of sets''. The theory of a set with an element is the same as the theory of a set with a function into it from $1$. Thus, when we speak of an element of an $n$-internal set, we mean the same thing as an $n$-internal function from $1$ into it. Indeed, if we fully work out what it means to discuss an $n$-internal element of some $n$-internal set $X$ internal to $C_{n - 1}$ internal to $C_{n - 2}$ internal to ... internal to lex category $C_0$, we find this data amounts to an element of what would naturally be called $\Hom_{C_0}(1, \Hom_{C_1}(1, \ldots \Hom_{C_{n - 1}}(1, X)\ldots))$. That is, it amounts to a fully globally defined element of $X$.

\begin{definition}
An {\bf element} (or {\bf global element} for emphasis) of an $(n + 1)$-internal set $X$ is an element of the $n$-internal set $\Hom_C(1_C, X)$, where $C$ is the parent of $X$.

(The base case of this recursive definition is that an element of an ordinary set (i.e., 0-internal set) is an element of it in the ordinary sense.)
\end{definition}

Note that any two $(n + 1)$-internal sets $A$ and $B$ with the same parent $C$ give rise to an $n$-internal set $\Hom_C(A, B)$. In particular, from any $(n + 1)$-internal set $B$ with parent $C$, we get an $n$-internal set $\Hom_C(1_C, B)$. For convenience, let us give this process a name:

\begin{definition}
For any $(n + 1)$-internal set $B$ with parent $C$, we use $\Glob(B)$ to refer to the $n$-internal set $\Hom_C(1_C, B)$.
\end{definition}

Note that the set of elements of an $n$-internal set is the result of applying $\Glob$ to it $n$ times.

It will also be useful to make this observation:

\begin{observation}
$\Glob$ acts functorially, in the sense that an $(n + 1)$-internal function between $A$ and $B$ naturally induces a corresponding $n$-internal function between $\Glob(A)$ and $\Glob(B)$, and furthermore, this process preserves composition.
\end{observation}

\subsection{Maps From Sets To Internal Sets}

We've seen how to define $n$-internal functions, whose domain and codomain are both $n$-internal sets for the same $n$. (These are ordinary functions when $n = 0$ and morphisms in a lex category when $n = 1$).

The next concept we will need is that of a map from a set to an internal set (or even multiply internal set). Note that the domain here is a $0$-internal set but the codomain is a $1$-tuply or $n$-internal set.

The definition is straightforward:

\begin{definition}
Given a set $S$ and a multiply internal set $X$, we define a {\bf map from $S$ to $X$} to be an function from $S$ to the set of elements of $X$. (In particular, if $X$ is singly internal, this is a function from $S$ to $\Hom_C(1_C, X)$, where $C$ is the parent of $X$).

When $X$ is an $n$-internal set, we refer to this sort of map as a {\bf $\mathbf{(0, n)}$ map} to indicate that its domain is a $0$-internal set but its codomain is an $n$-internal set.
\end{definition}

The definition is straightforward, but the motivation of it may raise some questions. The ultimate motivation is that this is what will make it easiest to talk about what we wish to talk about in this work. The definition is justified by its ubiquitous convenience for discussing the concepts, results, and insights to come later in this writeup.

That said, a little further motivation can be given like so: For any set $S$, the theory of a set and a function into it from $S$ is essentially algebraic, and we find that models of this theory internal to $C$ are the same as objects $X$ in $C$ along with a function $S \to \Hom_C(1, X)$.

The theory of $(0, n)$ maps is itself essentially algebraic, and so, as ever, we can apply $\rm{Internal}$ to it.

\begin{definition}
An internal $(0, n)$ map is a {\bf $\mathbf{(1, n + 1)}$ map}, going from $1$-internal sets to $(n + 1)$-internal sets. And by iteration of $\rm{Internal}$, we obtain the concepts of {\bf $\mathbf{(m, m + n)}$ maps} for each $m$ and $n$, going from an $m$-internal set $S$ to an $(m + n)$-internal set $X$, in each case meaning an $m$-internal function from $S$ to the $m$-internal set naturally described as $\Hom(1, \Hom(1, \ldots \Hom(1, X)\ldots))$, which has the same parent as $S$.

We refer to this general notion of an $(m, m + n)$ map [which is to say, a multiply internal $(0, n)$ map] as an {\bf $\mathbf{n}$-diving map}, to indicate that its codomain is $n$ levels of internality further down than its domain.
\end{definition}

Put another way, an $n$-diving map from $m$-internal set $A$ to $(m + n)$-internal set $B$ is an $m$-internal map from $A$ to $\Glob^n(B)$.

Note that an $n$-diving map is always such that its codomain's parent is a descendant of its domain's parent, and the data of the $n$-diving map in its domain's parent.

Next, let us observe that these diving maps compose associatively in a natural way, and thus give us a full-on category of multiply internal sets, which we may call $\MultiplyInternalSet$.

\begin{definition}
Given an $n$-diving maps $f : A \to B$ and a $p$-diving map $g : B \to C$, where $A$ is an $m$-internal set, $B$ is an $(m + n)$-internal set, and $C$ is an $(m + n + p)$-internal set, corresponding to an $n$-internal function $F : A \to \Glob^{n}(B)$ and an $(m + n)$-internal function $G : B \to \Glob^{p}(C)$ respectively, we define their composition $g \circ f$ as the $(n + p)$-diving map from $A$ to $C$ corresponding to the $m$-internal function $\Glob^{n}(G) \circ F : A \to \Glob^{n + p}(C)$.
\end{definition}

\begin{theorem}
The preceding notion of composition is associative, has identities, and is compatible with the existing notion of composition of multiply internal functions with the same parent.
\end{theorem}
\begin{proof}
The latter two claims are thoroughly straightforward. For the associativity claim, observe that given diving maps $f : A \to B$, $g : B \to C$, and $h : C \to D$, diving $m$, $n$, and $p$ levels respectively, with corresponding non-diving multiply internal functions $F : A \to \Glob^{m}(B)$, $G : B \to \Glob^{n}(C)$, and $H : C \to \Glob^{p}(D)$, respectively, the non-diving data corresponding to the composition $h \circ (g \circ f)$ is $\Glob^{m + n}(H) \circ (\Glob^{m}(G) \circ F)$, while the non-diving data corresponding to the composition $(h \circ g) \circ f$ is $\Glob^{m}(\Glob^{n}(H) \circ G) \circ F$. These two are equivalent by the functoriality of $\Glob$ and the associativity of composition of multiply internal functions.
\end{proof}

\begin{definition}
$\MultiplyInternalSet$ is the category whose objects are all multiply internal sets and whose morphisms are diving maps between them, composing as above.
\end{definition}

One observation which is often convenient for discussing diagrams in $\MultiplyInternalSet$ is the following:

\begin{lemma}
Let $A$ and $B$ be multiply internal lex categories, with $B$ a descendant of $A$. Let $1_A$ and $1_B$ be their terminal objects, respectively. Then there is a unique map from $1_A$ to $1_B$, and furthermore, every map from $1_A$ to a multiply internal set with parent $B$ factors uniquely through the map from $1_A$ to $1_B$.
\end{lemma}

As a result of this lemma, there is no harm in conflating all such $1$ objects with different parents within a diagram in $\MultiplyInternalSet$ into a single object with no specific label as to its parent, so long as the only role of these objects is to be the domain (rather than codomain) of maps.

\begin{comment}

\subsection{Scrap writing}

TODO: Incorporate or get rid of this.

The very simplest case is this question: Given a set $S$ and an object $X$ in some category $C$ (we will think of $X$ as a ``set internal to'' $C$), what counts as a morphism from $S$ to $X$?

The trouble here is that $S$ and $X$ live in different worlds. $S$ lives in the category $\Set$, and $X$ lives in the category $C$. Ordinarily, morphisms connect two things that live in the same world.

Nonetheless, for our purposes, there is a particular way of making sense of this which will be most appropriate, which will come up over and over. Namely, we'll bring $S$ and $X$ to the same world, and answer the question there: both $S$ and $X$ can naturally be interpreted as living in $C^{\op} \to \Set$, the presheaves over $C$ [$S$ as the constantly $S$ functor; $C$ as its Yoneda embedding $\Hom(-, C)$], and so we can ask what a morphism from $S$ to $X$ in that context is.

This amounts to taking ``morphism from $S$ to $X$'' to mean a function from $S$ to the global elements of $X$. If $C$ happens to have a terminal object (and for our purpose, basically the only categories we will ever consider are lex-categories, so it will), this amounts to the same thing as a function from $S$ to $\Hom_C(1, X)$.

This sort of thing can be pre-composed with ordinary set functions, or post-composed with morphisms in $C$, to get other morphisms from sets to internal sets.

These morphisms can ALSO be post-composed with lex-functors between lex-categories [that is, given a morphism $m$ from set $S$ to object $X$ in lex-category $C$, and a lex-functor $f$ taking $X$ in $C$ to $f(X)$ in category $D$, we get automatically a corresponding morphism $m;f$ from $S$ to $f(X)$].

All these kinds of compositions act associatively.
We can thus consider commutative diagrams whose arrows are a mix of functions from sets to sets, morphisms from object to object inside a category, morphisms from a set to an object of a category, and lex-functors taking an object of one category to an object of another category. Since everything still composes associatively, we can reason about these commutative diagrams in essentially just the ordinary way as for any commutative diagrams.

Now, though I've given this definition, let me motivate it a little more. Because there is something which makes this definition right for our purposes, while other similar definitions of other similar notions would be completely misguided.

(For example, we might have considered just as well defining morphisms in the other direction as well, from internal set $X$ to actual set $S$, via the same sort of embeddings in $C^{\op} \to \Set$, but this would be misguided, for our purposes. Or we could've considered defining morphisms from $S$ to $X$ via the other embedding, in $C \to \Set$ instead, and again this would be misguided.) 

The idea is this: In the ordinary world of sets, we already know what functions from $S$ to $T$ are. But if we hold $S$ fixed, and ask for that fixed $S$ about the notion of ``A set $T$, and a function from $S$ to $T$'' (i.e., the objects of the slice category $S/\Set$), this notion is an essentially algebraic notion: there is an essentially algebraic theory whose models are precisely these (and whose category of such models including their homomorphism structure is $S/\Set$).

As an essentially algebraic theory, we can interpret this not into just $\Set$, but into any lex-category $C$ (or into any category at all, via its Yoneda embedding into its lex-category of presheaves, though this is not super-important for our purposes). And when we do so, we find that the models of this theory in a category $C$ are precisely objects $X$ in $C$ along with the notion of a morphism from $S$ to $X$ given above; that is, an object $X$ in $C$ along with a function from $S$ to $\Hom_C(1, X)$.

Because to us everything is about essentially algebraic theories, the fact that there is this essentially algebraic notion of ``function out of $S$'' is what makes this the right notion.

Note that this doesn't symmetrize; the notion of ``function into S'' for a fixed set $S$ is NOT an essentially algebraic notion, and therefore we do not get an appropriate way to interpret it in arbitrary categories for free.

\subsection{Adjunction}
Note also that there is an adjunction here.
That is, whenever we have a one-directional notion of "Morphisms from As to Bs" for different kinds of objects A and B, it may be that it is representable on either side (that a morphism from A to B is as good as a morphism from f(A) to B in the universe where B lives, or as good as a morphism from A to g(B) in the universe where A lives), and if it is representable on both sides, we have an adjunction.

In our case, a morphism from a set S to an "internal set" X (i.e., object X in an arbitrary lex-category C) can be represented by bringing X into the world of real sets via $\Hom_C(1, X)$ and looking at the set-functions $S -> \Hom_C(1, X)$. Or it can be represented by bringing S into the world of internal sets, by constructing the essentially algebraic theory of functions out of S, and then interpreting this theory into C.

Formally, the adjunction is like so: Take IntSet (the category of internal sets) to be the category whose objects are pairs (C, X) where X is an object in lex-category C, and such that morphisms from (C, X) to (D, Y) are pairs (f, m) where f is a functor : C -> D and m is a morphism in $\Hom_D(f(X), Y)$. Composition of these morphisms proceeds in the obvious way, and is indeed associative. (If you are familiar with "fibered categories" represented via the "Grothendieck construction", this category IntSet is essentially the generic Grothendieck construction, the one whose projection to Cat is the fibered category representation of the identity functor from Cat to Cat).

Then we have a functor from IntSet to Set which sends (C, X) to $\Hom_C(1, X)$. And this functor has a left adjoint from Set to IntSet which sends a set S to the free lex-category interpreting the lex-theory of "a set with a function into it from S".

Does that make sense so far? (None of this adjunction stuff is strictly necessary for what's to come, but still I think it is good to pour here all my thoughts about the yoga of dealing with internal constructions.)

\subsection{Internal structures}
If that does make sense, the next step is to start talking about internal structures more general than just sets, and then to talk about multiply-internal sets and structures. 

Remember above, I said "We can thus consider commutative diagrams whose arrows are a mix of functions from sets to sets, morphisms from object to object inside a category, morphisms from a set to an object of a category, and lex-functors taking an object of one category to an object of another category. Since everything still composes associatively, we can reason about these commutative diagrams in essentially just the ordinary way as for any commutative diagrams.".

This will let us talk not just about morphisms from S to X for sets S and internal sets X (in some category C), but also about morphisms from S to X for structures S and internal structures X (in some category C).

For example, a group is a set (let's call it G) along with some arrows (the group operations; unit from 1 to G, binary multiplication from G x G to G, inverse from G to G), satisfying some properties. It's a particular diagram (in the language of diagrams that are allowed to invoke finite limit structure; in this case, the products 1 and G x G) of sets.

A homomorphism between groups is a morphism of such diagrams; that is, an arrow between domain and codomain sets, that induces a commutative square for each pair of corresponding arrows in the domain and codomain structure.

In just the same way, an internal group is an internal set along with such arrows, satisfying such properties. And we have internal group homomorphisms as such arrows making such things commute, yes, yes.

But also, since we have a notion of maps from sets to internal sets, we also have a notion of homomorphisms from groups to internal groups. A homomorphism from a group with underlying set G to an internal group with underlying object G' is an arrow from G to G' (in the sense we defined above; i.e., a function from G to global elements of G'; i.e., a function from G to Hom(1, G')), such that all the relevant induced squares commute.

Indeed, we have the full general yoga for groups that we had for sets:

Homomorphisms from groups to internal groups can be pre-composed with homomorphisms between groups, or post-composed with internal homomorphisms between internal groups in the same category, or post-composed with lex-functors taking an internal group in one category to an internal group in another category. And all these compositions act associatively.

And our notion of morphisms from groups to internal groups arises from an adjunction:

We have our category of actual groups Group, and we have a category of internal groups IntGroup (whose morphisms are combinations of lex-functors between categories taking one group to another, and internal group homomorphisms within one category between two of its internal groups).

There is a functor from IntGroup to Group given by taking global sections.

And this IntGroup to Group functor has a left adjoint: For every fixed group G, the theory of "A group, along with a homomorphism into it from G" is an essentially algebraic theory, and thus has a free model, the free category with an internal group extending G.

("Again, this is probably my model theory bias, but I think it will be clearer for a lot of people if you specify that this essentially algebraic theory is nothing but the theory of groups, with the language augmented with a constant symbol for every element of G and the theory augmented with the diagram of G."

Yeah, that makes sense; I'll make sure to note that. This is indeed exactly the "diagram" construction from model theory… except you model theorists toss into your diagram the negated sentences that hold (and other such things using arbitrary Boolean connectives), and we will not want to do that. So the theory is for our purposes to be augmented with all the equations in lex-language that hold in G, but it's important to note that lex-language does not include negated equalities.

But, yeah, just as the "diagram of M" is the theory whose models are the same thing as structures extending M in a particular way, and the "elementary diagram of M" is the theory whose models are the same thing as structures extending M in the manner of elementary extensions, yes, we're doing a very similar thing here, so if I spell that out as you say, hopefully people will find the idea familiar enough.)

Now, I specified this for "groups" here, but it works in just the same way for any kind of structure given in terms of diagrams of some shape (perhaps invoking in these diagrams also limits of some kind).

For any limit-theory, we have a notion of actual models of the theory, and also of internal models of the theory in categories with the appropriate limit-structure (for our purposes, we will always be looking at finite limit theories interpreted in categories with finite limits). We have a notion of homomorphisms from actual models to internal models, just given by the appropriate commutative diagram between sets and internal sets.

And this pre-composes with ordinary homomorphisms and post-composes with internal homomorphisms and limit-preserving functors, all in an associative way, all arising from an adjunction between the functor taking an actual structure to the theory of structures extending it, and the functor taking an internal structure to the actual structure given by its global elements.

I become very worried that I'm babbling unclearly, so please, give me feedback soon.

But the key idea in all of this is that once we have the notion of morphisms from sets to internal sets, we can do all the ordinary things we ever do with ordinary diagrams entirely of sets or ordinary diagrams entirely of internal sets, but just allowing for some of the morphisms to go from sets to internal sets.

Next we'll move on to discussing how this automatically extends to considering multiply internal sets and structures.

\subsection{Multiply Internal}
We already have sets (which I'll also call 0-internal sets) and internal sets (i.e., objects in (lex-)categories, which I'll also call 1-internal sets). And we have functions between sets (which I'll call (0,0)-maps) and morphisms between internal sets ( (1, 1)-maps) and also morphisms from sets to internal sets ( (0, 1)-maps).

Every concept we ever define is essentially algebraic.

So, in particular, the concept of "A (lex-)category C, along with an internal set X in C" is essentially algebraic.

(It's just the extension of the theory of (lex-)categories by specifying a designated object in that category)

And as an essentially algebraic theory, it has models not just in Set, but interpretable into any lex-category

Models of this theory in a lex-category D are called a set doubly internal to D; it consists of a category C internal to D, along with extra data picking out our doubly internal set X internal to C (edited) 

Unpacking this definition, more explicitly, a doubly internal set in D is this:

A category C internal to D (thus, objects Ob(C) and Mor(C) in D, along with the morphisms and equalities that equip these with category structure).

A global element of Ob(C) (that is, a map from 1 to Ob(C) in D).

Does that make sense so far? I feel like there's probably a cleaner way to phrase it, but it's also probably clear to you two already anyway.

We can phrase this in terms of globalization of structures as well, in that for C internal to D, we can create an actual category Glob(C), and X then amounts to picking an object from that category. These are just two different ways of looking at the same thing.

Put yet another way, an internal set in a category is a functor from 1 into that category. But because we have the notion of (0, 1)-maps in addition to (0,0)-maps, we have not only the notion of functors from 1 into categories, but also the notion of functors from 1 into internal categories. A functor from 1 into an internal category amounts to the same thing as  a doubly-internal set. 

\subsection{Globalization}
the globalization summary for Reid:

I assume in the following you are happy with the notion of a morphism from a set to an internal set. (Basically, a morphism from set S to internal set (i.e., object-in-category) X amounts to a function from S to the global elements of X)

Recall that our notion of morphisms from sets to internal sets can be seen as arising from an adjunction of functors between Set and IntSet, where IntSet is the category whose objects are pairs (C, X) [a (lex-)category C, and an object, aka internal set, X in C] and whose morphisms F : (C, X) -> (D, Y) are (lex-)functors F along with a morphism m : F(X) -> Y in D, composing in the expected way.

The right adjoint in this adjunction sends (C, X) to the set of global elements of X; i.e., $\Hom_C(1, X)$. This is called "globalization" of internal sets.

The left adjoint in this adjunction (though I should note it will cease to exist later in some contexts, since its construction depends on colimits) sends a set S to (C, X) where C is the lex-category representing the essentially algebraic theory of a set X along with a function from S to X; i.e., the free lex-category generated by an object X and S-many maps from 1 to X. By the model theory analogy Alex gave before, we could call this the "algebraic diagram" of S.)

All of this is pretty straightforward when thinking just about sets, but in fact, for ANY lex-theory (and the same thing works even with theories with infinitary operations, but let's just always talk about lex-theories) T, we can form the category of its models $Mod(T) = Set^T$ and the category of its internal models $IntMod(T) = IntSet^T$, where $C^T$ means "the category of lex-functors from T to C, and natural transformations between them".

And our adjunction will automatically lift in the same way to an adjunction between Mod(T) and IntMod(T), with a functor from IntMod(T) to Mod(T) that sends an internal model of T to an actual model of T, by taking global elements of all its underlying internal sets to turn them into actual sets. This is what I call "globalization", of structures in general.

(And also there is a functor from Mod(T) to IntMod(T) which takes an actual model M of T to the theory describing extensions of M, which is to say, the free lex-category with an internal model of T extending M. Again, we can call this the "algebraic diagram" functor.] 
Does that make sense?

But again, I want to say, even though globalization is important, I think in the past I've over-emphasized it at the beginning. You don't actually need to think about "globalization" as such at all, for defining introspective theories; you can just think about the more directly relevant notion of "maps from structures to internal structures".

Am I being confusing so far? I need feedback.

There is another notion which I'm going to call "internalization", but I'm not very good with names, so give me feedback on this as well:

To any theory T, we can straightforwardly associate another theory Internal(T) whose models are lex-categories with internal models of T.

This is the process which turns the theory of a set into the theory of a lex-category with an internal set, for example.

We can iterate this process, and get the theory of a lex-category with an internal "lex-category with an internal set", which was our notion of doubly-internal set.

And we can keep going ad infinitum, getting triply-internal sets, etc.
Does that all make sense?

If you are comfortable with Globalization, what this amounts to is that a multiply internal set is a lex-category with an internal lex-category, within whose globalization we pick another internal lex-category, within whose globalization we pick another internal lex-category, terminating at some point with picking an internal set rather than an internal lex-category.

So there's a kind of tree structure, whose nodes are strings "C;D;E;F;G" where C is a lex-category, D is a lex-category internal to C, E is a lex-category internal to D, etc.

And we can arrange for this tree to have also internal sets as its leaf nodes, as children of the categories they live in.
Does that make sense?

I fear I probably need to provide some clarification on the above but await feedback.

Ah, I feel like I'm wasting time explaining some things that maybe aren't important again. There's a trade-off between explaining everything I ever think about, so you have the same conceptual tools at the ready that I do, and just explaining what's necessary to get us going with understanding what an introspective theory, to start.

So, the adjunction, the category IntSet with its two kinds of morphisms, globalization, algebraic diagrams. That's all not strictly necessary to understand.
The important concepts to start are internal and multiple internal sets and structures, and maps from sets to multiply internal sets.

Alex  9:55 AM
You should probably write this down as a lemma and write down a proof. Lemma: Let T be an essentially algebraic theory. Then there is an essentially algebraic theory Internal(T) such that a model of Internal(T) is a lex category with an internal model of T. Proof: ...

Incidentally, when we say "model" here, we mean "model in Set". And then (blithely ignoring "up to equivalence" issues everywhere), uniqueness of Internal(T) comes down to the fact that an essentially algebraic theory is determined by its models in Set. One way to see this is to note that the Set models of an essentially algebraic theory T form a locally finitely presentable category, and its subcategory of finitely presentable objects is $T^op$, so we can recover T from the category of Set-models of T. Do I have this right?

Sridhar  10:00 AM
Re: https://srcat.slack.com/archives/CBE0GC6S1/p1530712556000035
Ah, ok. The proof is to take the "algebraic diagram" of the lex category T. So perhaps really the thing to write is the lemma that these "algebraic diagram"s exist. 

(As reminder, I use this algebraic diagram terminology like so: the AD of a model M of theory T is the theory of models of T extending M; that is, the theory T extended by constants for each member of M, and equations for each equation of T operations satisfied by these members of M)

So we have these notions of sets, internal sets, doubly internal sets, etc.

And we have these notions of (0, n) maps, by which I mean, a map from a set to an n-internal set. Remember, this works by taking the models of $Internal^n$(the theory of functions out of S) as the (0, n) maps out of S, for any fixed set S.

From the above it is clear that for any fixed set S and natural n, the theory of a (0, n) map out of S is essentially algebraic.

But an important realization that may not be obvious, and which I want you to ponder on, is this: it is in fact the case that for any fixed natural n, the theory of an arbitrary (0, n) map [i.e., the theory of a set S and a (0, n) map out of S] is also essentially algebraic.

If this is not obvious, it may be illustrative to sit down and actually just write out for yourself a presentation of the essentially algebraic theory of a (0, 1) or a (0, 2) map, for example, to see how this works. 

The reason it is important that the theory of a (0, n) map is essentially algebraic is because it means we can therefore construct the theory $Internal^m$(the theory of a (0, n) map) as well.

Models of this theory are a kind of map from an m-internal sets to an (m + n)-internal set; I'll call this an (m, m + n) map.

(a, b) maps can be composed with (b, c) maps to get (a, c) maps, and this composition structure is associative.
This will fully justify writing the kinds of diagrams of maps between internal sets at different levels of internality that I intend to write. I will write diagrams with maps from a-internal sets to b-internal sets, meaning by this always an (a, b) map in the sense just defined.

Alex  10:40 AM
So the theory of a (0,1) map has sorts S, Ob, Mor, the structure of a category C on (Ob, Mor), a constant symbol x of sort Ob, and a function $f: S -> \Hom_C(1_C,x)$.

Sridhar  10:41 AM
Yup

Alex  10:42 AM
That is, theory of a (0,n) map out of S always has an S-indexed family of constant symbols living in some definable set, and we replace that family with an actual function to that definable set from a sort called S.

Can you justify the composition of (a,b) maps with (b,c) maps being well-defined and associative using an adjunction to de-internalize? Or do you have some other way of proving this?

(This is one of those things that couldn't not be true, but you will have to write down a proof at some point)

Sridhar  10:51 AM
Re: https://srcat.slack.com/archives/CBE0GC6S1/p1531147360000383: Yup.
In slightly more detail, the theory of a (0, n) map extends the theory of an n-internal set (the codomain), so it contains a tower of internal categories terminating in the codomain n-internal set X. And then we obtain a definable set from this X by essentially applying Hom(1, -) over and over, climbing up the tower of internal categories, [i.e., repeated Globalization], and then using an actual function from S to this definable set.

Basically, our "S -> $Globalization^n$(X)" definition is immediately essentially algebraic, as Globalization is an essentially algebraic procedure.

[This is as opposed to the "$Internal^n$(Algebraic Diagram of S) -> X" definition, as the Internal(...) operation (and the Algebraic Diagram operation, for that matter) is not an essentially algebraic procedure, but rather a kind of colimit construction. Hence, my previous (over-)emphasis on Globalization. (Maybe there's a way to understand Internal and Algebraic Diagram as lex constructions using categories with specified finite limits instead of categories with all finite limits, but I haven't bothered working it out)] (edited) 

Alex
That is, theory of a (0,n) map out of S always has an S-indexed family of constant symbols living in some definable set, and we replace that family with an actual function to that definable set from a sort called S.

Sridhar  11:41 AM
Re: https://srcat.slack.com/archives/CBE0GC6S1/p1531147588000194:
Yes. In fact, there's a whole theory of abstract "tree-categories" (the first thing I came up with) that generalizes all this, but seems to only distract people to explain. I'll probably still talk about it at some point. But for now:

The relevant idea for our purposes is this: An (m, m + x) map is a kind of morphism between m-tuply-internal sets living in the same (m-1)-tuply-internal category.

[Perhaps I'll just say "m-whatever" from now on as shorthand for "m-tuply-internal-whatever"].

In more detail, an (m, m + x) map from A to B amounts to a morphism of m-sets between A [which is itself an m-set] and $Glob^x$(B) [the m-set that results from taking the (m + x)-set B, constructing its (m + x - 1)-set of global elements Glob(B), then constructing ITS (m + x - 2)-set of global elements $Glob^2$(B), etc., iterated x times].
In other words, an (m, m + x) map from A to B is the same data as an (m, m) map from A to $Glob^x$(B).

[This representation can be convenient, because (m, m) maps are perfectly ordinary straightforward things; the theory of (m, m) maps is just $Internal^m$(the theory of functions), and thus they compose associatively, etc, automatically].

But also, every morphism of m-sets gives rise to an associated morphism of (m - 1)-sets, its action on global elements. [The case of this where m = 1 is just the observation "A morphism between objects takes global elements of the domain to global elements of the codomain", and each further case is just the re-internalization of this observation.].

Thus, to compose an (m, m + x) map f: A -> B [a kind of morphism of m-sets from A to $Glob^x$(B)] with an (m + x, m + x + y) map g : B -> C [a kind of morphism of (m + x)-sets, from B to $Glob^y$(C)], we apply the previous paragraph's idea x many times to the latter map to obtain g', a morphism of m-sets from $Glob^x(B)$ to $Glob^(x + y)(C)$. Now we can compose f with g' as just ordinary composition (of morphisms between m-sets), to obtain a morphism of m-sets from A to $Glob^(x + y)(C)$, which is precisely what it is to be an (m, m + x + y) morphism from A to C.

This defines our notion of composition for these internalization-level-shifting maps.

In other words, composing f of type (m, m + x) from A to B with g of type (m + x, m + x + y) from B to C amounts to taking the ordinary (m, m) composition of f with $GlobalAction^x(g)$. (edited) 

Alex
Can you justify the composition of (a,b) maps with (b,c) maps being well-defined and associative using an adjunction to de-internalize? Or do you have some other way of proving this?

Sridhar  11:56 AM
As for associativity, this follows immediately once we realize that everything is defined in a lex way, and thus preserved by GlobalAction. Actually, all that matters is the functoriality of GlobalAction:

Given f, g, and h, as level-shifting maps which we wish to compose, with f shifting down x levels of internalization and g shifting down y levels of internalization, and writing F, G, and H for the equivalent data thought of as non-level-shifting maps, we have that (f . g) . h amounts to $F \circ GlobalAction^x(G) \circ GlobalAction^(x + y)(H)$, while f (g h) amounts to $F \circ  GlobalAction^x (G \circ GlobalAction^y(H))$.

Because GlobalAction is functorial and thus distributes across the relevant composition, these two expressions are equal.

Does that make sense? I'm sort of making up ad hoc haphazard notation for what I'm trying to say here, with my lowercase and uppercase distinction, but I think maybe you can follow what I mean?

I think things have reached (or even passed) the point where I should make the first blogposts soon, cleaning up much of what we've discussed so far. We're almost out of the weeds of what we need to know about internal sets and level-shifting maps between them.

We're very nearly ready to move on to actually discussing introspective theories. I'll just spend a little more time discussing the category of level-shifting-maps between arbitrary-level internal-sets, to try and establish familiarity and fluency with its yoga.

Alex  1:13 PM
Great, I'm with you on all of the above

Sridhar  1:14 PM
Great!

So now I want to draw attention back to the tree of all sets and (lex-)categories and multiply internal sets and lexcats.

We've defined a notion of morphisms between the various sets in this tree, where it makes sense to ask about the morphisms from m-set A to (m + x)-set B, just in case B is a proper descendant of A's parent.
https://srcat.slack.com/archives/CBE0GC6S1/p1530486943000002

Sridhar
So there's a kind of tree structure, whose nodes are strings "C;D;E;F;G" where C is a lex-category, D is a lex-category internal to C, E is a lex-category internal to D, etc.

And we can arrange for this tree to have also internal sets as its leaf nodes, as children of the categories they live in.

Does that make sense?

Sridhar  1:26 PM
Since we have an associative notion of composition of these morphisms, we have indeed a genuine category whose objects are the (any level)-set nodes in this tree.
I'll call this category TreeSet or something. Feel free to suggest a better name.

There are various constructions and properties and whatever relating this category structure to the tree structure, but the main thing for now is just to keep in mind that a morphism from A to B can exist just in case B is a proper descendant of A's parent (which is a category node rather than a set node), its data amounting to a diagram of some sort within that parent category (meaning a globally defined diagram within that internal category, to whatever extent that category is multiply internal).
Keep this in mind as we go along and it will help you be aware of what kinds of things make sense and how, and what doesn't make sense, in the yoga of level-shifting maps across multiply internal sets. A useful sanity check, like dimension checking in physics.

Dale  1:30 PM
I'm happy with everything that's happened so far.

Sridhar  1:30 PM
Great.
1:31
My task for now is to blogpost up what has been done so far. Then I think we can move on to discussing introspective theories.

There's still some more to say about the yoga of internal sets, etc, but I think we can handle it bit by bit as it comes up in context, from hereon.

Sridhar  1:40 PM
I want to re-stress that every thing we've ever discussed so far has been essentially algebraic definitions, constructions, proofs, etc, with the exception of the process of freely adding all remaining finite limits to a category with merely some specified finite limits, and the use of this in A) turning a lexcat T into the lexcat Internal(T), or B) turning a lexfunctor T -> Set [i.e., a model of T] into the lexcat of its Algebraic Diagram [i.e., the theory of extensions of that model].

A) and B) were used expositionally above, but ultimately these general constructions aren't necessary qua general constructions for anything we've done in defining and proving facts about (m, n) maps. All of our constructions concerning (m, n) maps are fully internalizable to any lex context. (edited) 

Sridhar  7:46 PM
Ok, a concept we're about to need is that of an internal (presheaf on a category).

The theory of presheaves on a category is essentially algebraic in a completely standard way. Given an internal category C, a presheaf on it is a slice P : Whatever -> Ob(C), along with an appropriate action of Mor(C) on this.

Hm, I should maybe spell this out in more detail, but for now, can I trust that you guys see how the theory of "a presheaf on a category" is understood as essentially algebraic? The key thing is to interpret Set-valued functions not directly in those terms, but rather by slice representation. (edited) 

A function between sets or a morphism in a multiply internal category is an (n, n) map. We've also spoken about functions from sets to objects, and the internalized like, which are (n, n + m) maps; that is, maps that takes us deeper into internalization.
A presheaf, though, is sort of like an (n, n - 1) map. If C is internal to V which is internal to W, then a presheaf on C is like a map from Ob(C) to Ob(V), which is from an object in V to an object in W, climbing rather than descending in level of internalization. (edited) 

Sridhar  10:32 PM
More generally, this is true of slices. A slice over object X in category V is like a map from X to Ob(V).

Sridhar  1:44 PM
For example, given a slice over X and a map f : Y -> X, we can pull the slice back along f to get a slice over Y. In the particular case where Y = 1, this turns global elements of X into objects of the category X lives in.

Sridhar  1:53 PM
Two warnings:
The sense in which a slice over X in category V represents a map from X to Ob(V) is of course only up to isomorphism of objects in V; there's no guarantee that our chosen pullbacks will ever be able to represent hitting particular elements of Ob(V) on the nose, nor do we want to care about such things anyway. Our Ob(V)'s evilly carry equality structure beyond isomorphism, but we'll ignore this. The old evil wart.
Also, the sense in which a slice over X in category V acts like an (n, n - 1) map from X to Ob(V) doesn't give us any general theory of (n, n - 1) maps with arbitrary codomain. Perhaps our formal theory of (m, n) maps could be generalized in some fashion, but for now, I say slices act like (n, n - 1) maps only to guide your intuition; not because they actually fit into our formal theory of (n, m) maps

Sridhar  7:41 PM
Let me point out one little observation-lemma about internal-sets-etc that it will be good to be comfortable with as we head into proofs:

Sridhar  7:52 PM
Suppose X is an object in W internal to V.
Then, when it comes to considering what the global elements of X are, it doesn't matter whether we use $1_V$ or $1_W$, in the sense that $\Hom_{TreeSet}(1_V, X)$ amounts to the same thing as $\Hom_{TreeSet}(1_W, X)$.
This is because both of these end up meaning $\Hom_V(1_V, \Hom_W(1_W, X))$.

The above is the binary case, but in just the same way, if X is an object internal to $V_1$ internal to $V_2$ internal to $V_3$ internal to..., and we consider $\Hom(1_{V_i}, X)$ for any i, we get the same thing. $\Hom_{TreeSet}(1, X)$ is well-defined regardless of which "1" we use.

For sake of a name to refer to this by, let's call this "1-conflatability". (edited) 

Sridhar  9:20 PM
Hm, returning to the discussion of slices, it will perhaps be useful to speak more about how to "compose" an (n, m) map with a slice (thought of as an (m, m - 1) map) to get an (n, m - 1) map.

Sridhar  1:16 AM
For what it's worth, generally, the most salient thing about an (n, n + m) map is the difference m, not the base level n. (After all, an (n, n + m) map is supposed to be just the same thing as a (0, m) map, but in the internal logic of some multiply internal category). Accordingly, I will also call these things "m-diving maps", emphasizing the salient quantity; it's a map which dives down a further m levels of internality, between domain and codomain.

\subsection{Abstract Tree-categories}
TODO. Optional, perhaps will not write.

\end{comment}
% We no longer need this, although it may still be useful to discuss for some purposes.

\filestart

\section{Category-theoretic preliminaries}

\subsection{Higher categorical terminology conventions}
We assume familiarity with sets, functions, categories, functors, natural transformations, limits, presheaves, $\Set$, $\Cat$, all in the ordinary sense. At times, we may also call upon some comfort with concepts such as 2-categories or abstract Kan extensions. It will also be very useful to have some familiarity with functorial semantics and internal algebraic structures such as internal categories.

We will take all categories to be locally small (which is to say, we will take $\Set$ to be large enough to include the hom-set between any two objects of any category we work with). Generally speaking, we are interested in the categories we work with being small as well, except for those particular large categories such as $\Set$, $\Set^X$, $\Cat$, etc, but wherever this is important we will make some explicit note.

We write $m \circ n$ or just $mn$ for composition of morphisms $n : X \to Y$, $m : Y \to Z$ in a category. We may occasionally write $n ; m$ to mean $m \circ n$.

We use the term \defined{lexcategory} for a category with finite limits. We use the term \defined{lexfunctor} for a functor preserving finite limits, whose domain and codomain are both lexcategories. By $\LexCat$, we mean the 2-category of lexcategories, lexfunctors, and natural transformations. We will not generally be explicit about making distinctions between $f(a \times b)$ and $f(a) \times f(b)$, etc, when $f$ is a lexfunctor, but shall instead write with the ordinary fluency for working with limit-preserving functors.

We will speak frequently of category-valued presheaves (i.e., contravariant functors into the category of categories) and natural transformations between these. Technically, what we mean by these are not \quote{functors} and \quote{natural transformations} in the traditional sense, but what some call "pseudofunctors" and \quote{pseudonatural transformations}, or \quote{2-functors} and \quote{2-natural transformations}, as the category of categories should be viewed as a 2-category (by which we mean the non-strict concept some call \quote{bicategory}), lacking a notion of equality between its 1-cells and only having a notion of isomorphism between them instead. That is, wherever one might traditionally ask for an (automatically coherent) system of equalities, this is replaced by a coherent system of isomorphisms. We take the convention that this is what terminology such as \quote{functor} and \quote{natural transformation} already means, in such a context. But we will try our best to construct arguments in such a way as that this is not a bother that needs to be explicitly worried about.

Similarly, we do not worry about distinguishing between terms like \quote{isomorphic} and \quote{equivalent} in statements like \quote{category $C$ is isomorphic/equivalent to category $D$}, always meaning by such a statement an adjoint equivalence. Everything means the weakest thing it could mean, unless we explicitly say we are dealing with something stricter.

Similarly, if we ever describe diagrams involving functors between categories as commuting, we really mean that these diagrams commute up to natural isomorphism. If we make claims about uniqueness in such a context, we mean the space of choices with the relevant isomorphisms is contractible. And so on. Again, our convention is that this is what such terminology already means, in any categorical context where one has such concepts of isomorphism around, unless we have taken care to say we are working with stricter notions instead (see more on strictness below). Unless we have said we are talking about strict notions, we never distinguish between equivalent categorical structures.

(That all said, nothing we do is higher-dimensional than 2-categorical, so everything could in theory be strictified in some fashion, if so desired.)

\subsection{Indexed, \repsmall/, and internal structures}
We will now give a series of related definitions, concerning what are called indexed structures. The notions being described in this section are all relatively old hat, none of them are newly invented by us, but we wish to pin them down with particular names to establish a vocabulary for easily talking about the things we wish to talk about in the rest of this document.

As we give these definitions, we will also observe a basic stock of theorems about them which. Again, we make no claim to originality with these preliminaries. They simply may be useful to remind the reader of, or to give labels to in order to reference as we use them.

The reader who is already very familiar with these notions and just unfamiliar with our conventions of vocabulary is advised to just skim these preliminaries on initial read and then return as needed when faced with unfamiliar vocabulary. Frankly, the reader who is not very familiar with these notions is also given similar advice. It is probably best to read a bit of the preliminaries to get the lay of the land, then go off and read the actual content and come back as needed. But who knows? To each reader, their own reading style may be best. \TODOinline{Figure out exactly the roadmap or reading advice we want to give.}

The key notion upon which everything else builds is the following:

\begin{definition}
Let $T$ be an arbitrary category. By a $T$-\defined{indexed set}, we mean a presheaf on $T$; that is, a contravariant functor from $T$ to $\Set$. By a \defined{function} or \defined{map} or any such thing between $T$-indexed sets, we mean a natural transformation between the corresponding presheaves.
\end{definition}

The category of $T$-indexed sets and maps between them is thus the presheaf category $\Set^{\op{T}}$. We may also refer to this as $\Psh{T}$.

We may refer to the data of an indexed set at any object $t$ of the category over which it is indexed as its data \defined{defined over} $t$, or which is $t$-\defined{indexed}, or as its $t$-\defined{aspect}. We can refer to the $t$-aspect of an indexed set $P$ as the set $P(t)$ or $P_t$.

Note that data defined over $t$ is automatically transferred to corresponding data defined over $s$ by any morphism from $s$ to $t$ in $T$, by the action of the presheaf. More explicitly, given morphism $m : s \to t$ in $T$, we may write $P(m) : P(t) \to P(s)$ for the corresponding function in $\Set$, or $P_m$.

In contexts where it is clear what presheaf $P$ we have in mind, we may also write $\pullAlong{m}$ for $P(m)$. Also, in contexts where it would cause no confusion to speak in this way, given some $t$-indexed datum $d \in P(t)$ and a morphism $m : s \to t$, we use the same name $d$ also to refer to the corresponding $s$-indexed datum which more explicitly would be called $P(m)(d)$ or $\pullAlong{m} d$. It will be especially common for us to abuse language in this name-reusing way when $t$ is a terminal object.

In the particular case where $t$ is a terminal object, we may refer to the aspect at $t$ of an indexed set as its \defined{global aspect}. By the Yoneda lemma, this global aspect data $P(1)$ of a presheaf $P$ on category $T$ is the same as the data of a map from the terminal object $1$ to $P$, which is the same as the data of a map from the constantly $1$ presheaf to $P$. This is also the same as the data of the limit of $P$, thought of a $\op{T}$-indexed diagram. In this way, even if $T$ does not have a terminal object, we may still speak of the global aspect of $T$-indexed sets.

\begin{definition}
We say an indexed set is \defined{\repsmall/} (or $T$-\repsmall/, when we wish to emphasize which indexing category we are talking about) if the corresponding presheaf is representable. \TODOinline{I am in the middle of changing instances of "small" in this document to "\repsmall/". They have not all changed yet, so you will see the word \quote{small} still used in much of the document to mean \repsmall/.}
\end{definition}

It is perhaps a bit misleading to use \quote{small}-derived terminology here, as this notion is not closed under subobjects. Indeed, what might be considered the smallest indexed set, the one which constantly takes the value of the empty set, is never \repsmall/ in this technical representability sense. But the analogy to the familiar distinction between \quote{small} sets and non-\quote{small} proper classes is often a fruitful one, motivating this terminology. Cf. Definition 7.3.3 in \autocite{jacobs1999categorical}, which uses the word \quote{small} in essentially the same way. When the indexing category $T$ has finite limits (or even just splittings of idempotents), note that this notion of \repsmall/ is equivalent to the standard notion \quote{tiny}; see \url{https://ncatlab.org/nlab/show/tiny+object#in_presheaf_categories}.

The word \quote{small} of course has a conventional meaning of set-sized (as opposed to proper-class-sized). When we wish to be clear that this is what we mean and not risk confusion with \repsmall/, we will say explicitly \defined{\setsmall/}.

\begin{convention}
Via the Yoneda embedding (which we denote $\yoneda$), we identify $T$ itself as the full subcategory of \repsmall/ $T$-indexed sets within the category of all $T$-indexed sets. In this way, we may speak, for example, of functions from objects of $T$ to $T$-indexed sets. That is, when $t$ is an object of $T$, we will readily write $t$ also to mean the Yoneda embedding of $t$, when we wish to treat it as a \repsmall/ $T$-indexed set; we will usually not explicitly write $\yoneda(t)$. And conversely, given a \repsmall/ $T$-indexed set $P$, we freely write also $P$ to name an object in $T$ representing $P$, rather than explicitly writing $\yoneda^{-1}(P)$.
\end{convention}

\begin{theorem}
Note that \repsmall/ sets, construed as objects of $\Psh{T}$, are closed under any limits which exist in $T$. In particular, if $T$ is a lexcategory, \repsmall/ sets are closed under finite limits. (This is essentially the observation that the Yoneda embedding preserves limits.)
\end{theorem}

\begin{definition}
Note that given an arbitrary functor $f : S \to T$, this induces by composition a functor $\pullAlong{f} : \Psh{T} \to \Psh{S}$.

That is, from a $T$-indexed set $P$, we may construct the following $S$-indexed set $\pullAlong{f} P$:

% https://q.uiver.app/?q=WzAsMyxbMCwwLCJcXG9we1N9Il0sWzIsMCwiXFxvcHtUfSJdLFs0LDAsIlxcU2V0Il0sWzAsMSwiXFxvcHtmfSJdLFsxLDIsIlAiXV0=
\[\begin{tikzcd}
	{\op{S}} && {\op{T}} && \Set
	\arrow["{\op{f}}", from=1-1, to=1-3]
	\arrow["P", from=1-3, to=1-5]
\end{tikzcd}\]
\end{definition}

\begin{theorem}\label{YonedaPullalong}
Given $f : S \to T$ and an object $s$ in $S$ and a $T$-indexed set $P$, we have that $\Hom(s, \pullAlong{f}P) = \Hom(f(s), P)$, with this correspondence being natural in both $s$ and $P$.
\end{theorem}
\begin{proof}
Keep in mind that in these Hom-expressions, $s$ and $f(s)$ have implicitly been construed as $S$-indexed sets via the Yoneda embedding. That is, more explicitly, our claim is $\Hom(\yoneda(s), \pullAlong{f}P) = \Hom(\yoneda(f(s)), P)$. To establish this claim, we can apply the Yoneda lemma to both sides of the equation to reduce it to $(\pullAlong{f}P)(s) = P(f(s))$, which is the definition of $\pullAlong{f}$.

This completes the proof. (In fancy categorical jargon, we have demonstrated that $\yoneda \circ f : S \to \Psh{T}$ is the relative left adjoint of $\pullAlong{f} : \Psh{T} \to \Psh{S}$, relative to the Yoneda embedding $\yoneda : S \to \Psh{S}$.)
\end{proof}

\begin{corollary}\label{YonedaPullalongLemma}
Given $f$, $s$, and $P$ as in \magicref{YonedaPullalong}, we have that every morphism $m : s \to \pullAlong{f} P$ factors through a morphism in the range of $\pullAlong{f}$. That is, $m = \pullAlong{f}(m') \circ \eta$ for some $m' : f(s) \to P$ and $\eta : s \to \pullAlong{f}(f(s))$.
\end{corollary}
\begin{proof}
This is corollary to \magicref{YonedaPullalong} by the general yoga of relative adjoints.

Specifically, consider the following naturality diagram for the correspondence in \magicref{YonedaPullalong}, where $m'$ is the morphism in $\Hom(f(s), P)$ corresponding to $m \in \Hom(s, \pullAlong{f}P)$ and $\eta$ is the morphism in $\Hom(s, \pullAlong{f}f(s))$ corresponding to $\id_{f(s)} \in \Hom(f(s), f(s))$.

% https://q.uiver.app/?q=WzAsOCxbMCwwLCJcXEhvbShmKHMpLCBmKHMpKSJdLFszLDAsIlxcSG9tKGYocyksIFApIl0sWzAsMywiXFxIb20ocywgXFxwdWxsQWxvbmd7Zn1mKHMpKSJdLFszLDMsIlxcSG9tKHMsIFxccHVsbEFsb25ne2Z9UCkiXSxbMSwxLCJcXGlkX3tmKHMpfSJdLFsyLDEsIm0nIl0sWzEsMiwiXFxldGEiXSxbMiwyLCJcXHB1bGxBbG9uZ3tmfShtJykgXFxjaXJjIFxcZXRhID0gbSJdLFswLDEsIm0nIFxcY2lyYyAtIl0sWzIsMywiXFxwdWxsQWxvbmd7Zn0obScpIFxcY2lyYyAtIl0sWzIsMCwiIiwxLHsibGV2ZWwiOjIsInN0eWxlIjp7ImhlYWQiOnsibmFtZSI6Im5vbmUifX19XSxbMywxLCIiLDEseyJsZXZlbCI6Miwic3R5bGUiOnsiaGVhZCI6eyJuYW1lIjoibm9uZSJ9fX1dLFs0LDUsIiIsMSx7InN0eWxlIjp7InRhaWwiOnsibmFtZSI6Im1hcHMgdG8ifX19XSxbNCw2LCIiLDEseyJzdHlsZSI6eyJ0YWlsIjp7Im5hbWUiOiJtYXBzIHRvIn19fV0sWzYsNywiIiwxLHsic3R5bGUiOnsidGFpbCI6eyJuYW1lIjoibWFwcyB0byJ9fX1dLFs1LDcsIiIsMSx7InN0eWxlIjp7InRhaWwiOnsibmFtZSI6Im1hcHMgdG8ifX19XV0=
\[\begin{tikzcd}
	{\Hom(f(s), f(s))} &&& {\Hom(f(s), P)} \\
	& {\id_{f(s)}} & {m'} \\
	& \eta & {\pullAlong{f}(m') \circ \eta = m} \\
	{\Hom(s, \pullAlong{f}f(s))} &&& {\Hom(s, \pullAlong{f}P)}
	\arrow["{m' \circ -}", from=1-1, to=1-4]
	\arrow["{\pullAlong{f}(m') \circ -}", from=4-1, to=4-4]
	\arrow[Rightarrow, no head, from=4-1, to=1-1]
	\arrow[Rightarrow, no head, from=4-4, to=1-4]
	\arrow[maps to, from=2-2, to=2-3]
	\arrow[maps to, from=2-2, to=3-2]
	\arrow[maps to, from=3-2, to=3-3]
	\arrow[maps to, from=2-3, to=3-3]
\end{tikzcd}\]
\end{proof}

\begin{theorem}\label{AspectIsSliceGlobal}
Let $\Sigma$ be the forgetful functor from a slice category $T/t$ to its ambient category $T$. Then the $t$-aspect of a $T$-indexed set $P$ is in correspondence with the global aspect of $\pullAlong{\Sigma} P$.
\end{theorem}
\begin{proof}
This is corollary to \magicref{YonedaPullalong}, which tells us $\Hom_{\Psh{T/t}}(1_{T/t}, \pullAlong{\Sigma} P)$ is in correspondence with $\Hom_{\Psh{T}}(\Sigma 1_{T/t}, P)$, where $1_{T/t}$ is the terminal object in $T/t$. As this terminal object is given by the identity morphism into $t$, we have that $\Sigma 1_{T/t} = t$. Thus, this equation is telling us that the global aspect of $\pullAlong{\Sigma} P$ corresponds to the $t$-aspect of $P$, as desired.
\end{proof}

\begin{theorem}\label{KanExtensionOfAdjoint}
If $f \adjointTo g$, then $\pullAlong{f} \adjointTo \pullAlong{g}$. Thus $\pullAlong{f} = \Lan_{\op{g}}$, while $\pullAlong{g} = \Ran_{\op{f}}$.
\end{theorem}
\begin{proof}
\TODOinline{This is simply the fact that adjunction is preserved by 2-functors and reversed by each of co and op, thus preserved by $\Hom(\op{-}, C)$ for a fixed category $C$. We don't really need to remark on this theorem here in the final draft, but I'm including it for my own reference, so I can stop getting things backwards in my head.}
\end{proof}

\begin{theorem}\label{RepsmallRightAdjoint}
If $f$ has a right adjoint $g$, then $\pullAlong{f}$ takes \repsmall/ sets to \repsmall/ sets. Specifically, $\pullAlong{f}(t) = g(t)$.
\end{theorem}
\begin{proof}
$\pullAlong{f}$ takes any representable presheaf with representing object $t$ in $T$ to the representable presheaf $\Hom_T(f(-), t) = \Hom_S(-, g(t))$.
\end{proof}

\begin{theorem}\label{PullalongIsLex}
Any functor of the form $\pullAlong{f}$ preserves finite limits.
\end{theorem}
\begin{proof}
This can be seen in several ways. Perhaps most familiarly, this can be seen from the fact that (co)limits in a a functor category are computed pointwise where the pointwise (co)limits exist, and of course set-sized (co)limits all exist in $\Set$. Secondly, when the domain of $f$ is a small category, it can be seen from the fact that $\pullAlong{f}$ has left and right adjoints (the left and right Kan extensions along $\op{f}$), so that $\pullAlong{f}$ in fact preserves ALL (co)limits that happen to exist, regardless of size. We can also note that $\pullAlong{f}(P) = \Hom(f(-), P)$, which is manifestly limit preserving (though this argument does not generalize as easily to colimit-preservation).
\end{proof}

We also define more generally the concept of a function between indexed sets having \repsmall/ fibers:
\begin{definition}\label{RepsmallFibersDefn}
A function $f : A \to B$ between $T$-indexed sets has \defined{\repsmall/ fibers} if the pullback of $f$ along any map into $B$ from a \repsmall/ set is itself \repsmall/ (thus, lives within a slice category of $T$). That is, we say $f$ has \repsmall/ fibers just in case for every pullback diagram of the following sort within the category of $T$-indexed sets, if $t$ is \repsmall/, then so is $s$:

% https://q.uiver.app/?q=WzAsNCxbMSwwLCJBIl0sWzEsMSwiQiJdLFswLDAsInMiXSxbMCwxLCJ0Il0sWzAsMSwiZiJdLFsyLDBdLFsyLDNdLFszLDFdLFsyLDEsIiIsMix7InN0eWxlIjp7Im5hbWUiOiJjb3JuZXIifX1dXQ==
\[\begin{tikzcd}
	s & A \\
	t & B
	\arrow["f", from=1-2, to=2-2]
	\arrow[from=1-1, to=1-2]
	\arrow[from=1-1, to=2-1]
	\arrow[from=2-1, to=2-2]
	\arrow["\lrcorner"{anchor=center, pos=0.125}, draw=none, from=1-1, to=2-2]
\end{tikzcd}\]
\end{definition}

(Beware that when $T$ does not have finite limits, this definition does not have the properties which might be expected. For example, we might expect that any morphism between $T$-\repsmall/ sets should have $T$-\repsmall/ fibers, which would not be true if $T$ itself did not have pullbacks. If $T$ does not have binary products, it will not even be true that a map into the terminal object $1$ has \repsmall/ fibers whenever its domain is \repsmall/.)

\begin{theorem}\label{RepsmallSumOfRepsmallFibers}
If $f : A \to B$ has \repsmall/ fibers and $B$ is \repsmall/, then $A$ is \repsmall/ too.
\end{theorem}
\begin{proof}
Apply the definition of \repsmall/ fibers to the trivial case of pulling $f$ back along $\id_B$.
\end{proof}

The following two theorems follow from the composition of pullback squares into larger pullback squares (or pullback rectangles, one might say):

\begin{theorem}
Maps with \repsmall/ fibers are closed under composition.
\end{theorem}
\begin{proof}
% https://q.uiver.app/?q=WzAsNixbMSwwLCJBIl0sWzEsMSwiQiJdLFswLDAsInMiXSxbMCwxLCJ0Il0sWzEsMiwiQyJdLFswLDIsInUiXSxbMCwxLCJmIl0sWzIsMF0sWzIsM10sWzMsMV0sWzIsMSwiIiwyLHsic3R5bGUiOnsibmFtZSI6ImNvcm5lciJ9fV0sWzEsNCwiZyJdLFszLDVdLFs1LDRdLFszLDQsIiIsMSx7InN0eWxlIjp7Im5hbWUiOiJjb3JuZXIifX1dXQ==
\[\begin{tikzcd}
	s & A \\
	t & B \\
	u & C
	\arrow["f", from=1-2, to=2-2]
	\arrow[from=1-1, to=1-2]
	\arrow[from=1-1, to=2-1]
	\arrow[from=2-1, to=2-2]
	\arrow["\lrcorner"{anchor=center, pos=0.125}, draw=none, from=1-1, to=2-2]
	\arrow["g", from=2-2, to=3-2]
	\arrow[from=2-1, to=3-1]
	\arrow[from=3-1, to=3-2]
	\arrow["\lrcorner"{anchor=center, pos=0.125}, draw=none, from=2-1, to=3-2]
\end{tikzcd}\]

When we presume $g$ to have \repsmall/ fibers, we find that $t$ is \repsmall/. Then when we presume $f$ to have \repsmall/ fibers, we find that $s$ is \repsmall/. The composition of the individual pullback squares yields a pullback rectangle, which allows us to conclude that the composition $g \circ f$ has \repsmall/ fibers.

The above illustrates the argument for binary composition, by simply composing pullbacks. The argument for $n$-ary composition for any finite $n$ works inductively in the same way (note that the base $0$-ary case works in the same way as well; the pullback of an identity morphism is an identity morphism, and an identity morphism with small codomain has small domain).
\end{proof}

\begin{theorem}
Maps with \repsmall/ fibers are closed under pullback along arbitrary maps.
\end{theorem}
\begin{proof}
% https://q.uiver.app/?q=WzAsNixbMiwwLCJBIl0sWzIsMSwiQiJdLFsxLDAsIkQiXSxbMSwxLCJDIl0sWzAsMSwidCJdLFswLDAsInMiXSxbMCwxLCJmIl0sWzIsMF0sWzIsMywiZiciXSxbMywxXSxbMiwxLCIiLDIseyJzdHlsZSI6eyJuYW1lIjoiY29ybmVyIn19XSxbNCwzXSxbNSw0XSxbNSwyXSxbNSwzLCIiLDAseyJzdHlsZSI6eyJuYW1lIjoiY29ybmVyIn19XV0=
\[\begin{tikzcd}
	s & D & A \\
	t & C & B
	\arrow["f", from=1-3, to=2-3]
	\arrow[from=1-2, to=1-3]
	\arrow["{f'}", from=1-2, to=2-2]
	\arrow[from=2-2, to=2-3]
	\arrow["\lrcorner"{anchor=center, pos=0.125}, draw=none, from=1-2, to=2-3]
	\arrow[from=2-1, to=2-2]
	\arrow[from=1-1, to=2-1]
	\arrow[from=1-1, to=1-2]
	\arrow["\lrcorner"{anchor=center, pos=0.125}, draw=none, from=1-1, to=2-2]
\end{tikzcd}\]

Any pullback of $f'$ (along some arbitrary map) is a pullback of $f$ itself (along an extended map with the same domain). Thus, if $f$ has yields \repsmall/ objects whenever pulled back along a map with \repsmall/ domain, so does its pullback $f'$.
\end{proof}

\begin{theorem}\label{RepSmallRightAdjointFibers}
If $L : Q \to T$ is a functor with a right adjoint, on a category $Q$ with pullbacks, and $f$ is a map between $T$-indexed sets with $T$-\repsmall/ fibers, then $\pullAlong{L} f$ has $Q$-\repsmall/ fibers.
\end{theorem}
\begin{proof}
Let us say $f: A \to B$, and let an arbitrary map $m : q \to \pullAlong{L}(B)$ be given, where $q$ is an object of $Q$. We must show that the pullback of $\pullAlong{L} f$ along $m$ also lies within $Q$. For sake of a name, let us call the domain of this pullback $P$.

% https://q.uiver.app/?q=WzAsNCxbMSwwLCJcXHB1bGxBbG9uZ3tMfUEiXSxbMSwxLCJcXHB1bGxBbG9uZ3tMfUIiXSxbMCwxLCJxIl0sWzAsMCwiUCJdLFswLDEsIlxccHVsbEFsb25ne0x9ZiJdLFszLDJdLFsyLDEsIm0iLDJdLFszLDBdLFszLDEsIiIsMSx7InN0eWxlIjp7Im5hbWUiOiJjb3JuZXIifX1dXQ==
\[\begin{tikzcd}
	P & {\pullAlong{L}A} \\
	q & {\pullAlong{L}B}
	\arrow["{\pullAlong{L}f}", from=1-2, to=2-2]
	\arrow[from=1-1, to=2-1]
	\arrow["m"', from=2-1, to=2-2]
	\arrow[from=1-1, to=1-2]
	\arrow["\lrcorner"{anchor=center, pos=0.125}, draw=none, from=1-1, to=2-2]
\end{tikzcd}\]

First, observe via \magicref{YonedaPullalongLemma} that $m$ factors as $\pullAlong{L}(m') \circ \eta$ for some $m' : L(q) \to B$ and $\eta : q \to \pullAlong{L}L(q)$.

% https://q.uiver.app/?q=WzAsMyxbMiwwLCJcXHB1bGxBbG9uZ3tMfUIiXSxbMSwwLCJcXHB1bGxBbG9uZ3tMfUwocSkiXSxbMCwwLCJxIl0sWzEsMCwiXFxwdWxsQWxvbmd7TH0gbSciLDJdLFsyLDEsIlxcZXRhIiwyXSxbMiwwLCJtIiwxLHsib2Zmc2V0Ijo1LCJjdXJ2ZSI6Mn1dLFsxLDUsIiIsMSx7InNob3J0ZW4iOnsidGFyZ2V0IjoyMH0sInN0eWxlIjp7ImhlYWQiOnsibmFtZSI6Im5vbmUifX19XV0=
\[\begin{tikzcd}
	q & {\pullAlong{L}L(q)} & {\pullAlong{L}B}
	\arrow["{\pullAlong{L} m'}"', from=1-2, to=1-3]
	\arrow["\eta"', from=1-1, to=1-2]
	\arrow[""{name=0, anchor=center, inner sep=0}, "m"{description}, shift right=5, curve={height=12pt}, from=1-1, to=1-3]
	\arrow[shorten >=2pt, Rightarrow, no head, from=1-2, to=0]
\end{tikzcd}\]

Thus, the pullback yielding $P$ we are interested in can be decomposed as follows:

% https://q.uiver.app/?q=WzAsNixbMiwwLCJcXHB1bGxBbG9uZ3tMfUEiXSxbMiwxLCJcXHB1bGxBbG9uZ3tMfUIiXSxbMSwxLCJcXHB1bGxBbG9uZ3tMfUwocSkiXSxbMSwwLCJcXHB1bGxBbG9uZ3tMfShBIFxcdGltZXNfe0J9IEwocSkpIl0sWzAsMSwicSJdLFswLDAsIlAiXSxbMCwxLCJcXHB1bGxBbG9uZ3tMfWYiXSxbMiwxLCJcXHB1bGxBbG9uZ3tMfSBtJyIsMl0sWzMsMF0sWzQsMiwiXFxldGEiLDJdLFs1LDNdLFs1LDRdLFszLDJdLFs0LDEsIm0iLDEseyJvZmZzZXQiOjUsImN1cnZlIjoyfV0sWzIsMTMsIiIsMSx7InNob3J0ZW4iOnsidGFyZ2V0IjoyMH0sInN0eWxlIjp7ImhlYWQiOnsibmFtZSI6Im5vbmUifX19XSxbNSw5LCIiLDEseyJsZXZlbCI6MSwic3R5bGUiOnsibmFtZSI6ImNvcm5lciJ9fV0sWzMsNywiIiwxLHsibGV2ZWwiOjEsInN0eWxlIjp7Im5hbWUiOiJjb3JuZXIifX1dXQ==
\[\begin{tikzcd}
	P & {\pullAlong{L}(A \times_{B} L(q))} & {\pullAlong{L}A} \\
	q & {\pullAlong{L}L(q)} & {\pullAlong{L}B}
	\arrow["{\pullAlong{L}f}", from=1-3, to=2-3]
	\arrow[""{name=0, anchor=center, inner sep=0}, "{\pullAlong{L} m'}"', from=2-2, to=2-3]
	\arrow[from=1-2, to=1-3]
	\arrow[""{name=1, anchor=center, inner sep=0}, "\eta"', from=2-1, to=2-2]
	\arrow[from=1-1, to=1-2]
	\arrow[from=1-1, to=2-1]
	\arrow[from=1-2, to=2-2]
	\arrow[""{name=2, anchor=center, inner sep=0}, "m"{description}, shift right=5, curve={height=12pt}, from=2-1, to=2-3]
	\arrow[shorten >=2pt, Rightarrow, no head, from=2-2, to=2]
	\arrow["\lrcorner"{anchor=center, pos=0.125}, draw=none, from=1-1, to=1]
	\arrow["\lrcorner"{anchor=center, pos=0.125}, draw=none, from=1-2, to=0]
\end{tikzcd}\]

The right half of the above diagram is $\pullAlong{L}$ (known to preserve pullbacks by \magicref{PullalongIsLex}) applied to the following pullback diagram in $\Psh{T}$:

% https://q.uiver.app/?q=WzAsNCxbMSwwLCJBIl0sWzEsMSwiQiJdLFswLDEsIkwocSkiXSxbMCwwLCJBIFxcdGltZXNfe0J9IEwocSkiXSxbMCwxLCJmIl0sWzIsMSwibSciLDJdLFszLDJdLFszLDBdLFszLDUsIiIsMCx7ImxldmVsIjoxLCJzdHlsZSI6eyJuYW1lIjoiY29ybmVyIn19XV0=
\[\begin{tikzcd}
	{A \times_{B} L(q)} & A \\
	{L(q)} & B
	\arrow["f", from=1-2, to=2-2]
	\arrow[""{name=0, anchor=center, inner sep=0}, "{m'}"', from=2-1, to=2-2]
	\arrow[from=1-1, to=2-1]
	\arrow[from=1-1, to=1-2]
	\arrow["\lrcorner"{anchor=center, pos=0.125}, draw=none, from=1-1, to=0]
\end{tikzcd}\]

Note that, as $f$ has $T$-\repsmall/ fibers and $L(q)$ is an object of $T$ (i.e., $T$-\repsmall/), we find that $A \times_{B} L(q)$ is also $T$-\repsmall/.

By \magicref{RepsmallRightAdjoint}, it follows that $\pullAlong{L}(A \times_{B} L(q))$ is $Q$-\repsmall/, as is $\pullAlong{L}L(q)$.

Thus, the left half of our above diagram is a pullback of morphisms within $Q$:

% https://q.uiver.app/?q=WzAsNCxbMSwxLCJcXHB1bGxBbG9uZ3tMfUwocSkiXSxbMSwwLCJcXHB1bGxBbG9uZ3tMfShBIFxcdGltZXNfe0J9IEwocSkpIl0sWzAsMSwicSJdLFswLDAsIlAiXSxbMiwwLCJcXGV0YSIsMl0sWzMsMV0sWzMsMl0sWzEsMF0sWzMsNCwiIiwxLHsibGV2ZWwiOjEsInN0eWxlIjp7Im5hbWUiOiJjb3JuZXIifX1dXQ==
\[\begin{tikzcd}
	P & {\pullAlong{L}(A \times_{B} L(q))} \\
	q & {\pullAlong{L}L(q)}
	\arrow[""{name=0, anchor=center, inner sep=0}, "\eta"', from=2-1, to=2-2]
	\arrow[from=1-1, to=1-2]
	\arrow[from=1-1, to=2-1]
	\arrow[from=1-2, to=2-2]
	\arrow["\lrcorner"{anchor=center, pos=0.125}, draw=none, from=1-1, to=0]
\end{tikzcd}\]

As $Q$ is closed under pullbacks, it follows that $P$ is $Q$-\repsmall/, completing our proof.
\end{proof}

\begin{definition}\label{IndexedStructuresDefn}
We can talk about any kind of $T$-indexed structure or $T$-indexed maps between such structures, as the appropriate diagram of $T$-indexed sets and functions. For example, we can talk about $T$-indexed groups and group homomorphisms between them. When the $T$-indexed sets involved (the sorts within the structure, including the domains and codomains of all the maps defining the structure) are all \repsmall/, we say the entire structure is \defined{\repsmall/}, or equivalently, we say it is \defined{internal} to $T$\footnote{This \quote{$T$-internal gadgets} terminology makes most sense when $T$ is thought of as a kind of structure such that structure-preserving maps from $T$ to $S$ take $T$-internal gadgets to $S$-internal gadgets. Thus, if the definition of gadgets invokes maps whose domains are defined using finite limits, we will use this terminology of $T$-internal gadgets only in contexts where we are taking $T$ as a category with finite limits (for example, when speaking of internal categories). If the definiton of gadgets invokes maps whose domains are defined using finite products, we will use this terminology of $T$-internal gadgets only in contexts where we are taking $T$ as a category with finite products (for example, when speaking of internal groups). If the definition of gadgets invokes maps whose domains are defined using countably infinite products, then to speak of $T$-internal gadgets, $T$ must be carrying countably infinite product structure, etc.}. By the Yoneda lemma, this amounts to a diagram of objects and morphisms within $T$ itself.
\end{definition}

Observe that, as $\pullAlong{f}$ for an arbitrary functor $f : S \to T$ preserves finite limits (by \magicref{PullalongIsLex}), it not only takes $T$-indexed sets to $S$-indexed sets but also acts as a functor from $T$-indexed structures to $S$-indexed structures more generally, for any notion of structure definable using finite limits. For example, $\pullAlong{f}$ takes $T$-indexed groups to $S$-indexed groups, and so on. Furthermore, by \magicref{RepsmallRightAdjoint}, if $f$ has a right adjoint, then $\pullAlong{f}$ will take \repsmall/ structures to \repsmall/ structures.

\subsection{Indexed categories}
\begin{definition}
In the same vein as all this, by a $T$-\defined{indexed category}, we mean a category-valued presheaf on $T$; that is, a contravariant functor from $T$ to $\Cat$, and by an \defined{indexed functor} (or simply \defined{functor}) between $T$-indexed categories, we mean a natural transformation between such presheaves.\footnote{The machinery of indexed categories is equivalent to the machinery of fibered categories, a presentation some prefer, but we refrain from that presentation for now. Many of the features which make fibered categories most useful do not strongly apply to our ultimate interest largely in internal structures, while adding distracting complexity to the exposition. The current choice of presentation seemed the simpler one for our purposes, but the reader who disagrees may translate everything into the language of fibered categories if they prefer.}. (In keeping with our general convention, note that \quote{functor to $\Cat$} and \quote{natural transformation between functors to $\Cat$} here really refer to pseudofunctors and pseudonatural transformations, respectively, as $\Cat$ is a 2-category). We say this indexed category is an \defined{indexed lexcategory} (aka, \defined{has finite limits}) if this presheaf factors through the inclusion of $\LexCat$ into $\Cat$; that is, if it takes every object to a lexcategory and every morphism to a lexfunctor. We say an indexed functor between indexed lexcategories \defined{preserves finite limits} if it arises from a natural transformation between the corresponding $\LexCat$-valued presheaves. And in the same way as all this, we can speak of \defined{natural transformations} between functors between indexed categories, or any other familiar categorical structure or property.
\end{definition}

One might have thought our definition of $T$-indexed category-like structures would simply be a special case of our previous definition of $T$-indexed set-like structures as suitable diagrams within $\Psh{T}$ (that is, as suitable diagrams of $\Set$-valued functors). That is indeed the essence of this definition. However, the fact that we take indexed categories to be given by pseudofunctors into the 2-category $\Cat$, instead of treating $\Cat$ as a 1-category, provides a subtle but technically convenient generalization beyond directly demanding mere diagrams of $\Set$-valued functors.

Still, all the same notational conventions apply to indexed categories. E.g., given a $T$-indexed category $C$, we write $C(t)$ (or $C_t$) for the category which is the $t$-aspect of $C$ at an object $t$ of $T$, we write $C(m) : C(t) \to C(s)$ (or $C_m$) for the functor induced by a morphism $m: s \to t$ in $T$, we may write $\pullAlong{m}$ instead of $C(m)$ in contexts where it is clear that we are referring to the action of $C$, etc.

We now might like to speak about a category being \repsmall/, in the sense that its collection of objects and its collection of morphisms are both \repsmall/. This is the essence of the definition we will indeed adopt (at \magicref{RepsmallCategoryDefn}) but there is one pitfall to be aware of here, related to the just mentioned subtlety. We generally speak about categories in such a way as that they do not come with a particular notion of their set of objects, as such. That is, two categories may be equivalent (in the technical sense of \quote{equivalent} within the 2-category $\Cat$) though presented with different ostensible sets of objects. For example, a category presented as comprised of one terminal object, and a category presented as comprised of two isomorphic terminal objects, are equivalent categories; there is no pseudofunctor from the 2-category of categories, functors, and natural isomorphisms to $\Set$ which would send the first of these to a one-element set and the second to a two-element set. We are to treat them as the \quote{same} category. So to speak about a category as having a particular set of objects, we must imagine it as carrying more fine-grained equality structure on its objects than we normally do.

Though a category does not have a well-defined set of objects, it \emph{does} have a well-defined set of morphisms between any two given objects. Thus, there is no such difficulty in defining when an indexed category is locally small.
\begin{definition}\label{LocallyRepmallDefn}
Given a $T$-indexed category $C$, an object $t$ of $T$ and any two objects $a$ and $b$ in $C(t)$, we can define a $T$-indexed set whose aspect at objects $r$ of $T$ is the set $\{ (m, n) \mid m \in \Hom_T(r, t), n \in \Hom_{C(r)}(\pullAlong{m} a, \pullAlong{m} b) \}$, with the obvious corresponding action on morphisms of $T$. If the $T$-indexed set defined in this way is \repsmall/ for every object $t$ of $T$ and objects $a$ and $b$ in $C(t)$, then we say $C$ is \defined{locally \repsmall/}.
\end{definition}

Note that this is the same as saying that $\langle \cod, \dom \rangle : \Mor(C) \to \Ob(C) \times \Ob(C)$ has \repsmall/ fibers in the sense of \magicref{RepsmallFibersDefn}, except for that we do not need to think of $\Ob(C)$ as carrying an equality relation as such.

\subsection{Strict categories}
These bothers around the ill-defined set of objects of a general indexed category shall take us down some technical digressions for a bit, before we return to our big picture ideas. (Please keep in mind, the nuances of this section mostly do not matter for a big picture understanding. The main part of this document where such details might matter is in being rigorous in our chapter on geminal categories. We recommend that on a first read, the reader ignore all discussion of strictification or distinction between strict and non-strict concepts, in order to pick up the big picture ideas. The reader can then pay attention to these details on later more scrupulous re-reads as desired.)

\begin{definition}\label{StrictCategoryDefn}
Specifically, let us say a \defined{strict category} is a set of objects (including the ability to speak about equality of objects in a potentially finer-grained sense than isomorphism) and a set of morphisms, with the usual operations and satisfying the usual equations.\footnote{In certain contexts where quotient sets are not available, it is sometimes useful to allow strict categories to also come with a sort for 2-cell isomorphisms between their objects; that is, to take the collection of morphisms between any two objects in a strict category to comprise a setoid rather than a set. That is the more flexible and in some sense morally superior definition, but we will not need its extra complexity for now.} We may also speak of a \defined{strict functor}, meaning a homomorphism of such structure that preserves all of it on-the-nose. Strict categories and the strict functors between them comprise the 1-category $\StrictCat$. Similarly we can speak of natural transformations between strict functors, and so on for strict counterparts of any other categorical notion. Note that we can also straightforwardly talk about functors from strict categories to categories.
\end{definition}

Every strict category [or functor or etc], gives rise to a category [or functor or etc] in whatever ordinary sense one would like to think of these. We may say the strict category [or etc] presents the category [or etc] which results. Beware, non-isomorphic strict categories can both present the same (up to equivalence) category! That is, the relation \quote{strict structure $S$ presents non-strict-structure $C$} is invariant up to isomorphism of strict structures $S$ but invariant more broadly up to equivalence of non-strict structures $C$.

Just as every strict category presents a non-strict category, conversely, one would ordinarily say every category is presented by at least one strict category.\footnote{The situation is more nuanced for turning functors between arbitrary categories equivalent to given strict categories into strict functors between the given strict categories, but that will not concern us for now.} One might, if one likes, say that the only distinction between categories and strict categories is that we gather categories up into a 2-category and speak of categories up to equivalence in such, while we gather strict categories up into a 1-category and speak of strict categories up to isomorphism in such.

\begin{definition}
We now go further in defining a \defined{strict lexcategory}. Here, we mean more than just a strict category for which finite limits exist. We also mean that, when taking special \quote{basic limits}, the relevant limit is not merely defined up to isomorphism, but is given as a particular object (in keeping with the fact that objects can be distinguished more finely-grained than up to isomorphism, within a strict category). A \defined{strict lexfunctor} is accordingly one which preserves these chosen basic limits not merely up to isomorphism, but on-the-nose. Strict lexcategories and the strict lexfunctors between them comprise the 1-category $\StrictLexCat$. In the same way, we can also speak of a \defined{strict category with finite products}, or any similar such categorical structure.
\end{definition}

This business of \defined{basic limits} will require more explanation, another technical subtlety. What I mean by this is like so: Consider for example the concept of a category with a terminal object. And now consider the concept of a category with a pair of terminal objects, a terminal object A and a terminal object B. Ordinarily, we would like to say these are equivalent concepts or equivalent theories. They give rise to equivalent 2-categories (of categories with terminal objects, functors taking terminal objects to terminal objects, and natural transformations between these). However, the concept of a strict category with a single chosen terminal object, and the concept of a strict category with two chosen terminal objects A and B, are not equivalent concepts. We can ask questions in the one case that we cannot in the other; for example, in the latter case, we can distinguish between those models in which A and B are equal objects and those models in which A and B are not equal objects. This is reflected also in these giving rise to non-equivalent categories of models (of strict categories with the designated terminal objects, and functors preserving designated terminal objects on the nose). So when we go strictify the concept of a category with a terminal object, we really must make a choice as to how we choose to designate the terminal object; once or multiply.

This issue was illustrated above for terminal objects, but arises again, perhaps even more perniciously, for categories with finite products or finite limits or the like. Here, we find that the essentially algebraic theory of \quote{A strict category with a chosen terminal object and a binary operation sending any pair of objects to a chosen product} is not precisely the same as the essentially algebraic theory of \quote{A strict category with an $n$-ary operation on objects assigning chosen $n$-ary products, for each finite $n$}. Or the essentially algebraic theory of \quote{A strict category with a chosen terminal object and chosen (binary) pullbacks} is not precisely the same as the essentially algebraic theory of \quote{A strict category with a chosen terminal object, chosen binary products, and chosen (binary) equalizers}, particularly when we ask for homomorphisms between such structures which preserve their operations on-the-nose.

So in general, when we wish to talk about the appropriate notion of \quote{strict lexcategory} (or \quote{strict category with finite products} or \quote{strict cartesian closed category} or any such thing), we must make some decision as to how exactly to formalize this. We must make some choice of a basic stock of limit operations (or representing object operations more generally) of the desired sort, such that all the other desired limits (or representing objects) can be constructed from these basic operations. Different choices will yield slightly different strict concepts, albeit equivalent for all non-strict purposes.

None of the results in this work are ever particularly sensitive to what choice of basic such operations we take in strictifying a theory. We shall simply suppose some such choice has been made whenever needed, and refer to its operations as our basic limits. The one notable presumption we will make is that there are only finitely many basic limit operations involved in defining a strict lexcategory (or any such finitely axiomatizable thing); beyond that, any choice is fine. If the reader insists that we commit to a specific choice, let us for harmony with Palmgren and Vickers \TODOinline{Check this cite; they probably just use terminal object and pullback} say a strict lexcategory is defined by having a chosen terminal object and a chosen (binary) pullback operator.

\begin{definition}
Of course, we can speak of \defined{indexed strict categories} now (or indexed strict lexcategories, indexed strict categories with finite products, etc), straightforwardly via \magicref{IndexedStructuresDefn}, as the appropriate diagram of indexed sets and functions between them. And we can speak of such indexed strict categories as being \repsmall/, just in case their indexed sets of objects and of morphisms are both \repsmall/.
\end{definition}

\begin{definition}\label{RepsmallCategoryDefn}
We will now say an indexed category is \defined{\repsmall/} if it is equivalent to some indexed strict category which is \repsmall/. Note that we do not demand that, as part of its structure, any particular such strict category is selected; merely, that it is possible to do so. However, we may use the terminology \defined{internal category}, to mean the selection of a specific \repsmall/ indexed strict category; similarly, an \defined{internal lexcategory} will mean the selection of a specific \repsmall/ indexed strict lexcategory (including chosen basic limits), and so on.
\end{definition}

\begin{definition}\label{LocallyRepsmallStrictDefn}
We also say an indexed strict category is \defined{locally \repsmall/} if the map $\langle \dom, \cod \rangle$ from its set of morphisms to its set of pairs of objects has \repsmall/ fibers (in other words, though its set of objects may not be \repsmall/, everything that exists between any two particular objects is \repsmall/). 

We can repeat in this language the observation made at the end of \magicref{LocallyRepmallDefn}. Given an indexed category $C$ which is equivalent to some indexed strict category $C'$, we have that $C$ is locally \repsmall/ just in case $C'$ is locally \repsmall/. Note that, although an indexed category may be equivalent to non-isomorphic indexed strict categories, they will all agree on whether they are locally \repsmall/.

Note that a \repsmall/ strict category indexed over a category with finite limits is a fortiori locally \repsmall/, as expected, as the collection of morphisms between any particular pair of objects is given by an equalizer between sets already presumed \repsmall/ in a \repsmall/ strict category.
\end{definition}

We note without detailed proof (\TODOinline{Give or cite proofs}) some strictification results which will be useful to us later.

\begin{theorem}\label{StrictifyCategory}
Every category is equivalent to some strict category.
\end{theorem}
\begin{proof}
In traditional foundations, this is tautology, as categories are defined as presented by strict categories. In other foundations, any results of ours dependent on this should be interpreted or modified accordingly.
\end{proof}

\begin{theorem}\label{StrictifyLexcategory}
Every lexcategory is equivalent to some strict lexcategory.
\end{theorem}
\begin{proof}
One way to see this is by first noting that, for a given category $C$ with finite limits, we first of all have that $C$ is equivalent to some strict category $C'$ by \magicref{StrictifyCategory}. To further equip $C'$ as a strict lexcategory, we just need to make some choice of designated limits for each basic limit within $C'$. This is readily done using the Axiom of Choice.

In contexts without Choice, this theorem needs to be approached more carefully, with a more deliberate choice of $C'$ rather than an arbitrary one, but can still be carried out. In particular, we can create the free strict lexcategory $D$ extending $C'$ while preserving all basic limit cones (in the weaker, non-on-the-nose sense, as $C'$ does not itself have chosen limits). The map from $C'$ into this $D$ will be full, faithful, and essentially surjective on objects; thus, an equivalence of categories.
\end{proof}

\begin{theorem}\label{StrictifyLexfunctor}
Given a strict lexcategory $D$, a lexcategory $C$, and a lexfunctor $f : C \to D$, there is some strict lexcategory $C'$ and strict lexfunctor $f' : C' \to D$ such that $f$ and $f'$ are equivalent within the slice 2-category $\Cat/D$.
\end{theorem}

\begin{theorem}\label{StrictifyIndexedCategory}
Any indexed category is equivalent to some indexed strict category.
\end{theorem}

\begin{theorem}\label{StrictifyIndexedLexcategory}
Any indexed lexcategory is equivalent to some indexed strict lexcategory.
\end{theorem}

\begin{theorem}\label{StrictifyInternalCategoryToInternalLexcategory}
Any internal category which has finite limits (qua indexed category) can be further equipped as an internal lexcategory (without modifying the internal category structure).
\end{theorem}
\begin{proof}
Let the internal category $C$, internal to $T$, be given, and suppose its $t$-aspect has finite limits for each object $t$ of $T$. That is, the category whose objects are $\Hom(t, \Ob(t))$ and whose morphisms are $\Hom(t, \Mor(t))$, with suitable composition structure from the diagram internal to $T$ defining $C$, has finite limits.

Then in particular, for each basis finite limit shape, we can consider the case where $t$ is taken to be the set of diagrams of such shape within $C$ (for example, for binary products, we can consider $t = \Ob(C) \times \Ob(C)$, or for binary equalizers, we can consider $t$ taken to be the kernel pair (that is, pullback along itself) of $\langle \cod, \dom \rangle : \Mor(C) \to \Ob(C) \times \Ob(C)$). There will then be, within the $t$-aspect of $C$, a corresponding generic diagram of this shape, which will have some limit within $C$ as $C$ has finite limits. The selection of any particular such limit (that is, a particular value in $\Hom(t, \Ob(C))$ to serve as the apex of the limit cone, and particular further values in $\Hom(t, \Mor(C))$ to serve as the projection morphisms of the limit cone) gives us the morphisms within $T$ which serve as a limit-assigning operation on $C$ for this particular shape of basic limit. After making such a choice for each of the basic limit operations (of which we can presume there are only finitely many), we ultimately have equipped $C$ as an internal lexcategory.
\end{proof}
Note that it is NOT true that any indexed strict category which has finite limits (qua indexed category) can furthermore be equipped as an indexed strict lexcategory (without modification to the indexed strict category structure)! The former has reindexing functors which need only preserve finite limits in a non-strict-sense, while the latter's chosen basic limits must be such that all reindexing functors preserve basic limits on-the-nose. So it is rather remarkable that we get this for free once our indexed strict category is furthermore repsmall.

\subsection{Self-indexing}
\begin{definition}
Note that, from any lexcategory $T$ (or even just a category with pullbacks), we obtain a $T$-indexed lexcategory by considering the functor $T/-$ which assigns to each object $t$ of $T$ the slice category $T/t$, and whose action on morphisms is given by pullback. We refer to this as the \defined{self-indexing} of $T$.
\end{definition}

Note in the above that our flexibility in considering an indexed category as a pseudofunctor into $\Cat$, rather than a strict functor into $\StrictCat$, pays off in letting us not worry about how to choose specific pullback slices in a strictly functorial way. \TODOinline{That said, we will actually eventually use the construction under which this is made strictly functorial, at one point internally in our proof of Lob's theorem, so perhaps we should write out that construction in these Preliminaries as well.}

The self-indexing $T/-$ of a lexcategory $T$ is not in general \repsmall/, nor even locally \repsmall/. Given two globally defined objects $A$ and $B$ of the self-indexed category, their corresponding hom-set $\Hom_{T/-}(A, B)$ amounts to the presheaf $\Hom_{T}(A \times -, B)$ on $T$, which is to say, the exponential $B^A$ within $\Psh{T}$. This indexed set is \repsmall/ just in case an exponential object $B^A$ already exists within $T$. This extends in the same way to non-globally-defined objects of the self-indexed category (considered as globally defined over some slice category of $T$ instead, a la \magicref{AspectIsSliceGlobal}), and so the self-indexing of $T$ is locally \repsmall/ just in case $T$ is locally cartesian closed. Even if we do not have local cartesian closure in full, note that when $A = 1$, the exponential $B^A$ always is given by $B$ itself, so that hom-sets whose domain is $1$ are always \repsmall/ within the self-indexed category, with $\Hom_{T/-}(1, B)$ being the same as $B$ itself. In this way, the global sections presheaf upon the self-indexed category yields the canonical equivalence between the self-indexed category and the category of \repsmall/ sets. \TODOinline{Here, we are talking about a presheaf on an indexed category. Perhaps that should wait till after the section on doubly indexed sets}

\begin{definition}
In the same way, we can also speak of an \defined{indexed category with finite products}, and indeed, from any category with finite products $T$ (or even just a category with binary products), we obtain a $T$-indexed category with finite products by considering the functor $T//-$ which assigns to each object $t$ of $T$ the full subcategory of $T/t$ consisting of projection slices (slices given by the projection $: t \times s \to t$ for some object $s$ of $T$), and whose action on morphisms is again given by pullback (the pullback of a projection being another projection in a canonical way). We refer to this as the \defined{simple self-indexing} of $T$. Note that $T//t$ can also be thought of as the Kleisli category for the $t \times -$ comonad; that is, the objects of $T//t$ are the same as the objects of $T$, while a morphism $: s_1 \to s_2$ in $T//t$ is the same as a morphism $: t \times s_1 \to s_2$ in $T$, with suitable composition structure.

For a category with finite limits (or just pullbacks and binary products), the simple self-indexing can be thought of as a full subcategory of the self-indexing; specifically, the full subcategory whose objects in each aspect are restricted to those of $T$ itself.
\end{definition}

By analogous reasoning to before, the simple self-indexing $T//-$ of a category with finite products $T$ is locally \repsmall/ just in case $T$ is cartesian closed.

\begin{theoremEnd}[category=IntrospLemmas]{lemma}\label{Lemma1}
If $Y$ is a category with initial object $0$ and $X$ is a (2-)category, then to any functor $f : Y \to X$, we can associate a corresponding functor $f'$ from $Y$ to the slice category $f(0)/X$.

Furthermore, if $D$ and $C$ are parallel functors from $Y$ to $X$, then a natural transformation from $D$ to $C$ amounts to the same thing as a map $\introS$ from $D(0)$ to $C(0)$ along with a natural transformation from $D'$ to $\introS^{*} \circ C'$, where $\introS^{*} : C(0)/X \to D(0)/X$ is the functor between these slice categories given by composition with $\introS$.

(Dually, for contravariant functors $f : \op{Y} \to X$ (such as with indexed structures), acting on a category $Y$ with a terminal object $1$, we obtain a corresponding contravarint functor $f'$ from $Y$ to the co-slice category $X/f(1)$. And then the dual further result as well.)
\end{theoremEnd}
\begin{proof}
The first half of the lemma is thoroughly straightforward. $f : Y \to X$ induces a functor between the arrow category of $Y$ and the arrow category of $X$, and since $Y$ sits inside its arrow category as the slice category $0/Y$, this gives us a functor from $Y$ to the arrow category of $X$, which when followed by the domain projection back to $X$ is constantly $f(0)$. This is our $f'$.

The second half is also straightforward to mechanically verify when $X$ is a 1-category. This lemma should be understood as a triviality. But we will take some care to write out in detail an abstract demonstration that works just as well when $X$ is a 2-category (or indeed, when all categories involved are of whatever higher dimension), so that (in keeping with our linguistic convention) the functors involved are pseudofunctors, the natural transformations are pseudonatural transformations, etc, without having to get our hands dirty manually fussing about higher-dimensional coherence data.

See \hyperref[proof:prAtEnd\pratendcountercurrent]{full proof} on page~\pageref{proof:prAtEnd\pratendcountercurrent}.
\end{proof}
\begin{proofEnd}[no link to proof]
We pick up from where we left off previously in the proof of \cref{Lemma1}.

Throughout the remainder of this proof, all references to \quote{category}, \quote{functor}, etc, are in the sense of whatever dimension of higher-categories encapsulates both $Y$ and $X$.

Let $Z$ be the category obtained by augmenting $Y$ with a new object $0_Z$ and unique maps from $0_Z$ to each object of $Y$. We have an inclusion functor $i : Y \to Z$, and this inclusion is fully faithful, in the sense that the induced map $\Hom_Y(y_1, y_2) \to \Hom_Z(i(y_1), i(y_2))$ is an equivalence for all $y_1, y_2 \in \Ob(Y)$.

The unique maps from $0_Z$ to each object in the range of $i$ constitute a diagram of this form:

\[\begin{tikzcd}
	& 1 \\
	Y && Z
	\arrow["\unique", from=2-1, to=1-2]
	\arrow["{0_Z}", from=1-2, to=2-3]
	\arrow[""{name=0, anchor=center}, "i"', from=2-1, to=2-3]
	\arrow[Rightarrow, from=1-2, to=0]
\end{tikzcd}\]

What's more, because of how $Z$ was constructed by freely augmenting $Y$ with a new object and co-cone from it to the inclusion of $Y$, this diagram satisfies the universal property that for any other similar diagram
\[\begin{tikzcd}
	& 1 \\
	Y && Z'
	\arrow["\unique", from=2-1, to=1-2]
	\arrow[from=1-2, to=2-3]
	\arrow[""{name=0, anchor=center}, from=2-1, to=2-3]
	\arrow[Rightarrow, from=1-2, to=0]
\end{tikzcd}\]
there is a unique functor from $Z$ to $Z'$ commutatively relating the two diagrams. In jargon, this universal property is summarized by saying $Z$ (along with the data of $0_Z$ and $i$) is the co-comma of the unique functor from $Y$ to $1$ and the identity functor from $Y$ to $Y$.

Now, observe that $i$ has a left adjoint, the functor $q : Z \to Y$ such that $q \circ i$ is the identity on $Y$ and such that $q$ of the initiality co-cone for $0_Z$ in $Z$ is the initiality co-cone for $0$ in $Y$. That is, $q$ is the functor obtained by the co-comma property for $Z$ as applied to this diagram expressing the initiality co-cone of $0$ in $Y$:

\[\begin{tikzcd}
	& 1 \\
	Y && Y
	\arrow["\unique", from=2-1, to=1-2]
	\arrow["{0}", from=1-2, to=2-3]
	\arrow[""{name=0, anchor=center}, "\id"', from=2-1, to=2-3]
	\arrow[Rightarrow, from=1-2, to=0]
\end{tikzcd}\]

It is straightforward to verify that this $q$ is indeed left adjoint to $i$, as any data in $Z$ is either from the fully faithful inclusion of $Y$ or from the initiality co-cone for $0_Z$, and $\Hom_Y(q(i(y_1)), y_2) \iso \Hom_Y(y_1, y_2) \iso \Hom_Z(i(y_1), i(y_2))$ naturally in $y_1, y_2$ from $Y$, and $\Hom_Y(q(0_Z), y) = \Hom_Y(0, y) \iso 1 \iso \Hom_Z(0_Z, i(y))$ naturally in $y$ from $Y$.

Now consider any two parallel functors $D, C : Y \to X$. Because $q \circ i$ is the identity on $Y$, we have that $\Nat(D, C) \iso \Nat(D \circ q \circ i, C)$, where $\Nat$ denotes the space of natural transformations between these functors. But because $q \dashv i$, we in turn have that $\Nat(D \circ q \circ i, C) \iso \Nat(D \circ q, C \circ q)$.

Finally, let us consider what a natural transformation between $D \circ q$ and $C \circ q$ amounts to. This is the same as a functor from $Z$ to the arrow category of $X$ whose domain and codomain projections to $X$ yield $D \circ q$ and $C \circ q$. But by the co-comma property of $Z$, this functor out of $Z$ corresponds to data of the following form:

\[\begin{tikzcd}
	& 1 \\
	Y && {\arrowcat{X}}
	\arrow["\unique", from=2-1, to=1-2]
	\arrow[from=1-2, to=2-3]
	\arrow[""{name=0, anchor=center}, from=2-1, to=2-3]
	\arrow[Rightarrow, from=1-2, to=0]
\end{tikzcd}\]

such that the rightmost arrow of this diagram corresponds to some arrow $\introS$ in $X$ whose domain is $(D \circ q)(0_Z) = D(0)$ and whose codomain is $(C \circ q)(0_Z) = C(0)$, and such that the bottom arrow of this diagram corresponds to a natural transformation from $D \circ q \circ i \iso D$ to $C \circ q \circ i \iso C$. The 2-cell in the above diagram then corresponds to the remaining data necessary for us to construe this natural transformation from $D$ to $C$ as simply the codomain projection of a natural transformation between $D'$ and $\introS^{*} \circ C'$, the functors from $Y$ to $D(0)/X$ as mentioned in the statement of this lemma.

\TODOinline{Phew! That made a mountain out of a molehill. But perhaps people sometimes appreciate such written-out detail.}
\end{proofEnd}

In order to state the next lemma, some terminology:

\begin{definition}
If $T$ is a lexcategory, then for each object of $t$, we can construct the free lexcategory extending $T$ with a global element of $t$. Call this $T[1 \to t]$. Also, for any $f : s \to t$ in $T$, we can get a map from $T[1 \to t]$ to $T[1 \to s]$ by sending the generic global element of $t$ in $T[1 \to t]$ to the result of applying $f$ to the generic global element of $s$ in $T[1 \to s]$. This action is clearly functorial. Thus, $T[1 \to -]$ comprises a $T$-indexed object of $T/\LexCat$.

We can replace all references to finite limit structure above with finite product structure. In this case, let us use the name $T[[1 \to -]]$ for the resulting $T$-indexed object of $T/\FiniteProductCat$.
\end{definition}

By \magicref{Lemma1}, we can see $T/-$ as a contravariant functor from a lexcategory $T$ to $\LexCat/T$. And similarly for $T//-$ in terms of finite product structure.

\begin{theoremEnd}[category=IntrospLemmas]{lemma}\label{SelfIndexingIsFree}
$T[1 \to -]$ is equivalent to $T/-$, when the latter is viewed as a contravariant functor from a lexcategory $T$ to $\LexCat/T$ via \magicref{Lemma1}.

(And in just the same way, for a category with finite products $T$, we have that $T[[1 \to -]]$ is equivalent to $T//-$.)
\end{theoremEnd}
\begin{proof}
This is a standard observation (see 1.10.15 of Bart Jacobs' \quote{Categorical logic and type theory}, although this claims it without proof), and also simple enough to show. See \hyperref[proof:prAtEnd\pratendcountercurrent]{full proof} on page~\pageref{proof:prAtEnd\pratendcountercurrent}.
\end{proof}
\begin{proofEnd}[no link to proof]
To start off, we construe $T/-$ as not merely a $T$-indexed lexcategory, but furthermore a $T$-indexed lexcategory-under-$T$ (equivalently, $T/1$) by appeal to the first half of \cref{Lemma1} as applied to $T/- : \op{T} \to \LexCat$. Note that, for any particular object $t$ of $T$, this identifies $T/1$ inside $T/t$ via pullback along the unique map from $t$ to $1$; thus, $T$ is identified as the subcategory of \quote{constant} data within $T/t$. We may explicitly refer to this inclusion lexfunctor as $i_t : T \to T/t$. (Beware that in general, $i_t$ is faithful but not full!)

Now, what we want to show that given any fixed diagram in $\LexCat$ of this form
\[\begin{tikzcd}
	& T \\
	{T/t} && V
	\arrow["{i_t}"', from=1-2, to=2-1]
	\arrow["f", from=1-2, to=2-3]
\end{tikzcd}\]
there is a correspondence between commutative triangles extending this diagram with a lexfunctor from $T/t$ to $V$, and elements of $\Hom_V(1, f(t))$.

Well, suppose given $m : 1 \to f(t)$ in $V$. Consider now how the action of $f$ on arrows induces a functor $f' : T/t \to V/f(t)$. Because $f$ preserves finite limits, and finite limits in a category determine the finite limits in its slice categories, this functor $f'$ also preserves finite limits. Furthermore, by pullback along $m$, we get a lexfunctor $m^* : V/f(t) \to V$. Thus, altogether, we get a lexfunctor $m^* \circ f' : T/t \to V$. What's more, the following diagram commutes:

\[\begin{tikzcd}
	& T \\
	{T/t} && V \\
	& {V/f(t)}
	\arrow["{i_t}"', from=1-2, to=2-1]
	\arrow["f", from=1-2, to=2-3]
	\arrow["{f'}"', from=2-1, to=3-2]
	\arrow["{m^*}"', from=3-2, to=2-3]
\end{tikzcd}\]

To see that this diagram commutes, observe that it can be decomposed like so:

\[\begin{tikzcd}
	& T \\
	{T/t} & V & V \\
	& {V/f(t)}
	\arrow["{i_t}"', from=1-2, to=2-1]
	\arrow["f", from=1-2, to=2-3]
	\arrow["{f'}"', from=2-1, to=3-2]
	\arrow["{m^*}"', from=3-2, to=2-3]
	\arrow["f"', from=1-2, to=2-2]
	\arrow["{i_{f(t)}}"', from=2-2, to=3-2]
\end{tikzcd}\]

Here, $i_{f(t)}$ is the inclusion lexfunctor from $V$ into $V/f(t)$ in the analogous way to $i_t : T \to T/t$. The left half of this diagram commutes because, $f$ being a lexfunctor, the operations \quote{Pull back along the map from $t$ to $1$ and then apply $f$ to the resulting slice} and \quote{Apply $f$ and then pull back along the map from $f(t)$ to $1$} are the same. The right half of this diagram commutes because $m^* \circ i_{f(t)}$ is identity, because this amounts to pulling back along two morphisms in a row, ultimately pulling back along a path from $1$ to $1$, and any morphism from $1$ to $1$ is the identity. \TODOinline{Phrase this better?}

Thus, every element of $\Hom_V(1, f(t))$ induces a corresponding commutative triangle of lexfunctors from $i_t : T \to T/t$ to $f : T \to V$.

Conversely, any such commutative triangle induces an element of $\Hom_V(1, f(t))$. To see this, we just need to see that there is some global element of $i_t(t)$ already in $T/t$.

Note that $i_t(t)$ is one of the projections from $t \times t$ to $t$. Furthermore, the terminal object in $T/t$ is the identity slice from $t$ to $t$. So the following commutative triangle serves as a global element of $i_t(t)$ within $T/t$:

\[\begin{tikzcd}
	t && {t \times t} \\
	& t
	\arrow["id"', from=1-1, to=2-2]
	\arrow["{i_t(t)}", from=1-3, to=2-2]
	\arrow["{\langle \id, \id \rangle}", from=1-1, to=1-3]
\end{tikzcd}\]

Let us use the name $g_t$ for this global element of $i_t(t)$ within $T/t$.

Now, we need to show that these two processes are inverse. \TODO

Finally, we must show that the re-indexings given by morphisms in $T$ via the definition of $T[1 \to -]$ correspond to the same reindexings as defined via $T/-$. \TODO

\TODOinline{The details of this are probably in Uemura, and we can just cite that on top of Bart Jacobs and be done}
\end{proofEnd}

\begin{corollary}\label{SelfIndexingIsFreeCorollary}
The analogues of \magicref{SelfIndexingIsFree} automatically also follow for any categorical structure extending the structure of a lexcategory which is automatically transferred to slice categories and preserved by pullback; that is, any structure which automatically transfers from an instance that structure also to its self-indexing (e.g., for the concepts of locally cartesian closed categories or for elementary toposes or for regular categories). And in just the same way also for any categorical structure extending the structure of a category with finite limits, which automatically transfers from an instance of that structure to its simple self-indexing (e.g., for cartesian closed categories).
\end{corollary}

\subsection{Double or multiple indexing}
\TODOinline{Reader beware, this section is only needed for understanding some of the discussion at the beginning of the chapter on Geminal Categories. Apart from that, this section does not come up significantly. TODO: Emphasize the notion of multiply internal structures, the main thing this section is used to discuss.}

At this point, for any algebraic-categorical notion $S$ (e.g., the notion of a commutative ring, or the notion of a lexcategory), we also have a definition of the notion of a pair of a category and an instance of notion $S$ indexed over that category.

We can thus iterate this process. In particular, we can speak of a $T$-indexed (category $C$ and $C$-indexed set $P$). We can call this also a $(T, C)$-indexed set $P$. Let us observe in more detail what this amounts to.

What this means is that, in addition to having a category $T$ and a $T$-indexed category $C$, we also have for every object $t$ in $T$, some corresponding $C(t)$-indexed set. Thus, we obtain for each $t$-indexed object $c$ of $C$ a corresponding set we may denote $P(t)(c)$ or $P(t, c)$ or $P_t(c)$ (the $t$-defined $c$-defined elements of $P$). And for each morphism $n : c \to d$ in $C(t)$, we have a reindexing function $P(t, n) : P(t, d) \to P(t, c)$. These reindexings act functorially in that $P(t, n_1 \circ \ldots \circ n_k) = P(t, n_k) \circ \ldots \circ P(t, n_1)$ for any sequence of composable morphisms $n_1, \ldots, n_k$ in $C(t)$.

But furthermore, we must have functorial reindexing maps for $P$ along morphisms of $T$. This means, for any map $m : s \to t$ in $T$, we must have for every $t$-defined object $c$ of $C$ a reindexing function $P(m, c) : P(t, c) \to P(s, C(m)(c))$.  We may just write $P(m)$ to refer generically to any $P(m, c)$. These reindexings act functorially in that $P(m_1 \circ \ldots \circ m_k) = P(m_k) \circ \ldots \circ P(m_1)$ for any sequence of composable morphisms $m_1, \ldots, m_k$ in $T$.

Finally, the reindexings along morphisms of $T$ must preserve in a suitable sense the reindexings along morphisms of $C$. This means the following square of reindexings commutes, for any morphisms $m : s \to t$ in $T$ and $n : c \to d$ in $C(t)$:

% https://q.uiver.app/?q=WzAsNCxbMCwwLCJQKHQsIGQpIl0sWzAsMSwiUCh0LCBjKSJdLFsxLDAsIlAocywgQyhtKShkKSkiXSxbMSwxLCJQKHMsIEMobSkoYykpIl0sWzAsMSwiUCh0LCBuKSIsMl0sWzAsMiwiUChtKSJdLFsxLDMsIlAobSkiLDJdLFsyLDMsIlAocywgQyhtKShuKSkiXV0=
\[\begin{tikzcd}
	{P(t, d)} & {P(s, C(m)(d))} \\
	{P(t, c)} & {P(s, C(m)(c))}
	\arrow["{P(t, n)}"', from=1-1, to=2-1]
	\arrow["{P(m)}", from=1-1, to=1-2]
	\arrow["{P(m)}"', from=2-1, to=2-2]
	\arrow["{P(s, C(m)(n))}", from=1-2, to=2-2]
\end{tikzcd}\]

Using this coherence condition, any reindexing in $C$ followed by a reindexing in $T$ (the left-bottom path) can be turned into an equivalent reindexing in $T$ followed by a reindexing in $C$ (the top-right path). Thus, for any string of reindexings (alternating between reindexings in $C$ and reindexings in $T$), there is a unique reindexing in $T$ followed by a reindexing in $C$ which it is forced equivalent to by the coherence condition and functoriality.

Thus, we can resummarize all of these conditions like so: We create a category denoted $\Groth_{T} C$ (or just $\Groth C$) whose objects are pairs $(t, c)$ where $t$ is an object in $T$ and $c$ is an object in $C(t)$. A morphism in $\Groth C$ from $(s, c)$ to $(t, d)$ is given by a pair $(m, n)$ where $m : s \to t$ in $T$ and $n : c \to C(m)(d)$ in $C(s)$. This represents a reindexing along $m$ followed by a reindexing along $n$, and so by consideration of the previous paragraph, we get also the appropriate composition rule validating our desired coherence condition and automatically ensuring associativity. Specifically, the appropriate composition rule is that $(a, n)$ followed by $(m, b)$ composes to $((a ; m), (C(m)(n) ; b))$, as can be visualized from our above-noted coherence condition like so:

% https://q.uiver.app/?q=WzAsNixbMiwwLCJQKHQsIGQpIl0sWzIsMSwiUCh0LCBjKSJdLFs0LDAsIlAocywgQyhtKShkKSkiXSxbNCwxLCJQKHMsIEMobSkoYykpIl0sWzAsMCwiXFxidWxsZXQiXSxbNCwyLCJcXGJ1bGxldCJdLFswLDEsIlAodCwgbikiLDFdLFswLDIsIlAobSkiXSxbMSwzLCJQKG0pIl0sWzIsMywiUChzLCBDKG0pKG4pKSIsMV0sWzQsMCwiUChhKSJdLFszLDUsIlAoYikiXSxbNCwxXSxbMSw1XV0=
\[\begin{tikzcd}
	\bullet && {P(t, d)} && {P(s, C(m)(d))} \\
	&& {P(t, c)} && {P(s, C(m)(c))} \\
	&&&& \bullet
	\arrow["{P(t, n)}"{description}, from=1-3, to=2-3]
	\arrow["{P(m)}", from=1-3, to=1-5]
	\arrow["{P(m)}", from=2-3, to=2-5]
	\arrow["{P(s, C(m)(n))}"{description}, from=1-5, to=2-5]
	\arrow["{P(a)}", from=1-1, to=1-3]
	\arrow["{P(b)}", from=2-5, to=3-5]
	\arrow[from=1-1, to=2-3]
	\arrow[from=2-3, to=3-5]
\end{tikzcd}\]

Then, a $(T, C)$-indexed set is just the same as as a $(\Groth_T C)$-indexed set. This also gives us easily the right notion of maps between $(T, C)$-indexed sets. They are just maps between the corresponding $(\Groth_T C)$-indexed sets (i.e., natural transformations between presheaves on $\Groth_T C$). In this way, we can speak about $(\Groth_T C)$-indexed structures more generally than mere sets.

We will only rarely need to consider any of this multi-indexing, and to the extent we do, almost always will only consider $(T, C)$-indexed sets $P$ in cases where $C$ is in fact $T$-small, and furthermore $P$ is $T$-small. \TODOinline{Define what it means for $P$ to be $T$-small. Show how this leads to a simpler representation of $P$ as living internally to $T$}

The construction $\Groth_T C$ is called the Grothendieck construction. By projecting out first coordinates, we get a functor from $\Groth_T C$ to $T$; functors which arise in this way are called Grothendieck fibrations, or just fibrations. It turns out, given a fibration just as an abstract functor between categories, one can recover the indexed category which gave rise to it.

Thus, the data of an indexed category is equivalent to the data of a fibration. The entire machinery of indexed categories can therefore equivalently be presented in terms of fibrations. For this reason, fibrations are also called fibered categories. In particular, one can give a more intrinsic account of the conditions under which an arbitrary functor is a fibration. Furthermore, any natural transformation between $T$-indexed categories induces a corresponding map between the corresponding fibrations in $\Cat/T$, and again the natural transformation can be recovered from this map, and again a more intrinsic account can be given of which maps arise in this way. Some things are easier to describe in a fibration-based presentation. Other things are more difficult. For our purposes (using the general language of indexed structures ultimately for the goal of understanding specifically \repsmall/ or internal structures), we felt the indexed category presentation was the most apt. Thus, we will not describe the theory of fibered categories further. We use the Grothendieck construction only for the correspondence between $(\Groth_T C)$-indexed sets and $(T, C)$-indexed sets.

Of course, this construction can be iterated further now. A $(T, C)$-indexed category $D$ is a $(\Groth_T C)$-indexed category (i.e., a contravariant functor from $\Groth_T C$ to $\Cat$), and thus gives rise to another category $\Groth_{\Groth_T C} D$. Structures indexed over $\Groth_{\Groth_T C} D$ can be called $(T, C, D)$-indexed structures. And so on ad infinitum. Do not worry, we will not need to explicitly consider this to any further depth of indexing.



\TODOinline{Discuss the concept of being $T$-\repsmall/ vs. $C$-\repsmall/ when $(T, C)$-indexed}

Note also that any structure singly-indexed over $T$ can automatically be thought of as doubly-indexed over $T$ and $C$, where the indexing over $C$ is trivial. This is basically by the fact that the Grothendieck construction for $T$ and $C$ comes with a projection functor to $T$, so that all $T$-indexed structures thus induce, via this functor, a $T$-and-$C$-indexed structure. Thus, we can readily speak of maps between $T$-indexed structures and $(T, C)$-indexed structures, by treating the former as implicitly $(T, C)$-indexed themselves.

Indeed, more generally in the multiply indexed context, any structure indexed over some prefix of a string of categories is automatically indexed over the full string of categories. And in the same way, this allows us to speak of maps between structures indexed by different strings of categories. This is the main reason for us to bring all this up, just so that we can speak of maps between structures at different levels of indexing.

(Keep in mind also that an honest-to-goodness actual structure, living in $\Set$, is like the zero-ary case of indexing; it's indexed by the empty string of categories $()$, but can be seen in a trivial way as $T$-indexed for any category $T$).

Note that a map from a $T$-indexed structure $A$ to a $(T, C)$-indexed structure $B$ thus amounts to a map from $A$ to $\Hom_C(1, B)$, whenever $C$ has a terminal object. So all this high-faluting multiply indexed stuff just amounts to another way of thinking about maps into global aspects.

\subsection{Unorganized draft category-theoretic lemmas}

\TODOinline{Everything from here on out in the preliminaries is in a draft state, not yet readable}

\subsubsection{Laxly indexed strict categories}
\begin{definition}\label{LaxlyIndexedStrictCategory}
Given a strict category $C$, we define the notion of a \defined{laxly indexed strict category} on it. This is the concept of a lax functor from $\op{C}$ to the category of strict categories. In detail it is, like so: A laxly $C$-indexed strict category $L$ consists of the following data: A strict category $L(c)$ for each object $c$ of $C$. Furthermore, a strict functor $L(m) : L(d) \to L(c)$ for each morphism $m : c \to d$ in $C$. Furthermore, a natural transformation from $\id_{L(c)}$ to $L(\id_c)$ for each object $c$ in $C$. Furthermore, a natural transformation from $L(n) \circ L(m)$ to $L(m \circ n)$ for each composable pair of morphisms $n : b \to c$ and $m: c \to d$ in $C$. Furthermore, there are some coherence conditions to impose on these natural transformations, but they do not matter for our purposes so we do not bother stating them. These are the conditions which would ensure that, if these natural transformations were in fact invertible, this comprised a pseudofunctor.
\end{definition}

In fact, absolutely none of the conditions here which concern equalities between elements of any $\Mor(L(c))$ for any object $c$ in $C$ matter to us. It doesn't matter for our purposes that composition in any category $L(c)$ of a laxly indexed strict category be associative or satisfy the identity law or any of that. All that matters to us is that certain operations form elements of $\Mor(L(c))$ with the appropriate domains and codomains.

We only need this notion for one example in one situation. The one example we use is the following construction:

\begin{construction}\label{CoreOfSelfIndexing}
Given any strict lexcategory $C$, there is a straightforward way to create a laxly indexed strict category indexed by $C$ where $L(c)$ is the core of the slice category $C/c$ (that is, the subcategory of $C/c$ containing all its objects, but only those of its morphisms which are isomorphisms), while $L(m)$ is given by pullback along $m$ (the chosen pullbacks of our strict lexcategory $C$). Call this the \defined{core of the self-indexing} of $C$. The universal property of pullbacks ensures that this is pseudofunctorially indexed, and thus a fortiori laxly indexed. However, as chosen pullbacks satisfy no further demanded coherence beyond just being pullbacks, there is no reason to expect the natural transformations witnessing pseudofunctoriality here to in fact be strict identities (that is, chosen pullback along a composition of morphisms needn't strictly match the composition of chosen pullbacks along the individual morphisms).

Note that, when $C$ is internal to lexcategory $T$, this construction can be carried out internal to $T$ as well.
\end{construction}

There are methods by which one can work to strictify a pseudofunctor in such situations, but we do not need these methods for the one instance in which we deal with this. Just the bare minimum structure of a laxly indexed strict category, sans even the coherence conditions on its indexed morphisms, suffices for us.

A laxly $C$-indexed discrete strict category is the same as a $C$-indexed set, and thus we also get degenerate examples of laxly $C$-indexed strict categories from any $C$-indexed set.

\subsubsection{Comma categories}

\begin{theorem}[The Comma-Kan Lemma] \label{CommaKan}
Suppose, within some 2-category $C$, we have the following comma object and left Kan extensions:

% https://q.uiver.app/?q=WzAsNyxbNCwyLCJYIl0sWzQsNCwiXFxvbWVnYSJdLFsyLDQsIlkiXSxbMiwyLCIoZl9YL2ZfWSkiXSxbMSwxLCJBIl0sWzAsMCwiQiJdLFs1LDFdLFswLDEsImZfWCJdLFsyLDEsImZfWSIsMl0sWzMsMiwiXFxwaV9ZIiwyXSxbMywwLCJcXHBpX1giXSxbMCwyLCJcXGdhbW1hIiwyLHsibGV2ZWwiOjJ9XSxbNCwzLCJzIiwxXSxbNCw1LCJxIiwxXSxbNSwwLCJcXExhbl9xIChzOyBcXHBpX1gpIl0sWzUsMiwiXFxMYW5fcShzOyBcXHBpX1kpIiwyXSxbMywxNCwiXFxlcHNpbG9uX1giLDEseyJzaG9ydGVuIjp7InNvdXJjZSI6MjB9fV0sWzMsMTUsIlxcZXBzaWxvbl9ZIiwxLHsic2hvcnRlbiI6eyJzb3VyY2UiOjIwfX1dXQ==
\[\begin{tikzcd}
	B \\
	& A &&&& {} \\
	&& {(f_X/f_Y)} && X \\
	\\
	&& Y && \omega
	\arrow["{f_X}", from=3-5, to=5-5]
	\arrow["{f_Y}"', from=5-3, to=5-5]
	\arrow["{\pi_Y}"', from=3-3, to=5-3]
	\arrow["{\pi_X}", from=3-3, to=3-5]
	\arrow["\gamma"', Rightarrow, from=3-5, to=5-3]
	\arrow["s"{description}, from=2-2, to=3-3]
	\arrow["q"{description}, from=2-2, to=1-1]
	\arrow[""{name=0, anchor=center, inner sep=0}, "{\Lan_q (s; \pi_X)}", from=1-1, to=3-5]
	\arrow[""{name=1, anchor=center, inner sep=0}, "{\Lan_q(s; \pi_Y)}"', from=1-1, to=5-3]
	\arrow["{\epsilon_X}"{description}, shorten <=3pt, Rightarrow, from=3-3, to=0]
	\arrow["{\epsilon_Y}"{description}, shorten <=4pt, Rightarrow, from=3-3, to=1]
\end{tikzcd}\]

Furthermore, suppose $f_X$ preserves the Kan extension $\Lan_q (s; \pi_X)$. (We notably do NOT make any such assumption on $f_Y$).

Then $\Lan_q s : B \to (f_X / f_Y)$ exists and furthermore is preserved by both $\pi_X$ and $\pi_Y$.
\end{theorem}
\begin{proof}
We may compute as follows: By the universal property of the comma object $(f_X / f_Y)$, the set of 1-cells from $B$ to this comma object whose projections match our two Kan extensions is given by the set of 2-cells between the top and bottom path from $B$ to $\omega$ in the above diagram. Since $f_X$ preserves the top Kan extension, the top path is itself a left Kan extension, and using its universal property, we find that the 2-cells from the top path to the bottom path are the same as 2-cells between two different paths from $A$ to $\omega$; specifically, from $s; \pi_X; f_X$ to $q; \Lan_q(s; \pi_Y); f_Y$. Such a 2-cell is given by the composition of $\gamma$ and $\epsilon_Y$.

Thus, we do indeed get a 1-cell $m : B \to (f_X / f_Y)$ whose composition with each projection $\pi$ matches $\Lan_q(s; \pi)$. What remains is only to show that $m$ is indeed $\Lan_q s$.

Let an arbitrary $k : B \to (f_X/f_Y)$ be given. By the universal property of the comma category again, we have that 2-cells from $m$ to $k$ are in correspondence with choices of 2-cells from $m ; \pi$ to $k ; \pi$ for each projection $\pi$, such that both resulting composite 2-cells from $m; \pi_X; f_X$ to $k ; \pi_y; f_Y$ are equal. But each $m ; \pi = \Lan_q(s; \pi)$, so a 2-cell from this to $k ; \pi$ amounts to a 2-cell from $s; \pi$ to $q; k ; \pi$. A choice of such 2-cells satisfying the coherence condition is, again by the universal property of the comma category $(f_X / f_Y)$, the same thing as a 2-cell from $s$ to $q; k$. Thus, we have shown $\Hom(m, k) = \Hom(s, (q; k))$, which establishes $m$ as satisfying the universal property defining $\Lan_q s$. This completes the proof.
\end{proof}

Dually, we can turn all 2-cells around in this theorem, replacing the left Kan extensions with right Kan extensions and now having a condition that the functor on the codomain side of our comma category must preserve the corresponding Kan extension.

\begin{corollary}\label{CommaCategoryColimits}
Given a cospan of functors $f_X, f_Y$ from respective categories $X$ and $Y$, if $X$ and $Y$ both have colimits of a particular shape and $f_X$ preserves colimits of that shape, then $(f_X / f_Y)$ has colimits of that shape, preserved by both projections $\pi_X$ and $\pi_Y$.

Furthermore, an arbitrary functor into $(f_X / f_Y)$ then preserves colimits of that shape iff its composition with both projections preserves such colimits.

(Dually, we have the corresponding statements where all instances of \quote{colimit} are turned into \quote{limit} and the first statement's limit preservation condition is put on $f_Y$ rather than $f_X$.)
\end{corollary}
\begin{proof}
The first statement follows from \magicref{CommaKan} by taking $A$ to be the generic category of the indicated shape and taking $B$ to be the terminal category, considering how colimits correspond to left Kan extensions along such a functor.

The second statement then follows from the fact that in addition to the the forgetful functor from $(f_X / f_Y)$ into $X \times Y$ preserving colimits as just shown, such a forgetful functor from a comma category to a product category is always conservative (that is, if the image of a morphism under this functor is invertible, the morphism was already invertible in the comma category). A conservative functor which preserves colimits automatically has the desired property.
\end{proof}

\begin{corollary}\label{LexCatComma}
Comma objects exist in $\LexCat$, constructed in the same way as in $\Cat$ (thus, preserved by the forgetful functor into $\Cat$).
\end{corollary}
\begin{proof}
From \magicref{CommaCategoryColimits}, we see that when $f_X, f_Y$ are a co-span of finite limit preserving functors between categories which have finite limits, then the comma category $(f_X / f_Y)$ (the comma object in $\Cat$) is also a lexcategory and its projections are lexfunctors. Thus, this $(f_X / f_Y)$ and its projections exist within $\LexCat$. That these continue to comprise a comma object span within $\LexCat$ follows immediately from the fact that the forgetful functor $|-|$ from $\LexCat$ to $\Cat$ induces bijections between the sets of 2-cells $\Hom(f, g)$ and $\Hom(|f|, |g|)$ for any parallel 1-cells $f$ and $g$ in $\LexCat$ (that is, the 2-cells in $\LexCat$ between lexfunctors are just ordinary natural transformations, with no further property or structure).
\end{proof}

The conclusion of \magicref{CommaKan} will often be particularly useful to us in conjunction with the following lemma.

\begin{lemma}\label{CommaMap}
Let $C$ and $D$ be 2-categories, with $|-| : C \to D$ a 2-functor, and suppose for all objects $X$, $Y$, and $\omega$ of $C$ along with 1-cell $f_X : X \to \Omega$ in $C$ and 1-cell $f_Y : |Y| to |\omega|$ in $D$, we have a span in $C$ whose image under $|-|$ is the comma object $(|f_X| / f_Y)$ in $D$ along with its projections.

Furthermore, suppose $X$ is initial in $C$. Then the identity map on $|X|$ is initial within $D(X, X)$.
\end{lemma}
\begin{proof}
Take $Y = \omega = X$, let $f_X$ be the identity on $X$, and let $f_Y : |X| \to |X|$ in $D$ be arbitrary. \TODO
\end{proof}

For turning the initial AU into our PA sigma 1 models, etc, after we've established that AUs are closed under comma category constructions, let X = Y = omega =the initial AU in which some AU definable morphisms are invertible, let f_X = identity, and f_y = arbitrary lex. It follows that the comma category is also an AU, and because everything is component wise, the projections send the relevant AU-definable morphisms to invertible morphisms. Since projections from a comma object are jointly conservative, the comma category is also an AU where the AU definable morphisms of note are invertible. Thus, X maps into the comma category in a unique AU way; the projections following this mapping are identity, and we get a natural transformation from identity on X to f_Y.

\subsubsection{Theories, models, functors, etc}
\begin{TODOblock}\label{ModelTerminology}
We need to standardize our terminology on theories, models, and internal models. Is a lexfunctor from $T$ to $S$ an internal model of $T$ in $S$? Or is a lexfunctor from $T$ the globalization of an internal category in $S$ an internal model of $T$ in $S$? Is a model always Set-valued? Etc. Let us do a Ctrl+F for "model" to make sure we are not confusing on this point. It may be best to speak scrupulously of lexcategories and lexfunctors at all times, and not of theories or models, except in the fixed phrase "introspective theory".

We can use interior vs. included.
\end{TODOblock}

When we say "lex theory", this is a synonym for lexcategory (emphasizing the correspondence between lexcategories and essentially algebraic theories).

Given a lex theory $T$ thought of as \quote{the theory of gadgets}, we can identify any particular \quote{gadget} $G$ with a corresponding lexfunctor $g : T \to \Set$, or vice versa. Even though these are equivalent data, it may be linguistically natural to avoid identifying them. (For example, if $T$ is the free lexcategory with an internal category, then a functor $: T \to Set$ corresponds to a strict category, yet it could be confusing to refer to it simultaneously as a functor and as a category.). The linguistic device we use for disambiguating this when necessary is by calling the gadget $G$ the \defined{down-reification} of the functor $g : T \to \Set$, and conversely, calling the functor $g : T \to \Set$ the \defined{up-reification} of the gadget.

In the same way, we may say of a diagram $G$ (of a particular shape involving some particular finite limits) within a lexcategory $S$ as corresponding to a lexfunctor $g$ into $S$ from the free lexcategory containing a diagram of such a shape. Again, these are equivalent data but we may wish to linguistically differentiate between, say, \quote{a pullback square $G$ in $S$} (consisting of four objects and four morphisms) and \quote{a functor $g$ into $S$ from the countably infinite category which is the free lexcategory on a pullback square}. Again, in these scenarios, we say the lexfunctor is the up-reification of the diagram, or the diagram is the down-reification of the lexfunctor.

Given a category $T$ and a $T$-indexed (or $T$-internal) category $C$, we may speak of a diagram internal to $C$, as shorthand meaning a diagram internal to the global aspect of $C$. Similarly, in such a situation, we may speak of a functor into $C$ from another (non-indexed) category $S$, as shorthand meaning a functor from $S$ to the global aspect of $C$. We can make sense of this shorthand internally too: If $C$ is a $T$-indexed category, then by a category $C'$ internal to $C$, we mean that $C'$ is internal to the global aspect of $C$. Given another $T$-indexed category $D$, by a $T$-indexed functor from $D$ to $C'$, we mean a $T$-indexed map that assigns to each object of $D$ an object in the $T$-indexed category corresponding to the aspect of $C'$ at the terminal object of $C$. \TODOinline{Yeah, this belongs in the multi-indexing section. It's getting to be a mess. Let's try not to use this kind of shorthand, instead.}

\TODOinline{Introduce terminology for "the walking X" meaning "the free lexcategory with an internal X". "Internal X" always means "Map into me from the lex theory of Xes", and NOT "Internal lexcategory in me, and map into THAT from the lex theory of Xes". The latter should be called doubly internal. The theory of Xes has an internal X. Hm. Maybe we need good terminology for "the theory of blahs" vs "a blah", when blah is itself a lexcategory-extending notion.}

\subsubsection{Interpreting essentially algebraic theories into strict lexcategories}
\TODOinline{Clean this section up, motivate it, use it}

Consider the free/forgetful adjunction between strict categories and strict lexcategories. For any strict lexcategory, we can consider the unit of this adjunction at it. If this unit is injective on objects, we say this strict lexcategory \defined{imposes no equations on objects}. When we say \defined{strict lex theory}, we mean a strict lexcategory that imposes no equations on objects. That is, it does not have an object $X$ such that $X \times X = X$ strictly, or any such thing. These are the appropriate strict lexcategories to model essentially algebraic theory for interpretation into other strict lexcategories, in the following sense: 

If $T_{strict}$ is a strict lexcategory which imposes no equations on objects, and $S_{strict}$ is a strict lexcategory, and these present non-strict lexcategories $T$ and $S$, then the strict category $\StrictLexCat(T_{strict}, S_{strict})$ presents the category $\LexCat(T, S)$.
\begin{proof}
???? \TODO

There is a clear functor from the category presented by $\StrictLexCat(T_{strict}, S_{strict})$ to $\LexCat(T, S)$. We must show that this is essentially surjective on objects, full, and faithful.

So, for essential surjectivity on objects: Suppose given a lexfunctor from T to S. Perhaps this guarantees us an equivalent strict functor from T_{strict} to S_{strict}?

Then perhaps we can replace a strict functor which weakly preserves limits by an equivalent strict lexfunctor whenever the domain doesn't impose equations on objects? We take the strict functor from T_{strict} to S_{strict}, and turn it into a strict lexfunctor from Free(T_{Strict}) to S_{strict}. Ok, what we want is a strict lexfunctor from T_{Strict} to Free(T_{strict}).

If that's so, then the next step is fullness and faithfulness. Given two strict lexfunctors, suppose given an arbitrary natural transformation between them in LexCat. This is some selection of morphisms of S satisfying some conditions. This is unaffected by concerns about object duplication or basic limit preservation on the nose. So the conditions defining such a natural transformation in terms of selections of morphisms of S_{strict} are the same. Thus, fullness and faithfulness is guaranteed.
\end{proof}

Let's just make this true by definition instead. Let's say a strict lexcategory imposes no equations on objects if any strict functor out of it to another strict lexcategory which preserves limits weakly can be replaced with an equivalent functor which preserves limits on the nose.

Every strict lexcategory $T$ presents a category which can also be presented by a strict lexcategory which imposes no equations on objects.
\begin{proof}
??? \TODO

Take the free strict lexcategory on the underlying category of $T$ in which the basic limit comes in $T$ become limit cones (but not necessarily chosen limit cones; that is, we freely impose that the comparator map to the chosen limit is an isomorphism). This will be equivalent to T (proof??) and a map from it to someone else amounts to a strict lexfunctor out of $T$ which weakly preserves limits.
\end{proof}

Fix a lexcategory $T$, a set $D$, and a function $f : D \to \Ob(T)$ (since $T$ is non-strict, this really means a functor from the discrete category presented by $D$ to $\Ob(T)$, but that is fine). We say this function $f$ is dense if any full sublexcategory of $T$ containing (up to isomorphism) every object in the range of $f$ also contains (up to isomorphism) every object in $D$. Call this a lexcategory with strictly specified sorts.

The appropriate notion of a lex theory to have models in a strict lexcategory is a lexcategory with strictly specified sorts.

Consider the concept of a strict lexcategory $C$ and a lexfunctor from $T$ to the lexcategory presented by $C$. This is ALMOST an essentially algebraic concept, except for the fact that $T$ has not been given strictly specified sorts.

Let $T_{strict}$ be a strict lexcategory presenting the lexcategory $T$, and let $S_{strict}$ be a strict lexcategory presenting the lexcategory $S$.

The category $\StrictCat[T_{strict}, S_{strict}]$

Blah blah

Let us say the category of sketches is the minimal subcategory of the category of strict lexcategories closed under set-sized colimits indexed by posets and under freely adjoining objects, freely adjoining a morphism between existing objects, or freely making two parallel morphisms equal. A finite sketch is the same thing but with closure only under finite colimits (but also, we don't need the finite colimits here).

A sketch has the desired property of imposing no equations.

\TODOinline{Perhaps we should mark somehow those theorems which are specific to $\Set$ and perhaps use Choice vs those theorems which are meant to internalize}

\subsubsection{Existence of initial structures in $\Set$, $\LexCat$, etc}
\TODO

Some general theorem here which we will invoke in $\Set$, $\LexCat$ etc. Not to be invoked internal to toposes or anything, mind you, where different arguments are necessary than the following.

\begin{theorem}\label{Adamek}
The general theorem would presumably be the transfinite construction of initial algebras for pointed endofunctors preserving colimits indexed by ordinal $k$, on a category with colimits indexed by any ordinal $\leq k$. (For non-pointed endofunctors, we can do the same as long as we also have finite coproducts)
\end{theorem}

In our metatheory, we assume such a theorem and also we assume that initial small instances of some variety of category also always have unique homomorphsms into $\Set$ itself, if $\Set$ is that variety of category. (This is as opposed to the kind of initial algebras we get internal to a topos, arithmetic universe, etc, where $\Set$ is too large for the initiality to apply with respect to it)

Actually, we don't really need all that explicitly.

"Partial Horn logic and cartesian categories" by Palmgren and Vickers discusses the proof (internalizable to toposes, etc, and also adaptable to the infinitary case) that free models of essentially algebraic theories exist. Their construction makes it very clear that what is needed for the free model of a quasi-equational theory to exist is just an effective regular category with suitable W-types for modeling the terms (including partial terms) and equations of the theory. It is clear how quasi-equational theories generalize to arbitrary arity and the initial model construction as well.

This also discusses how to turn a quasi-equational theory T into the quasi-equational theory of a strict lexcategory with an internal model of T.

Applying the initial model to this gives us an initial strict lexcategory with an internal model of T; the strict lexcategory presented by T, we may say.

They also show, in essence, that the lexcategory this presents has the right universal property in $\LexCat$. And they show every lexcategory is presented by some quasi-equational theory (as is every strict lexcategory). So every result we might care about of this sort is implicit in there.

\subsubsection{Scribbles}
So, for example, let us try constructing, for a lexcategory $T$, the initial lexcategory with an internal initial model of $T$. We find some quasi-equational theory $Th(T)$ presenting $T$. This theory is k-sized, so we work from now on in k-ary quasi-equational theories. We now make the k-ary quasi-equational theory of a strict lexcategory with an internal initial model of $Th(T)$, and apply the initial algebra theorem. We get an initial strict lexcategory with an internal initial model of $Th(T)$; call this $S$.

Suppose we have three quasi-equational theories:
Th(T)
Th(Init-T')
Th(D)

These present strict lexcategories
T'
Init-T'
D'

These present lexcategories
T
Init-T
D

Th(T) is chosen to present T, with T' arising from this.
Th(Init-T') is chosen to present Init-T' as the free strict lexcategory with an internal initial model of Th(T). Init-T then arises from this.
Th(D) is chosen to present D, with D' arising from this.

For any initial model of Th(T) in D', we get a unique map from Init-T' to D' which restricts to the corresponding map from T' to D'.

Given two strict maps f, g from Init-T' to D' which both take T' to initial models in D', we want a unique transform from f to g. Transforms from f to g correspond by lemma 55 (even without the strictness assumption on f, g) to transforms between the corresponding models of Th(Init-T') in D'. We want [TODO] to show that these correspond to transforms between the corresponding models of Th(T) in D'. If we can do that, then since those models are initial, we get a unique map between them, as desired.
\fileend

\section{Introspective theories and GL-categories}

% Non-evil definition
\subsection{Definition}
\begin{definition} \label{IntrospectiveTheory}
A \defined{pre-introspective theory} is a lexcategory $T$, a $T$-indexed lexcategory $C$, and a lexfunctor $F$ from the self-indexing of $T$ to $C$, like so:

\[\begin{tikzcd}
	{\op{T}} && {\LexCat}
	\arrow["{T/-}"{name=0}, from=1-1, to=1-3, shift left=2]
	\arrow["{C}"{name=1, swap}, from=1-1, to=1-3, shift right=2]
	\arrow[Rightarrow, "{F}"', from=0, to=1]
\end{tikzcd}\]

If $C$ is small or locally small, then we say this is an \defined{introspective theory} or \defined{locally introspective theory}, respectively.
\end{definition}

We write out the triple $\langle T, C, F \rangle$ to refer to a pre-introspective theory when we wish to be fully explicit about its structure. But in typical abuse of language, we also often refer to it simply by the name of its underlying lexcategory $T$, when this would not cause confusion.

\begin{construction}
If $\langle T, C, F \rangle$ is a pre-introspective theory, and any lexfunctor $G : C \to D$ is given for some other $T$-indexed lexcategory $D$, then $\langle T, D, G \circ F \rangle$ is itself a pre-introspective theory, like so: 

\[\begin{tikzcd}
	{\op{T}} && {\LexCat}
	\arrow["{T/-}"{name=0}, from=1-1, to=1-3, shift left=5]
	\arrow["{C}"{name=1, description}, from=1-1, to=1-3]
	\arrow["{D}"{name=2, swap}, from=1-1, to=1-3, shift right=5]
	\arrow[Rightarrow, "{F}"', from=0, to=1]
	\arrow[Rightarrow, "{G}"', from=1, to=2]
\end{tikzcd}\]

Of course, this yields an introspective or locally introspective theory just in case $D$ is small or locally small, respectively.
\end{construction}

\begin{construction}
If $\langle T, C, F \rangle$ is a pre-introspective theory, and $t$ is any object in $T$, then the slice category $T/t$ can be equipped in a natural way as a pre-introspective theory as well. If we start from an introspective or locally introspective theory, then so respectively will be the result of this construction.
\end{construction}
\begin{proof}[Details]
Consider the forgetful functor $\Sigma : T/t \to T$. Composition and whiskering with this gives us the triple $\langle T/t, \Sigma C, \Sigma F \rangle$, like so:

\[\begin{tikzcd}
	{\op{\left(T/t\right)}} & {\op{T}} && {\LexCat}
	\arrow["{T/-}"{name=0}, from=1-2, to=1-4, shift left=2]
	\arrow["{C}"{name=1, swap}, from=1-2, to=1-4, shift right=2]
	\arrow["{\Sigma}", from=1-1, to=1-2]
	\arrow[Rightarrow, "{F}"', from=0, to=1]
\end{tikzcd}\]

This is a pre-introspective theory as the top composite is the same as the self-indexing $(T/t)/-$ of $T/t$ (that is, slice categories within slice categories are obtained by first applying the forgetful functor into the ambient category and then taking the ordinary slice category).

When $C$ is small or locally small, then so respectively is $\Sigma C$. The representing objects for $\Sigma C$ will simply be those for $C$, mapped into $T/t$ via $\Sigma$'s right adjoint (i.e., pulled back along the unique morphism from $t$ to $1$).
\end{proof}

When we abuse language and speak of $T/t$ as an introspective theory, the above construction is what we mean. That said, there is another natural way to equip $T/t$ as an introspective theory as well.

\begin{TODOblock}
There is another way to equip $T/t$ as an introspective theory, in which we use as our internal category the slice category $C/t$ so to speak. Construction 1.3 is actually equivalent to doing this and then applying Construction 1.2 to the internal forgetful lexfunctor from $C/t$ to $C$. This equivalence is easier to see once we see how the natural transformation taking $t$ to $\Hom_C(1, t)$ works.
\end{TODOblock}

\begin{construction}\label{SAndN}
Given a pre-introspective theory $\langle T, C, F \rangle$, we obtain immediately a lexfunctor from $T$ to the global aspect of $C$, given by the global aspect of $F$. Let us call this lexfunctor $S$, though all it amounts to is the same thing as $F$, restricted to acting on $T/1$.

We shall also obtain a natural transformation from the identity functor on $T$ to the functor $x \mapsto \Hom_C(1, S(x))$ [equivalently, $\Hom_C(1, F(x))$]. Let us call this natural transformation $N$. The specific construction of $N$ is like so:
\end{construction}
\begin{proof}[Details]
Given any functor $F$ between any $T$-indexed categories $D$ and $C$, we obtain a function from $\Hom_D(y, x)$ to $\Hom_C(F(y), F(x))$ natural in $x$ and $y$. For $F$ a lexfunctor, we can specialize this to a function from $\Hom_D(1, x)$ to $\Hom_C(1, F(x))$ natural in $x$. This is a $T$-indexed collection of functions, applicable to objects $x$ of $D$ defined over any object in $T$ and natural with respect to all morphisms of $D$ defined over any object in $T$. But in particular, this is applicable to objects within, and natural with respect to morphisms within, the global aspect of $D$.

If we now take $D$ to be the self-indexing of $T$, its global aspect is $T$ itself, and for each such object $x$, we may identify $\Hom_D(1, x)$ with $x$ itself, construed as a $T$-indexed set. Thus, we obtain a natural transformation from $x$ to $\Hom_C(1, F(x))$ as desired.
\end{proof}

It turns out, we can recover $F$ from arbitrary such $S$ and $N$ too. That is, \cref{SAndN} has an inverse.

\begin{construction}\label{InvertSAndN}
Given a lexcategory $T$, a $T$-indexed lexcategory $C$, a lexfunctor $S$ from $T$ to the global aspect of $C$, and a natural transformation $N$ from the identity functor on $T$ to the functor $x \mapsto \Hom_C(1, S(x))$, we can obtain a lexfunctor $F$ from the self-indexing of $T$ to $C$.

Specifically, let us first define yet another $T$-indexed lexcategory. We start with the observation that $C$, as a lexcategory (even if it is a $T$-indexed one) comes with a self-indexing $C/-$ of its own. That is, we get a $T$-indexed $C$-indexed lexcategory. The aspect of this defined over $1$ in $T$ gives us a lexcategory indexed by the global aspect of $C$. Composing this with our functor $S$ from $T$ to the global aspect of $C$, we get a $T$-indexed lexcategory. It is specifically the $T$-indexed lexcategory which assigns to each object $t$ in $T$ the global aspect of $C/S(t)$, and whose action on morphisms $f$ in $T$ is given by pullback along $S(f)$ in the global aspect of $C$. For sake of a name, let us call this $C/S$.

We will create our lexfunctor $F$ as a composition of lexfunctors from the self-indexing of $T$ to $C/S$ and from $C/S$ to $C$.

To obtain a lexfunctor from the self-indexing of $T$ to $C/S$, consider any data in the category $T/t$ for arbitrary object $t$ in $T$. This amounts to corresponding data in $T$ (slices over $t$ and commutative triangles over $t$), to which we can apply $S$ to get slices and commutative triangles over $S(t)$ in the global aspect of $C$. As $S$ is a lexfunctor, and finite limits in slice categories can be constructed out of finite limits in the underlying category, this yields a lexfunctor from $T/t$ to $C/S$. \TODOinline{Naturality? This depends on the lexness of S too, it seems.}

Next, note that if we had an element of $\Hom_C(1, S(t))$, pullback along that would give a lexfunctor from $C/S(t)$ to $C$. And indeed, this is precisely what $N$ gives us: an element of $\Hom_C(1, S(t))$ defined over $t$. Thus, we get a lexfunctor from $C/S$ to $C$. \TODOinline{Naturality?}

Composing these two steps, we have our desired lexfunctor $F$ from the self-indexing of $T$ to $C$.
\end{construction}

\begin{theorem}\label{SAndNCorrespondence}
If we apply the two constructions \cref{SAndN} and \cref{InvertSAndN} one after another in either order, we end up with the same values we started with. Thus, we have a true one-to-one correspondence.
\end{theorem}
\begin{proof}
If we apply \cref{InvertSAndN} first and \cref{SAndN} subsequently, we end up with the same $S$ and $N$ we started with.

For the $S$ we end up with acts on data by treating this data as slices over $1$, applying the original $S$ to it, then pulling it back along a morphism from $1$ to $\Hom_C(1, S(1)) = \Hom_C(1, 1) = 1$. This morphism from $1$ to $1$ must be the identity, and so we have done nothing here but simply apply $S$ in the end.

And the $N$ we end up with acts on any object $x$ of $T$ by \TODO.

Conversely, if we apply \cref{SAndN} first and \cref{InvertSAndN} subsequently, we end up with the same $F$ we started with. \TODO
\end{proof}

As a result of \cref{SAndNCorrespondence}, we can give another definition equivalent to \cref{IntrospectiveTheory}.

\begin{definition}\label{IntrospectiveTheoryNS}
A \defined{pre-introspective theory} is a lexcategory $T$, a $T$-indexed lexcategory $C$, a lexfunctor $S$ from $T$ to the global aspect of $C$, and a map $N$ from $t$ to $\Hom_C(1, S(t))$, natural in $t$ in $T$.
\end{definition}

As before, introspective or locally introspective theories are those where $C$ is small or locally small. The value of this new definition is that there is much less data around. In particular, when we wish to turn this into a lex definition in section \TODO, we will find the coherence conditions much easier to manage. It will also be easier to define the appropriate notion of homomorphisms between pre-introspective theories by thinking about this definition.

This definition also allows us to quickly appreciate the significance of introspective theories from a functorial semantics point of view. An introspective theory is precisely an essentially algebraic theory (this is the role of $T$) extending the theory of lexcategories (this is the role of $C$), such that every model of the theory is equipped with a designated homomorphism (this is the role of $N$) into an internal model of the same theory (this is the role of $S$).

\filestart

\section{\Loeb/'s theorem}\label{LoebChapter}
\subsection{Preview}
In this chapter, we prove our most important theorem, justifying the significance of the simple concept of introspective theories. We show how every introspective theory automatically satisfies a general version of \Loeb/'s theorem, acting as the construction of general fixed points. We will also see how \Loeb/'s theorem is in full generality a phenomenon linked to presheaves, and not only constrained to representable presheaves.

The key results of this chapter are those covered in \magicref{PreIntrospDiagSection} and \magicref{IntrospDiagSection}, culminating in \magicref{IntrospLoeb}, our most important theorem. All material in those sections is original to this work.

The material on the \Loeb/ property in general categories in \magicref{LoebPropertySection} includes some observations which can also be found (either explicitly or implicitly) in existing literature. We give our own exposition of this material, which felt important to include in a clean and complete exposition of the significance of our key results. In particular, we confirm how these general \Loeb/ property results continue to be applicable in our particular introspective theory context, even without common presumptions such as cartesian closure, and even with care taken to distinguish the roles of $T$, $C$, and $\Psh{C}$ in a general introspective theory $\langle T, C \rangle$.

The discussion in \magicref{LawvereFPTSection} concerns Lawvere's fixed point theorem, which of course is not original to us, but we also include some reframing and generalization of this which is due to us rather than Lawvere. The discussion in \magicref{LawvereFPTReduxSection} compares our reframing to some other reframings of Lawvere's fixed point theorem in the existing literature.

The sole theorem in \magicref{NoSmallSelfIndexing} is the same theorem as proven in \autocite{PittsTaylor1989}. We re-note it here simply to observe that it follows as a special case of our more general \magicref{IntrospLoeb}.

\subsection{The \Loeb/ property in abstract}\label{LoebPropertySection}
\begin{definition}\label{LoebPropertyDefn}
Let $D$ be any category with a terminal object and let $\Box : D \to D$ be a terminal-object-preserving endofunctor on $D$. We say $\Box$ has the \defined{\Loeb/ property} if, for every object $\Omega$ of $D$ and every morphism $\loebNeg : \Box \Omega \to \Omega$, there exists a morphism $\omega : 1 \to \Omega$ making the following square commute:
% https://q.uiver.app/#q=WzAsNCxbMCwxLCIxIl0sWzEsMCwiXFxCb3ggXFxPbWVnYSJdLFsxLDEsIlxcT21lZ2EiXSxbMCwwLCJcXEJveCAxIl0sWzEsMiwiXFxsb2ViTmVnIl0sWzAsMiwiXFxvbWVnYSIsMl0sWzMsMCwiIiwwLHsibGV2ZWwiOjIsInN0eWxlIjp7ImhlYWQiOnsibmFtZSI6Im5vbmUifX19XSxbMywxLCJcXEJveCBcXG9tZWdhIl1d
\[\begin{tikzcd}[ampersand replacement=\&]
	{\Box 1} \& {\Box \Omega} \\
	1 \& \Omega
	\arrow["\loebNeg", from=1-2, to=2-2]
	\arrow["\omega"', from=2-1, to=2-2]
	\arrow[Rightarrow, no head, from=1-1, to=2-1]
	\arrow["{\Box \omega}", from=1-1, to=1-2]
\end{tikzcd}\]

In other words, for every $\Omega \in D$ and $\loebNeg : \Box \Omega \to \Omega$, there is a fixed point of $\omega \mapsto \loebNeg \circ (\Box \omega) : \Hom_D(1, \Omega) \to \Hom_D(1, \Omega)$.

If such fixed points are furthermore always unique, we say $\Box$ has the \defined{\Loeb/ property with uniqueness}. (Note that the \Loeb/ property with uniqueness is the same as saying that the unique map $: \Box 1 \to 1$ is an initial $\Box$-algebra.)
\end{definition}

\begin{observation}\label{KnasterTarskiExample}
For example, the identity endofunctor on the category of complete lattices and monotonic maps between them has the \Loeb/ property (this amounts to the Knaster-Tarski fixed point theorem). However, this does not have the \Loeb/ property with uniqueness.
\end{observation}

In this chapter, we will establish that for every introspective theory $\langle T, C \rangle$, each aspect of each of $\Box_{T/-}$, $\Box_C$, and $\Box_{\Psh{C}}$ has the \Loeb/ property with uniqueness. That such a strong result follows from such a minimal and simple categorical structure motivates much of our interest in the concept of introspective theories.

But before we establish this version of \Loeb/'s theorem for introspective theories in particular, we will develop the theory of the \Loeb/ property and its consequences a little further in abstract. 

\begin{theorem}\label{LoebTransfer}
Let $D$ and $E$ be categories with terminal objects, and let $M : D \to E$ and $N : E \to D$ be functors preserving terminal objects. Suppose $NM : D \to D$ has the \Loeb/ property. Then so does $MN : E \to E$. Furthermore, if $NM$ has the \Loeb/ property with uniqueness, then so does $MN$.
\end{theorem}
\begin{proof}
This is by the general theorem\sTODOinline{Note as a lemma in Preliminaries?} that fixed points of a composition of functions are in bijection with fixed points of any cyclic rearrangement of that composition (as $f$ and $g$ themselves restrict to inverse maps between fixed points of $gf$ and $fg$). In particular, letting $comp_x(y) = x \circ y$, the fixed points of $\omega_E \mapsto comp_{\loebNeg}(M(N(\omega_E)))$ are in bijection with the fixed points of $\omega_D \mapsto N (comp_{{\loebNeg}}(M(\omega_D)))$, which is to say, of $\omega_D \mapsto comp_{N({\loebNeg})}(N (M (\omega_D)))$.

The latter fixed points must exist (or exist uniquely) if $NM : D \to D$ has the \Loeb/ property (or the \Loeb/ property with uniqueness, respectively), and thus in such cases so do the former fixed points, establishing the corresponding property for $MN : E \to E$.
\end{proof}

\sTODOinline{The above is a special case of Eppendahl's observations on algebra-initial coalgebras in \quote{Coalgebra-to-Algebra Morphisms}.}

\begin{theorem}\label{LoebPropertyLexUniqueness}
Let $D$ be any lexcategory\sTODOinline{Or any category with terminal object and equalizers, but for now we will go ahead and presume binary products as well for convenience}, and let $\Box : D \to D$ be a terminal-object-preserving endofunctor on $D$. If $\Box$ has the \Loeb/ property, then it furthermore has the \Loeb/ property with uniqueness.
\end{theorem}
\begin{proof}
We must show that, given any two commutative squares as below (with the same $\loebNeg$ on the right hand side of each), the morphisms $\omega$ and $\psi$ are equal:

% https://q.uiver.app/?q=WzAsOCxbMywxLCIxIl0sWzQsMCwiXFxCb3ggXFxPbWVnYSJdLFs0LDEsIlxcT21lZ2EiXSxbMywwLCJcXEJveCAxIl0sWzAsMCwiXFxCb3ggMSJdLFsxLDAsIlxcQm94IFxcT21lZ2EiXSxbMSwxLCJcXE9tZWdhIl0sWzAsMSwiMSJdLFsxLDIsIlxcbG9lYk5lZyJdLFswLDIsIlxccHNpIiwyXSxbMywwLCIiLDAseyJsZXZlbCI6Miwic3R5bGUiOnsiaGVhZCI6eyJuYW1lIjoibm9uZSJ9fX1dLFszLDEsIlxcQm94IFxccHNpIl0sWzQsNSwiXFxCb3ggXFxvbWVnYSJdLFs1LDYsIlxcbG9lYk5lZyJdLFs0LDcsIiIsMix7ImxldmVsIjoyLCJzdHlsZSI6eyJoZWFkIjp7Im5hbWUiOiJub25lIn19fV0sWzcsNiwiXFxvbWVnYSIsMl1d
\[\begin{tikzcd}
	{\Box 1} & {\Box \Omega} && {\Box 1} & {\Box \Omega} \\
	1 & \Omega && 1 & \Omega
	\arrow["\loebNeg", from=1-5, to=2-5]
	\arrow["\psi"', from=2-4, to=2-5]
	\arrow[Rightarrow, no head, from=1-4, to=2-4]
	\arrow["{\Box \psi}", from=1-4, to=1-5]
	\arrow["{\Box \omega}", from=1-1, to=1-2]
	\arrow["\loebNeg", from=1-2, to=2-2]
	\arrow[Rightarrow, no head, from=1-1, to=2-1]
	\arrow["\omega"', from=2-1, to=2-2]
\end{tikzcd}\]

Let $h : H \to 1$ be the equalizer of $\omega$ and $\psi$.  We will have that $\omega = \psi$ just in case $h$ is an isomorphism. As this $h$ is monic, making $H$ a subobject of $1$, we will have that $h$ is an isomorphism just in case there is any map from $1$ to $H$.

Thanks to the \Loeb/ property, this in turn occurs just in case there is some map from $\Box H$ to $H$. And by the definition of $H$ as an equalizer, this occurs just in case there is some map from $\Box H$ to $1$ which gives equal results when composed with $\omega$ and with $\psi$.

But the map $\Box h : \Box H \to 1$ does indeed have this property, as seen in the following commutative diagram (where the top left square commutes because $h; \omega = h; \psi$):

% https://q.uiver.app/?q=WzAsOCxbMCwyLCIxIl0sWzEsMSwiXFxCb3ggXFxPbWVnYSJdLFsxLDIsIlxcT21lZ2EiXSxbMCwxLCJcXEJveCAxIl0sWzAsMCwiXFxCb3ggSCJdLFsxLDAsIlxcQm94IDEiXSxbMiwwLCIxIl0sWzIsMSwiXFxPbWVnYSJdLFsxLDIsIlxcbG9lYk5lZyJdLFswLDIsIlxccHNpIiwyXSxbMywwLCIiLDAseyJsZXZlbCI6Miwic3R5bGUiOnsiaGVhZCI6eyJuYW1lIjoibm9uZSJ9fX1dLFszLDEsIlxcQm94IFxccHNpIl0sWzQsMywiXFxCb3ggaCIsMl0sWzQsNSwiXFxCb3ggaCJdLFs1LDEsIlxcQm94IFxcb21lZ2EiLDJdLFs1LDYsIiIsMCx7ImxldmVsIjoyLCJzdHlsZSI6eyJoZWFkIjp7Im5hbWUiOiJub25lIn19fV0sWzYsNywiXFxvbWVnYSJdLFsxLDcsIlxcbG9lYk5lZyJdXQ==
\[\begin{tikzcd}
	{\Box H} & {\Box 1} & 1 \\
	{\Box 1} & {\Box \Omega} & \Omega \\
	1 & \Omega
	\arrow["\loebNeg", from=2-2, to=3-2]
	\arrow["\psi"', from=3-1, to=3-2]
	\arrow[Rightarrow, no head, from=2-1, to=3-1]
	\arrow["{\Box \psi}", from=2-1, to=2-2]
	\arrow["{\Box h}"', from=1-1, to=2-1]
	\arrow["{\Box h}", from=1-1, to=1-2]
	\arrow["{\Box \omega}"', from=1-2, to=2-2]
	\arrow[Rightarrow, no head, from=1-2, to=1-3]
	\arrow["\omega", from=1-3, to=2-3]
	\arrow["\loebNeg", from=2-2, to=2-3]
\end{tikzcd}\]

This completes the proof.
\end{proof}

\begin{remark}
Note that \magicref{LoebPropertyLexUniqueness} makes essential use of the structure available in a lexcategory. We can see this by considering the example from \magicref{KnasterTarskiExample}, which has the \Loeb/ property but not the \Loeb/ property with uniqueness. This is possible as the category of complete lattices and arbitrary monotonic maps lacks equalizers.
\end{remark}

The application of these abstract results to locally introspective theories in particular is like so:

\begin{theorem}\label{LoebTransferIntrosp}
If $\langle T, C \rangle$ is a locally introspective theory and $t$ is an object of $T$ such that at least one of $\Box_{T/-}$, $\Box_C$, or $\Box_{\Psh{C}}$ has the \Loeb/ property (without presumed uniqueness) at its $t$-aspect, then all three have the \Loeb/ property with uniqueness at their $t$-aspect.

If this happens for every $t \in T$, we say this locally introspective theory itself has the \defined{\Loeb/ property}.
\end{theorem}
\begin{proof}
By \magicref{LoebTransfer}, when considering the definitions of the various $\Box$ operators given via the triangle at \magicref{BoxDefn}, we find that if any of these $\Box$ operators have the \Loeb/ property at their $t$-aspect, then all three do. By \magicref{LoebPropertyLexUniqueness}, we can furthermore conclude the \Loeb/ property with uniqueness.
\end{proof}

\sTODOinline{
I am removing the following for now as possibly a distraction.
\begin{remark}
In the above proof, we used \magicref{LoebTransfer} to transfer the \Loeb/ property from $\Box_C$ to $\Box_{T/-}$. Later, via \magicref{IntrospAsGeminal}, we will learn an alternative reason why we can automatically transfer properties which hold of $C$ in all introspective theories $\langle T, C \rangle$ to properties which hold of $T$ in all introspective theories $\langle T, C \rangle$.
\end{remark}
}

\magicref{LoebPropertyLexUniqueness} is only a special case of a much broader and important theorem which we now discuss. For convenience of exposition, we take $\Box$ in the following to be a lexfunctor, though this can be relaxed.

\begin{theorem}\label{CoalgToAlg}
Let $D$ be any lexcategory and let $\Box : D \to D$ be a lexfunctor. Let $E$ be any \repsmall/ $D$-indexed category. (Note that $\Box$ acting on $E$ induces also another \repsmall/\sTODOinline{Is this \repsmall/ when $\Box$ is not lex? It may be that we actually are using some kind of left Kan extension here, and the result will not be \repsmall/ in such cases. At any rate, it seems the better thing to do is to return to how I originally framed this from line 312 in commit commit 375fa9c.} $D$-indexed category $\Box E$ \sTODOinline{Note in preliminaries how the choice of internal category presenting these indexed categories doesn't matter, as $\Box$ acts on the internal functors between these too}, as well as a functor from each $d$-defined aspect of $E$ to the $(\Box d)$-defined aspect of $\Box E$, for $d \in D$. In particular, as $\Box$ is terminal-object-preserving, $\Box$ acts as a functor from the global aspect of $E$ to the global aspect of $\Box E$.) 

Suppose also given a $D$-indexed functor $f : \Box E \to E$, and let the endofunctor $F$ on the global aspect of $E$ be given by first applying $\Box$ to arrive in the global aspect of $\Box E$, then applying $f$ to arrive back in the global aspect of $E$.

If $\Box$ has the \Loeb/ property, then there is an $F$-hylomorphism (as in \magicref{HylomorphismDefn}) between any $F$-coalgebra $W : w \to F(w)$ and any $F$-algebra $M : F(m) \to m$ in the global aspect of $E$. And if $\Box$ furthermore has the \Loeb/ property with uniqueness, then this hylomorphism is unique.
\end{theorem}
\begin{proof}
A hylomorphism from $W$ to $M$ is a fixed point of $x \mapsto M \circ F(x) \circ W : \Hom_E(w, m) \to \Hom_E(w, m)$. But as $F(x) = f(\Box x)$, this is the same as a fixed point for $x \mapsto g(\Box x)$ where $g(-)$ is defined by $M \circ f(-) \circ W : \Box \Hom_E(w, m) \to \Hom_E(w, m)$. 

The hylomorphisms from $W$ to $M$ are thus the same as the fixed points given by the \Loeb/ property with respect to this $g$. This completes the proof.
\end{proof}

\sTODOinline{The above theorem \magicref{CoalgToAlg} generalizes a little further than we've stated it. It is not quite necessary that $E$ be \repsmall/ overall, nor that $D$ have any limits beyond a terminal object. It would suffice for the hom-sets between globally defined objects of $E$ to be \repsmall/ (this is the structure of an enriched category). But it seems not worth the extra hassle of explaining what $\Box E$ means in that situation right now, and every use we will make of this is anyway covered by the description above.}

We now demonstrate how \magicref{LoebPropertyLexUniqueness} can be seen as a special case of \magicref{CoalgToAlg} (at least, in the case where $\Box$ is a lexfunctor):
\begin{corollary}\label{LoebPropertyLexUniquenessRedux}
Let $D$ be any lexcategory\sTODOinline{Or any category with terminal object and equalizers, but for now we will go ahead and presume binary products as well for convenience}, and let $\Box : D \to D$ be a lexfunctor. If $\Box$ has the \Loeb/ property, then it furthermore has the \Loeb/ property with uniqueness.
\end{corollary}
\begin{proof}
Let $E$ be an arbitrary object of $D$ (thus, a \repsmall/ $D$-indexed set) and let us construe this also as a \repsmall/ $D$-indexed discrete category. Let $f : \Box E \to E$ be an arbitrary map in $D$, and as above, let us take $F : \Hom_D(1, E) \to \Hom_D(1, E)$ to be given as the composition of $\Box : \Hom_D(1, E) \to \Hom_D(1, \Box E)$ with $f \circ - : \Hom_D(1, \Box E) \to \Hom_D(1, E)$.

As $E$ is a discrete category, observe that any $F$-coalgebra or $F$-algebra in the global aspect of $E$ amounts to a fixed point of $f \circ \Box(-) : \Hom_D(1, E) \to \Hom_D(1, E)$. The \Loeb/ property tells us such fixed points exist, while \magicref{CoalgToAlg} tells us there is a hylomorphism between any such fixed points. But as $E$ is a discrete category, such a hylomorphism amounts to just an equality between the two elements of $\Hom_D(1, E)$. Thus, any two such fixed points are equal, which is to say, we have the \Loeb/ property with uniqueness.
\end{proof}
(It is perhaps easy to miss how the presumption of equalizers in $D$ has been used in the argument for \magicref{LoebPropertyLexUniquenessRedux}. At one point within its invocation of \magicref{CoalgToAlg}, the argument considers the object $\Box \Hom_E(w, m)$ for parallel $w, m \in \Hom_D(1, E)$. As such, it depends upon the fact that $\Hom_E(w, m)$ is a \repsmall/ $D$-indexed set. This object $\Hom_E(w, m)$ of $D$ is given by an equalizer between parallel maps from $1$ to $E$ in $D$; this is where the fact that $D$ is a lexcategory is essential.)

\begin{corollary}\label{InitialTerminalCoincidence}
Consider the same setup as of \magicref{CoalgToAlg}, and presume $\Box$ has the \Loeb/ property with uniqueness (as we now know follows automatically from the \Loeb/ property on a lexcategory). Then any fixed point of $F$ (in the sense of an object $e$ of the global aspect of $E$ along with an isomorphism between $e$ and $F(e)$) is simultaneously an initial $F$-algebra and a terminal $F$-coalgebra. In particular, any two such fixed points are isomorphic, via a unique $F$-algebra isomorphism.
\end{corollary}
\begin{proof}
In that context, \magicref{CoalgToAlg} says that every $F$-coalgebra has a \emph{unique} hylomorphism into every $F$-algebra. In the particular case that the coalgebra is invertible, this can be read as a morphism between algebras, and establishes that the coalgebra's inverse is an initial algebra. Dually, for any invertible algebra, this establishes its inverse as a terminal coalgebra.
\end{proof}

\sTODOinline{Link this to \magicref{IntrospTyConFixedPoints}. Note after our later bootstrapping section that this uniqueness means that ultimately, the apparent dependence on various arbitrary choices in our proof of the bootstrapping theorem doesn't matter in the end. We get the same result no matter what.}

\begin{remark}
The argument we have given for \magicref{CoalgToAlg} and thus for \magicref{InitialTerminalCoincidence} is essentially the same as that given for Lemma 7.6 in \autocite{birkedal2011first}. For convenience for our purposes, we have framed this in terms of internal categories, though in \autocite{birkedal2011first} it is more properly framed as about enriched categories more generally. On the other hand, this argument is given in \autocite{birkedal2011first} in a context where the uniqueness of the \Loeb/ property has already been presumed, whereas we have noted that this argument can also be given in a context where only the weaker \Loeb/ property without uniqueness has been presumed, and then this argument can be used to in fact derive said uniqueness in a lexcategory.

Arguments establishing that the weaker \Loeb/ property entails the \Loeb/ property with uniqueness in contexts with identity types have been noted in the literature on guarded recursion. For example, as Theorem V.8 in \autocite{birkedal2013universes} and as Theorem 9.5 in \autocite{birkedal2021Multimodal}. However, we are unaware of any prior observation in the literature that this uniqueness can also be understood as a special case of the existence of coalgebra-to-algebra hylomorphisms, unifying those arguments.
\end{remark}

\begin{theorem}\label{CCCLoebUniqueness}
The identity endofunctor on a cartesian closed category has the \Loeb/ property with uniqueness just in case the category is the trivial terminal category.
\end{theorem}
\begin{proof}
Taking $\Box$ to be this identity endofunctor and applying the \Loeb/ property with uniqueness to the morphism $\id_{B^A} : \Box(B^A) \to (B^A)$, for arbitrary objects $A$ and $B$, we find that each $B^A$ has a unique global element, which is to say, there is a unique map between any pair of objects. Thus all objects become isomorphic to the terminal object.
\end{proof}

\begin{corollary}\label{CCCLoebLex}
The identity endofunctor on a cartesian closed category with equalizers has the \Loeb/ property just in case the category is the trivial terminal category.
\end{corollary}
\begin{proof}
By combining \magicref{CCCLoebUniqueness} and \magicref{LoebPropertyLexUniqueness}.
\end{proof}

\begin{observation}
The example from \magicref{KnasterTarskiExample} shows that it is possible for the identity endofunctor on a nontrivial cartesian closed category to have the \Loeb/ property, so long as neither uniqueness nor equalizers are presumed.
\end{observation}

\subsection{Lawvere's fixed point theorem}\label{LawvereFPTSection}
Let us refresh the reader on Lawvere's fixed point theorem \autocite{lawvere1969diagonal}, which captures the general structure of many diagonalization arguments and their relationship to cartesian closed structure. We shall first review a proof of Lawvere's fixed point theorem close in spirit to Lawvere's framing of his result.

Then we will note a slight generalization for which essentially the same argument applies. Then in the next section we will turn this generalization into a result in the context of general pre-introspective theories. Then we will specialize further down to introspective theories, and observe a wonderful \quote{bootstrapping} phenomenon which arises there, which shall ultimately provide us with the \Loeb/ property in that context, which is our main result.

\openNamed{theorem}{Lawvere's Fixed Point Theorem}\label{LawveresFixedPointTheorem}
Let $T$ be an arbitrary category. Let $X$ be an object of $T$ and let $\Omega$ be any $T$-indexed set. Suppose also given some map $\App' : X \to \Omega^X$ (equivalent to the data of a map $\App : X \times X \to \Omega$).

Let $\point$ be any object of $T$. By a \quote{point} of a $T$-indexed set, we shall mean an element of its aspect at $\point$ (equivalent to the data of a map into it from $\point$).

Suppose $\App$ has the surjectivity-like property that, for every map $F : X \to \Omega$, there is a point $f$ of $X$, such that for every point $x$ of $X$, we have that $\App(f, x) = F(x)$.

Then for any map $\loebNeg : \Omega \to \Omega$, there exists a point $\omega$ of $\Omega$ such that $\omega = \loebNeg(\omega)$. That is to say, $\loebNeg$ has a fixed point.
\closeNamed{theorem}
\begin{proof}
Let $F : X \to \Omega$ be the following composition:

% https://q.uiver.app/?q=WzAsNCxbMCwwLCJYIl0sWzIsMCwiWCBcXHRpbWVzIFgiXSxbMywwLCJcXE9tZWdhIl0sWzQsMCwiXFxPbWVnYSJdLFswLDEsIlxcbGFuZ2xlIFxcaWRfWCwgXFxpZF9YIFxccmFuZ2xlIl0sWzEsMiwiXFxBcHAiXSxbMiwzLCJcXGxvZWJOZWciXV0=
\[\begin{tikzcd}
	X && {X \times X} & \Omega & \Omega
	\arrow["{\langle \id_X, \id_X \rangle}", from=1-1, to=1-3]
	\arrow["\App", from=1-3, to=1-4]
	\arrow["\loebNeg", from=1-4, to=1-5]
\end{tikzcd}\]

That is, for any generalized element $x$ of $X$, we have that $F(x) = \loebNeg(\App(x, x))$.

We know there exists a point $f$ of $X$ which corresponds with $F$ in the manner of our surjectivity-like supposition on $\App$. Now consider the instance of this surjectivity-like supposition where $x = f$. This tells us that $\App(f, f) = F(f)$. But $F(f) = \loebNeg(\App(f, f))$.

Thus, taking $\omega = \App(f, f)$, we have that $\omega = \loebNeg(\omega)$ as desired.
\end{proof}

Let us make a few remarks on the scope of generality of this theorem.

Lawvere originally states this theorem specifically for the case where $T$ is a cartesian closed category, but later in \autocite{lawvere1969diagonal} notes that this implies the theorem just as well for the case where $T$ is merely a category with finite products, as any category can be embedded as a full subcategory of a cartesian closed category in a way which preserves any products or exponentials already present (via the Yoneda embedding). \autocite{lawvere1969diagonal} does not explicitly consider examples where the original category of interest $T$ lacks finite products, such that $X \times X$ is not an object of $T$, nor consider taking $\Omega$ to be merely a $T$-indexed set rather than an object of $T$, but of course these are covered in the same way by the same insight that we can work in $\Psh{T}$ instead of $T$.

Having observed that we can just as well frame the theorem with any of its objects drawn from $\Psh{T}$ rather than $T$, the reader might then well wonder why in our framing we have allowed some objects to be in $\Psh{T}$ but still constrained others (such as $X$) to come from $T$. We chose this particular framing partly as this is closest to the applications we have in mind, and also partly for what amount to stylistic reasons. In particular, having stated the theorem in this form, interpreting the surjectivity condition on $\App$ only requires quantification over the set of morphisms from object $X$ to presheaf $\Omega$ (i.e., the set $\Omega(X)$), instead of requiring quantification over the class of natural transformations from a presheaf $X$ to another presheaf $\Omega$ (which is potentially a proper class, if $T$ is proper-class-sized). But this is not really of much importance, and again the more general form of the theorem follows readily from the ostensibly less general one.

\autocite{lawvere1969diagonal} also only states this theorem in the particular case where $\point$ is a terminal object. In general, we can always pass from $T$ to a slice category $T/\point$, and in so doing we will turn what was $\point$-defined data in $T$ into globally defined data in $T/\point$ (a la \magicref{AspectIsSliceGlobal}). So constraining $\point$ to be a terminal object does not constrain the theorem excessively. However, it does constrain the theorem slightly, in that interpreting the surjectivity precondition in $T/\point$ in this way results in a stronger (that is, less often satisfied) surjectivity precondition than in the more flexible framing of the theorem we have given: The surjectivity condition in $T/\point$ would amount to requiring that for every $F : \point \times X \to \Omega$ in $\Psh{T}$, we could find a corresponding $f$. However, we have only required surjectivity with respect to the more constrained set of $F : X \to \Omega$ in $\Psh{T}$.

We do not actually need this extra flexibility for proving our main result. For our purposes, just like Lawvere's, it would suffice to always take $\point$ to be a terminal object. But we note the availability of this flexibility all the same (if only for the purpose of comparison at the end of this chapter to other variants on Lawvere's fixed point theorem recently noted in the literature, such as \magicref{MagmoidalFixedPointTheorem}).

Even this loosened surjectivity presumption is still far overkill as far as the needs of the argument go. All that really matters is for one specific definable value to be in the range of $\App'$. But in general practice and for our particular purposes, this is always established because of some such surjectivity condition anyway, so that seems the most useful framing in which to give the theorem.

Having said all that about the wide applicability of \magicref{LawveresFixedPointTheorem}, we actually will need to generalize it slightly further for our purposes. Having given the above discussion of the traditional theorem to prime the reader's intuitions through familiarity, we now put forward the following simple generalization:

\openNamed{theorem}{Self-Related Point Theorem}\label{SelfRelatedPointTheorem}
Let $T$ be an arbitrary category. Let $\point$ and $X$ be objects of $T$ and let $\Omega$ be any $T$-indexed set. Suppose also given some map $\App' : X \to \Omega^X$ (equivalent to the data of a map $\App : X \times X \to \Omega$).

As before, we shall use \quote{point of} as shorthand for \quote{element of the $\point$-aspect of}.

Suppose also given a binary relation $R$ on the points of $\Omega$. (We needn't presume $R$ to be symmetric or transitive or any such thing.). And suppose $\App$ has the surjectivity-like property that, for every morphism $F : X \to \Omega$, there is a point $f$ of $X$, such that for every point $x$ of $X$, we have $R(\App(f, x), F(x))$.

Then there exists a point $\omega$ of $\Omega$ such that $R(\omega, \omega)$. That is to say, $R$ has a self-related point.
\closeNamed{theorem}
\begin{proof}
Let $F : X \to \Omega$ be the following composition:

% https://q.uiver.app/?q=WzAsNCxbMCwwLCJYIl0sWzIsMCwiWCBcXHRpbWVzIFgiXSxbMywwLCJcXE9tZWdhIl0sWzQsMF0sWzAsMSwiXFxsYW5nbGUgXFxpZF9YLCBcXGlkX1ggXFxyYW5nbGUiXSxbMSwyLCJcXEFwcCJdXQ==
\[\begin{tikzcd}
	X && {X \times X} & \Omega & {}
	\arrow["{\langle \id_X, \id_X \rangle}", from=1-1, to=1-3]
	\arrow["\App", from=1-3, to=1-4]
\end{tikzcd}\]

That is, for any generalized element $x$ of $X$, we have that $F(x) = \App(x, x)$.

We know there exists a point $f$ of $X$ in accordance with our surjectivity-like supposition on $\App'$. Now consider the instance of the surjectivity-like supposition where $x = f$. This tells us that $R(\App(f, f), F(f))$. But $F(f) = \App(f, f)$.

Thus, we have found a point of $\Omega$ which is related to itself by $R$, as desired.
\end{proof}

It may not be obvious that this generalizes \magicref{LawveresFixedPointTheorem}. The following shows how this is so:

\openNamed{corollary}{Relatedly-Fixed Point Theorem}\label{RelatedlyFixedPointTheorem}
Consider the same setup as of \magicref{SelfRelatedPointTheorem}, and furthermore, suppose given $\loebNeg : \Omega \to \Omega$.

Then there exists a point $\omega$ of $\Omega$ such that $R(\omega, \loebNeg(\omega))$. We might describe this as \quote{$\omega$ is an $R$-fixed point of $\loebNeg$}.
\closeNamed{corollary}
\begin{proof}
Consider the binary relation $R_{\loebNeg}$ on points of $\Omega$ given by $R_{\loebNeg}(\omega_1, \omega_2) = R(\omega_1, \loebNeg(\omega_2))$.

We have been given the supposition that, for every morphism $F : X \to \Omega$, there is a point $f$ of $X$, such that for every point $x$ of $X$, we have $R(\App(f, x), F(x))$.

As this holds for arbitrary $F : X \to \Omega$, this also holds when an arbitrary $F$ is replaced by $\loebNeg \circ F : X \to \Omega$. That is to say, for every $F : X \to \Omega$, there is a point $f$ of $X$, such that for every point $x$ of $X$, we have $R(\App(f, x), (\loebNeg \circ F)(x))$, which is to say, $R_{\loebNeg}(\App(f, x), F(x))$.

But this is precisely the surjectivity supposition we need in order to invoke \magicref{SelfRelatedPointTheorem} with $R_{\loebNeg}$ in place of $R$. Doing so, we obtain a point $\omega$ of $\Omega$ such that $R_{\loebNeg}(\omega, \omega)$, which is to say $R(\omega, \loebNeg(\omega))$, as desired.
\end{proof}

Now we can see that \magicref{LawveresFixedPointTheorem} is of course the instance of \magicref{RelatedlyFixedPointTheorem} where the relation $R$ is taken to be equality. But \magicref{RelatedlyFixedPointTheorem} is strictly more general in allowing the use of an arbitrary relation.

(As for the relation between \magicref{RelatedlyFixedPointTheorem} and \magicref{SelfRelatedPointTheorem}, each is an instance of the other. We above obtained \magicref{RelatedlyFixedPointTheorem} as a corollary of \magicref{SelfRelatedPointTheorem}. But also conversely, \magicref{SelfRelatedPointTheorem} is the special case of \magicref{RelatedlyFixedPointTheorem} where $g$ is taken to be $\id_{\Omega}$.)

At any rate, we shall find the added flexibility of allowing a relation in place of equality to be valuable in the next sections, as we begin to specialize towards our application in introspective theories.

\subsection{Presheaf diagonalization for pre-introspective theories}\label{PreIntrospDiagSection}
\openNamed{theorem}{Pre-introspective Diagonalization}\label{PreIntrospDiag}
Let $\langle T, C, \introS, \introN \rangle$ be a pre-introspective theory. Let $\point_T$ be the terminal object of $T$ and let $\point_C$ be the terminal object of $C$.\footnote{We use this $\star$ notation rather than $1$ notation so that we can make the observation that this theorem's proof actually applies more generally, not depending on any limit structure. It would suffice to let $T$ be any category, let $C$ be a $T$-indexed category, let $\introS$ be a functor from $T$ to the global aspect of $C$, let $\introN$ be a map from $t$ to $\Hom_C(\point_C, \introS(t))$, natural in $t \in T$, let $\point_T$ be any object of $T$, and let $\point_C$ be any globally defined object of $C$. Knowing that the proof makes no use of limit structure may make it easier to follow.} Throughout the following, we use \quote{point of} as shorthand for \quote{element of the $\point_T$-aspect of}.

Furthermore, let $P$ be a $(T, C)$-indexed set, in the sense of \magicref{PreliminariesMultipleIndexing}. We will write in the following $P(c)$ to mean the $T$-indexed set $t \mapsto P(t, c)$, for globally defined objects $c$ of $C$.

Suppose also given some object $\Omega \in T$ with a map $\quotient : \Omega \to P(\point_C)$ such that the induced function $\quotient \circ - : \Hom(X \times X, \Omega) \to \Hom(X \times X, P(\point_C))$ is surjective.

Suppose also given some object $X \in T$ and map $\alpha : X \to P(\introS(X))$. We also make a surjectivity-like assumption on $\alpha$. Specifically, we suppose that for every global element $p$ of $P(\introS(X))$, there is a point $x$ of $X$ such that $\alpha(x) = p$, as points of $P(\introS(X))$.

Finally, let $\loebNeg$ be a globally defined element of $P(\introS(\Omega))$.

Then we obtain a point $\omega$ of $\Omega$, such that $\quotient(\omega) = \pullAlong{\introN_{\Omega}(\omega)} \loebNeg$.
\closeNamed{theorem}
\begin{proof}
We shall show how this is an instance of \magicref{SelfRelatedPointTheorem}.

We define $\App : X \times X \to \Omega$ like so: Consider the two projection maps $\pi_1, \pi_2 : X \times X \to X$, as the two generic $(X \times X)$-defined elements of $X$. We thus obtain also $(X \times X)$-defined elements $\alpha(\pi_1)$ of $P(\introS(X))$ and $\introN_{X}(\pi_2)$ of $\Hom_C(\point_C, \introS(X))$. Combining these via the presheaf action of $P$, we get $\pullAlong{( \introN_{X}(\pi_2) )} ( \alpha(\pi_1) )$ as an $(X \times X)$-defined element of $P(\point_C)$. By the surjectivity presumption on $\quotient$, we find a preimage of this under the action of $\quotient : \Omega \to P(\point_C)$. We take this preimage to be our $\App : X \times X \to \Omega$. Thus, for any generalized elements $x_1, x_2$ of $X$ with the same domain, we have that $\quotient(\App(x_1, x_2)) = \pullAlong{( \introN_{X}(x_2) )} ( \alpha(x_1) )$.

We must now establish an appropriate surjectivity supposition on $\App$ for invoking \magicref{SelfRelatedPointTheorem}. 

Let an arbitrary $F : X \to \Omega$ be given. We then have that $\introS(F) : \introS(X) \to \introS(\Omega)$ in the global aspect of $C$. We can apply the action of $P$ along this morphism to $\loebNeg$ (a global element of $P(\introS(\Omega))$), thus obtaining a global element $\pullAlong{\introS(F)} \loebNeg$ of $P(\introS(X))$. By the surjectivity-like assumption on $\alpha$ we made, we now have a corresponding point $f$ of $X$, such that $\alpha(f) = \pullAlong{\introS(F)} \loebNeg$ (the right side here having been reinterpreted from a global element into a point).

It follows that for every point $x$ of $X$, we have that $\pullAlong{\introN_{X}(x)} \alpha(f) = \pullAlong{\introN_{X}(x)} \pullAlong{\introS(F)} \loebNeg$.

Note that by the definition of $\App$, we have that $\quotient(\App(f, x)) = \pullAlong{( \introN_{X}(x) )} \alpha(f)$.

Also note that by \magicref{SWithN}, we have that $\introS(F) \circ_C \introN_{X}(x) = \introN_{\Omega}(F(x))$. Thus, by the functoriality of $P$, we have that $\pullAlong{\introN_{X}(x)} \pullAlong{\introS(F)} \loebNeg =  \pullAlong{\introN_{\Omega}(F(x))} \loebNeg$.

Combining these last three paragraphs, we have that $\quotient(\App(f, x)) =  \pullAlong{\introN_{\Omega}(F(x))} \loebNeg$.

If we define the relation $R(\omega_1, \omega_2)$ as the equation $\quotient(\omega_1) = \pullAlong{\introN_{\Omega}(\omega_2)} \loebNeg$ accordingly, we have now established the surjectivity supposition required in order to invoke \magicref{SelfRelatedPointTheorem}. From this invocation, we get a point of $\Omega$ which is related by $R$ to itself, which is just what we desired, completing the proof.
\end{proof}
\begin{corollary}\label{PreIntrospDiagSpecialization}
In many cases we are interested in (though not all!), we furthermore take $P(\point_C)$ to be $T$-\repsmall/ and take $\Omega$ to be $P(\point_C)$, with $\quotient : \Omega \to P(\point_C)$ as the identity map between these. We then automatically have that the aspect of $\quotient$ at ${X \times X}$ is surjective as required.
\end{corollary}

\begin{theorem}\label{PreIntrospDiagFromIso}
Suppose given a locally introspective theory $\langle T, C, \introS, \introN \rangle$ and an object $P$ in the global aspect of $\Psh{C}$.

If there is any object $X$ of $T$ with an isomorphism from $X$ to $P(\introS(X))$, then, within the global aspect of $\Psh{C}$, for every $\loebNeg : \Box P \to P$, we obtain an $\omega : 1 \to P$, such that $\omega = g \circ \omega'$, where $\omega' = \Box_{\Psh{C}}(\omega) : 1 \to \Box P$. In other words, we obtain the instance of the \Loeb/ property constrained specifically to $P$.

We get the same result also if, within $\Glob{C}$, there is any object $Y$ along with an isomorphism from $Y$ to $\introS(P(Y))$.
\end{theorem}
\begin{proof}
Any isomorphism $\alpha : X \to P(\introS(X))$ (or even just a retraction) will automatically satisfy the surjectivity-like precondition allowing us to invoke \magicref{PreIntrospDiag} via \magicref{PreIntrospDiagSpecialization}, which takes $\Omega$ as $P(\point_C)$ and $\quotient$ as identity. Everything follows immediately from this, but has just been Yoneda-ized in its phrasing.

Specifically, keep in mind via the Yoneda lemma that the data of a map from $c \in C$ to $P \in \Psh{C}$ is the same as an element of $P(c)$. In this way, our $\loebNeg : \Box P \to P$ can be seen as indeed an element of $P(\Box P) = P(\introS(P(\point_C))) = P(\introS(\Omega))$, as required.

The invocation of \magicref{PreIntrospDiag} via \magicref{PreIntrospDiagSpecialization} will give us a global element $\omega$ of $\Omega = P(\point_C)$ such that $\omega = \pullAlong{\omega'}{\loebNeg}$, where $\omega' = \introN_{\Omega}(\omega)$ is a global element of $\Box P(\point_C)$. Again, by the Yoneda lemma, such an $\omega$ corresponds to a map from $1$ to $P$ in the global aspect of $\Psh{C}$, such an $\omega'$ corresponds to a map from $1$ to $\Box P$ (specifically, $\omega' = \Box \omega$, by \magicref{SMatchesN}), and our equation relating $\omega$ and $\omega'$ is that that $\omega = \loebNeg \circ \omega'$.

For the last remark about starting from a fixed point for $\introS(P(-))$ rather than a fixed point for $P(\introS(-))$, observe that if we have a $Y$ isomorphic to $\introS(P(Y))$, then by taking $X$ to be $P(Y)$, we obtain an $X$ isomorphic to $P(\introS(X))$.\footnote{This is a special case of the bijective correspondence between fixed points of cyclic rearrangements of compositions, which we also observed within the proof of \magicref{LoebTransfer}.}

\sTODOinline{That is, let $F$ and $G$ be arbitrary covariant or contravariant functors, not necessarily of the same variance as each other. Note that fixed points up to isomorphism of $F \circ G$ are in correspondence with fixed points up to isomorphism of $G \circ F$, with the functors $G$ and $F$ carrying out the two directions of the correspondence. (For that matter, we can also observe that values $X$ which retract onto $F(G(X))$ induce values $Y$ which retract onto $G(F(Y))$, although this is no longer a 1 : 1 correspondence). In this particular case, this means that fixed points up to isomorphism of the contravariant endofunctor $P(\introS(-))$ on $T$ are in correspondence with fixed points up to isomorphism of the contravariant endofunctor $\introS(P(-))$ on the global aspect of $C$. \TODOinline{Make something in the preliminaries about fixed points of compositions and cyclic change of composition, then cite it here and elsewhere in this chapter}}
\end{proof}

\sTODOinline{We note in passing that the above argument can be understood as working just the same in the context of a merely pre-introspective finite product theory. This just requires some care when interpreting the $\Box$ notation and discussing maps from objects of $C$ to objects of $\Psh{C}$, given the concerns from magicref{BoxNotationSmallnessConcerns}. Without local introspectiveness, $C$ is no longer a full subcategory of $\Psh{C}$, and thus, $\Box_{\Psh{C}}$ is no longer an endofunctor on $\Psh{C}$, but can still be understood as a functor from $\Psh{C}$ to $C$. Furthermore, in such a context, we can still make sense of maps from objects of $C$ to objects of $\Psh{C}$ in the manner of the Yoneda lemma, or by seeing both $C$ and $\Psh{C}$ as full subcategories of the wider category of arbitrary $(T, C)$-indexed sets with no representability conditions.

Also nothing above depends on having equalizers, so it would work in a "pre-introspective finite product theory", but I got rid of this concept from the document for now.

Indeed, I'm getting rid of this whole note from the document for now.}

\subsection{Bootstrapping to \Loeb/'s theorem for introspective theories}\label{IntrospDiagSection}
This last theorem gives us an instance of the \Loeb/ property, but comes with the precondition of a certain isomorphism.

Incredibly, we can bootstrap away this isomorphism precondition, in the context of an introspective theory. That is, in the context of an introspective theory, we can use one particular instance of \magicref{PreIntrospDiag} itself to provide the very isomorphisms necessary in order to then re-invoke \magicref{PreIntrospDiag} via \magicref{PreIntrospDiagFromIso}.

Our plan is to consider the $(T, C)$-indexed set $P$ such that $P(t, c)$ is the set of isomorphism classes of $C(t)/c$, with the action of $P$ on morphisms of $C$ being given by pullback (while the action of $P$ on morphisms of $T$ is given by the reindexing action of the $T$-indexed category $C$).\footnote{Note that this $(T, C)$-indexed set $P$ is NOT presumed to be $T$-representable! Indeed, we cannot generally hope for this, as we do not presume $T$ to have any regularity or exactness properties such that we could carry out internal to $T$ such quotienting constructions as would yield the object of isomorphism classes of $C$.}

\sTODOinline{Perhaps note the following nuances or perhaps not:

Note that this $P$ will not in general be $T$-\repsmall/, because $T$ does not in general have quotient objects. However, this will cause no difficulties.

Note also that this $P$ is well-defined even though $C$ is taken only as a category and not a strict category, as even though a category does not have a well-defined set of objects, it still has a well-defined set of isomorphism classes of objects.}

In more detail, for any fixed $t$ and any morphism $m : c_1 \to c_2$ of $C(t)$, the action $P(t, m) : P(t, c_2) \to P(t, c_1)$ is given by pullback in the lexcategory $C(t)$ along $m$; that is, this is given by $\pullAlong{m} : C(t)/c_2 \to C(t)/c_1$ considered as taking isomorphism classes of objects to isomorphism classes of objects\sTODOinline{Maybe move all this to the Preliminaries in the section on doubly-indexed sets}. Note that this reindexing along morphisms in $C$ is indeed strictly functorial, because we are working with isomorphism classes of objects rather than with objects simpliciter.

We now choose any internal category $C_{strict}$ in $T$ which presents $C$ (by definition, such an internal category exists in an introspective theory; there may be multiple non-isomorphic such internal categories presenting $C$, but any will do for our purposes) and we take $\Omega$ to be $\Ob(C_{strict})$, with $\quotient : \Omega \to P(\point_C)$ sending each object of each aspect of $C_{strict}$ to its isomorphism class within the corresponding aspect of $C$. Note that every component of $\quotient$ as a natural transformation between presheaves on $T$ is surjective (because $C$ is presented by $C_{strict}$, the isomorphism classes of $C$ and of $C_{strict}$ are the same, and there is clearly a surjection from the objects of $C_{strict}$ (at any aspect) onto the isomorphism classes of $C_{strict}$ (at the same aspect)). Thus in particular the component of $\quotient$ at the object $X \times X$ of $T$ is surjective. \sTODOinline{Mention something about how this is a well-defined map of indexed sets; that is, $\quotient$ interacts appropriately with pullback}. 

We take $X$ to be the subobject of $\Mor(C_{strict})$ comprising those morphisms whose codomain is $\introS(\Mor(C_{strict}))$. That is, the object given by the following equalizer diagram.

% https://q.uiver.app/?q=WzAsNCxbMCwwLCJYIl0sWzEsMCwiXFxNb3IoQ197c3RyaWN0fSkiXSxbMywwLCJcXE9iKENfe3N0cmljdH0pIl0sWzIsMSwiMSJdLFswLDEsImkiLDIseyJzdHlsZSI6eyJ0YWlsIjp7Im5hbWUiOiJtb25vIn19fV0sWzEsMiwiXFxjb2QiXSxbMSwzLCIhIiwyXSxbMywyLCJcXGludHJvUycoXFxNb3IoQ197c3RyaWN0fSkpIiwyXV0=
\[\begin{tikzcd}
	X & {\Mor(C_{strict})} && {\Ob(C_{strict})} \\
	&& 1
	\arrow["i"', tail, from=1-1, to=1-2]
	\arrow["\cod", from=1-2, to=1-4]
	\arrow["{!}"', from=1-2, to=2-3]
	\arrow["{\introS'(\Mor(C_{strict}))}"', from=2-3, to=1-4]
\end{tikzcd}\]

In the above diagram, we have labelled an arrow with the name $\introS'(\Mor(C_{strict}))$. By this we mean some arbitrary globally defined object of $C_{strict}$ which presents the globally defined object $\introS(\Mor(C_{strict}))$ of $C$. We pedantically caution that there may actually be multiple non-equal global elements of $\Ob(C_{strict})$ which present objects isomorphic to $\introS(\Mor(C_{strict}))$. But any arbitrary choice of some such element will be fine to use as the arrow in this diagram for our purposes.\footnote{Indeed, it is readily seen that even two non-equal such choices will still lead to isomorphic $X$es. Or more precisely, isomorphic results as an object of $T$, though not isomorphic as a subobject of $\Mor(C_{strict})$, as the specific choice of inclusion map $i : X \to \Mor(C_{strict})$ will vary. But again, any so-arising choice will be fine for our purposes.}

Note that, by virtue of being an equalizer, the inclusion map $i : X \to \Mor(C_{strict})$ in $T$ is monic, and thus (as $\introS$ is a lexfunctor) so also is $\introS(i) : \introS(X) \to \introS(\Mor(C_{strict}))$ in $C$. From this, we can define our $\alpha : X \to P(\introS(X))$ and establish its surjectivity condition. Specifically, observe that pullback along $\introS(i)$ gives us a functor $\pullAlong{\introS(i)} : C/\introS(\Mor(C_{strict})) \to C/\introS(X)$. If we focus on the action of $\pullAlong{\introS(i)}$ on objects, consider its input object as presented by an object of $C_{strict}/\introS'(\Mor(C_{strict}))$ (whose objects comprise $X$), and consider its output object modulo isomorphism, this yields $\pullAlong{\introS(i)} : X \to P(\introS(X))$, which we take as our definition of $\alpha$.

As for the surjectivity condition, let $F$ be an arbitrary global element of $P(\introS(X))$; that is, an arbitrary isomorphism class of objects of $C/\introS(X)$. The pushforward (i.e., composition) action of $\introS(i)$ gives us a functor from $C/\introS(X) \to C/\introS(\Mor(C_{strict}))$, taking $F$ to $\introS(i) \circ F$, an isomorphism class of objects of the global aspect of $C/\introS(\Mor(C_{strict}))$. This will be presented by at least one globally defined element $f$ of $X$ (keeping in mind the definition of $X$); there may be multiple non-equal such $f$ but any will do. Observe that $\alpha(f)$ is the isomorphism class of $C/\introS(X)$ corresponding to $\introS(i) \circ F$ pulled back along $\introS(i)$. This isomorphism class is the same as that of $F$ itself, because of the monicity of $\introS(i)$, like so:

% https://q.uiver.app/?q=WzAsNixbMSwxLCJcXGludHJvUyhYKSJdLFsxLDIsIlxcaW50cm9TKFxcTW9yKENfe3N0cmljdH0pKSJdLFsxLDAsIlxcYnVsbGV0Il0sWzAsMiwiXFxpbnRyb1MoWCkiXSxbMCwxLCJcXGludHJvUyhYKSJdLFswLDAsIlxcYnVsbGV0Il0sWzAsMSwiXFxpbnRyb1MoaSkiXSxbMiwwLCJGIl0sWzMsMSwiXFxpbnRyb1MoaSkiLDJdLFs0LDMsIlxcaWQiLDIseyJsZXZlbCI6Miwic3R5bGUiOnsiaGVhZCI6eyJuYW1lIjoibm9uZSJ9fX1dLFs0LDAsIlxcaWQiLDIseyJsZXZlbCI6Miwic3R5bGUiOnsiaGVhZCI6eyJuYW1lIjoibm9uZSJ9fX1dLFs0LDEsIiIsMix7InN0eWxlIjp7Im5hbWUiOiJjb3JuZXIifX1dLFs1LDQsIkYiLDJdLFs1LDIsIlxcaWQiLDAseyJsZXZlbCI6Miwic3R5bGUiOnsiaGVhZCI6eyJuYW1lIjoibm9uZSJ9fX1dLFs1LDAsIiIsMCx7InN0eWxlIjp7Im5hbWUiOiJjb3JuZXIifX1dXQ==
\[\begin{tikzcd}
	\bullet & \bullet \\
	{\introS(X)} & {\introS(X)} \\
	{\introS(X)} & {\introS(\Mor(C_{strict}))}
	\arrow["{\introS(i)}", from=2-2, to=3-2]
	\arrow["F", from=1-2, to=2-2]
	\arrow["{\introS(i)}"', from=3-1, to=3-2]
	\arrow["\id"', Rightarrow, no head, from=2-1, to=3-1]
	\arrow["\id"', Rightarrow, no head, from=2-1, to=2-2]
	\arrow["\lrcorner"{anchor=center, pos=0.125}, draw=none, from=2-1, to=3-2]
	\arrow["F"', from=1-1, to=2-1]
	\arrow["\id", Rightarrow, no head, from=1-1, to=1-2]
	\arrow["\lrcorner"{anchor=center, pos=0.125}, draw=none, from=1-1, to=2-2]
\end{tikzcd}\]

Thus, $\alpha(f) = F$ as an element of $P(\introS(X))$, establishing the required surjectivity condition on $\alpha$.

\bigskip
Thus, all presumptions are satisfied for us to be able to apply \magicref{PreIntrospDiag} with these definitions, for an arbitrary globally defined element $g$ of $P(\introS(\Omega))$.

In particular, let $G$ be an arbitrary globally defined object of $\Psh{C}$. (In fact, it suffices for $G$ merely to have reindexing along isomorphisms rather than arbitrary morphisms of $C$; that is, for $G$ to be an object of $\Psh{\core{C}}$, where $\core{C}$ is the subcategory of $C$ containing just its invertible morphisms.)

This will be presented by an object of $T/\Ob(C_{strict})$ (the map into $\Ob(C_{strict})$ whose fiber at any object $c_{strict}$ of $C_{strict}$ is the set $G(c)$, where $c$ is the object of $C$ presented by $c_{strict}$). By applying $\introS$ to this, we get a globally defined object of $C/\introS(\Ob(C_{strict}))$, which is to say, a global element of $P(\introS(\Omega))$. Take this to be our $\loebNeg$.

Invoking \magicref{PreIntrospDiag} (on the introspective theory $\langle T, C, \introS, \introN \rangle$, with all other inputs ($P$, $\Omega$, $\quotient$, $X$, $\alpha$, and $\loebNeg$) as described with the same name above), we now get a globally defined element $\omega$ of $\Omega = \Ob(C_{strict})$ such that $\quotient(\omega) = \pullAlong{\introN_{\Omega}(\omega)} \loebNeg$. This equation is saying precisely that $\omega$ presents an object $Y$ of $C$ such that $Y$ is isomorphic to $\introS(G(Y))$. \sTODOinline{Maybe more about how $\pullAlong{\introN_{\Omega}(\omega)} \loebNeg$ is $\introS(G(Y))$}

Thus, we have proven the following:
\begin{theorem}\label{IntrospTyConFixedPoints}
For any introspective theory $\langle T, C \rangle$, and any globally defined object $G$ of $\Psh{C}$, or even of $\Psh{\core{C}}$, there is some object $Y \in \Glob{C}$ along with an isomorphism from $Y$ to $\introS(G(Y))$.
\end{theorem}

Combining this with \magicref{PreIntrospDiagFromIso} to eliminate the latter's isomorphism precondition, we now reach the following conclusion:
\openNamed{theorem}{L\"ob's Theorem for Introspective Theories}\label{IntrospLoeb}
Suppose given an introspective theory $\langle T, C, \introS, \introN \rangle$.

Then, within $\Glob{\Psh{C}}$, for every object $P$ and morphism $\loebNeg : \Box P \to P$, we obtain an $\omega : 1 \to P$, such that $\loebNeg \circ (\Box \omega) = \omega$.

In other words, the global aspect of $\Box_{\Psh{C}}$ has the \Loeb/ property. Keeping in mind the equivalences of \magicref{LoebTransferIntrosp}, we may conclude that the global aspects of $\Box_{T/-}$, $\Box_C$, and $\Box_{\Psh{C}}$ all have the \Loeb/ property with uniqueness.
\closeNamed{theorem}

\begin{observation}
We can consider the particular case where $P$ is $C$-\repsmall/, just as $\Box P$ is. In other words, where $P(-) = \Hom_C(-, c)$ is the representable presheaf on $C$ represented by some object $c$ of $C$. All traditional accounts of \Loeb/'s theorem are along these lines. But note that we can also just as well consider this \magicref{IntrospLoeb} for non-representable presheaves $P$, a significant generalization of the traditional viewpoint.
\end{observation}

\begin{corollary}\label{IntrospLoebAtEachAspect}
For any introspective theory $\langle T, C \rangle$, every aspect of $\Box_T$, $\Box_C$, and $\Box_{\Psh{C}}$ has the \Loeb/ property with uniqueness.

In other words, every introspective theory has the \Loeb/ property, in the terminology of \magicref{LoebTransferIntrosp}.
\end{corollary}
\begin{proof}
By \magicref{SliceBoxIsAspectBox}, each aspect of any of these $\Box$ functors is the global aspect of the corresponding $\Box$ functor on the corresponding slice introspective theory. Thus, we simply invoke \magicref{IntrospLoeb} on this slice introspective theory.
\end{proof}

The above is our key result. The fact that the simple definition of introspective theories is enough to lead to their satisfying the \Loeb/ property with uniqueness motivates much of our interest in the concept of introspective theories.

\begin{observation}
The fixed points produced by \magicref{IntrospTyConFixedPoints} are furthermore unique up to canonical isomorphism, by combining \magicref{IntrospLoeb} with \magicref{InitialTerminalCoincidence}. \sTODOinline{Give more details}
\end{observation}

\subsection{The self-indexing cannot be \repsmall/, except trivially}\label{NoSmallSelfIndexing}
\sTODOinline{Perhaps move this into another section instead of giving it a dedicated section.}
We note an important corollary of the above:

\begin{theorem}\label{LocallyCartesianLoeb}
Let $T$ be any lexcategory, and equip it as an introspective theory $\langle T, C, \introF \rangle = \langle T, T/-, \id \rangle$ by taking $C$ to be $T$'s self-indexing and $\introF$ to be the identity (a la \magicref{TrivialPreIntrosp}). Recall from \magicref{LocallySmallSelfIndexing} that this will be locally introspective (that is, the self-indexing will be locally \repsmall/) precisely when $T$ is locally cartesian closed.

This will furthermore be introspective (that is, the self-indexing will be \repsmall/) only when $T$ is the trivial terminal category.
\end{theorem}
\begin{proof}
For a lexcategory $T$ equipped as a pre-introspective theory in this way, the operation $\Box_T$ acts as the identity.

And by \magicref{IntrospLoeb}, if $T$ is an introspective theory, then $\Box_T$ will have the \Loeb/ property with uniqueness.

But by \magicref{CCCLoebUniqueness}, the identity endofunctor on a cartesian closed lexcategory has the \Loeb/ property with uniqueness only when the category is the trivial terminal category.
\end{proof}
This \quote{no-go} result was demonstrated in \autocite{PittsTaylor1989} by an essentially identical argument to the argument we have given, when the abstractions in our argument are unwound to this special case.

But by generalizing to introspective theories, we are able to expand from this negative result (there are no nontrivial lexcategories whose self-indexing is \repsmall/) to a positive result (there are many nontrivial examples of introspective theories, which all end up satisfying the \Loeb/ property with uniqueness and all the further corollaries of this noted in \magicref{LoebPropertySection}).

\begin{observation}
From the above, we see that, though the \Loeb/ property holds for all introspective theories automatically, it does not hold automatically for merely locally introspective theories (as there are many locally cartesian closed categories which are nontrivial. Counterexamples could also be constructed from non-well-founded transitive relations using \magicref{KripkeLocallyIntrosp}.). However, we have also seen there are some natural examples of locally introspective but not fully introspective theories with the property that arbitrarily loose sub-introspections of them can be made into introspective theories, as in the relationship between our archetypal examples \magicref{KripkeLocallyIntrosp} and \magicref{KripkeIntrosp}, or the relationship between our archetypal examples \magicref{StepIndexingLocallyIntrosp} and \magicref{StepIndexingIntrosp}. Such locally introspective theories will thus inherit the \Loeb/ property from their sub-introspections.

\sTODOinline{Perhaps discuss the idea of a locally introspective theory where every object, or indeed every slice, is contained in some "full sub-introspective theory" of it, and how this inherits the Loeb property (at every aspect). This can then be related to our archetypal examples of locally introspective theories based on presheaf categories.}
\end{observation}

\subsection{As applied to our archetypal examples}
Here we discuss the application of \sTODOinline{\magicref{IntrospTyConFixedPoints} and }\magicref{IntrospLoeb} to our archetypal examples of introspective theories:

\subsubsection{ZF-Finite examples}\label{ZFFiniteLoebDiscussion}
Recall from \magicref{SigmaModelComplex} that we have a natural introspective theory $\langle \ZfinSigma, \InnerZfin \rangle$, where $\ZfinSigma$ is the lexcategory of $\Sigma_1$-definable hereditarily finite sets and $\Sigma_1$-definable functions between them up to provable equivalence in ZF-Finite, and $\InnerZfin$ is the lexcategory internal to $\ZfinSigma$ of arbitrary definable sets and arbitrary definable functions between them up to provable equivalence in ZF-Finite.

Recall from the discussion at \magicref{ZFFiniteModal} that the global aspect of $\InnerZfin$ can be identified with $\Zfin$ (the actual category of arbitrary definable sets and functions between them in ZF-Finite) and that the $\Box$ operator acts on this by sending each \quote{the object of $X$es} to \quote{the object of definitions of $X$es within ZF-Finite}. In the particular case where the object in question is subterminal (thus representing a proposition), this amounts to the traditional provability operator sending the proposition $X$ to the proposition \quote{There is a proof in ZF-Finite of X}.

Thus, as applied to these subterminal objects, our \Loeb/ property with uniqueness for this introspective theory is indeed the namesake \Loeb/ property of traditional logic: It tells us that if there is a proof that the provability of $X$ entails $X$, then there is in fact an unconditional proof of $X$. \Goedel/'s second incompleteness theorem follows as the special case of this where $X$ is a manifest falsehood, and \Goedel/'s first incompleteness theorem then readily follows from the second incompleteness theorem.

But we may consider non-subterminal objects as well, and here our \Loeb/ property with uniqueness gives us a form of guarded recursion in the context of such logical theories as ZF-Finite. Specifically, for any definable function from definitions of $X$es to actual $X$es, there is a unique (up to provable equivalence) definition of an $X$ which is provably equivalent to the given function applied to its own definition.\footnote{Here, all mentions of definability and provability are with respect to the particular theory ZF-Finite, though analogous constructions of introspective theories can be carried out for other logical theories as well, such as any computably enumerable extension of ZF-Finite, as we later discuss at \magicref{NothingSpecialToZFFinite}.}

We are not aware of guarded recursion having been strongly investigated in this context before. We aspire to explore working with this form of guarded recursion further in future work. For now we simply observe it as a vast generalization of the traditional purely propositional interpretation of \Loeb/'s theorem in logic.

\subsubsection{Kripke frame example}
\newcommand{\PshUnderQInf}{\mathrm{Psh}'(Q)}

Recall the introspective theory $\langle \PshUnderQInf, C' \rangle$ from \magicref{KripkeIntrosp}, constructed from a well-founded transitive relation $<$ on a set $P$, with $Q$ being $P$ augmented with a new maximum element $\infty$ and construed as a preorder category using the $<$ relation. The $\PshUnderQInf$ here is a full sublexcategory of $\Psh{Q}$, defined by certain cardinality constraints, but these cardinality constraints can be taken to be arbitrarily loose such that any \setsmall/ number of particular desired objects of $\Psh{Q}$ can be found within $\PshUnderQInf$.

The global aspect of $C'$ here is a certain full sublexcategory of $\Set^{|P|}$ (again defined by cardinality constraints, which may again be taken to be arbitrarily loose such that any \setsmall/ number of particular desired objects can be found within this). Recall from the discussion at \magicref{KripkeFrameModal} that the $\Box$ operator acts on this such that $\Box F(x)$ is the product of $F(y)$ over all $y < x$, where $F \in \Set^{|P|}$ and $x, y \in P$. For subterminal $F$ acting as propositions, this corresponds to the traditional interpretation of the $\Box$ operator in a Kripke frame, such that $\Box F$ is true at a world just in case $F$ is true at all lower worlds.

In this context, the \Loeb/ property with uniqueness which we are given by \magicref{IntrospLoeb} tells us that we may define functions by transfinite recursion: Given at each $x \in P$ a function $g$ from $\prod_{y < x} F(y)$ to $F(x)$, we obtain a uniquely determined function $G$ whose domain is $P$ such that each $G(x)$ is given by $g$ applied to the values of $G$ at $y < x$.

In the particular case where $F$ is subterminal representing a proposition (that is, an arbitrary subset of $|P|$), this amounts to the principle of transfinite induction or \quote{strong induction}: It tells us that a proposition holds of all of $P$ so long as it holds of any particular $x \in P$ once it holds of all $y < x$.

Of course, these principles of transfinite recursion/induction over well-founded transitive relations are well-known and easy to establish directly, without all the machinery of introspective theories. (The induction principle here is after all the very defining characteristic of well-foundedness.) But it is remarkable to observe how these phenomena are in this way unified with the phenomena of \Loeb/'s theorem in traditional logic (as discussed at \magicref{ZFFiniteLoebDiscussion}), not just in the form of the \Loeb/ property result but in the particular derivation of it as well.

\subsubsection{Step-indexing example}
The application of our \Loeb/'s theorem with uniqueness results to the introspective theory \magicref{StepIndexingIntrosp} corresponding to step-indexing in the topos of trees is similar to the one just discussed. Recall from the discussion at \magicref{StepIndexingModal} that we here have a $\Box$ operator on (an arbitrarily loose full sublexcategory of) $\Psh{\omega}$, where $\omega$ is the poset of natural numbers, such that $\Box F(0) = 1$ and $\Box F(n + 1) = n$, for $n \in \omega$ and $F \in \Psh{\omega}$.

Our \Loeb/ property with uniqueness thus tells us that we may define functions on the natural numbers by the most familiar kind of recursion: Given any specified value at $0$, and any specified way to transform a value at $n$ into a value at $n + 1$ for each $n \in \omega$, there is a unique function on the natural numbers taking on the specified value at $0$ and whose value at each $n + 1$ is derived from its value at $n$ in the specified way.

In the particular special case where we are dealing with subterminal objects of $\Psh{\omega}$, these amount to downwards closed subsets of $\omega$, and the above specializes to the principle of ordinary induction for these: Given a downwards closed subset of $\omega$, if it contains $0$ and is closed under successor, then it contains all of $\omega$.

Again, all of this is quite familiar and easy to demonstrate directly without any invocation of the machinery of introspective theories (these amount to the characteristic properties of the natural numbers as a natural numbers object within $\Set$). But again, it is remarkable that we can in this way see these as strongly unified with the analogous properties and the derivation of those properties for our other archetypal examples, including the case of \magicref{ZFFiniteLoebDiscussion} which has no direct relationship to presheaves over a well-founded structure.

\subsection{Relating variations on Lawvere's fixed point theorem}\label{LawvereFPTReduxSection}
Although not important for our main narrative, we note here some further comments on the relation of Lawvere's fixed point theorem to generalizations of ours or others.

First, we observe that \magicref{LawveresFixedPointTheorem} can be straightforwardly re-obtained as a special case of our \magicref{PreIntrospDiag}.
\begin{proof}
First, we handle the special case of \magicref{LawveresFixedPointTheorem} where $T$ has finite limits and $\Omega$ is an object of $T$.

This is a special case of \magicref{PreIntrospDiag} where we take the pre-introspective theory $\langle T, C, \introF \rangle$ to be the trivial one where $C$ is the self-indexing $T/-$ and $\introF$ is the identity.

Furthermore, $P$ is taken to be the $(T, C)$-indexed set represented by $\Omega$; that is, such that $P(t, c) = \Hom_T(t \times c, \Omega)$. Note that $P(\introS(t))$ for objects $t$ of $T$ is therefore the $T$-indexed set $\Omega^t$. In particular, $P(1)$ is thus isomorphic to $\Omega$. As in \magicref{PreIntrospDiagSpecialization}, we can take $\quotient$ to be this isomorphism (one can think of it as an identity if one likes), and this will then automatically be surjective on its $X \times X$ aspect.

We take $\alpha : X \to P(\introS(X)) = \Omega^X$ to be given by the map $\App' : X \to \Omega^X$ presumed in \magicref{LawveresFixedPointTheorem}. The surjectivity presumption from \magicref{LawveresFixedPointTheorem} then becomes the surjectivity presumption of \magicref{PreIntrospDiag}. 

And to give a $g$ in the global aspect of $P(\introS(\Omega)) = \Omega^\Omega$ is precisely the data presumed by the name $g$ in \magicref{LawveresFixedPointTheorem}.

This matches all the presumptions of \magicref{PreIntrospDiag} up with corresponding presumptions from \magicref{LawveresFixedPointTheorem}, and the conclusion we then obtain from \magicref{PreIntrospDiag} is readily seen to be the same as the conclusion from \magicref{LawveresFixedPointTheorem}.

The above shows how to obtain \magicref{LawveresFixedPointTheorem} as an instance of \magicref{PreIntrospDiag} when $T$ is a lexcategory and $\Omega$ is an object of $T$. We then obtain \magicref{LawveresFixedPointTheorem} in full (that is, for arbitrary categories $T$ and $T$-indexed sets $\Omega$) from this special case, by first replacing $T$ with $\Psh{T}$, as noted in our discussion following our presentation of \magicref{LawveresFixedPointTheorem}.
\end{proof}

We also note in passing that another interesting generalization of \magicref{LawveresFixedPointTheorem} was recently remarked upon in \autocite{roberts2021substructural}. The following (or rather, its contrapositive) was given as Theorem 11 there. We shall present our own proof.

\openNamed{theorem}{Magmoidal Fixed Point Theorem}\label{MagmoidalFixedPointTheorem}
Let $T$ be an arbitrary category with objects $\point$ and $\Omega$, and let $B : T \times T \to T$ be a bifunctor on $T$ such that we have a transformation $\delta_t : t \to B(t, t)$ natural in $t$ from $T$. As ever, use \quote{point of} to mean \quote{element of the $\point$-aspect of}.

Suppose given an object $X$ of $T$ and an $\alpha : B(X, X) \to \Omega$ with the pointwise surjectivity property that for every $F : X \to \Omega$, there is a point $f$ of $X$, such that for every point $x$ of $X$, we have that the following diagram commutes:

% https://q.uiver.app/?q=WzAsNSxbMCwwLCJcXHBvaW50Il0sWzEsMCwiQihcXHBvaW50LCBcXHBvaW50KSJdLFsyLDAsIkIoWCwgWCkiXSxbMywwLCJcXE9tZWdhIl0sWzEsMSwiWCJdLFswLDEsIlxcZGVsdGFfe1xccG9pbnR9Il0sWzEsMiwiQihmLCB4KSJdLFsyLDMsIlxcYWxwaGEiXSxbMCw0LCJ4IiwyXSxbNCwzLCJGIiwyXV0=
\[\begin{tikzcd}
	\point & {B(\point, \point)} & {B(X, X)} & \Omega \\
	& X
	\arrow["{\delta_{\point}}", from=1-1, to=1-2]
	\arrow["{B(f, x)}", from=1-2, to=1-3]
	\arrow["\alpha", from=1-3, to=1-4]
	\arrow["x"', from=1-1, to=2-2]
	\arrow["F"', from=2-2, to=1-4]
\end{tikzcd}\]

Then for every $g : \Omega \to \Omega$, there is a point $\omega$ of $\Omega$ such that $\omega = g(\omega)$. That is to say, a fixed point of $g$.
\closeNamed{theorem}
\begin{proof}
Take $\App : X \times X \to \Omega$ to be defined like so: For each object $t$ of $T$, we define $\App_t : \Hom(t, X) \times \Hom(t, X) \to \Hom(t, \Omega)$ by giving $\App_t(m, n)$ as the following composition:

% https://q.uiver.app/?q=WzAsNCxbMSwwLCJCKHQsIHQpIl0sWzAsMCwidCJdLFszLDAsIkIoWCwgWCkiXSxbNCwwLCJcXE9tZWdhIl0sWzEsMCwiXFxkZWx0YV97dH0iXSxbMCwyLCJCKG0sIG4pIl0sWzIsMywiXFxhbHBoYSJdXQ==
\[\begin{tikzcd}
	t & {B(t, t)} && {B(X, X)} & \Omega
	\arrow["{\delta_{t}}", from=1-1, to=1-2]
	\arrow["{B(m, n)}", from=1-2, to=1-4]
	\arrow["\alpha", from=1-4, to=1-5]
\end{tikzcd}\]

That this definition of $\App_t$ is natural in $t$ follows from the naturality of $\delta$ and the functoriality of $B$. Specifically, naturality with respect to $h: s \to t$ is seen as follows:

% https://q.uiver.app/?q=WzAsNixbMSwwLCJCKHQsIHQpIl0sWzAsMCwidCJdLFszLDAsIkIoWCwgWCkiXSxbNCwwLCJcXE9tZWdhIl0sWzAsMSwicyJdLFsxLDEsIkIocywgcykiXSxbMSwwLCJcXGRlbHRhX3t0fSJdLFswLDIsIkIobSwgbikiXSxbMiwzLCJcXGFscGhhIl0sWzQsNSwiXFxkZWx0YV9zIiwyXSxbNCwxLCJoIl0sWzUsMCwiQihoLCBoKSIsMV0sWzUsMiwiQihtIGgsIG4gaCkiLDJdXQ==
\[\begin{tikzcd}
	t & {B(t, t)} && {B(X, X)} & \Omega \\
	s & {B(s, s)}
	\arrow["{\delta_{t}}", from=1-1, to=1-2]
	\arrow["{B(m, n)}", from=1-2, to=1-4]
	\arrow["\alpha", from=1-4, to=1-5]
	\arrow["{\delta_s}"', from=2-1, to=2-2]
	\arrow["h", from=2-1, to=1-1]
	\arrow["{B(h, h)}"{description}, from=2-2, to=1-2]
	\arrow["{B(m h, n h)}"', from=2-2, to=1-4]
\end{tikzcd}\]

The desired result now follows by \magicref{LawveresFixedPointTheorem}.
\end{proof}
\magicref{LawveresFixedPointTheorem} is of course the special case of \magicref{MagmoidalFixedPointTheorem} where $B$ is the familiar cartesian product and $\delta$ is the familiar diagonal transformation. Thus, in \autocite{roberts2021substructural}, \magicref{MagmoidalFixedPointTheorem} is considered as a generalization of Lawvere's fixed point theorem. But as we've just seen, \magicref{MagmoidalFixedPointTheorem} is also a special case of Lawvere's fixed point theorem, appropriately construed (as in our formulation of \magicref{LawveresFixedPointTheorem} which removes the $\point = 1$ constraint), despite the seeming mismatch between general bifunctors and specifically cartesian products. As noted before, there is no need for $X \times X$ to be $T$-\repsmall/, and if such closure of our underlying category is insisted upon, we can just as well always pass to $\Psh{T}$ first.

\sTODOinline{Remarks on Cantor's theorem, Liar's paradox, and Y combinator as examples of Lawvere's fixed point theorem. Cantor's theorem is a contrapositive statement using surjection. Liar's paradox is a contrapositive statement using a retraction/isomorphism. Y combinator is a positive statement using a retraction, and also involves passing to a slice category. Note that the reason we presume in Cantor's theorem that negation on $\Omega$ has no fixed points is because of another instance of Lawvere's fixed point theorem, via Liar's paradox!}

\sTODOinline{Point out the error in Yonofsky's discussion of Kleene's recursion theorem and how our more general related-point formulation allows us to correct this.}

\fileend

\filestart

\section{Geminal categories: The free introspective theory}\label{GeminalChapter}
\subsection{Preview}
In this chapter, we build the machinery to give an explicit yet tractably compact description of the initial introspective theory (which we call the theory of \quote{geminal categories}). This is the key result of this chapter.

We also show the remarkable result that any strict introspective theory can itself be equipped in a natural way as a model of this initial introspective theory; that is, any strict introspective theory can be seen as a geminal category.

(This last statement is easy to misinterpret, so let me be a bit more clear as to what I mean by this. I do not mean the trivial statement that every introspective theory extends the initial introspective theory. Rather, I mean that the theory of strict introspective theories extends the initial introspective theory (even though the theory of strict introspective theories is not itself an introspective theory).)

We will also discuss a partial converse of sorts, a way to extract an introspective theory from a geminal category, with the extracted introspective theory having a certain terminality property (that is, we construct a sort of co-free introspective theory induced by the given geminal category).

This chapter requires some preliminary concepts to be established in \magicref{MultiplyInternal} and \magicref{StrictIntrospSection}. The basic definitions concerning geminal categories are then given in \magicref{GeminalFirstDefnSection} through \magicref{GeminalSecondDefnSection}. After all this machinery has been built, the key result that the theory of geminal categories is in fact the initial introspective theory is ultimately demonstrated in \magicref{InitialIntrospectiveTheorySection}. We then discuss co-free constructions in \magicref{CofreeGeminalSection}.

\subsection{Multiply internal structures}\label{MultiplyInternal}
Before we get to the main material of this chapter, it will be helpful to introduce the concept of \quote{multiply internal} structures, which are used heavily throughout this chapter.

First, a small remark on notation: Recall that if we have a lexfunctor $F : C \to D$ and a structure $S$ internal to $C$, then we obtain a structure $F(S)$ of the same sort internal to $D$. Often, we shall write $F[S]$ for this instead of $F(S)$, to emphasize this particular operation as visually distinct from all the other ways in which parentheses can be used.

\bigskip
\sTODOinline{Relate the following to the $\cartwith{\theoryT}$ operation. An (n+1)-tuply internal model of $\theoryT$ is an internal model of $\cartwith{}^n T$.}

\begin{definition}
Let $C_0$ be a lexcategory, and let $C_1$ be the global aspect of a lexcategory internal to $C_0$. Now suppose given some structure $S$ internal to $C_1$. We may say that this structure $S$ is \defined{doubly internal} to $C_0$.

We may iterate this process. Suppose now that $C_2$ is the global aspect of some lexcategory internal to $C_1$, which in turn remains the global aspect of some lexcategory internal to $C_0$. We can now speak of structures internal to $C_2$ as being \defined{triply internal} to $C_0$.

And in general, given a sequence $C_0, C_1, C_2, \ldots, C_n$ where each $C_{i + 1}$ is the global aspect of a lexcategory internal to $C_i$, we may speak of structures internal to $C_n$ as being $(n + 1)$\definedManualIndexSort{-tuply internal}{tuply internal} to $C_0$ (and in the same way $n$-tuply internal to $C_1$, $(n - 1)$-tuply internal to $C_2$, and so on). That is, we recursively define an $(n + 1)$-tuply internal structure as a structure internal to the global aspect of an $n$-tuply internal lexcategory, with the base case being that the only $0$-tuply internal lexcategory of some $C$ is $C$ itself.

(Multiply internal structures can equivalently be thought of as multiply indexed structures (in the sense of \magicref{PreliminariesMultipleIndexing}) satisfying suitable \repsmallness/ conditions, but they are probably more easily understood in the presentation just given.)
\end{definition}

\begin{definition}
Observe that whenever $C$ is a lexcategory and $D$ is a $C$-indexed locally \repsmall/ lexcategory, the global sections functor $\Hom_D(1, -)$ can be seen as an indexed lexfunctor from $D$ to the self-indexing $C/-$; in particular, the global aspect of this lets us see $\Hom_D(1, -)$ as a lexfunctor from the global aspect of $D$ to $C$ itself. Let us write $\Gamma_D : \Glob{D} \to C$ to refer to this last lexfunctor, or drop the subscript and write simply $\Gamma$ where there is no need to disambiguate which $D$ we are referencing. (In particular, when writing $\Gamma[S]$ with no subscript on the $\Gamma$, we always mean $\Gamma_X[S]$ where $S$ is singly internal to $X$, though $X$ may in turn be internal or multiply internal to some other category.)\sTODOinline{If we never use unsubscripted $\Gamma$ anymore, we should get rid of these mentions of it.}
\end{definition}

Thus, if $S$ is some structure internal to the global aspect of $D$, we find that $\Gamma_D[S]$ is a structure of the same sort internal to $C$. In this way, any doubly-internal structure $S$ yields a singly-internal structure $\Gamma[S]$, and more generally, any $(n + 1)$-tuply internal structure $S$ yields an $n$-tuply internal structure $\Gamma[S]$.

Note that any lexcategory $C$ can also be thought of as a lexcategory internal to $\Set$, and thus $\Gamma_C$ in this instance is the same as $\Glob{-} : C \to \Set$. In this case, we may write $\LabeledGlob{C}$ for this map, to emphasize that we are specifically dealing with a global sections lexfunctor whose domain is $C$ and whose codomain is $\Set$.

\begin{definition}\label{InducedHomoDefn}
Recall \magicref{TermModelIsInitialForLex}, which tells us that, for any lexcategory $B$, the global sections functor $\LabeledGlob{B}$ is initial among all lexfunctors from $B$ to $\Set$. Thus, for any lexfunctor $F : B \to C$, we obtain a unique natural transformation as in the following diagram:

% https://q.uiver.app/#q=WzAsMyxbMCwwLCJCIl0sWzIsMCwiXFxTZXQiXSxbMSwxLCJDIl0sWzAsMSwiXFxMYWJlbGVkR2xvYntCfSJdLFswLDIsIkYiLDJdLFsyLDEsIlxcTGFiZWxlZEdsb2J7Q30iLDJdLFszLDIsIiEiLDAseyJzaG9ydGVuIjp7InNvdXJjZSI6MjAsInRhcmdldCI6MjB9fV1d
\[\begin{tikzcd}
	B && \Set \\
	& C
	\arrow[""{name=0, anchor=center, inner sep=0}, "{\LabeledGlob{B}}", from=1-1, to=1-3]
	\arrow["F"', from=1-1, to=2-2]
	\arrow["{\LabeledGlob{C}}"', from=2-2, to=1-3]
	\arrow["{!}", shorten <=3pt, shorten >=3pt, Rightarrow, from=0, to=2-2]
\end{tikzcd}\]

In this way, for any $B$-internal structure $S$, we obtain a homomorphism from $\LabeledGlob{B}(S)$ to $\LabeledGlob{C}(F[S])$. We refer to this homomorphism as $\InducedHomo{F}{S}$.

This process can be carried out in the internal logic of a lexcategory as well. That is, if $F : B \to C$ is an internal lexfunctor between $V$-internal lexcategories, and $S$ is some structure internal to the global aspect of $B$ (thus doubly internal to $V$), we get a $V$-internal homomorphism $\InducedHomo{F}{S} : \Gamma_B[S] \to \Gamma_C[F(S)]$ in the same way.
\end{definition}

\begin{observation}\label{GlobOfGlob}
If $C$ is a lexcategory and $B$ is the global aspect of some $C$-indexed locally representable lexcategory $B'$, then $\LabeledGlob{B}(-) = \Hom_{\LabeledGlob{C}(B')}(1, -)$ $ = \Hom_{C}(1, \Hom_{B'}(1, -)) $ $ = \LabeledGlob{C}(\Gamma_{B'}(-))$. Thus, $\LabeledGlob{B}$ and $\LabeledGlob{C} \circ \Gamma_{B'}$ are isomorphic. As the former is initial among lexfunctors from $B$ to $\Set$, so is the latter, and thus in this case the natural transformation described in \magicref{InducedHomoDefn} becomes an isomorphism:

% https://q.uiver.app/#q=WzAsMyxbMCwwLCJCID0gXFxHbG9ie0InfSJdLFsyLDAsIlxcU2V0Il0sWzEsMSwiQyJdLFswLDEsIlxcTGFiZWxlZEdsb2J7Qn0iXSxbMCwyLCJcXEdhbW1hX3tCJ30iLDJdLFsyLDEsIlxcTGFiZWxlZEdsb2J7Q30iLDJdLFszLDIsIiEiLDAseyJzaG9ydGVuIjp7InNvdXJjZSI6MjAsInRhcmdldCI6MjB9fV1d
\[\begin{tikzcd}
	{B = \Glob{B'}} && \Set \\
	& C
	\arrow[""{name=0, anchor=center, inner sep=0}, "{\LabeledGlob{B}}", from=1-1, to=1-3]
	\arrow["{\Gamma_{B'}}"', from=1-1, to=2-2]
	\arrow["{\LabeledGlob{C}}"', from=2-2, to=1-3]
	\arrow["{!}", shorten <=3pt, shorten >=3pt, Rightarrow, from=0, to=2-2]
\end{tikzcd}\]

That is to say, $\InducedHomo{\Gamma_{B'}}{S} : \Glob{S} \to \Glob{\Gamma_{B'}[S]}$ is always an isomorphism.
\end{observation}

\begin{lemma}\label{InducedGlobalCommute}
If $F : C \to D$ is a strict lexfunctor, and $Q$ is a $C$-internal lexcategory, then $F \circ \Gamma_Q = \Gamma_{F(Q)} \circ \InducedHomo{F}{Q}$. That is to say, the following outer diagram commutes, as evidenced by the inner chase of an arbitrary datum $m$ in $\Glob{Q}$:

% https://q.uiver.app/#q=WzAsOCxbMCwwLCJcXEdsb2J7UX0iXSxbMywwLCJcXEdsb2J7RihRKX0iXSxbMCwzLCJDIl0sWzMsMywiRCJdLFsxLDEsIm0iXSxbMiwxLCJGKG0pIl0sWzEsMiwiXFxIb21fUSgxLCBtKSJdLFsyLDIsIkYoXFxIb21fUSgxLCBtKSkgPSBcXEhvbV97RihRKX0oMSwgRihtKSkiXSxbMCwxLCJcXEluZHVjZWRIb21ve0Z9e1F9Il0sWzAsMiwiXFxHYW1tYV9RIiwyXSxbMSwzLCJcXEdhbW1hX3tGKFEpfSJdLFs0LDYsIiIsMix7InN0eWxlIjp7InRhaWwiOnsibmFtZSI6Im1hcHMgdG8ifX19XSxbNCw1LCIiLDAseyJzdHlsZSI6eyJ0YWlsIjp7Im5hbWUiOiJtYXBzIHRvIn19fV0sWzUsNywiIiwwLHsic3R5bGUiOnsidGFpbCI6eyJuYW1lIjoibWFwcyB0byJ9fX1dLFs2LDcsIiIsMix7InN0eWxlIjp7InRhaWwiOnsibmFtZSI6Im1hcHMgdG8ifX19XSxbMiwzLCJGIiwyXV0=
\[\begin{tikzcd}
	{\Glob{Q}} &&& {\Glob{F(Q)}} \\
	& m & {F(m)} \\
	& {\Hom_Q(1, m)} & {F(\Hom_Q(1, m)) = \Hom_{F(Q)}(1, F(m))} \\
	C &&& D
	\arrow["{\InducedHomo{F}{Q}}", from=1-1, to=1-4]
	\arrow["{\Gamma_Q}"', from=1-1, to=4-1]
	\arrow["{\Gamma_{F(Q)}}", from=1-4, to=4-4]
	\arrow[maps to, from=2-2, to=3-2]
	\arrow[maps to, from=2-2, to=2-3]
	\arrow[maps to, from=2-3, to=3-3]
	\arrow[maps to, from=3-2, to=3-3]
	\arrow["F"', from=4-1, to=4-4]
\end{tikzcd}\]
\end{lemma}

\begin{definition}\label{TransferNDefn}
Note that any structure $S$ which is $n$-tuply internal to a lexcategory $C$ (for $n \geq 1$) is ultimately described by some kind of diagram within $C$, and thus taken by a lexfunctor $F : C \to D$ to a structure of the same sort $n$-tuply internal to $D$ as well. It is natural to refer to this as $F[S]$ in the same way as for singly internal $S$.

This operation $\TransferN{n}{F}{S}$ for multiply internal $S$ can be inductively understood like so: The base case is when $S$ is singly internal to the domain of $F$, in which case $\TransferN{1}{F}{S}$ is just $F(S)$ in the ordinary way. On the other hand, if $S$ is $n$-tuply internal to $\dom(F)$ for $n \geq 2$, then there is some $\dom(F)$-internal lexcategory $B'$ such that $S$ is $(n - 1)$-tuply internal to $\Glob{B'}$. In this case, we have also the lexfunctor $\InducedHomo{F}{B'} : \Glob{B'} \to \Glob{F[B']}$, and thus we can understand $\TransferN{n}{F}{S}$ as meaning $\TransferN{n - 1}{\InducedHomo{F}{B'}}{S}$, reducing us from the $n$-tuply internal case to the $(n - 1)$-tuply internal case.
\end{definition}

\subsection{Strict introspective theories}\label{StrictIntrospSection}
It will be technically convenient for us to work in this chapter with a slightly less \quote{presentation-free} variant of our notion of introspective theories.

\begin{definition}\label{StrictIntrospDefn}
A \defined{strict introspective theory} is a strict lexcategory $T$, a lexcategory $C$ internal to $T$, a strict lexfunctor $\introS$ from $T$ to the global aspect of $C$, and a natural transformation $\introN$ from $\id_T$ to $\Hom_C(1, \introS(-))$.
\end{definition}

As usual, to name a strict introspective theory, we can enumerate the entire ordered tuple $\langle T, C, \introS, \introN \rangle$, or sometimes we just note $\langle T, C \rangle$ or $T$ explicitly and leave the rest implicit.

The definition of a strict introspective theory differs from the definition of an ordinary introspective theory (\magicref{DefnIntrospSN}) in the following ways: $T$ is made strict (thus, its internal structures can be considered up to equality instead of mere isomorphism), we demand the selection of $C$ as a particular $T$-internal lexcategory up to equality (instead of simply up to presenting equivalent indexed categories), and we take $\introS$ as a strict lexfunctor (thus, $\introS$ preserves chosen basis limits on-the-nose).

Clearly, any strict introspective theory presents some introspective theory. Conversely, we have the following:

\begin{theorem}\label{StrictifyingIntrosp}
Any introspective theory $\langle T, C, \introS, \introN \rangle$ is presented by some strict introspective theory.
\end{theorem}
\begin{proof}
Suppose given an introspective theory $\langle T, C, \introS, \introN \rangle$. By definition of the \repsmallness/ of $C$, we can choose some lexcategory $C_{int}$ internal to $T$ which presents the $T$-indexed category $C$. (That is, even though $C$ itself is only specified up to equivalence of indexed categories, we can choose a specific presentation of it by a \repsmall/ indexed strict category $C_{int}$ which is specified  more fine-grainedly up to isomorphism of indexed strict categories.)

Now using \magicref{FreeStrictifyLexcategory}, let $T_{strict}$ be some strict lexcategory which presents $T$ and which has the freeness property that any lexfunctor from $T$ to a strict lexcategory $L$ is presented by some strict lexfunctor from $T_{strict}$ to $L$. Because $T_{strict}$ presents $T$, we can choose some specific internal lexcategory $C_{strict}$ in $T_{strict}$ (this $C_{strict}$ being specified up to equality!) which presents $C_{int}$. Because $C_{strict}$ presents $C_{int}$ which in turn presents $C$, $\introS$ can be viewed as a (non-strict) lexfunctor from $T$ to the global aspect of $C_{strict}$. Now using the freeness property of $T_{strict}$, we obtain a strict lexfunctor $\introS_{strict}$ from $T_{strict}$ to the global aspect of $C_{strict}$, such that $\introS_{strict}$ presents $\introS$.

Finally, we deal with $\introN$. Natural transformations are essentially unaffected by strictness considerations. That is, given parallel strict functors $A_{strict}$ and $B_{strict}$, natural transformations between these are in bijection with natural transformations between the non-strict functors these present. So our original $\introN$ corresponds to a unique natural transformation between the identity on $T_{strict}$ and $\Hom_{C_{strict}}(1, \introS_{strict}(-))$.

Thus, we have obtained a strict introspective theory $\langle T_{strict}, C_{strict}, \introS_{strict}, \introN \rangle$ presenting the introspective theory $\langle T, C, \introS, \introN \rangle$.
\end{proof}

Strict introspective theories are slightly more convenient than introspective theories for phrasing the results of this chapter, because strict introspective theories are themselves an essentially algebraic notion. That is, there is an essentially algebraic theory such that the models of this theory are the strict introspective theories. (This is in precisely the same way that the theory of strict categories is essentially algebraic, while the theory of categories construed up to equivalence is not quite essentially algebraic.)

As with any essentially algebraic theory, we get automatically a corresponding notion of homomorphism.

\begin{definition}\label{StrictIntrospHomoDefn}
A \defined{homomorphism} between strict introspective theories $\langle T_1, C_1, \introS, \introN \rangle$ and $\langle T_2, C_2, \introS, \introN \rangle$ is a strict lexfunctor $H : T_1 \to T_2$ such that $H[C_1] = C_2$, and $\InducedHomo{H}{C_1} \circ \introS = \introS \circ H$, and $H[\introN_t] = \introN_{H(t)}$ for each object $t$ of $T_1$.

The condition relating $H$ to $\introS$ is illustrated like so:

% https://q.uiver.app/#q=WzAsNCxbMCwwLCJUXzEiXSxbMiwwLCJUXzIiXSxbMCwxLCJcXEdsb2J7Q18xfSJdLFsyLDEsIlxcR2xvYntIW0NfMV19ID0gXFxHbG9ie0NfMn0iXSxbMCwxLCJIIl0sWzAsMiwiXFxpbnRyb1MiLDJdLFsyLDMsIlxcSW5kdWNlZEhvbW97SH17Q18xfSIsMl0sWzEsMywiXFxpbnRyb1MiXV0=
\[\begin{tikzcd}
	{T_1} && {T_2} \\
	{\Glob{C_1}} && {\Glob{H[C_1]} = \Glob{C_2}}
	\arrow["H", from=1-1, to=1-3]
	\arrow["\introS"', from=1-1, to=2-1]
	\arrow["{\InducedHomo{H}{C_1}}"', from=2-1, to=2-3]
	\arrow["\introS", from=1-3, to=2-3]
\end{tikzcd}\]

The condition relating $H$ to $\introN$ is that the following two natural transformations are equal:

% https://q.uiver.app/#q=WzAsOCxbMiwwLCJUXzEiXSxbMywwLCJUXzIiXSxbMCwwLCJUXzEiXSxbMSwxLCJcXEdsb2J7Q18xfSJdLFswLDIsIlRfMSJdLFsxLDIsIlRfMiJdLFszLDIsIlRfMiJdLFsyLDMsIlxcR2xvYntDXzJ9Il0sWzAsMSwiSCIsMl0sWzIsMCwiXFxpZCJdLFsyLDMsIlxcaW50cm9TIiwyXSxbMywwLCJcXEhvbV97Q18xfSgxLCAtKSIsMl0sWzQsNSwiSCIsMl0sWzUsNiwiXFxpZCJdLFs1LDcsIlxcaW50cm9TIiwyXSxbNyw2LCJcXEhvbV97Q18yfSgxLCAtKSIsMl0sWzksMywiXFxpbnRyb04iLDIseyJzaG9ydGVuIjp7InNvdXJjZSI6MjB9fV0sWzEzLDcsIlxcaW50cm9OIiwyLHsic2hvcnRlbiI6eyJzb3VyY2UiOjIwfX1dXQ==
\[\begin{tikzcd}
	{T_1} && {T_1} & {T_2} \\
	& {\Glob{C_1}} \\
	{T_1} & {T_2} && {T_2} \\
	&& {\Glob{C_2}}
	\arrow["H"', from=1-3, to=1-4]
	\arrow[""{name=0, anchor=center, inner sep=0}, "\id", from=1-1, to=1-3]
	\arrow["\introS"', from=1-1, to=2-2]
	\arrow["{\Hom_{C_1}(1, -)}"', from=2-2, to=1-3]
	\arrow["H"', from=3-1, to=3-2]
	\arrow[""{name=1, anchor=center, inner sep=0}, "\id", from=3-2, to=3-4]
	\arrow["\introS"', from=3-2, to=4-3]
	\arrow["{\Hom_{C_2}(1, -)}"', from=4-3, to=3-4]
	\arrow["\introN"', shorten <=3pt, Rightarrow, from=0, to=2-2]
	\arrow["\introN"', shorten <=3pt, Rightarrow, from=1, to=4-3]
\end{tikzcd}\]

That the codomains of these two natural transformations are equal follows from the previous conditions.
\end{definition}

Such homomorphisms are closed under composition and thus we obtain the category of strict introspective theories.

As the category of models of an essentially algebraic theory, this category must have an initial object. That is, there is a strict introspective theory with a unique homomorphism into any other strict introspective theory. In this chapter, we will find a tractable explicit description of this initial strict introspective theory.

\subsection{Defining geminal categories}\label{GeminalFirstDefnSection}
\sTODOinline{Note that throughout the following, essentially nothing about the actual lex structure of lexcategories specifically matters. All that matters is the concepts of multiply internal structures. This would all work just as well for other \quote{tree-categories}.}

\newcommand{\bent}[1]{#1}
\newcommand{\straight}[1]{\widehat{#1}}

We will give two different presentations of the definition of \quote{geminal categories}. First, in this section, we give a definition using several infinite sequences of data and of equations. These infinite sequences will be highly redundant in that their first few entries suffice to derive all their later entries, but the advantage of this verbose definition is that it has convenient symmetries. Later, at \magicref{CompactGeminalCatDefn}, we will see a much more compact definition eliminating these redundancies.

\begin{definition}[Geminal category]\label{VerboseGeminalCatDefn}
A \defined{geminal category}\footnote{Another evocative name for this concept might be \quote{nesting doll category}.} internal to lexcategory $C_0$ consists of several ingredients:

\begin{itemize}
    \item 
    The first ingredient is an infinite sequence $C_1, C_2, C_3, \ldots$, in which each $C_i$ (for $i \geq 1$) is the global aspect of a lexcategory $C'_i$ internal to $C_{i - 1}$.
\end{itemize}

Thus, each $C'_{i + n}$ is $n$-tuply internal to $C_i$.

(Throughout the following, it will be useful to keep in mind that we are using these general naming habits: Primed names are used for internal structures, while the corresponding unprimed names indicate the corresponding global aspects. Furthermore, names subscripted with index $i$ arise from structure internal to $C_{i - 1}$.)

\begin{itemize}
    \item
    The second ingredient comprising a geminal category is an infinite sequence of internal lexfunctors $F'_1, F'_2, F'_3, \ldots$, where each $F'_i : C'_i \to \Gamma[C'_{i + 1}]$ is internal to $C_{i - 1}$ (for $i \geq 1$).
\end{itemize}

Pictorially, this can be envisioned like so: 

% https://q.uiver.app/#q=WzAsMTEsWzEsMCwiQydfMSJdLFsyLDAsIlxcR2FtbWFfMVtDJ18yXSJdLFsyLDEsIkMnXzIiXSxbMywxLCJcXEdhbW1hXzJbQydfM10iXSxbMCwwLCJDXzA6Il0sWzAsMSwiQ18xOiJdLFswLDIsIkNfMjoiXSxbMywyLCJDJ18zIl0sWzQsMiwiXFxHYW1tYV8zW0MnXzRdIl0sWzAsMywiXFxsZG90cyJdLFs0LDMsIlxcbGRvdHMiXSxbMCwxLCJGJ18xIl0sWzIsMywiRidfMiJdLFs3LDgsIkYnXzMiXSxbNSw0LCJcXEdhbW1hXzEiXSxbNiw1LCJcXEdhbW1hXzIiXSxbOSw2LCJcXEdhbW1hXzMiXV0=
\[\begin{tikzcd}
	{C_0:} & {C'_1} & {\Gamma_1[C'_2]} \\
	{C_1:} && {C'_2} & {\Gamma_2[C'_3]} \\
	{C_2:} &&& {C'_3} & {\Gamma_3[C'_4]} \\
	\ldots &&&& \ldots
	\arrow["{F'_1}", from=1-2, to=1-3]
	\arrow["{F'_2}", from=2-3, to=2-4]
	\arrow["{F'_3}", from=3-4, to=3-5]
	\arrow["{\Gamma_1}", from=2-1, to=1-1]
	\arrow["{\Gamma_2}", from=3-1, to=2-1]
	\arrow["{\Gamma_3}", from=4-1, to=3-1]
\end{tikzcd}\]

Here, the first row is structure internal to $C_0$, the second row is structure internal to $C_1$ (thus, doubly internal to the ambient $C_0$), the third row is structure internal to $C_2$ (thus, triply internal to the ambient $C_0$), and so on. We also for convenience use the abbreviation $\Gamma_i$ for $\Gamma_{C'_i} : C_i \to C_{i - 1}$ for $i \geq 1$, illustrating these in the vertical line on the left of the picture.

In keeping with our naming convention, we shall also use $F_i : C_i \to \Glob{\Gamma_i[C_{i + 1}]}$ to refer to the global aspect of $F'_i$. As \magicref{GlobOfGlob} tells us that $\InducedHomo{\Gamma_i}{C'_{i + 1}} : \Glob{C'_{i + 1}} \to \Glob{\Gamma_i[C'_{i + 1}]}$ is an isomorphism, we can also define $\straight{F}_i : C_i \to C_{i + 1}$ as the unique map making the following diagram commute:

% https://q.uiver.app/#q=WzAsMyxbMCwwLCJDX2kgPSBcXEdsb2J7QydfaX0iXSxbMiwyLCJDX3tpICsgMX0gPSBcXEdsb2J7Qydfe2kgKyAxfX0iXSxbMiwwLCJcXEdsb2J7XFxHYW1tYV9pW0MnX3tpICsgMX1dfSJdLFswLDIsIkZfaSJdLFsxLDIsIlxcSW5kdWNlZEhvbW97XFxHYW1tYV9pfXtDJ197aSArIDF9fSIsMl0sWzAsMSwiXFxzdHJhaWdodHtGfV9pIiwyXV0=
\[\begin{tikzcd}
	{C_i = \Glob{C'_i}} && {\Glob{\Gamma_i[C'_{i + 1}]}} \\
	\\
	&& {C_{i + 1} = \Glob{C'_{i + 1}}}
	\arrow["{F_i}", from=1-1, to=1-3]
	\arrow["{\InducedHomo{\Gamma_i}{C'_{i + 1}}}"', from=3-3, to=1-3]
	\arrow["{\straight{F}_i}"', from=1-1, to=3-3]
\end{tikzcd}\]

These $\straight{F}_i$ are convenient as they line up straightforwardly:

% https://q.uiver.app/#q=WzAsNCxbMCwwLCJDXzEiXSxbMSwwLCJDXzIiXSxbMiwwLCJDXzMiXSxbMywwLCJcXGxkb3RzIl0sWzAsMSwiXFxzdHJhaWdodHtGfV8xIl0sWzEsMiwiXFxzdHJhaWdodHtGfV8yIl0sWzIsMywiXFxzdHJhaWdodHtGfV8zIl1d
\[\begin{tikzcd}
	{C_1} & {C_2} & {C_3} & \ldots
	\arrow["{\straight{F}_1}", from=1-1, to=1-2]
	\arrow["{\straight{F}_2}", from=1-2, to=1-3]
	\arrow["{\straight{F}_3}", from=1-3, to=1-4]
\end{tikzcd}\]

Finally, the last ingredients we require are some equations:

\begin{itemize}
    \item 
     We require that $\TransferN{j - i}{\straight{F}_i}{C'_j} = C'_{j + 1}$ and $\TransferN{j - i}{\straight{F}_i}{F'_j} = F'_{j + 1}$ for $j > i \geq 1$.

     (We are using \magicref{TransferNDefn} here to apply $\straight{F}_i$ to structures multiply internal to its domain $C_i$.)
     
    \item
    Furthermore, we require that the following diagram of lexfunctors internal to $C_{i - 1}$ commutes, for each $i \geq 1$. We call this equation $E_i$.
    
% https://q.uiver.app/#q=WzAsNSxbMCwwLCJDJ19pIl0sWzAsMiwiXFxHYW1tYV9pIFtDJ197aSArIDF9XSJdLFsyLDIsIlxcR2FtbWFfe1xcR2FtbWFfe2l9W0MnX3tpICsgMX1dfSBbXFxiZW50e0Z9X2lbQydfe2kgKyAxfV1dIl0sWzMsMCwiXFxHYW1tYV9pIFtDJ197aSArIDF9XSJdLFszLDIsIlxcR2FtbWFfaSBbXFxHYW1tYV97aSArIDF9IFtDJ197aSArIDJ9XV0iXSxbMCwxLCJGJ19pIiwyXSxbMCwzLCJGJ19pIl0sWzEsMiwiXFxJbmR1Y2VkSG9tb3tGJ19pfXtDJ197aSArIDF9fSIsMl0sWzMsNCwiXFxHYW1tYV9pIFtGJ197aSArIDF9XSJdLFsyLDQsIiIsMix7ImxldmVsIjoyLCJzdHlsZSI6eyJoZWFkIjp7Im5hbWUiOiJub25lIn19fV1d
\[\begin{tikzcd}
	{C'_i} &&& {\Gamma_i [C'_{i + 1}]} \\
	\\
	{\Gamma_i [C'_{i + 1}]} && {\Gamma_{\Gamma_{i}[C'_{i + 1}]} [\bent{F}_i[C'_{i + 1}]]} & {\Gamma_i [\Gamma_{i + 1} [C'_{i + 2}]]}
	\arrow["{F'_i}"', from=1-1, to=3-1]
	\arrow["{F'_i}", from=1-1, to=1-4]
	\arrow["{\InducedHomo{F'_i}{C'_{i + 1}}}"', from=3-1, to=3-3]
	\arrow["{\Gamma_i [F'_{i + 1}]}", from=1-4, to=3-4]
	\arrow[Rightarrow, no head, from=3-3, to=3-4]
\end{tikzcd}\]

That is, we require that $\InducedHomo{F'_i}{C'_{i + 1}} \circ F'_i = \Gamma_i[F'_{i + 1}] \circ F'_i$. This could be glossed in abuse of notation as \quote{$F'_i \circ F'_i = F'_{i + 1} \circ F'_i$}.

(To derive the identity in the bottom-right of the above diagram, first note that $\bent{F}_i[C'_{i + 1}] = \InducedHomo{\Gamma_i}{C'_{i + 1}} [ \straight{F}_i[C'_{i + 1}]] = \InducedHomo{\Gamma_i}{C'_{i + 1}} [C'_{i + 2}]$. 

Thus, $\Gamma_{\Gamma_{i}[C'_{i + 1}]} [F'_i[C'_{i + 1}]] =$ $\Gamma_{\Gamma_{i}[C'_{i + 1}]} [\InducedHomo{\Gamma_i}{C'_{i + 1}} [C'_{i + 2}]] = \Gamma_i [\Gamma_{i + 1} [C'_{i + 2}]]$, where the last step is by \magicref{InducedGlobalCommute}.)
\end{itemize}

This concludes the definition of a geminal category internal to $C_0$.
\end{definition}

By a \defined{geminal category} simpliciter, we mean of course the case where $C_0 = \Set$. (Note that in this case, $C'_1$ can be identified with its global aspect $C_1$, in the same way that any structure internal to $\Set$ can be identified with its global aspect, as the global elements functor from $\Set$ to $\Set$ is the identity.). We wrote out here the definition for general $C_0$, instead of specifically for $C_0 = \Set$, in order to emphasize certain symmetries in this definition.

When being fully explicit, we reference a geminal category by enumerating $\langle C'_1, C'_2, C'_3, \ldots; F'_1, F'_2, F'_3, \ldots \rangle$. Given such a geminal category $K$, we may write $\underlying{K}$ to refer to its underlying lexcategory $C'_1$.

All aforementioned structure apart from $C_0$ itself has been given as $i$-tuply internal to $C_0$ for some $i > 0$. Thus, all of this structure is indeed given by diagrams within $C_0$.

Indeed, this definition of geminal category is manifestly essentially algebraic. That is, there is an essentially algebraic theory such that models of that theory internal to $C_0$ are the same thing as geminal categories internal to $C_0$.

Our ultimate goal will be to show that this theory of geminal categories is the initial introspective theory. This is the whole motivation for our study of geminal categories. But to show this result, we must develop some other machinery first.

\subsection{Geminal category homomorphisms}
As geminal categories are defined by an essentially algebraic theory, we automatically get a notion of homomorphism between geminal categories. It amounts to the following:

\begin{definition}\label{VerboseGeminalCatHomoDefn}
Given two geminal categories $\langle C'_1, C'_2, C'_3, \ldots; F'_1, F'_2, F'_3, \ldots \rangle$ and $\langle D'_1, D'_2, $ $D'_3, \ldots; \phi'_1, \phi'_2, \phi'_3, \ldots \rangle$, a \defined{homomorphism} from the former to the latter consists of a strict lexfunctor $H : C'_1 \to D'_1$ such that $H[C'_i] = D'_i$ and $H[F'_i] = \phi'_i$ for each $i > 1$, while also the following diagram commutes:

% https://q.uiver.app/?q=WzAsNCxbMCwwLCJDJ18xIl0sWzIsMCwiRCdfMSJdLFsyLDIsIlxcR2xvYntIW0MnXzJdfSA9IFxcR2xvYntEXzInfSJdLFswLDIsIlxcR2xvYntDXzInfSJdLFswLDEsIkgiXSxbMCwzLCJGJ18xIiwyXSxbMywyLCJcXEluZHVjZWRIb21ve0h9e0MnXzJ9IiwyXSxbMSwyLCJcXHBoaSdfMSJdXQ==
\[\begin{tikzcd}
	{C'_1} && {D'_1} \\
	\\
	{\Glob{C_2'}} && {\Glob{H[C'_2]} = \Glob{D_2'}}
	\arrow["H", from=1-1, to=1-3]
	\arrow["{F'_1}"', from=1-1, to=3-1]
	\arrow["{\InducedHomo{H}{C'_2}}"', from=3-1, to=3-3]
	\arrow["{\phi'_1}", from=1-3, to=3-3]
\end{tikzcd}\]
\end{definition}

(In the above, $H[C'_i]$ and $H[F'_i]$ makes use of \magicref{TransferNDefn} to denote the application of $H$ to multiply internal structures.)

\begin{theorem}\label{GeminalContainsGeminal}
Given any geminal category $K = \langle C'_1, C'_2, C'_3, \ldots; F'_1, F'_2, F'_3, \ldots \rangle$, we have also that $\langle C'_2, C'_3, C'_4, \ldots; F'_2, F'_3, F'_4 \ldots \rangle$ comprises a geminal category internal to $\underlying{K} = C'_1$. We refer to this internal geminal category as $\InteriorGeminal{K}$.

We furthermore have that $F'_1$ acts as a geminal category homomorphism from $K$ to the global aspect of $\InteriorGeminal{K}$. We refer to this homomorphism as $\IntoSelf{K} : K \to \Gamma[\InteriorGeminal{K}]$.
\end{theorem}
\begin{proof}
This is all direct by definition.

For the first part, each condition imposed upon each $C'_{i}$ or $F'_{i}$ in the definition of a geminal category comes with an analogous condition imposed upon $C'_{i + 1}$ or $F'_{i + 1}$. Thus, it is immediate that the given $\InteriorGeminal{K}$ satisfies the conditions to be a geminal category internal to $\underlying{K}$.

For the second part, the definition of a geminal category directly imposes upon $F'_1$ precisely the conditions which are necessary for $F'_1$ to comprise a geminal category homomorphism from $K$ to the global aspect of $\InteriorGeminal{K}$. In particular, equation $E_1$ from \magicref{VerboseGeminalCatDefn} is identical to the commutative diagram from \magicref{VerboseGeminalCatHomoDefn}, in this context.
\end{proof}

Via the yoga of functorial semantics, \magicref{GeminalContainsGeminal} states how the theory of geminal categories can be equipped as an introspective theory. In detail, this is given like so:

\begin{construction}\label{GLCatTheoryIsIntrosp}
Let $\GLCatTheory$ be the free strict lexcategory with an internal geminal category (that is, in the terminology of \magicref{QuasiTheoryTheory}, we take $\GLCatTheory$ to be the classifying strict lexcategory $\classifying{\theoryT}$, where $\theoryT$ is the theory of geminal categories). 

Thus, strict lexfunctors from $\GLCatTheory$ to any other strict lexcategory $D$ correspond to geminal categories internal to $D$, while natural transformations between such lexfunctors correspond to homomorphisms between these $D$-internal geminal categories.

Let $K$ denote the $\GLCatTheory$-internal geminal category corresponding to the identity functor on $\GLCatTheory$.

By \magicref{GeminalContainsGeminal} in the internal logic of $\GLCatTheory$, we obtain also a geminal category $\InteriorGeminal{K}$ internal to $\underlying{K}$, as well as a homomorphism $\IntoSelf{K} : K \to \Gamma[\InteriorGeminal{K}]$.

Thus, there is some strict lexfunctor $\introS$ from $\GLCatTheory$ to the global aspect of $\underlying{K}$, corresponding to $\InteriorGeminal{K}$. Furthermore, there is some natural transformation $\introN$ from the identity functor on $\GLCatTheory$ to $\Hom_{\underlying{K}}(1, \introS(-))$, corresponding to $\IntoSelf{K}$.

Putting this together, we have a strict introspective theory $\langle \GLCatTheory, \underlying{K}, \introS, \introN \rangle$.
\end{construction}

\sTODOinline{Note that there is a more limited analogue of the above, where we observe that the free X with an internal geminal Y is itself a geminal X, whenever Y extends X and X extends the notion of a lexcategory. The difficulty with turning this into an introspective theory is that the property we really depend on from the free lexcategory $L$ with an internal gadget of some sort is not just the 1-categorical property that it has a unique homomorphism to every other gadget, but the 2-categorical property that the category of homomorphisms from it to another lexcategory and natural transformations between those is equivalent to the 1-category of internal gadgets within that codomain lexcategory. This is true for lexcategories, but not necessarily for other doctrines, and I believe this is related to how $\Hom(1, -)$ is always a lexfunctor (thus turning internal models into genuine models,) but not always an X-functor. The subtle role of this in the above proof should be highlighted.}

\subsection{Compactly defined geminal categories}\label{GeminalSecondDefnSection}
The above all amounts to an infinitary presentation of the theory of geminal categories. For this reason, we call it the \quote{verbose presentation} of geminal categories. However, it turns out this same theory can be finitely axiomatized as well.

\begin{definition}[Geminal category, compact presentation]\label{CompactGeminalCatDefn}
A \defined{compactly presented geminal category} internal to lexcategory $C_0$ consists of the structure $C'_i$, $F'_i$, and equations $E_i$ of the verbose presentation, but only for $i \in \{1, 2\}$.

(Here, in interpreting the codomain of $F'_2$, we take $C'_3$ to be $F_1[C'_2]$, and in interpreting the equation $E_2$, we take $F'_3$ to be $F_1[F'_2]$)

That is, a compactly presented geminal category internal to $C_0$ consists of the following six pieces of data:

\begin{itemize}
    \item A lexcategory $C'_1$ internal to $C_0$, whose global aspect we call $C_1$. We refer to $\Gamma_{C'_1} : C_1 \to C_0$ as $\Gamma_1$.
    
    \item A lexcategory $C'_2$ internal to $C_1$, whose global aspect we call $C_2$. We refer to $\Gamma_{C'_2} : C_2 \to C_1$ as $\Gamma_2$.
    
    \item A lexfunctor $F'_1 : C'_1 \to \Gamma_1[C'_2]$, internal to $C_0$. We call the global aspect of this $F_1 : C_1 \to C_2$.
    
    \item A lexfunctor $F'_2 : C'_2 \to \Gamma_2[C'_3]$, internal to $C_1$.
    
    (Here, $C'_3$ is defined as $F_1[C'_2]$.)
    
    \item The equation $\InducedHomo{F'_1}{C'_{2}} \circ F'_1 = \Gamma_1[F'_2] \circ F'_1$, internal to $C_0$. We call this equation $E_1$.
    
    \item The equation $\InducedHomo{F'_2}{C'_{3}} \circ F'_2 = \Gamma_2[F'_3] \circ F'_2$, internal to $C_1$. We call this equation $E_2$.
    
    (Here, $F'_3$ is defined as $F_1[F'_2]$.)
\end{itemize}
\end{definition}

As usual, we reference a compactly presented geminal category by enumerating the ordered tuple $\langle C'_1, C'_2; F'_1, F'_2 \rangle$.

Clearly, the structure defining a compactly presented geminal category is part of the structure in our verbose definition of a geminal category. But in fact, these are equivalent definitions.

\begin{theorem}\label{GeminalCompactIsVerbose}
The structure of a compactly presented geminal category uniquely determines the further structure of a geminal category (as originally defined in \magicref{VerboseGeminalCatDefn})).
\end{theorem}
\begin{proof}
Throughout the following, as before, we define each $\straight{F}_i$ from the corresponding $F'_i$ by $\straight{F}_i = \left(\InducedHomo{\Gamma_i}{C'_{i + 1}}\right)^{-1} \circ \Glob{F_i} : C_i \to C_{i + 1}$.

By definition, in a geminal category, we must have that $C'_j = \straight{F}_1[C'_{j - 1}]$ and $F'_j = \straight{F}_1[F'_{j - 1}]$ for each $j > 2$.

Accordingly, if we are given the structure in \magicref{CompactGeminalCatDefn}, and we are to extend it to all the further structure in \magicref{VerboseGeminalCatDefn}, we may use the above particular recurrences to inductively define $C'_j$ and $F'_j$ for each $j > 2$, ultimately in terms of the base cases of $j \in \{1, 2\}$ which we have been given. Adopt these definitions throughout the following accordingly.

The equations given to us directly in the compact presentation are the equations $E_1$ and $E_2$ of the verbose presentation. Furthermore, we again obtain the equation $E_i$ for each $i > 2$ inductively by applying $\straight{F}_1$ to $E_{i - 1}$.

What remains is only to see that each $\straight{F}_i$ takes $C'_j$ to $C'_{j + 1}$ and takes $F'_j$ to $F'_{j + 1}$, for $j > i \geq 1$.

We prove this by induction on $i$. For the base case of $i = 1$, we have ensured this by construction. As for the inductive step, suppose we know this already holds for $i$. Then for $j > i + 1$ we have $\straight{F}_{i + 1}[C'_j] = \straight{F}_{i + 1} [\straight{F}_i [C'_{j - 1}]] = \InducedHomo{\straight{F}_i}{C'_{i + 1}} [\straight{F}_i [C'_{j - 1}]] = \straight{F}_i [\straight{F}_i [C'_{j - 1}]] = \straight{F}_i [C'_j] = C'_{j + 1}$, where the second step is by the global aspect of $E_i$ (along with some applications of \magicref{GlobOfGlob}), the third step is by \magicref{TransferNDefn}, and the other steps are by our induction hypothesis. And similarly with $F'$ in place of $C'$ throughout as well.
\end{proof}

\begin{corollary}\label{CompactGeminalCatHomoDefn}
In \magicref{VerboseGeminalCatHomoDefn}, the conditions $H[C'_i] = D'_i$ and $H[F'_i] = \phi'_i$ automatically follow for all $i > 2$ once they hold for $i = 2$.
\end{corollary}

Thus, we can go back and forth between thinking of geminal categories in either the verbose or compact presentation as we please, whichever is most convenient at any moment.

\subsection{Geminal categories from introspective theories}

\begin{construction}\label{IntrospAsGeminal}
From a strict introspective theory $\langle T, C, \introS, \introN \rangle$, we obtain a geminal category $\langle T, C; \introS, \introN_{C} \rangle$, whose underlying lexcategory is $T$. This is the canonical way to view an introspective theory as a geminal category.
\end{construction}
\begin{proof}
It is immediate in the definition of a strict introspective theory that $C$ is a lexcategory internal to $T$, and $\introS$ is a lexfunctor from $T$ to $\Glob{C}$. This gives us the first three out of the six ingredients of \magicref{CompactGeminalCatDefn}.

As for $\introN_{C}$ (meaning the components of the natural transformation $\introN$ at the objects of the diagram within $T$ which defines the $T$-internal lexcategory $C$), this gives us a $T$-internal lexfunctor from $C$ to $\Hom_C(1, \introS[C]) = \Gamma[\introS[C]]$. This is the fourth ingredient of \magicref{CompactGeminalCatDefn}.

What remains are to verify equations $E_1$ and $E_2$. In this context, $E_1$ is a special case of \magicref{SMatchesN}, while $E_2$ is given by the naturality of $\introN$ with respect to the components of $\introN_C$ themselves.

This completes the construction. We observe furthermore that strict introspective theory homomorphisms are automatically geminal category homomorphisms between the geminal categories obtained by this construction.
\end{proof}

There is another closely related construction which is of even more importance:

\begin{construction}\label{IntrospContainsGeminal}
From a strict introspective theory $\langle T, C, \introS, \introN \rangle$, we obtain a $T$-internal geminal category $\langle C, \introS[C]; \introN_{C}, \introS[\introN_{C}] \rangle$, whose underlying lexcategory is $C$.
\end{construction}
\begin{proof}
This is the result of first obtaining the geminal category $\gamma = \langle T, C, \introS, \introN_C \rangle$ from \magicref{IntrospAsGeminal}, and then forming $\InteriorGeminal{\gamma}$.
\end{proof}

We now are ready to prove our main result about geminal categories.

\subsection{The free introspective theory}\label{InitialIntrospectiveTheorySection}
\begin{theorem}\label{InitialIntrospectiveTheory}
The strict introspective theory given in \magicref{GLCatTheoryIsIntrosp} is the initial strict introspective theory.
\end{theorem}
\begin{proof}
We must show there is a unique homomorphism from the strict introspective theory $\langle \GLCatTheory, K \rangle$ of \magicref{GLCatTheoryIsIntrosp} to any other strict introspective theory $\langle T, D \rangle$.

Such a homomorphism is comprised of a strict lexfunctor $H : \GLCatTheory \to T$ satisfying certain conditions. By the nature of $\GLCatTheory$, this amounts to a geminal category $\langle D'_1, D'_2, D'_3, \ldots; F'_1, F'_2, F'_3, \ldots \rangle$ internal to $T$ satisfying certain conditions.

One particular geminal category internal to $T$ is the one that is given by $\gamma = \langle D, \introS[D]; \introN_D, \introS[\introN_D] \rangle$, as noted at \magicref{IntrospContainsGeminal}. In verbose terms, this geminal category is $\langle D, \introS[D], \introS[\introS[D]], \ldots;$ $ \introN_D, \introS[\introN_D], \introS[\introS[\introN_D]], \ldots \rangle$, with each successive component being $\introS$ applied to the previous component.

What remains is to show that the lexfunctor $H : \GLCatTheory \to T$ corresponding to this $\gamma$ uniquely satisfies the conditions of \magicref{StrictIntrospHomoDefn}.

The condition \quote{$H[C_1] = C_2$} in \magicref{StrictIntrospHomoDefn} says in this context that we must use a geminal category whose underlying lexcategory is $D$.

The condition concerning $\introN$ in \magicref{StrictIntrospHomoDefn}, along with the definition of $\introN$ in \magicref{GLCatTheoryIsIntrosp}, says that we must use a geminal category whose first lexfunctor component is $\introN_{D}$.

Finally, the commutative diagram concerning $\introS$ in \magicref{StrictIntrospHomoDefn}, along with the definition of $\introS$ in \magicref{GLCatTheoryIsIntrosp}, says we must use a geminal category such that each successive component of this geminal category is $\introS$ applied to the previous component.

The conjunction of these conditions clearly is uniquely satisfied by $\gamma$. This completes the proof.
\end{proof}

\begin{observation}\label{EveryIntrospModelsInitialIntrospRemark}
Given the result of \magicref{InitialIntrospectiveTheory}, we can rephrase \magicref{IntrospAsGeminal} as telling us that every strict introspective theory is a model of the initial introspective theory, so to speak. In other words, there is a lexfunctor interpreting the initial introspective theory into the theory of strict introspective theories. This is quite remarkable!
\end{observation}

\subsection{Geminal gadgets}
We have now successfully described the initial introspective theory. But we can also take our free construction results a bit further than this.

Specifically, every introspective theory is, among other things, an essentially algebraic theory extending the theory of strict lexcategories. That is, we have a functor from the category of introspective theories to the category of extensions of the essentially algebraic theory of strict lexcategories (essentially, this functor takes $\langle T, C, \introS, \introN \rangle$ to $\langle T, C \rangle$). This functor has a left adjoint.

Put in other words, for any essentially algebraic theory $Th$ such that models of $Th$ come with an underlying strict lexcategory, there is a free strict introspective theory $\langle T, C, \introS, \introN \rangle$ with a designated $T$-internal model of $Th$ with underlying lexcategory $C$.

For simplicity as a first introduction, everything done previously was the special case where $Th$ was simply the theory of strict lexcategories itself. But now we describe the more general results, which follow by almost exactly the same reasoning as used before:

Specifically, let models of $Th$ be called \quote{gadgets}, and maps between them called \quote{gadget homomorphisms}. Then the free introspective theory extending $Th$ is the theory of \quote{geminal gadgets}, with the definition of a \quote{geminal gadget} being exactly as in either definition of a \quote{geminal category}, but with all instances of lexcategories and lexfunctors replaced by gadgets and gadget homomorphisms.

This is by exactly the same arguments as we have just given. All the results and arguments given earlier in this chapter apply just as well mutatis mutandis when lexcategories and lexfunctors are replaced by gadgets and gadget homomorphisms, except for \magicref{IntrospAsGeminal} (it will not be the case that an arbitrary strict introspective theory can be viewed as a geminal gadget). However, the analogue of the construction \magicref{IntrospContainsGeminal} still holds (i.e., given an introspective theory $\langle T, C \rangle$ such that $C$ is the underlying lexcategory of a gadget, then the geminal category structure which $C$ is equipped with by \magicref{IntrospContainsGeminal} furthermore underlies geminal gadget structure).

\sTODOinline{Perhaps also discuss the straightforward notion of a non-strict geminal category or gadget: One for which $C_1$ is a non-strict lexcategory or gadget, and $F_1$ needn't be strict either (preserves finite limits but not necessarily on the nose), but everything else remains strict. Every non-strict geminal gadget straightforwardly admits a presentation by a strict one, by using a presentation of $C_1$ with no nontrivial equations on objects in a suitable sense.}

\sTODOinline{
\subsection{Modal logic in geminal categories}
\begin{observation}
In every introspective theory $\langle T, C \rangle$, we have a $T$-internal endofunctor $\Box_C$ on $C$ modeling the modal logic GL. Thus, this is in particular true of the initial introspective theory, which by \magicref{InitialIntrospectiveTheory} is the theory of geminal categories. Thus, any geminal category $K$ comes with such an operation $\Box_{|K|}$ on $|K|$, and this operation is furthermore preserved by geminal category homomorphisms.
\end{observation}

\sTODOinline{Guide readers by showing how we have the axioms of GL modal logic in a geminal category, but do NOT have A |- []A. Weave our archetypal examples into here.}
}

\subsection{Co-free introspective theories and geminal categories}\label{CofreeGeminalSection}

We have above discussed how to create free introspective theories, which can be thought of as produced by a certain left adjoint functor\footnote{Specifically, left adjoint to the forgetful functor from strict introspective theories to strict lexcategories with a designated internal lexcategory.}. In this section, we discuss some right adjoint constructions, which can be thought of as \quote{co-free}.

\begin{construction}
Construing strict introspective theories as geminal categories via \magicref{IntrospAsGeminal} gives us a functor from the category of strict introspective theories to the category of geminal categories (or more generally, a functor from the category of $V$-internal strict introspective theories to the category of $V$-internal geminal categories, for any fixed lexcategory $V$). This functor has a right adjoint.

As this works for arbitrary lexcategories $V$, this right adjoint admits an explicit description, as a purely lex construction.
\end{construction}
\openDetails
For linguistic convenience, we shall in the following take $V$ as $\Set$, but it will be clear that the same explicit construction works for any ambient lexcategory $V$.

We are tasked with showing that, for any geminal category $C_1$, there is a suitably terminal introspective theory with a geminal category homomorphism to $C_1$.

We will first give the details of the construction, and then give the proof that it has the terminality property.

Recall from \magicref{ModalAxiom4Section} that any introspective theory $\langle T, C \rangle$ comes with a $T$-internal endolexfunctor $\Box_C = \introF(\Hom_C(1, -)): C \to C$, along with a $T$-internal natural transformation $4 : \Box_C \to \Box_C^2$ built from $\introN$. As this applies in particular to the introspective theory of \TODO (where $\introF(\Hom_C(1, -)) = $ \TODO), we find that any geminal category $C_1 = \langle C_1, C'_2; F_1, F'_2 \rangle$ \TODOinline{Should make $C_1$ into $C'_1$ throughout} comes with an endolexfunctor $\Box_{C_1} = \Gamma_{C'_2} \circ \straight{F}_1 : C_1 \to C_1$, along with a natural transformation $4 : \Box_{C_1} \to \Box_{C_1} \Box_{C_1}$ corresponding to the action of $F'_2$.

Via \magicref{BoxCoalgebrasInGeminal}, the category of $\Box_{C_1}$-coalgebras is a strict lexcategory. Among these $\Box_{C_1}$-coalgebras, there are some coalgebras $m : c \to \Box_{C_1} c$ with the property that the following diagram commutes:

% https://q.uiver.app/?q=WzAsNCxbMCwwLCJjIl0sWzAsMiwiXFxCb3ggYyJdLFsyLDAsIlxcQm94IGMiXSxbMiwyLCJcXEJveF4yIGMiXSxbMCwxLCJtIiwyXSxbMCwyLCJtIl0sWzIsMywiXFxCb3ggbSJdLFsxLDMsIjRfYyIsMl1d
\[\begin{tikzcd}
	c && {\Box c} \\
	\\
	{\Box c} && {\Box^2 c}
	\arrow["m"', from=1-1, to=3-1]
	\arrow["m", from=1-1, to=1-3]
	\arrow["{\Box m}", from=1-3, to=3-3]
	\arrow["{4_c}"', from=3-1, to=3-3]
\end{tikzcd}\]

Let $K$ be the full subcategory of those $\Box_{C_1}$-coalgebras with the specified property. It is readily seen that this $K$ is closed under the finite limits of the category of $\Box_{C_1}$-coalgebras, and thus is itself a strict lexcategory. (Indeed, $K$ can be defined as the equalizer of two strict lexfunctors from the $\Box_{C_1}$-coalgebras to the $\Box_{C_1}^2$-coalgebras.)

Note that we have the following commutative diagram of internal lexfunctors in $C_1$:

% https://q.uiver.app/#q=WzAsNCxbMCwwLCJDJ18yIl0sWzAsMiwiXFxCb3hfe0NfMX0gQydfMiJdLFszLDAsIlxcQm94X3tDXzF9IEMnXzIiXSxbMywyLCJcXEJveF97Q18xfSBcXEJveF97Q18xfSBDJ18yIl0sWzAsMSwiRidfMiIsMl0sWzAsMiwiRidfMiJdLFsyLDMsIlxcQm94X3tDXzF9IEYnXzIgPSBcXEdhbW1hX3tDJ18yfSBbXFxzdHJhaWdodHtGfV8xW0YnXzJdXSA9IFxcR2FtbWFfe0MnXzJ9W0YnXzNdIl0sWzEsMywiNF97QydfMn0gPSBcXEluZHVjZWRIb21ve0YnXzJ9e0MnXzJ9IiwyXV0=
\[\begin{tikzcd}
	{C'_2} &&& {\Box_{C_1} C'_2} \\
	\\
	{\Box_{C_1} C'_2} &&& {\Box_{C_1} \Box_{C_1} C'_2}
	\arrow["{F'_2}"', from=1-1, to=3-1]
	\arrow["{F'_2}", from=1-1, to=1-4]
	\arrow["{\Box_{C_1} F'_2 = \Gamma_{C'_2} [\straight{F}_1[F'_2]] = \Gamma_{C'_2}[F'_3]}", from=1-4, to=3-4]
	\arrow["{4_{C'_2} = \InducedHomo{F'_2}{C'_2}}"', from=3-1, to=3-4]
\end{tikzcd}\]

This diagram commutes by equation $E_2$. But this is also the commutative diagram which establishes that the internal lexcategory $F'_2: C'_2 \to \Box_{C_1} C'_2$ within the category of $\Box_{C_1}$-coalgebras is furthermore within its subcategory $K$.

Note that the map $\straight{F}_1 : C_1 \to C_2$ is such that its range \TODO. Thus it acts as a map from $C_1$ to the global aspect of this $K$-internal lexcategory. By composing this with the projection functor from $K$ to $C_1$, we get a map $\introS$ from $K$ to the global aspect of this $K$-internal lexcategory.

Finally, for $\introN$, \TODO.

\closeDetails

\begin{proof}

Let $C_1 = \langle C_1, C'_2; F_1, F'_2 \rangle$ be a geminal category, let $T = \langle T, C, \introS, \introN \rangle$ be a strict introspective theory (which we can also construe as a geminal category via \magicref{IntrospAsGeminal}), and let $H : T \to C_1$ be a geminal category homomorphism.

Let us use the suggestive name $\Box_{C_1}$ for the endolexfunctor $\Gamma_{C'_2} \circ F_1 : C_1 \to C_1$.

By virtue of $H$ being a geminal category homomorphism, we have that $H \circ \Box_T = \Box_{C_1} \circ H$. In detail, this is seen via the following commutative diagram:

% https://q.uiver.app/?q=WzAsNixbMywwLCJDXzEiXSxbMywxLCJcXEdsb2J7QydfMn0iXSxbMywyLCJDXzEiXSxbMCwwLCJUIl0sWzAsMSwiXFxHbG9ie0N9Il0sWzAsMiwiVCJdLFswLDEsIkZfMSIsMl0sWzEsMiwiXFxHYW1tYV97QydfMn0iLDJdLFswLDIsIlxcQm94X3tDXzF9IiwwLHsiY3VydmUiOi01fV0sWzMsMCwiSCJdLFszLDQsIlxcaW50cm9TIl0sWzQsMSwiXFxJbmR1Y2VkSG9tb3tIfXtDfSIsMV0sWzQsNSwiXFxHYW1tYV9DIl0sWzUsMiwiSCIsMl0sWzMsNSwiXFxCb3hfVCIsMix7ImN1cnZlIjo1fV1d
\[\begin{tikzcd}
	T &&& {C_1} \\
	{\Glob{C}} &&& {\Glob{C'_2}} \\
	T &&& {C_1}
	\arrow["{F_1}"', from=1-4, to=2-4]
	\arrow["{\Gamma_{C'_2}}"', from=2-4, to=3-4]
	\arrow["{\Box_{C_1}}", curve={height=-30pt}, from=1-4, to=3-4]
	\arrow["H", from=1-1, to=1-4]
	\arrow["\introS", from=1-1, to=2-1]
	\arrow["{\InducedHomo{H}{C}}"{description}, from=2-1, to=2-4]
	\arrow["{\Gamma_C}", from=2-1, to=3-1]
	\arrow["H"', from=3-1, to=3-4]
	\arrow["{\Box_T}"', curve={height=30pt}, from=1-1, to=3-1]
\end{tikzcd}\]

In the above diagram, the left side is the definition of $\Box_T$ and the right side is the definition of $\Box_{C_1}$. The top rectangle is one of the conditions in \magicref{VerboseGeminalCatHomoDefn} and the bottom rectangle is by \magicref{InducedGlobalCommute}.

Thus, the whiskering of $\introN : \id_T \to \Box_T$ along $H$ yields a natural transformation from $H$ to $H \circ \Box_T = \Box_{C_1} \circ H$. Illustrated like so:

% https://q.uiver.app/?q=WzAsNixbNCwwLCJDXzEiXSxbNCwxLCJcXEdsb2J7QydfMn0iXSxbNCwyLCJDXzEiXSxbMCwwLCJUIl0sWzEsMSwiXFxHbG9ie0N9Il0sWzAsMiwiVCJdLFswLDEsIkZfMSIsMl0sWzEsMiwiXFxHYW1tYV97QydfMn0iLDJdLFswLDIsIlxcQm94X3tDXzF9IiwwLHsiY3VydmUiOi01fV0sWzMsMCwiSCJdLFszLDQsIlxcaW50cm9TIl0sWzQsMSwiXFxJbmR1Y2VkSG9tb3tIfXtDfSIsMV0sWzQsNSwiXFxHYW1tYV9DIl0sWzUsMiwiSCIsMl0sWzMsNSwiXFxpZCIsMix7ImxldmVsIjoyLCJzdHlsZSI6eyJoZWFkIjp7Im5hbWUiOiJub25lIn19fV0sWzE0LDQsIlxcaW50cm9OIiwyLHsic2hvcnRlbiI6eyJzb3VyY2UiOjIwfX1dXQ==
\[\begin{tikzcd}
	T &&&& {C_1} \\
	& {\Glob{C}} &&& {\Glob{C'_2}} \\
	T &&&& {C_1}
	\arrow["{F_1}"', from=1-5, to=2-5]
	\arrow["{\Gamma_{C'_2}}"', from=2-5, to=3-5]
	\arrow["{\Box_{C_1}}", curve={height=-30pt}, from=1-5, to=3-5]
	\arrow["H", from=1-1, to=1-5]
	\arrow["\introS", from=1-1, to=2-2]
	\arrow["{\InducedHomo{H}{C}}"{description}, from=2-2, to=2-5]
	\arrow["{\Gamma_C}", from=2-2, to=3-1]
	\arrow["H"', from=3-1, to=3-5]
	\arrow[""{name=0, anchor=center, inner sep=0}, "\id"', Rightarrow, no head, from=1-1, to=3-1]
	\arrow["\introN"', shorten <=4pt, Rightarrow, from=0, to=2-2]
\end{tikzcd}\]

This natural transformation from $H$ to $\Box_{C_1} \circ H$ acts as a functor $\beta$ from $T$ to the category of $\Box_{C_1}$-coalgebras, such that $\beta$ followed by the forgetful functor to $C_1$ yields $H$. As $H$ is a strict lexfunctor, this $\beta$ is also a strict lexfunctor, when the category of $\Box_{C_1}$-coalgebras is construed as a strict lexcategory in the manner of \magicref{BoxCoalgebrasInGeminal}.

Not only that, but every coalgebra $m : c \to \Box_{C_1} c$ in the range of $\beta$ has the property that the following diagram commutes \TODO:

% https://q.uiver.app/?q=WzAsNCxbMCwwLCJjIl0sWzAsMiwiXFxCb3ggYyJdLFsyLDAsIlxcQm94IGMiXSxbMiwyLCJcXEJveF4yIGMiXSxbMCwxLCJtIiwyXSxbMCwyLCJtIl0sWzIsMywiXFxCb3ggbSJdLFsxLDMsIjRfYyIsMl1d
\[\begin{tikzcd}
	c && {\Box c} \\
	\\
	{\Box c} && {\Box^2 c}
	\arrow["m"', from=1-1, to=3-1]
	\arrow["m", from=1-1, to=1-3]
	\arrow["{\Box m}", from=1-3, to=3-3]
	\arrow["{4_c}"', from=3-1, to=3-3]
\end{tikzcd}\]

\sTODOinline{Explain above diagram}

\TODOinline{Write out details of how, by restricting to coalgebras for $\Box_{C'_1}$ with the suitable property, we get an introspective theory, which is terminal among all introspective theories which map into this geminal category}

Let $K$ be the full subcategory of $\Box_{C_1}$ coalgebras with the specified property. It is readily verified that $K$ is closed under the chosen basic limits of $\Box_{C_1}$ \TODOinline{As an equalizer between strict lexfunctors}, and thus $K$ can be seen as a full sub-strict-lexcategory of the $\Box_{C_1}$ coalgebras.

Note that the diagram defining $C'_2$ as an internal lexcategory in $C_1$ is in fact a diagram within the subcategory $K$, as \TODOinline{Cite relevant part of definition of a geminal category}. Thus, we may consider $C'_2$ as a $K$-\repsmall/ lexcategory as well.

Define the strict lexfunctor $\introS : K \to \Glob{C'_2}$ by composing the forgetful functor from $K$ to $C_1$ with $F_1 : C_1 \to \Glob{C'_2}$.

Note that $\Hom_{C_1}(1, \introS(-)) : K \to K$ takes each $m : c \to \Box_{C_1} c$ in $K$ to $4_c : \Box_{C_1} c \to \Box_{C_1} \Box_{C_1} c$, by \TODO.

Finally, define the natural transformation $\introN$ as sending each $m : c \to \Box_{C_1} c$ in $K$ to the following coalgebra map from $m$ to $\Box_{C_1} m$ (the coalgebra map whose underlying map in $C_1$ is $m$ itself). This square commutes because of the defining property of $K$: \TODO.

Thus, we obtain a strict introspective theory $K = \langle K, C'_2, \introS, \introN \rangle$.

Let the strict lexfunctor $k$ be the inclusion of $K$ into the $\Box_{C_1}$ coalgebras, followed by the forgetful functor from the $\Box_{C_1}$ coalgebras to $C_1$. Observe that $k$ acts as a geminal category homomorphism $: K \to C_1$, by \TODO.

Note that the construction of $K$ and $k$ depends only upon the geminal category $C_1$. Furthermore, observe for any strict introspective theory $T$ and geminal category homomorphism $H: T \to C_1$ as above, the $\beta$ as described above becomes the unique strict introspective theory homomorphism from $T$ to $K$ such that $\beta H = k$, by \TODO. 

Thus, the operation turning $C_1$ into $K$ is the right adjoint to the functor turning strict introspective theories into geminal categories.
\end{proof}

\begin{corollary}
As left adjoints preserve initial objects, the above tells us that the initial strict introspective theory (which we described in \magicref{InitialIntrospectiveTheory}) is also the initial geminal category.
\end{corollary}

\begin{construction}
Consider the functor from the category of geminal categories to the category of strict lexcategories with a designated internal geminal category, given by sending any geminal category $G$ to the strict lexcategory $|G|$ with internal geminal category $\InteriorGeminal{G}$. This functor has a right adjoint.

In other words, for any strict lexcategory $C_0$ with an internal geminal category $\gamma$, there is a geminal category $G$ equipped with a strict lexfunctor $H : \underlying{G} \to C_0$ satisfying $H[\InteriorGeminal{G}] = \gamma$, which is terminal among all so equipped geminal categories (in the sense that for any other such geminal category $K$ with strict lexfunctor $J : \underlying{K} \to C_0$ satisfying $J[\InteriorGeminal{K}] = \gamma$, there is a unique geminal category homomorphism $M : K \to G$ such that $H \circ M = J$).

This co-free $G$ admits an explicit description as $C_0 \times \Glob{\underlying{\gamma}}$ suitably equipped. (Indeed, just as before, this whole construction can be carried out internal to an arbitrary ambient lexcategory.)
\end{construction}
\openDetails
Let $\gamma = \langle C'_1, C'_2, C'_3, \ldots; F'_1, F'_2, F'_3, \ldots \rangle$. Throughout the following, let unprimed names of lexcategories denote global aspects of primed names and let $\straight{F}_1 : C_1 \to C_2$ be defined from $F'_1$ in our usual fashion. For convenience, we will also say simply \quote{lexfunctor} in the following to mean \quote{strict lexfunctor}.

Let $G_0 = C_0 \times C_1$. We have that $\gamma \times \InteriorGeminal{\gamma}$ is a geminal category $\langle G'_1, G'_2, G'_3, \ldots; \phi'_1, \phi'_2, \phi'_3, \ldots \rangle$ internal to $G_0$, with each $G'_n = C'_n \times C'_{n + 1}$ and $\phi'_n = F'_n \times F'_{n + 1}$, for $n \geq 1$.

Let lexfunctor $\phi_0 : G_0 \to G_1$ be defined by $\phi_0(c_0, c_1) = (c_1, \straight{F}_1(c_1))$. It's straightforward to then verify that $G = \langle G_0, G'_1, G'_2, \ldots; \phi_0, \phi'_1, \phi'_2, \ldots \rangle$ is a geminal category (with the only nontrivial aspect being the equation $E_0$, as it were). \sTODOinline{Verify the trivial bits to be sure they're trivial, and the nontrivial bit to be sure it's also obvious enough to not need further comment}

We also clearly have a projection lexfunctor $H$ from $\underlying{G} = C_0 \times C_1$ to $C_0$, and by unfolding definitions, this does indeed satisfy $H[\InteriorGeminal{G}] = \gamma$.
\closeDetails
Having described the construction's details, we now prove that this construction has the stated terminality property:

\begin{proof}
Suppose given any geminal category $K = \langle K_0, K'_1, K'_2, \ldots; P_0, P'_1, P'_2, \ldots \rangle$ and lexfunctor $J : K_0 \to C_0$ such that $J[\InteriorGeminal{K}] = \gamma$. As ever, we will use unprimed names for the global aspects of primed names.

A lexfunctor $M$ from $\underlying{K} = K_0$ to $\underlying{G} = C_0 \times C_1$ is given by a pair of lexfunctors $J_0 : K_0 \to C_0$ and $J_1 : K_0 \to C_1$. Since $H$ is simply projection of the $C_0$ component, we will have that $H \circ M = J$ precisely when $J_0 = J$. Thus, specifying such $M$ is given by specifying $J_1$ alone. We must show that there is a unique $J_1$ making this $M$ into a geminal category homomorphism from $K$ to $G$.

Keeping in mind \magicref{VerboseGeminalCatHomoDefn}, adapted to this context, we see the conditions for such $M$ to be a geminal category homomorphism. First of all, we must have that $M[\InteriorGeminal{K}] = \InteriorGeminal{G}$, which is to say, $J[\InteriorGeminal{K}] = \gamma$ (which has already been presumed) and $J_1[\InteriorGeminal{K}] = \InteriorGeminal{\gamma}$. On top of this, the final condition for $M$ to be a geminal category homomorphism is that the following diagram commutes:

% https://q.uiver.app/#q=WzAsNCxbMCwwLCJLXzAiXSxbMiwwLCJDXzAgXFx0aW1lcyBDXzEiXSxbMiwyLCJDXzEgXFx0aW1lcyBDXzIiXSxbMCwyLCJLXzEiXSxbMCwxLCJNID0gXFxsYW5nbGUgSiwgSl8xIFxccmFuZ2xlIl0sWzAsMywiUF8wIiwyXSxbMywyLCJcXEluZHVjZWRIb21ve019e0snXzF9IiwyXSxbMSwyLCJcXHBoaV8wIFxcOyAgW1xcdGV4dHt3aGljaCBpc30gXFw7IChjXzAsIGNfMSkgXFxtYXBzdG8gKGNfMSwgXFxzdHJhaWdodHtGfV8xKGNfMSkpXSJdXQ==
\[\begin{tikzcd}
	{K_0} && {C_0 \times C_1} \\
	\\
	{K_1} && {C_1 \times C_2}
	\arrow["{M = \langle J, J_1 \rangle}", from=1-1, to=1-3]
	\arrow["{P_0}"', from=1-1, to=3-1]
	\arrow["{\InducedHomo{M}{K'_1}}"', from=3-1, to=3-3]
	\arrow["{\phi_0 \;  [\text{which is} \; (c_0, c_1) \mapsto (c_1, \straight{F}_1(c_1))]}", from=1-3, to=3-3]
\end{tikzcd}\]

This diagram commutes just in case both of the following diagrams commute, which separately consider its $C_1$ and $C_2$ components:

% https://q.uiver.app/?q=WzAsNCxbMCwwLCJLXzAiXSxbMiwwLCJDXzAgXFx0aW1lcyBDXzEiXSxbMiwyLCJDXzEiXSxbMCwyLCJLXzEiXSxbMCwxLCJNID0gXFxsYW5nbGUgSiwgSl8xIFxccmFuZ2xlIl0sWzAsMywiUF8wIiwyXSxbMywyLCJcXEluZHVjZWRIb21ve0p9e0snXzF9IiwyXSxbMSwyLCIoY18wLCBjXzEpIFxcbWFwc3RvIGNfMSJdLFswLDIsIkpfMSIsMV1d
\[\begin{tikzcd}
	{K_0} && {C_0 \times C_1} \\
	\\
	{K_1} && {C_1}
	\arrow["{M = \langle J, J_1 \rangle}", from=1-1, to=1-3]
	\arrow["{P_0}"', from=1-1, to=3-1]
	\arrow["{\InducedHomo{J}{K'_1}}"', from=3-1, to=3-3]
	\arrow["{(c_0, c_1) \mapsto c_1}", from=1-3, to=3-3]
	\arrow["{J_1}"{description}, from=1-1, to=3-3]
\end{tikzcd}\]

% https://q.uiver.app/#q=WzAsNSxbMCwwLCJLXzAiXSxbMiwwLCJDXzAgXFx0aW1lcyBDXzEiXSxbMiwyLCJDXzIiXSxbMCwyLCJLXzEiXSxbMSwxLCJDXzEiXSxbMCwxLCJNID0gXFxsYW5nbGUgSiwgSl8xIFxccmFuZ2xlIl0sWzAsMywiUF8wIiwyXSxbMywyLCJcXEluZHVjZWRIb21ve0pfMX17SydfMX0iLDJdLFsxLDIsIihjXzAsIGNfMSkgXFxtYXBzdG8gXFxzdHJhaWdodHtGfV8xKGNfMSkiXSxbMCw0LCJKXzEiLDFdLFs0LDIsIlxcc3RyYWlnaHR7Rn1fMSIsMV1d
\[\begin{tikzcd}
	{K_0} && {C_0 \times C_1} \\
	& {C_1} \\
	{K_1} && {C_2}
	\arrow["{M = \langle J, J_1 \rangle}", from=1-1, to=1-3]
	\arrow["{P_0}"', from=1-1, to=3-1]
	\arrow["{\InducedHomo{J_1}{K'_1}}"', from=3-1, to=3-3]
	\arrow["{(c_0, c_1) \mapsto \straight{F}_1(c_1)}", from=1-3, to=3-3]
	\arrow["{J_1}"{description}, from=1-1, to=2-2]
	\arrow["{\straight{F}_1}"{description}, from=2-2, to=3-3]
\end{tikzcd}\]

In each of the above diagrams, the top-right triangle trivially commutes, so the commutativity condition for the overall square is equivalent to the commutativity of the bottom-left triangle.

From the diagram for the $C_1$ component, we see that $J_1$ is uniquely determined as $\InducedHomo{J}{K'_1} \circ P_0$. All that remains is to verify that this choice of $J_1$ does indeed satisfy the condition $J_1[\InteriorGeminal{K}] = \InteriorGeminal{\gamma}$, as well as the condition of the commutative diagram for the $C_2$ component.

\bigskip
For the former condition, we have the chain of equations $J_1[\InteriorGeminal{K}]$ 

$ = \InducedHomo{J}{K'_1} [P_0[\InteriorGeminal{K}]] $ 

$= \InducedHomo{J}{K'_1} [\InteriorGeminal{\InteriorGeminal{K}}] $ 

$= \InteriorGeminal{J[\InteriorGeminal{K}]} $ 

$= \InteriorGeminal{\gamma}$. \sTODOinline{Are we sure all these equations are correct?}

\bigskip
And as for the final commutativity condition, this follows like so:

% https://q.uiver.app/#q=WzAsNixbMCwwLCJLXzAiXSxbMCwyLCJLXzEiXSxbNSwwLCJDXzEiXSxbNSwyLCJDXzIiXSxbMiwwLCJLXzEiXSxbMiwyLCJLXzIiXSxbMCwxLCJQXzAiLDJdLFsyLDMsIlxcc3RyYWlnaHR7Rn1fMSJdLFswLDQsIlBfMCIsMl0sWzQsMiwiXFxJbmR1Y2VkSG9tb3tKfXtLJ18xfSIsMl0sWzAsMiwiSl8xIiwwLHsiY3VydmUiOi01fV0sWzEsNSwiXFxJbmR1Y2VkSG9tb3tQXzB9e0snXzF9Il0sWzUsMywiXFxJbmR1Y2VkSG9tb3tKfXtLJ18yfSJdLFs0LDUsIlBfMSJdLFsxLDMsIlxcSW5kdWNlZEhvbW97Sl8xfXtLJ18xfSIsMix7ImN1cnZlIjo1fV0sWzEwLDQsIiIsMCx7InNob3J0ZW4iOnsic291cmNlIjoyMCwidGFyZ2V0IjoyMH0sInN0eWxlIjp7ImhlYWQiOnsibmFtZSI6Im5vbmUifX19XSxbNSwxNCwiIiwyLHsic2hvcnRlbiI6eyJ0YXJnZXQiOjIwfSwic3R5bGUiOnsiaGVhZCI6eyJuYW1lIjoibm9uZSJ9fX1dXQ==
\[\begin{tikzcd}
	{K_0} && {K_1} &&& {C_1} \\
	\\
	{K_1} && {K_2} &&& {C_2}
	\arrow["{P_0}"', from=1-1, to=3-1]
	\arrow["{\straight{F}_1}", from=1-6, to=3-6]
	\arrow["{P_0}"', from=1-1, to=1-3]
	\arrow["{\InducedHomo{J}{K'_1}}"', from=1-3, to=1-6]
	\arrow[""{name=0, anchor=center, inner sep=0}, "{J_1}", curve={height=-30pt}, from=1-1, to=1-6]
	\arrow["{\InducedHomo{P_0}{K'_1}}", from=3-1, to=3-3]
	\arrow["{\InducedHomo{J}{K'_2}}", from=3-3, to=3-6]
	\arrow["{P_1}", from=1-3, to=3-3]
	\arrow[""{name=1, anchor=center, inner sep=0}, "{\InducedHomo{J_1}{K'_1}}"', curve={height=30pt}, from=3-1, to=3-6]
	\arrow[shorten <=4pt, shorten >=4pt, Rightarrow, no head, from=0, to=1-3]
	\arrow[shorten >=4pt, Rightarrow, no head, from=3-3, to=1]
\end{tikzcd}\]

In the above diagram, the top equation is our definition $J_1 = \InducedHomo{J}{K'_1} \circ P_0$, and the bottom equation follows from this definition as well \sTODOinline{we may wish to clarify the reader's understanding of $\InducedHomo{J}{K'_2}$, which involves a doubly internal structure}. The left square commutes as part of the definition of $K$ being a geminal category, and the right square commutes because $J[P'_1] = F'_1$ (which was part of our presumption that $J[\InteriorGeminal{K}] = \gamma$). \sTODOinline{Make sure this all makes sense.}

This completes the proof of the terminality property of $G$.
\end{proof}

\sTODOinline{Observe that the above two co-free constructions can be combined, to find the terminal introspective theory with a suitable lexfunctor into a given lexcategory yielding a given internal geminal category}

\sTODOinline{Do a recap section}

\sTODOinline{Close this chapter with a discussion of how, in many contexts in mathematics, it is the geminal categories (the $\Glob{C}$) which seem to play a more primary role than the introspective theories (the $T$). For example, we are typically more interested in the full syntactic category of PA than the $\Sigma_1$-restricted syntactic category, or in presheaves on the discrete set of worlds (including arbitrary subsets of worlds as propositions) rather than specifically those presheaves which respect the accessibility relation (restricting us to only the \quote{open} propositions). Why, then, has our development of ideas in this document focused on introspective theories rather than geminal categories? This is partly a matter of taste. We could have taken geminal categories as the fundamental notion, and not considered any introspective theory other than the theory of geminal categories. We could still derive \Loeb/'s theorem for geminal categories and so on. But the definition of introspective theories is much cleaner than that of geminal categories, and permits this extraction of the definition of geminal categories as its initial model, which is a beautiful result, helping to motivate the definition of geminal categories. Having the three- or four-axiom concept of an introspective theory also sometimes helps us more easily recognize that an example is a geminal category, without having to verify the six axioms of a geminal category directly. Also, having the concept of introspective theories clarifies which presheaves we can apply Löb's theorem (or \magicref{IntrospTyConFixedPoints}) to. To emphasize this last point, it would be very useful to have an explicit example of a presheaf on a geminal category which we CANNOT apply Löb's theorem to (or perhaps cannot even make sense of the box operator applied to).

We should have a discussion section on comparing and contrasting introspective theories and geminal categories, and the motivation to study both and their relation.}

\sTODOinline{Demonstrate that we do NOT get Loeb's theorem internal to a geminal category G for arbitrary presheaves P on \underlying{G'}, thus demonstrating the necessity of the presheaf existing within an introspective theory. The presheaf needs to be parametrized by a parameter from an object of an enclosing introspective theory. So P(S(X)) |- []P(S(X)) is available.}

\fileend

% \section{Modal logic}
\subsection{The box operator}
The following notation will be very convenient for us going forward. It is also suggestive of connections with modal logic we will eventually explore.

Let $\langle T, C \rangle$ be a locally introspective theory. \TODOinline{Allow this notation for pre-introspective theories too?}

We say a presheaf on $C$ is locally $T$-small if the map from its category of elements to $C$ has $T$-small fibers. In other words, $P(c)$ is represented by an object of $T/t$ for each $t$-definable object $c$ of $C$.

Note that the category of locally $T$-small presheaves on $C$ is itself a $T$-indexed lexcategory. We will refer to this as $\Psh{C}$.

Thus, we have three $T$-indexed lexcategories of note: $T$ itself (considered as a $T$-indexed category through the self-indexing $T/-$), $C$, and $\Psh{C}$.

Between these, we also have a cycle of $T$-indexed lexfunctors, like so:

\[\begin{tikzcd}
	&& {T/-} \\
	\\
	C &&&& {\Psh{C}}
	\arrow["\introF"', from=1-3, to=3-1]
	\arrow["{c \; \mapsto \Hom_{C}(-, c)}"', from=3-1, to=3-5]
	\arrow["{P \mapsto P(1)}"', from=3-5, to=1-3]
\end{tikzcd}\]

Here, the bottom arrow is the Yoneda embedding, sending each object of $C$ to the corresponding representable presheaf. The right arrow takes a presheaf to its global elements. The left arrow is the $\introF$ which is part of the structure of an introspective theory.

In general, we will write $\Box$ for a roundtrip around this diagram, starting from any of its three nodes.

Thus, we will write $\Box$ for the $T$-indexed lexfunctor from $T$ to itself given by $t \mapsto \Hom_C(1, \introF(t))$.

We will ALSO write $\Box$ for the $T$-indexed lexfunctor from $C$ to itself given by $c \mapsto \introF(\Hom_C(1, c))$.

And we will ALSO write $\Box$ for the $T$-indexed lexfunctor from $\Psh{C}$ to $\Psh{C}$, which sends the presheaf $P$ to the presheaf represented by $\introF(P(1))$.

When we want to clarify precisely the domain we are operating on, we may write $\Box_T$, $\Box_C$, or $\Box_{\Psh{C}}$, as appropriate.

As the Yoneda embedding is naturally thought of as the inclusion of a full subcategory, identifying $C$ with the corresponding representable presheaves within $\Psh{C}$, we may also think of this last instance of $\Box$ as a $T$-indexed lexfunctor from $\Psh{C}$ to $C$.

Note that, having set up these various notions of $\Box$, we find that $\Box$ and each of the maps in the diagram \quote{commutes} in the appropriate sense; that is, they can be seen as preserving each other.

For example, $\introF$ preserves $\Box$, in that both $\introF(\Box(-))$ and $\Box(\introF(-))$ yield the same $T$-indexed functor from the self-indexing of $T$ to $C$. This is readily seen by unwinding their definitions: These are both $\introF(\Hom_C(1, \introF(-)))$.

We also have that taking $\Hom_C(1, -)$ preserves $\Box$ in the same way. $\Hom_C(1, \Box(-)) = \Box(\Hom_C(1, -)) = \Hom_C(1, \introF(\Hom_C(1, -)))$. More generally, consider the map $G$ which assigns to every locally $T$-small presheaf $P$ upon $C$ its object of global elements $P(1)$. Then $G$ preserves $\Box$ in the same way: $G(\Box P) = \Box(G(P)) = \Hom_C(1, \introF(P(1)))$, where we make sense of $G(\Box P)$ by identifying $\Box P$ within $c$ with the presheaf it represents.

Given object $c$ of $C$, and locally $T$-small presheaf $P$ on $C$, we will write $c \implies P$ to indicate the exponential presheaf $P^c$. That is, the presheaf $P(- \times c)$. We may also write $c \implies d$ where $d$ is an object of $C$ as well, to mean $c \implies P$ for the presheaf $P = \Hom_C(-, d)$ represented by $d$. Thus, $c \implies d$ is the presheaf $\Hom_C(- \times c, d)$.

Note that for any locally $T$-small presheaf $P$ on $C$ and object $c$ of $C$, we have that $P(c)$ can be identified with the presheaf $c \implies P$ evaluated at $1$, and thus $\introF(P(c))$ can be identified with $\Box(c \implies P)$. In particular, by considering the case when $P$ is represented by object $d$, we find that $\Box(c \implies d)$ is naturally identifiable with $\introF(\Hom_C(c, d))$.

The above was all discussed for $T$, $C$, and $\Psh{C}$ considered as $T$-indexed lexcategories, but this all descends to corresponding structure on their global aspects as well. Keep in mind, in the global aspect context, $\introF$ is the same as $\introS$, so wherever in the above we discussed $\introF$, a corresponding statement holds as well for $\introS$, when considering just the global aspect.

The choice of this $\Box$ notation for these purposes is meant to convey an analogy with the $\Box$ operator of modal logic, and in particular, with the provability operator of provability logic. We will explore this more in later remarks.

The key point here is that the rules of the $\Box$ operator in a Kripke normal modal logic are essentially the rules of a lex endofunctor on a category, and any of our $\Box$ operators is certainly lex as a composite of lex functors.

Furthermore, each of our $\Box$ operators comes with a natural transformation from $\Box$ to $\Box \Box$ corresponding to the so-called 4 axiom $\Box A \vdash \Box \Box A$.

For the $\Box$ operator on $T$ this is clear, as the natural transformation $\introN$ from identity to $\Box$ encodes the even stronger property $t \vdash \Box t$. The 4 axiom is the special case where $t$ here is itself of the form $\Box A$.

We do not have such a strong natural transformation from identity to $\Box$ as acting on the other corners of the triangle ($C$ or $\Psh{C}$). However, by taking the natural transformation from $t$ to $\Box_T t$, whiskering it on both sides along the trips from any other corner of the triangle into and out of $T$, and then applying the commutativity of $\Box$ with each leg of the triangle, we get a natural transformation from $\Box$ to $\Box \Box$ at each other corner of the triangle as well.

Thus, our $\Box$ follows all the rules of the modal logic K4, in each of these contexts. Later, we shall see that conversely, the general logic followed by $\Box$ in a locally introspective theory is no stronger than K4, while in an introspective theory, it is the modal logic GL. Indeed, in the very next chapter we will see how in an introspective theory we get the last ingredient for the modal logic GL, \Loeb's theorem.

% \filestart

\section{Examples in the wild}

\subsection{Preview}
In previous chapters, we have defined introspective theories and geminal categories. That is, we have axiomatized the theory of introspective theories and the theory of geminal categories. Now we look at some notable models of these axiomatic theories, which is to say, at some notable specific examples of introspective theories and of geminal categories. These examples are of a sort which might be considered to have been found \quote{in the wild}, instead of being freely syntactically constructed as the examples of the last chapter were.

There are two broad classes of models/examples of note in this chapter:

Firstly, there are those which are similar in flavor to the traditional instances of \Goedel/ian phenomena studied in logic. These are based on logical theories which have some internal ability to discuss themselves, such as Peano Arithmetic, or higher-order intuitionistic logic, or the like. Here, it has long been recognized that \Goedel/ian phenomena arise at the propositional level, but the full phenomenon of guarded recursion which we proved for introspective theories in \TODO in has not been noted in these contexts before. We also give an example of a model of this sort which goes well beyond computability or even countability, thus beyond many traditional approaches to presenting the \Goedel/ian phenomena in logic.

The second class of models/examples we consider are more similar in flavor to the traditional interpretation of the modal logic GL in well-founded orders. Here, the existence of guarded recursion is straightforward, but it is the unification with our general theory which is of note. Among these models are examples like the topos of trees, the canonical model discussed in the literature on guarded recursion. We also demonstrate similar but distinct models which support an interpretation of Boolean provability logic, as opposed to the fundamentally intuitionistic logic of the topos of trees.

\subsection{The main initiality-based construction}
\begin{construction}\label{SpecialInitialIntrosp}
Let $Special$ be a left comma-stable sub-2-category of $\LexCat$, in the sense of \magicref{CommaStableDefn}. Furthermore, suppose $Special$ has an initial object $T$, and that this $T$ has an internal lexcategory $C$ such that $\Glob{C}$ is itself an object of $Special$.

Then we obtain a unique lexfunctor $\introS \in Special(T, \Glob{C})$, by the initiality of $T$.

Furthermore, by \magicref{CommaKanStrongSigmesque}, we have that $\id_T$ is initial within $\LexCat(T, T)$. Thus, in particular, there is a unique natural transformation $\introN : \id_T \to \Hom_C(1, \introS(-))$. In this way, we obtain an introspective theory $\langle T, C, \introS, \introN \rangle$.
\end{construction}

\begin{theorem}\label{SpecialInitialIntrospIsInitial}
Let $Special$ and $\langle T, C, \introS, \introN \rangle$ be given as in \magicref{SpecialInitialIntrosp} above.

Consider also any other introspective theory $\langle T', C', \introS', \introN' \rangle$ such that $\introS' : T' \to \Glob{C'}$ lives in $Special$. By the initiality of $T$ within $Special$, we get a unique special lexfunctor $H : T \to T'$. If this $H$ is such that $H[C] = C'$, then this $H$ is also an introspective theory homomorphism (in the sense of \TODOinline{cite}). The condition that $H$ interacts appropriately with $\introS$ and $\introS'$ is automatic by the initiality of $T$ within $Special$. Furthermore, the condition that $H$ interacts appropriately with $\introN$ and $\introN'$ is automatic by the fact that $H$ is initial within $\LexCat(T, T')$, thanks to \magicref{CommaKanStrongSigmesque}.
\end{theorem}

\subsection{Self-initializing and super-initializing theories}
\subsubsection{The initial model as a geminal category}
\begin{construction}\label{InitoGeminalYieldsGeminal}
Suppose given some lexcategory $Th$ (the theory of \quote{gadgets}), along with a lexcategory $C$ internal to $Th$ (the underlying lexcategory of a gadget).

Furthermore, suppose given an initial gadget $G_1$ with an initial internal gadget $G_2$. That is, suppose given some lexcategory $V$ such that $\LexCat(Th, V)$ has an initial object (our $G_1$) and such that $\LexCat(Th, \Glob{G_1[C]})$ has an initial object (our $G_2$).

Because $G_1$ is initial, we automatically get a unique homomorphism $F_1 : G_1 \to \Gamma{G_2}$. And because $G_2$ is an initial $G$-internal gadget, we automatically get a unique $G_1$-internal homomorphism $F_2 : G_2 \to \Gamma[G_3]$ where $G_3 = F_1[G_2]$.

This setup is thus a geminal gadget internal to $V$ (with the equations $E_1$ and $E_2$ of \magicref{CompactGeminalCatDefn} automatically satisfied by the uniqueness observations in the previous paragraph).

Indeed, this is the unique way to equip $\langle G_1, G_2 \rangle$ as a geminal gadget.
\end{construction}

In practice, when an initial gadget has an initial internal gadget like above, this is usually not just some accident (caused by a paucity of globally defined structures, say), but rather, is due to the theory of gadgets itself encoding the construction of an internal initial gadget:
\begin{definition}
Suppose, as above, given some lexcategory $Th$ (the theory of \quote{gadgets}), along with a lexcategory $C$ internal to $Th$ (the underlying lexcategory of a gadget).

If every gadget has an initial internal gadget, and every gadget homomorphism preserves these initial internal gadgets, then we say the theory of gadgets is \defined{self-initializing}.

In other words, $Th$ is self-initializing if $\LexCat(Th, \Glob{C})$ has an initial object, and this initiality is preserved by $\InducedHomo{f}{C}$ for every lexfunctor $f$ out of $Th$.
\end{definition}

The above all admits a generalization worth noting:

\begin{construction}\label{SuperInitoGeminalYieldsGeminal}
Suppose given some lexfunctor $i : Th \to Th'$, along with a lexcategory $C$ internal to $Th$. Call $Th$ the theory of \quote{gadgets}, and $Th'$ the theory of \quote{supergadgets}. Via $i$, every supergadget has an underlying gadget, and via $C$, every gadget has an underlying lexcategory.

Furthermore, suppose given an initial gadget $G_1$ with an initial internal supergadget $G_2$. That is, suppose given some lexcategory $V$ such that $\LexCat(Th, V)$ has an initial object (our $G_1$) and such that $\LexCat(Th', \Glob{G_1[C]})$ has an initial object (our $G_2$).

Because $G_1$ is initial, we automatically get a unique gadget homomorphism $F_1: G_1 \to \Gamma{G_2}$. And because $G_2$ is an initial $G_1$-internal supergadget, we automatically get a unique $G_1$-internal supergadget homomorphism $F_2 : G_2 \to \Gamma[G_3]$ where $G_3 = F_1[G_2]$.

This setup is thus a geminal gadget internal to $V$ (with the equations $E_1$ and $E_2$ of \magicref{CompactGeminalCatDefn} automatically satisfied by the uniqueness observations in the previous paragraph).

Indeed, this is the unique way to equip $\langle G_1, G_2 \rangle$ as a geminal gadget $\langle G_1, G_2; F_1, F_2 \rangle$ such that $F_2$ comes from a supergadget homomorphism.
\end{construction}

And again, in practice, when an initial gadget has an initial internal supergadget like above, this is usually not just some accident caused by a paucity of globally defined structures, but rather, is due to the theory of gadgets itself encoding the construction of an internal initial supergadget:

\begin{definition}
Suppose, as above, given some lexfunctor $i : Th \to Th'$, along with a lexcategory $C$ internal to $T$. We call $T$ the theory of \quote{gadgets}, and $Th'$ the theory of \quote{supergadgets}. Via $i$, every supergadget has an underlying gadget, and via $C$, every gadget has an underlying lexcategory.

If every gadget has an initial internal supergadget, and every gadget homomorphism preserves these initial internal supergadgets, then we say the theory of gadgets (or more precisely, the extension of the theory of gadgets by the theory of supergadgets) is \defined{super-initializing}.

In other words, this situation is super-initializing if $\LexCat(Th', \Glob{C})$ has an initial object, and this initiality is preserved by $\InducedHomo{f}{C}$ for every lexfunctor $f$ out of $Th$.

Note in this case that $Th'$ will itself be self-initializing, as every supergadget is a fortiori a gadget (thus having an initial internal supergadget), and every supergadget homomorphism is a fortiori a gadget homomorphism (thus preserving initial internal supergadgets).
\end{definition}

The self-initializing situation is of course the special case of the super-initializing situation where $Th' = Th$ and $i$ is the identity.

There are a number of self- and super-initializing theories in the wild, which thus immediately give us examples of geminal categories in the wild.

For example: It is straightforward to show that every NNO-topos has internal initial models of every finitely axiomatizable lex theory, preserved by every NNO-topos homomorphism \TODOinline{Cite in preliminaries}.

It is a little more difficult, but also possible to show that more generally, every arithmetic universe has internal initial models of every finitely axiomatizable lex theory, preserved by every arithmetic functor. \TODOinline{Cite in preliminaries}

Thus, any finitely axiomatizable extension of the theory of arithmetic universes is self-initializing. More generally, given any $Th$ extending the theory of arithmetic universes, and any finitely axiomatizable $Th'$ extending $Th$, the extension of $Th$ to $Th'$ is super-initializing. 

This immediately gives us many examples of geminal categories using the above construction. For example, as one random example among myriad, we can obtain a geminal category $\langle G_1, G_2 \rangle$ where $G_1$ is the initial cartesian closed arithmetic universe and $G_2$ is its internal initial NNO-topos satisfying the internal axiom of choice.

\TODOinline{Note that when a theory is super-initializing, then we have that in the geminal category $\gamma = \langle G_1 G_2; F_1, F_2 \rangle$, $G_3 = F_1[G_2]$ is also an initial supergadget internal to $G_2$, and $F_3 = F_1[F_2]$ is a supergadget homomorphism. Thus, $\InteriorGeminal{\gamma}$ is itself given by the same construction of a geminal category for the self-initializing theory of supergadgets.}

\begin{TODOblock}
When the \initogeminal/ theory T is such that furthermore Set is a model of T, and we have the strong property noted above that EVERY homomorphism out of the initial model of T preserves its interior initial model's initiality, then we furthermore get a soundness result here: The map from the initial model M of T (the one defined by global elements in the lexcategory corresponding to T) to Set takes M's interior initial model of T to the actual M in Set, and so on.

Thus, we get a homomorphism from the globalization of M's interior initial model of T to the actual M. Since M is initial, this homomorphism onto it is a retraction. Thus, the map from M into the globalization of its internal model has a left inverse.

Note the following two caveats:

1. This composition needn't be identity in the other order.

Proof: Let p be a proposition which is independent from PA, and consider a term t such as "3 if p is true, 5 if p is false", well-defined classically. The map from M' to M to M' will take the term t living in M' to its interpretation as an actual particular number (either 3 or 5) and then to that actual particular number as a canonical term (either "3" or "5"), but that canonical term will not be equal to the original term since t is neither provably equal to 3 nor provably equal to 5.

2. If Set is not a model of T to begin with, there needn't even be a homomorphism from M's internal model to M in the first place.

Proof: Let T be a theory corresponding to PA + ~Con(PA) or the like. Then the initial model M of T is nontrivial, i.e. doesn't prove 1 = 0 (since PA + ~Con(PA) is consistent, since PA doesn't prove Con(PA), by G2IT). However, M's internal initial model of T is trivial, i.e. proves 1 = 0 (since T proves ~Con(PA) which in turn entails ~Con(T)). We cannot have a homomorphism from the latter to the former.
\end{TODOblock}

We have discussed all this just in the context of geminal categories, but this extends to give analogous constructions of introspective theories as well. We discuss these next.

\TODOinline{Maybe delete most of this section (except for the definitions of self- and super-initializing theories) and only give examples of introspective theories.}

\subsubsection{The theory of initial models as an introspective theory}
Throughout the following, we use the \quote{initial model} terminology of \magicref{LexModelTerminology}. As a reminder, we say a lexfunctor $f : T \to S$ is an initial model of $T$ if it is initial within the category $\LexCat(T, S)$.

By the 2-category $\initMod{T}$, we mean $\LexCat$ with its objects restricted to just those lexcategories with initial models of $T$, and its $1$-cells restricted to just those lexfunctors which preserve initial models of $T$. (The $2$-cells remain unchanged.)

\begin{theorem}\label{InitialModelWithInitialModel}
For every \setsmall/ lexcategory $Th$, there is an initial object within $\initMod{Th}$.
\end{theorem}
\begin{proof}
This is in exactly the same way that we have familiar constructions such as of the initial NNO-topos, the initial arithmetic universe, the initial lexcategory with countable products, etc.

In more detail, the category of strict lexcategories with internal initial models of $Th$, and strict lexfunctors strictly preserving these internal initial models, is the category of models of an infinitary quasi-equational theory (whose infinitary operations have arity bounded by a cardinal dependent on the size of $Th$), and thus has an initial object. This initial strict structure furthermore is initial in the non-strict context, because all the relevant operations (finite limits, initial models of $Th$) are given by universal properties, so that any functor out of the initial strict structure preserving these in a non-strict sense is canonically isomorphic to a functor preserving these strictly on the nose.
\end{proof}

\begin{construction}
If $Th$ is a self-initializing theory, then $\initMod{Th}$ is left comma-stable within $\LexCat$, via \magicref{InitialModelCommaStable}. Furthermore, it has an initial object $T$ via \magicref{InitialModelWithInitialModel}.

This $T$ by definition has an initial internal model of $Th$; that is, there is an initial $f \in \LexCat(Th, T)$. Furthermore, since $T$ is self-initializing, it contains an internal category $C$ such that $\Glob{f[C]}$ itself is an object of $\initMod{Th}$.

We can thus invoke \magicref{SpecialInitialIntrosp} to obtain a unique introspective theory $\langle T, f[C], \introS, \introN \rangle$ where $\introS$ is a map in $\initMod{Th}$.
\end{construction}

There is an extension of the above construction to super-initializing theories. However, it is a bit trickier. The key issue is to construct, for a super-initializing lexfunctor $i : Th \to Th'$, a lexcategory which captures simultaneously the properties which are shared by initial models of $Th$ and by initial models of $Th'$. We sketch out the construction as follows:

\begin{construction}
Let $i : Th \to Th'$ be a lexfunctor, such that models of $Th$ are considered gadgets, models of $Th'$ are considered supergadgets, and via $i$ every supergadget is thought of as having an underlying gadget. Given a lexcategory $L$, we will say that a $Th'$-initial model of $Th$ in $L$ is an internal supergadget $\beta$ in $L$, along with, for every internal gadget $\alpha$ in $L$, a chosen gadget homomorphism from $h_\alpha : \beta \to \alpha$, such that furthermore, these chosen homomorphisms are closed under postcomposition with supergadget homomorphisms (that is, for any supergadget homomorphism $f : \beta \to \beta'$ in $L$, we have that $f \circ h_{\beta} = h_{\beta'}$, as gadget homomorphisms). \TODOinline{Illustrate this diagrammatically}. Note that this structure is NOT given by a universal property! There may be multiple non-equivalent ways to choose such structure within $L$. \TODO \TODOinline{Note that for super-initializing, the relevant thing is an X that has a designated map to each Y, and these designated maps are closed under post-composing Y-homomorphisms. There is NOT uniqueness, though. This construction is all a little tricky, requiring modifying both the Comma-Kan lemma and the comma category sigma1esqueness lemma, in addition to describing why an initial such thing exists, and so on and so on, so maybe TODO it for the time being.}
\end{construction}

\subsubsection{A self-initializing theory with uncountable and uncomputable flavor}
\TODOinline{Consider the free topos with countable (co)products, which has an internal free topos with countable (co)products as well, with internal and external views coinciding on what things have or preserve countable coproducts.}

\TODOinline{Note that this contains true arithmetic, but is still subject to \Loeb/'s theorem and \Goedel/'s incompleteness theorems. Its just that its consistency sentence isn't expressible in first-order arithmetic or indeed by quantification over naturals.}

\sTODOinline{A self-initializing theory with countable but uncomputable flavor}

\subsection{The initial arithmetic universe}
\begin{construction}\label{IAUAsIntrospGeneral}
Let $\IAU$ be the initial arithmetic universe, and let $C$ be any arithmetic universe internal to $\IAU$. Then by the combination of \magicref{SpecialInitialIntrosp} and \magicref{CommaStableArithmetic}, we obtain an introspective theory $\langle \IAU, C, \introS, \introN \rangle$ in which $\introS : \IAU \to \Glob{C}$ is the unique such arithmetic functor, and the natural transformation $\introN : \id_T \to \Hom_C(1, \introS(-))$ is uniquely determined.
\end{construction}

\begin{observation}
Note that the above construction can be applied using ANY arithmetic universe internal to $\IAU$. One natural choice is where $C$ is taken to be the initial arithmetic universe internal to $IAU$ (which exists thanks to \TODOinline{cite from Preliminaries}). In this case, the natural transformation $\introN$ we obtain is the same as the one constructed in Lemma 5.15 of \autocite{van2020g}.
\end{observation}

We now use \magicref{IAUAsIntrospGeneral} to give a fuller account of our original guiding example of an introspective theory based on traditional logical theories, from \magicref{SigmaModelComplex}.

\begin{observation}
The introspective theory described in \magicref{SigmaModelComplex} is the maximal localization, in the sense of \magicref{LocalizeIntrosp}, of an introspective theory produced by \magicref{IAUAsIntrospGeneral}.
\end{observation}
\begin{proof}
Recall the categories $\Zfin$ and $\ZfinSigma$ from \magicref{SigmaModelComplex}. Here, $\Zfin$ is an exact category whose objects and morphisms correspond to definable classes and graphs of functions between these in the theory ZF-Finite, with morphisms taken modulo provable equality in ZF-Finite. While $\ZfinSigma$ is the subcategory of $\Zfin$ where the definability conditions are further restricted to $\Sigma_1$-definability.

It is readily verified that $\Zfin$ is an arithmetic universe. Thus, there is a unique arithmetic functor $!_{\Zfin} : \IAU \to \Zfin$. Let $M$ be the set of morphisms in $\IAU$ which are taken to isomorphisms by this $!_{\Zfin}$. By \magicref{ArithmeticLocalization}, this $!_{\Zfin}$ factors uniquely through the arithmetic localization $\IAU[M^{-1}]$. \TODOinline{Illustrate}. Using \magicref{IsLexLocalizationLemma}, it is straightforwardly, if tediously, verified that this $\IAU[M^{-1}]$ is in fact $\ZfinSigma$, with $!_{\Zfin}$ thus being the unique arithmetic functor from $\IAU$ to $\ZfinSigma$ followed by the inclusion from $\ZfinSigma$ to $\Zfin$. That is to say, the role played by the $\Sigma_1$ constraints in defining $\ZfinSigma$ is precisely to make $\ZfinSigma$ an arithmetic localization of $\IAU$.

Note also that, as $\Zfin$ and $\ZfinSigma$ are both computably enumerable arithmetic universes internal to $\Set$, we find, in keeping with \magicref{ComputableMeansIAUInternal}, that these are the images in $\Set$ of arithmetic universes internal to the initial arithmetic universe $\IAU$. That is, letting $\Glob_{IAU}$ be the unique arithmetic functor from $\IAU$ to $\Set$ (which is the same as the global sections funtor $\Hom_{\IAU}(1, -)$, thanks to \magicref{GlobalIsArithmeticOnIAU}), we have arithmetic universes $\Glob_{IAU}^{-1}[\Zfin]$ and $\Glob_{IAU}^{-1}[\ZfinSigma]$ such that the images of these under $\Glob_{IAU}$ are $\Zfin$ and $\ZfinSigma$, respectively.

Via \magicref{SpecialInitialIntrosp}, we thus obtain an introspective theory $\langle \IAU, \Glob_{IAU}^{-1}[\Zfin] \rangle$.

Note in this introspective theory that $\introS : \IAU \to \Glob{\Glob_{IAU}^{-1}[\Zfin]} = \Zfin$ is the unique arithmetic functor from $\IAU$ to $\Zfin$. Thus the set of morphisms in $\IAU$ sent to isomorphisms by this $\introS$ is the same as the $M$ defined above.

Now let $\langle \ZfinSigma, \InnerZfin \rangle$ be the introspective theory described in \magicref{SigmaModelComplex}.

It is readily verified that $\InnerZfin$ and $\Glob_{IAU}^{-1}[\Zfin]$ can be chosen so that the former is the image of the latter under the unique arithmetic functor $!_{\ZfinSigma} : \IAU \to \ZfinSigma$. Furthermore, it is readily verified that $\introS : \ZfinSigma \to \Glob{\InnerZfin}$ is an arithmetic functor. Thus by \magicref{SpecialInitialIntrosp}, the unique arithmetic functor $!_{\ZfinSigma} : \IAU \to \ZfinSigma$ is in fact an introspective theory homomorphism from $\langle IAU, \Glob_{IAU}^{-1}[\Zfin] \rangle$ to $\langle \ZfinSigma, \InnerZfin \rangle$.

Since $!_{\ZfinSigma} : \IAU \to \ZfinSigma$ was, as noted above, the same as the arithmetic localization $\IAU \to \IAU[M^{-1}]$, we may invoke \magicref{LocalizeIntrosp} to conclude that the introspective theory homomorphism from $\langle IAU, \Glob_{IAU}^{-1}[\Zfin] \rangle$ to $\langle \ZfinSigma, \InnerZfin \rangle$ is the same as the localization of the introspective theory $\langle IAU, \Glob_{IAU}^{-1}[\Zfin] \rangle$ at $M$, which by the observation three paragraphs ago is the maximal localization of this introspective theory.

This concludes the proof.
\end{proof}


\TODOinline{Note how the above therefore extends easily to considerations of any traditional logical theory, not just ZF-Finite. Any such theory gives rise to an arithmetic universe internal to IAU, and we can then take the corresponding localized introspective theory. For example, PA, or vNBG, or PA + whatever computably enumerable set of axioms, etc.}

\TODOinline{Note that basically nothing here is special about IAU. We could similarly construct introspective theories using initial objects of any structure left comma-stable over LexCat, given any structure of the same kind internal to the initial one. What's special about IAU is just that it happens to actually contain interesting structures internally (such as PA, ZFC, the initial internal AU, etc), whereas the initial lexcat, or initial regular category, or initial lexcat with finite pullback-stable colimits, or such things all don't have much interesting internally. An example that DOES have interesting internal structure, however, might be the free lexcategory with pullback-stable disjoint etc countable coproducts.}

\subsection{Models based on well-founded posets or semicategories}
There are two flavors of models here: Those which give introspective theories (these come from well-founded trees using a certain size restriction; e.g., considering a model based on the von Neumann universe/cumulative hierarchy), and those which give only locally introspective theories with \Loeb/'s theorem fixed points (these come from arbitrary well-founded trees; these are related to the models used in guarded recursion theory, but our distinction between the roles of $T$ and $C$ has previously gone unnoticed and allows us to interpret these models as not proving $\lnot \lnot \Box 0$). We discuss the latter construction first, as it is simpler, and a step en route to grasping the former construction.

Previous iterations of this document at this point gave an overly complicated way of describing something simple (though still good to understand):

First of all, let $Disc$ be an arbitrary (set-sized) category. This gives rise also to the category $\Psh{Disc}$ of presheaves on $Disc$, which is automatically a lexcategory, and indeed locally cartesian closed. By the observation of \cref{TrivialPreIntrosp}, this yields a locally introspective theory $\langle \Psh{Disc}, \Psh{Disc}/-, \id \rangle$.

Now, let $f : Disc \to Struct$ be an arbitrary functor from $Disc$ into an arbitrary (also set-sized) category $Struct$. This induces by composition a functor $f^* : \Psh{Struct} \to \Psh{Disc}$. This $f^*$ preserves pullbacks (as pullbacks are computed pointwise in presheaf categories. Indeed, $f^*$ furthermore preserves all limits, as it has a left adjoint given by left Kan extension). This $f^*$ also has a right adjoint (given by right Kan extension).

\TODOinline{The above tells us that any geometric morphism between locally cartesian closed categories induces in the same way a locally introspective theory.}

By now using \cref{IntrospPullback} with our functor $f^*$ as applied to our first locally introspective theory $\langle \Psh{Disc}, \Psh{Disc}/-, \id \rangle$, we get a second locally introspective theory $\langle \Psh{Struct}, \Psh{Disc}/- \circ f^*, \ldots \rangle$.

This is ALMOST the locally introspective theory we are interested in for Kripke semantics. But it needs to be massaged a bit more, in a manner requiring some further assumptions. \TODOinline{Note that if we stop right here, we get a natural notion of model corresponding to S4 Kripke frames.}

First, a lemmatic construction. Suppose given any arbitrary profunctor $H : X \profuncTo Y$. This $H$ induces by profunctor composition (with profunctors $:1 \profuncTo X$, which correspond to presheaves on $X$) correspondingly an ordinary functor $H \circ - : \Psh{X} \to \Psh{Y}$. Note that this functor $H \circ -$ has a right adjoint (right Kan lift of a profunctor along a profunctor).

(Alternatively, we can think of the above like so: Given (set-sized) categories $X$ and $Y$ and any arbitrary functor $H : X \to \Psh{Y}$, this extends uniquely to a (set-sized-)colimit preserving functor $: \Psh{X} \to \Psh{Y}$, as $\Psh{X}$ is the free cocompletion of $X$ (with respect to set sized colimits). This functor is the one we call $H \circ -$, and by the Special Adjoint Functor Theorem, it will have a right adjoint.)

\TODOinline{Wherever above I put a set-sized constraint on a category, it sounds like I am constraining the category to not be too large. But really what this amounts to is to say that the corresponding presheaf category we are considering must not be too small: they must include presheaves of sufficiently high cardinality relative to the original category.}

If given two such $H_1, H_2$ and a transformation $n : H_1 \to H_2$, this extends also to a transformation $n \circ -$ from $H_1 \circ -$ to $H_2 \circ -$ as ordinary functors $: \Psh{X} \to \Psh{Y}$.

Let us now suppose that $Struct$ (from before) is in fact the free category adding identities to some semicategory $Struct^-$. Then we have a bifunctor $\Hom_{Struct^-} : \op{Struct} \times Struct \to \Set$, as the morphisms of $Struct^-$ are not only closed under composition with each other, but also (trivially) under composition with identities on either side, and thus closed under composition on either side with the morphisms of $Struct$. 

This bifunctor $\Hom_{Struct^-} : \op{Struct} \times Struct \to \Set$ comes with an inclusion transformation to the bifunctor $\Hom_{Struct} : \op{Struct} \times Struct \to \Set$. These bifunctors can both be read as profunctors from Struct to Struct; the latter is in fact the identity bifunctor on Struct, and the former is what we will take to be our $H$ as above. The inclusion transformation thus will become an inclusion transformation $i$ from $H \circ -$ to identity as functors $: \Psh{Struct} \to \Psh{Struct}$.

These comprised the last ingredients we needed for proper Kripke semantics for irreflexive frames. Remember, we already had a locally introspective theory $\langle \Psh{Struct}, \Psh{Disc}/- \circ f^*\rangle$ from above. Let us call this $\langle \Psh{Struct}, C \rangle$ for convenience. We now modify it like so using: \cref{IntrospInternalMap}.

% https://q.uiver.app/?q=WzAsMyxbMCwwLCJcXG9we1xcUHNoe1N0cnVjdH19Il0sWzIsMCwiXFxMZXhDYXQiXSxbMSwyLCJcXG9we1xcUHNoe1N0cnVjdH19Il0sWzAsMSwiXFxQc2h7U3RydWN0fS8tIiwwLHsib2Zmc2V0IjotMn1dLFswLDEsIkMiLDIseyJvZmZzZXQiOjJ9XSxbMCwyLCJIIFxcY2lyYyAtIiwyXSxbMiwxLCJDIiwyXSxbMyw0LCIiLDAseyJzaG9ydGVuIjp7InNvdXJjZSI6MjAsInRhcmdldCI6MjB9fV0sWzQsMiwiQyBcXG9we2l9IiwxLHsic2hvcnRlbiI6eyJzb3VyY2UiOjIwfX1dXQ==
\[\begin{tikzcd}
	{\op{\Psh{Struct}}} && \LexCat \\
	\\
	& {\op{\Psh{Struct}}}
	\arrow[""{name=0, anchor=center, inner sep=0}, "{\Psh{Struct}/-}", shift left=2, from=1-1, to=1-3]
	\arrow[""{name=1, anchor=center, inner sep=0}, "C"', shift right=2, from=1-1, to=1-3]
	\arrow["{H \circ -}"', from=1-1, to=3-2]
	\arrow["C"', from=3-2, to=1-3]
	\arrow[shorten <=1pt, shorten >=1pt, Rightarrow, from=0, to=1]
	\arrow["{C \op{i}}"{description}, shorten <=7pt, Rightarrow, from=1, to=3-2]
\end{tikzcd}\]

Keeping in mind that $H \circ - : \Psh{Struct} \to \Psh{Struct}$ has a right adjoint, we may conclude that the result is a locally introspective theory. When we start off taking $Struct^-$ to be a Kripke frame presumed transitive but not reflexive, taking Disc to be the discrete category on the same objects as Struct-, and $f : Disc \to Struct$ to be the inclusion, then the result of the above process is the introspective theory which corresponds to Kripke semantics on $Struct^-$. \TODOinline{Write out in more detail what the construction comes down to and thus showing how it corresponds to traditional Kripke semantics.}

The result will be locally Loeb when the order on the objects of Struct- given by its morphisms is a converse well-founded order. \TODOinline{Expand on this}. We can then impose a suitable size constraint to get it to be fully introspective.

\begin{TODOblock}
Clarify the size constraint. Note that it is very common in mathematics to take the relative point of view on Set, in terms of Grothendieck universes or the like, so as to consider the topos $Set^K$ as built up from a bunch of full subtoposes defined by a global cardinality constraint: those presheaves whose cardinality at each object is constrained by an upper bound, and this upper bound is the same at each object. But there is no reason we must only consider such constant upper bounds. We can just as well consider all kinds of varying upper bounds. And by allowing the the upper bounds to vary in the appropriate way, growing sufficiently fast, we get that $Set^K$ is built up from a bunch of full subtoposes which are all introspective theories. It is like a shift of frame of reference, to allow the upper bounds to vary with suitable \quote{slope} instead of having to be constant. But it serves all the same purposes as the very standard move in mathematics, of taking a relative point of view on Set.

Note that while typical categorical arguments work within structural set theory, the above can be done most readily within a material set theory. Furthermore, while typical categorical arguments work within the internal logic of toposes with NNO or some such thing, the above requires us to move beyond this, and is done most readily using the Axiom of Replacement. Thus, ZF or IZF or the like. Specifically, take Set to be a material set theory and a strict lexcategory, and take a cardinality constraint at a node to be a set of sets (corresponding to a full subcategory of Set) satisfying the condition that this full subcategory is closed under finite limits. Then we furthermore impose the condition that the full subcategory at node X contains the small category of all discrete presheaves on < X and all natural transformations between them. Using transfinite induction, we can easily define a function from nodes to sets that has this property.

The reason we must use the Axiom of Replacement is essentially because the initial algebra/transfinite recursion properties of well-founded sets within a mere topos $T$ are only with respect to endomorphisms of the subobject functor (which is representable, and thus such endomorphisms are themselves represented by endomorphisms on $\Omega$, living internally to the category), and not with respect to natural transformations of the self-indexing more generally (which is not representable, and thus its endomorphisms are not given by some internal data). Even simple natural transformations of the self-indexing such as the powerset operation on indexed sets may not admit corresponding catamorphisms defined by induction (e.g., there is in general no slice above the natural numbers in which the fiber of n + 1 is the powerset of the fiber of n).
\end{TODOblock}

\begin{TODOblock}
The above results immediately imply that the theorems of modal logic which hold for all locally introspective theories are no stronger than those which hold for all transitive Kripke frames, and the theorems which hold for all introspective theories or the theorems which hold in all locally Loeb theories are no stronger than those which hold for all transitive converse well-founded Kripke frames. From this, we can readily conclude that the theorems which hold in all locally introspective theories are K4 and the theorems which hold in all introspective theories or the theorems which hold in all locally Loeb theories are GL. Does the last two of these coinciding help us embed every locally introspective theory into an introspective theory, in the same way as we did for the unconstrained vs constrained presheaf models of GL Kripke frames?
\end{TODOblock}

\TODOinline{LaTeXify the above better}

\begin{TODOblock}
Give topos of trees example as well. This is what happens when we take $f$ as the identity functor and $Struct = Disc$ as the free category on some semicategory (in particular, the semicategory of natural numbers with strict reverse ordering). Note that this is an example of an introspective theory in which the functor from the introspective theory to the global aspect of the geminal category is an equivalence of categories (probably an equivalence of geminal categories, even? Thus, what we were calling a GLS-category...). Our $\Box$ operator becomes, on this category, what Birkedal et al call the step operator. This has a left adjoint as well, what Birkedal et al call the constant set operator. It's likely that in general we have left adjoints for these models based on well-founded semicategories.

Actually, many of the things we cite to Birkedal are already anticipated in "Unifying Recursive and Co-recursive Definitions in Sheaf Categories" by Pietro Di Gianantonio Marino Miculan.
\end{TODOblock}

\subsection{Recap}
\TODO

\fileend

% \section{Applications}

TODO. (E.g., naive set theory)

%  \section{Future work}
 
 (appropriate lambda calculus, (infinity,1)-version)

% \section{Pedantries}

\subsection{Pedantries}
A minor formal point: In the definition of an introspective theory, in point 3, we speak about a "lex-functor".

Since we are viewing everything as essentially algebraic, our theory of "categories with finite limits" can't merely postulate existence of finite limits, but indeed must postulate CHOSEN finite limits (say, via chosen terminal objects and chosen binary pullbacks).

We can consider lex-functors as having to preserve chosen limits on-the-nose, or as only having to take limit diagrams to limit diagrams (not necessarily taking chosen limits to chosen limits).

The former concept is considered somewhat "evil", because ideally we shouldn't care about distinctions between isomorphic objects.

However, the notion of homomorphism which falls right out of the general framework of essentially algebraic theories is indeed the evil one: when lex-categories are considered as an essentially algebraic concept, the corresponding notion of homomorphism is lex-functors which preserve chosen limits-on-the-nose.

Lex-functors between categories with chosen limits which merely take limit diagrams to limit diagrams are also an essentially algebraically definable concept, though.
So we have a choice of which of these concepts to use in our formal definition of an introspective theory.
In the past, I've made different decisions here, but on this go-round, I'm going to say the notion of lex-functor we should use is the one which fits right into the usual framework of essentially algebraic theories: whenever I say "lex-functor", I will mean for it to take chosen limits to chosen limits.

This will not be a problem for any of the model constructions I am interested in, and will make everything else much easier. (edited) 

The whole need to worry about this kind of choice is only because our categories come with a notion of equality between objects which is finer-grained than isomorphism. This is because our theory of categories is essentially algebraic, and thus given by a lex 1-category itself, instead of by a lex 2-category.

We could rectify all this by trying to climb the dimensionality ladder, but since our categories are supposed to be capable of containing internal such categories as well, it's no use merely stepping to n-categories: an internal weak n-category (weak in the senes of not presuming any notion of equality on objects beyond the n-categorical notion of isomorphism) lives nontrivially only in an (n + 1)-category, not in an n-category. So if we really wanted to rectify this and Do No Evil, we'd have to step all the way up to infinity-categories (in the sense of (infinity, 1)-categories; i.e., morphisms at arbitrary depth, but we can take all 2- or higher morphisms to be invertible).
This is something worth working out at some point, surely, but it would so extraneously complicate and thus obfuscate this already empirically difficult-to-explain first introduction to everything that I leave it to future work. (Presumably or hopefully or morally, everything would just port over in just the same way to that context, given any decent formalization of (infinity, 1)-categories and internal (infinity, 1)-categories; there's only one point I worry about, which I will note when we come to it.)

(This is just picayune formalities and nit-picking, but these are the kind of picayune formalities and nit-picking some category theorists got hung up on in looking at my notes way back in the past, so: I note pre-emptively these formalities and disclaimers. We're going to go ahead and be Evil here, and we can figure out the non-Evil version of things later.)

% \section{Test}
Anything that appears here is just for my test LaTeX debugging.

$f \operatorname{Hom}(a, b)$

\end{document}
