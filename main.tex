\documentclass[12pt]{article}
% The above really belongs in the prelude, but Overleaf gets a little weird about that not being in the main document.
% The document class really belongs in the prelude, but Overleaf gets a little weird about that not being in the main document.

% Just some made up sizing for now. Can tinker with this later.
\usepackage[
  a4paper,
  margin=1in,
  headsep=5pt, % separation between header rule and text
]{geometry}

% Copied from Reid
\usepackage{amssymb,amsmath, amsthm}
\usepackage[utf8]{inputenc} % allow utf-8 input
\usepackage[T1]{fontenc}    % use 8-bit T1 fonts
\usepackage{url}            % simple URL typesetting
\usepackage{booktabs}       % professional-quality tables
\usepackage{amsfonts}       % blackboard math symbols
\usepackage{nicefrac}       % compact symbols for 1/2, etc.
\usepackage{microtype}      % microtypography

% Put in by me, Sridhar
\usepackage{color}          % Color definitions (used for `blue`, `red`)
\usepackage{newpxtext}      % Use a nice text font
\usepackage{enumitem}       % Bulleted lists
  \setenumerate{parsep=10pt} % Futz with the exact value later
\usepackage{comment}        % Multiline comments
\raggedbottom               % No more underfull vbox errors
\usepackage{quiver}         % Commutative diagrams
\usepackage{indentfirst}    % Indent first paragraph in a section
\usepackage{csquotes}       % Quotation marks without needing `` and ''
\usepackage[style=alphabetic]{biblatex}
\bibliography{references}   % Uses the file `references.bib` for bibliography information.
\usepackage{etoolbox}       % Generic LaTeX tools, including nice conditionals. Maybe unnecessary.
\usepackage{underscore}     % Underscore subscripts in text mode. Just useful because I do this in a hurry in TODO notes often.

\renewcommand\qedsymbol{$\blacksquare$} % Change QED symbol to black square so it is not confused with the modal box

% The following are packages that might need to be loaded before hyperref
\usepackage{imakeidx}       % Indexing
  \makeindex[intoc]         % Include index in table of contents
  
% And now, hyperref!
\usepackage[colorlinks=true, linkcolor=blue, citecolor=teal]
  {hyperref}                % Hyper-linked references (note that certain packages need to be loaded before this instead of after this, such as indexing)
  
% And now... Packages that might need to be loaded after hyperref.
\usepackage[conf={restate}]
  {proof-at-the-end}        % Allows for moving proofs of theorems to later in the document, with auto-linking, restating, etc
\usepackage[capitalize, nameinlink]
  {cleveref}                % References named by type and number
\newcommand*{\parensref}[1]{\hyperref[{#1}]{\nameref*{#1} (\cref*{#1})}}              % References named by explicit name and parenthesized type and number

% It's not ideal that I have to manually keep track of when to use \cref vs. when to use \nameref or \parensref.
% The following solution is based on https://tex.stackexchange.com/a/66096/231784.
%\newcommand{\magicref}[1]{%
%  \if\vcenter\getrefbykeydefault{#1}{name}{}\vcenter
%    \cref{#1}%
%  \else
%    \nameref{#1}%
%  \fi
%}
\newcommand{\magicref}[1]{%
  \if\vcenter\getrefbykeydefault{#1}{name}{}\vcenter
    \cref{#1}%
  \else
    \parensref{#1}%
  \fi
}
\newcommand{\magicparensref}[1]{%
  \if\vcenter\getrefbykeydefault{#1}{name}{}\vcenter
    \cref{#1}%
  \else
    \parensref{#1}%
  \fi
}


% For subfiles, in a toggle-able way
\newif\ifsubfiles

% For TODO counters, in a toggle-able way
\newif\ifDisplayTODOCount

\newcommand{\filestart}{
    \ifsubfiles
    %\documentclass[./main.tex]{subfiles}
    % Commented out while we don't use subfiles as this appearance of the documentclass command is making Overleaf think of this prelude as an individually compilable file. See https://www.overleaf.com/learn/how-to/Set_Main_Document
    \begin{document}
    \fi
}

\newcommand{\fileend}{
    \ifsubfiles
    \end{document}
    \fi
}
\newcommand{\fileinclude}[1]{
    \setcounter{TODOcounter}{0}
    \ifsubfiles
    \newpage \subfile{#1}
    \else
    \include{#1}
    \fi
    \ifDisplayTODOCount
    {\color{red} There are \theTODOcounter\; TO DOs remaining in this section.}
    \fi
}

% Commands defined by Sridhar for this document

% Remember: \newtheorem{command name}{displayed name}[parent counter] makes a theorem whose counter is subordinate to counter, but \newtheorem{command name}[shared counter]{displayed name} makes a theorem which shares the counter. Also remember that there is automatically a counter available named "section".

% Playing around
\newtheoremstyle{envstyle}
  {\topsep}
  {\topsep}
  {}
  {}
  {\scshape \color{purple}}
  {}
  {.5em}
  {}

%\theoremstyle{plain}
\theoremstyle{envstyle}
\newtheorem{theorem}{Theorem}[section]
\newtheorem{lemma}[theorem]{Lemma}
\newtheorem{construction}[theorem]{Construction}
\newtheorem{example}[theorem]{Example}
\newtheorem{observation}[theorem]{Observation}
\newtheorem{corollary}[theorem]{Corollary}
\newtheorem{corollarytoproof}[theorem]{Corollary (to proof)}
\newtheorem{conjecture}[theorem]{Conjecture}

%\theoremstyle{definition}
\theoremstyle{envstyle}
\newtheorem{definition}[theorem]{Definition}
\newtheorem{convention}[theorem]{Convention}
\newtheorem{warningenv}[theorem]{Warning}
\newtheorem{note}[theorem]{Note}

\newcommand{\defined}[1]{\textbf{#1}\index{#1}} % For indicating the defined term in a definition
\newcommand{\definedManualIndexSort}[2]{\textbf{#1}\index{#2@#1}}

\newcommand{\openNamed}[2]{\begin{#1}[#2]\index{#2}}
\newcommand{\openNamedManualIndexSort}[3]{\begin{#1}[#2]\index{#3@#2}}
\newcommand{\closeNamed}[1]{\end{#1}}

%\theoremstyle{remark}
\theoremstyle{envstyle}
\newtheorem*{remark}{Remark}

% This doesn't quite work yet. The word "Proof" is bolded?
% \renewenvironment{proof}{\paragraph{\scshape \color{blue} Proof:}}{\hfill$\square$}

\newtheoremstyle{redstyle}
  {\topsep}
  {\topsep}
  {\color{red}}
  {}
  {\scshape \color{red}}
  {}
  {.5em}
  {}
  
\theoremstyle{redstyle}
\newtheorem*{TODOblock}{TO DO}

% Warning! LaTeX will swallow the following space stupidly in this command. You may have to place a backslash-space after it instead.
\newcommand{\TODO}{{\color{red} TO DO}}

\newcommand{\TODOinline}[1]{\TODO\ {\color{red} #1}}
\newcommand{\UnprintedTODO}[1]{}

% Using \operatorname for constants is maybe a little wrong, as it suggests a closer semantic association to the right. For now, we just use \mathrm.
\newcommand{\const}[1]{\mathrm{#1}} % Typesetting named constants
\newcommand{\constcat}[1]{\const{#1}} % Typesetting named constant categories
\newcommand{\arrowcat}[1]{\operatorname{Arrow}(#1)} % Typesetting arrow categories
\newcommand{\Hom}{\operatorname{Hom}}
\newcommand{\Iso}{\operatorname{Iso}}
\newcommand{\Nat}{\operatorname{Nat}}
\newcommand{\dom}{\operatorname{dom}}
\newcommand{\cod}{\operatorname{cod}}
\newcommand{\op}[1]{#1^{\mathrm{op}}} % If we include the `physics` package, then this needs to be renewcommand, as \op is defined for \outerproduct by the `physics` package.
\newcommand{\Ob}{\operatorname{Ob}}
\newcommand{\Mor}{\operatorname{Mor}}
\newcommand{\Set}{\constcat{Set}}
\newcommand{\Cat}{\constcat{Cat}}
\newcommand{\FiniteProductCat}{\constcat{FiniteProductCat}}
\newcommand{\LexCat}{\constcat{LexCat}}
\newcommand{\StrictCat}{\constcat{StrictCat}}
\newcommand{\StrictCatTwo}{\StrictCat_2}
\newcommand{\StrictLexCat}{\constcat{StrictLexCat}}
\newcommand{\StrictLexCatTwo}{\StrictLexCat_2}
\newcommand{\id}{\const{id}}
\newcommand{\unique}{!}
\newcommand{\iso}{\simeq}
\newcommand{\Psh}[1]{\operatorname{Psh}(#1)}
\newcommand{\Lan}{\operatorname{Lan}}
\newcommand{\Ran}{\operatorname{Ran}}

\newcommand{\introF}{\mathcal{F}}
\newcommand{\introS}{\mathcal{S}}
\newcommand{\introN}{\mathcal{N}}
\newcommand{\glQuote}{\mathcal{J}}

\newcommand{\nat}{\mathbb{N}}

\newcommand{\GLCatTheory}{\mathrm{Th}(\mathrm{GL})}

\newcommand{\outertheory}{X}
\newcommand{\innertheory}{Y}

\renewcommand{\implies}{\Rightarrow}
\newcommand{\biimplies}{\iff}

\newcommand{\code}[1]{\lceil #1 \rceil}

\newcommand{\pullAlong}[1]{#1^*}

% Should maybe use the word "representable" for this?
% Hack to define a command name whose name has to end with a 
% slash, thus preventing spacing problems. (Otherwise, space 
% automatically gets swallowed after invoking the command.)
\def\repsmall/{repsmall}
\def\setsmall/{set-sized}
\def\catsmall/{small}
\def\Goedel/{G\"odel}
\def\Godel/{\Goedel/}
\def\Loeb/{L\"ob}
\def\Lob/{\Loeb/}
\def\sigmesque/{sigmesque}

\newcommand{\adjointTo}{\dashv}
\newcommand{\yoneda}{\mathrm{yoneda}}

\newcommand{\profuncTo}{\nrightarrow} % Profunctor \to arrows

\renewcommand{\quote}{\textquote} % I have no use for the existing \quote command in LaTeX, and would like that short name for this instead

\newcommand{\point}{\star}
\newcommand{\App}{App}

\newcommand{\later}{\blacktriangleright}
\newcommand{\El}{\operatorname{El}}

\newcommand{\Groth}{\int}

\newcommand{\theoryT}{\mathbb{T}}
\newcommand{\modelsIn}[2]{#1-\mathrm{Mod}(#2)}
\newcommand{\classifying}[1]{\mathcal{C}_{#1}}
\newcommand{\cartwith}[1]{\mathrm{Lex}\varpi #1}
\newcommand{\TheoryOfLexCat}{\theoryT_{\mathrm{lex}}}
\newcommand{\LexCatToTheory}[1]{\operatorname{Th}(#1)}

\newcommand{\quotient}{\mathcal{Q}}
\newcommand{\core}[1]{\operatorname{core}(#1)}

\title{Abstract Provability Structures}

\author{
  Sridhar Ramesh\\
  \texttt{sramesh@berkeley.edu}
}

\pagenumbering{roman}
\begin{document}
\maketitle

\begin{abstract}
The aim of these notes is to identify and draw attention to a certain simple and categorically natural kind of mathematical structure which both serves as an abstract environment for the reasoning used in establishing \Loeb's theorem in its traditional instances, and furthermore allows this and the associated theorems and fixed-point results of the \Goedel-\Loeb\ modal logic to be vastly generalized.

Some such \Loeb-style fixed point phenomena have been explored in the literature before, but these notes aim to highlight a particularly minimal, simple, general abstraction that covers several threads of work in the literature, abstracting both the work on the \Goedel-\Loeb\ incompleteness theorems via arithmetic universes a la Joyal, and the work on \Loeb's theorem as a guarded fixed point combinator and on guarded (co)inductive types by Birkedal et al.

Notably, as differentiated from much other work in the literature on type theories with guarded fixed points, we do not take the existence of guarded fixed points as a presumption of our framework by fiat, but rather achieve them as nontrivially derived from the framework. Also, in keeping with provability logic but as differentiated from much existing work on guarded fixed points, our models do not generally validate or have an inhabitant of the type $\neg \neg \Box 0$ (where $0$ stands for falsehood or the empty set, and $\neg$ stands for exponentiation with $0$ as base).

The core idea is the identification of those essentially algebraic theories satisfying the property that every model of these theories contains also, as part of its structure, a homomorphism into an internal model of the same theory.

This structure turns out to be viewable as a categorification of the Hilbert-Bernays derivability conditions of provability logic. In particular, it induces a bifunctor on each model of such a theory which is formally similar to the $\Box(A \implies B)$ operator of the \Goedel-\Loeb\ modal logic. Furthermore, as mentioned, various fixed-point theorems within these categories will be demonstrated, both at the level of terms and at the level of types. From these, the traditional instances of \Loeb's theorem in the context of provability of propositions falls out as a special case. Along the way, the relationship between \Loeb's theorem and presheaves is also highlighted, which has previously gone unremarked upon.

Our abstract framework admits free instances, and we establish the nature of these free instances in some detail. In so doing, the distinction and yet also close relationship between categorical models of provability logic which do and don't validate $X \vdash \Box X$ in general is formally clarified.

It should be noted that the bulk of this work was originally developed during my time in graduate school from 2006 to 2013, but I did not write it up suitably at that time.

\end{abstract}

\newpage
\pagenumbering{arabic}
\setcounter{page}{1}
\tableofcontents

% \filestart

\section{Introduction}
The aim of these notes is to identify and draw attention to a certain surprisingly simple and category-theoretically natural mathematical structure which both serves as an abstract environment for the reasoning used in establishing \Goedel/'s incompleteness theorems and \Loeb/'s theorem in their traditional instances (as in \autocite{goedel1931formal} and \autocite{loeb1955solution}), and furthermore allows these and the further theorems and fixed-point results of the \Goedel/-\Loeb/ modal logic of provability (as in \autocite{boolos1995logic}) to be vastly generalized.

Some such \Loeb/-style fixed point phenomena have been explored in the literature before, but our abstraction is of note as a particularly simple and general one. This abstraction for the first time formally unifies three distinct threads of work in the literature, having as special cases the interpretation of the \Goedel/-\Loeb/ incompleteness theorems via the initial arithmetic universe a la Joyal (as discussed in \autocite{van2020g}), the interpretation of \Loeb/'s theorem as a guarded fixed point combinator and associated work on guarded (co)inductive types via step-indexing in contexts such as the topos of trees (as in \autocite{birkedal2011first}), and the classical interpretation of the GL modal logic in well-founded transitive Kripke frames.

Our interest is in a \emph{minimal} categorical structure which naturally reflects the abstract structure of the G\"odelian argument. We emphasize that (as opposed to much of the literature on categorical abstractions of guarded recursion), our abstraction does not have \Loeb/'s theorem built into it directly as an assumption, but rather allows \Loeb/'s theorem to be derived from much more basic presumptions. Our abstraction is indeed so simple that it does not even make such common presumptions as cartesian closure, regularity, or coproducts, all of which turn out not to be needed for the derivation of \Loeb/'s theorem. (Indeed, not presuming cartesian closure is vital for allowing our abstraction to cover the initial arithmetic universe!)

The core idea is the identification of those essentially algebraic theories satisfying the property that every model of these theories contains also, as part of its structure, a homomorphism into an internal model of the same theory. We call these \quote{introspective theories}. This document is devoted to initiating the study of introspective theories.

We give two category-theoretic formalizations of the concept of an introspective theory (one directly corresponding to the above description (\magicref{DefnIntrospSN}), the other less so (\magicref{DefnIntrospIndexed})) and prove them equivalent (\magicref{SNCorrespondence}). We then derive a form of \Loeb/'s theorem, in terms of the existence of suitably guarded fixed points, for arbitrary introspective theories (\magicref{IntrospLoeb}). In this demonstration of \Loeb/'s theorem, the relationship between \Loeb/'s theorem and presheaves is also highlighted, including the applicability of \Loeb/'s theorem to non-representable presheaves, which has previously gone unremarked upon.

This derivation of \Loeb/'s theorem for introspective theories is our most important key result. The separate demonstrations of how each of the three traditional instances of \Loeb/'s theorem noted above correspond to certain constructions of introspective theories comprise other key results.

(Specifically, these three traditional instances are seen as instantiations of our abstract theory like so: An introspective theory corresponding to Joyal's work with the initial arithmetic universe is discussed in \magicref{IAUSection}. An introspective theory corresponding to step-indexing in the topos of trees is discussed in \magicref{StepIndexingSection}. Introspective theories corresponding to the classical interpretation of GL modal logic in well-founded transitive Kripke frames are discussed in \magicref{KripkeFrameSection}. These last two constructions are themselves unified and generalized much further in \magicref{ModelsBasedOnPresheafCategories}.)

I believe this is the first formal demonstration of how traditional logical contexts such as the syntactic category of Peano Arithmetic (discussed as an introspective theory at \magicref{ZFFiniteSection}) support guarded recursion not just at the level of propositions (where this amounts to \Loeb/'s theorem in its traditional sense), but also for general terms of arbitrary type, and also for types themselves. Similarly for contexts such as the initial arithmetic universe or the initial topos with natural numbers object (discussed in \magicref{SelfInitializingSection}).

In addition to such traditional finitary logical theories, we give a similar demonstration of the initial topos with countable products as inducing another model of our formal abstraction (in \magicref{ToposWithCountableProductsSection}). As this structure contains both uncomputable and uncountable data, yet is constructed in a very similar way to the traditional logical incarnations of the \Godel/-\Lob/ phenomenon, this should vividly dispel the oft-repeated canard that the \Godel/-\Lob/ phenomenon in logic is fundamentally about or constrained to computability. (As amounts to the same thing, this illustrates that the phenomenon is not constrained to structures internalizable in the initial arithmetic universe).

The concept of an introspective theory is itself essentially algebraic in nature, and thus admits free instances as well, and we give a tractable explicit description of the initial introspective theory in \magicref{GeminalChapter}. This explicit description of the initial introspective theory is another key result of ours. We also observe a remarkable surprising relationship between the initial introspective theory and the theory of introspective theories (\magicref{EveryIntrospModelsInitialIntrospRemark}), and some dual co-free constructions of introspective theories (\magicref{CofreeGeminalSection}).

Though introspective theories are our fundamental objects of interest, along the way, we consider also relaxations of the definition of introspective theories to encompass more general structures (in particular, the relaxation we call \quote{locally introspective theories}, defined at \magicref{DefnLocallyIntrosp}) which, while not supporting the derivation of the \Godel/-\Lob/ phenomena, allow us to state other theorems and constructions in their natural generality and note broader connections with other mathematics.

\sTODOinline{It should be noted that the core ideas of this work were originally developed during my\footnote{It seems suddenly silly to use the otherwise conventional \quote{our}, \quote{we}, etc, for these particular sentences.} time in graduate school from 2006 to 2013 at the University of California, Berkeley, under the advising of Dana Scott, for which I am forever grateful. It was originally developed as the research which was to become my doctoral dissertation, but I did not write it up suitably at that time. Life took its turns and twists, and thankfully I have been able to write this material up now.}

\subsection{Reading roadmap}
The Preliminaries from \magicref{PreliminariesTerminologyConventions} through \magicref{PreliminariesSelfIndexing} cover conventions and material which are used throughout the entire document, which the reader will certainly want to familiarize themselves with. The remainder of the Preliminaries can be read on an as needed basis.

The first chapters \magicref{DefnChapter} and \magicref{ModalChapter} establish the basic concepts of introspective theories, which all later chapters depend on. However, the later chapters \magicref{LoebChapter}, \magicref{GeminalChapter}, and \magicref{ExamplesChapter} can be read essentially independently of each other, in any order or fashion the reader likes. The only dependence between these is that the concept of geminal categories from \magicref{GeminalChapter} is invoked in one isolated section of \magicref{ExamplesChapter}, at \magicref{GeminalSelfInitializingSection}.

\fileend

\include{Category-theoretic preliminaries}

\filestart

\section{Introspective theories}

\subsection{Preview}
In this chapter, we introduce the central object of our interest, the notion of an \quote{introspective theory}.

An introspective theory is an essentially algebraic theory such that every model of the theory includes a lexcategory with an internal model of the same theory, as well as a homomorphism from the overall model into the global aspect of the internal model.

We will give two formal definitions of an introspective theory, and prove them equivalent. The second formal definition we give will directly correspond to the previous paragraph. The first formal definition we give will be a bit more compact, but framed in the language of indexed categories.

En route to discussing introspective theories, we also discuss some more general notions we call \quote{pre-introspective theories}, \quote{locally introspective theories}, and so on, which will be of some use to us as well.

% Non-evil definition
\subsection{First definition (indexed style)}

\begin{definition} \label{DefnPreIntrospIndexed}
A \defined{pre-introspective theory} is a lexcategory $T$, a $T$-indexed lexcategory $C$, and a lexfunctor $\introF$ from the self-indexing of $T$ to $C$, like so:

% https://q.uiver.app/?q=WzAsMixbMCwwLCJcXG9we1R9Il0sWzIsMCwiXFxMZXhDYXQiXSxbMCwxLCJULy0iLDAseyJvZmZzZXQiOi0yfV0sWzAsMSwiQyIsMix7Im9mZnNldCI6Mn1dLFsyLDMsIlxcaW50cm9GIiwyLHsic2hvcnRlbiI6eyJzb3VyY2UiOjIwLCJ0YXJnZXQiOjIwfX1dXQ==
\[\begin{tikzcd}
	{\op{T}} && \LexCat
	\arrow[""{name=0, anchor=center, inner sep=0}, "{T/-}", shift left=2, from=1-1, to=1-3]
	\arrow[""{name=1, anchor=center, inner sep=0}, "C"', shift right=2, from=1-1, to=1-3]
	\arrow["\introF"', shorten <=1pt, shorten >=1pt, Rightarrow, from=0, to=1]
\end{tikzcd}\]
\end{definition}

We write out the triple $\langle T, C, \introF \rangle$ to refer to a pre-introspective theory when we wish to be fully explicit about its structure. But in typical abuse of language, we also often refer to it simply by the name of its underlying lexcategory $T$ or of the pair $\langle T, C \rangle$, when this would not cause confusion. We will frequently use the same name $\introF$ as though it applies to all introspective theories simultaneously, in the same way that notation like $+$ or $\times$ is overloaded as applying over all rings simultaneously.

\DiscussSmallNotation
% If we use \wordsmall/ for \repsmall/, then have the following as a footnote on \repsmall/ here:
% \footnote{We remind the reader that we use this terminology in the sense of \magicref{RepsmallDefn} and \magicref{RepsmallCategoryDefn}, and it has nothing to do with being \setsmall/. Rather, it rather means that $C$ is presented by an internal category.}

\begin{definition} \label{DefnIntrospIndexed}
An \defined{introspective theory} is a pre-introspective theory $\langle T, C \rangle$ in which $C$ is \repsmall/\footnote{We reminder the reader that this means $C$ is presented by an internal category in $T$.}.
\end{definition}

We shall show in later chapters how this simple concept of an introspective theory already suffices to exhibit and capture all the fundamental phenomena of \Goedel/\ codes, diagonalization, the \Goedel/\ incompleteness theorems, and \Loeb/'s theorem. And we shall show that all the traditional instances of \Goedel/'s incompleteness phenomena arise from special cases of this purely algebraic structure. We will also demonstrate functorial fixed point results for this structure, and show some interesting applications of these.

We shall also introduce some further generalizations of this concept, in order to be able to state results along the way in their natural generality or point out connections to related work or interesting structures that are not quite introspective theories per se but are closely related. But throughout these notes, if at any time the abstractions seem daunting or distracting, remember that the concrete concept which matters most is the concept of an introspective theory as defined above.

The example-oriented reader may immediately demand an example of a pre-introspective theory, to orient themselves. Here is the simplest example (or class of examples) of a pre-introspective theory:

\begin{example} \label{TrivialPreIntrosp}
Let $T$ be any lexcategory. Then we have a pre-introspective theory $\langle T, T/-, \id \rangle$. That is, a pre-introspective theory in which $C$ is taken to be the self-indexing itself, with $\introF$ as the identity.
\end{example}

Alas, this simple example of a pre-introspective theory is almost never an introspective theory. That is to say, a lexcategory's self-indexing is almost never \repsmall/\footnote{Indeed, the only case in which this happens is the trivial one where $T$ is the terminal category! We will ultimately establish this result at \magicref{LocallyCartesianLoeb}.}.

Here, then, is a simple example of an introspective theory:

\begin{example}
Let $T$ be any lexcategory, and let $C$ be any \repsmall/ $T$-indexed lexcategory. Then we have an introspective theory $\langle T, C, \introF \rangle$ where each aspect of $\introF$ sends all objects to the terminal object.
\end{example}

This is indeed an introspective theory. But alas, although this last example can be as nontrivial as one likes in terms of the structure of $T$ and $C$, it is of course trivial in all its further structure.

Nontrivial introspective theories do exist and we will give some archetypal examples of them soon enough. But in order to do so, it will be convenient to first develop some further machinery on how (pre-)introspective theories may be presented.

\subsection{Second definition (non-indexed style)}

We shall now make an observation about an alternative but equivalent way to specify the data of a pre-introspective theory.

\begin{theorem}\label{SNCorrespondence}
Given a lexcategory $T$ and a $T$-indexed lexcategory $C$, specifying a pre-introspective theory $\langle T, C, \introF \rangle$ (i.e., specifying a $T$-indexed lexfunctor from the self-indexing $T/-$ to $C$) is equivalent to specifying a (non-indexed) lexfunctor $\introS : T \to \Glob{C}$, along with specifying maps $\introN_t$ from each $t \in T$ to $\Hom_C(1, \introS(t))$, naturally in $t$.

That is, keeping in mind that $\Hom_C(1, \introS(-)) : T \to \Psh{T}$, and recalling that we also identify $T$ with a full subcategory of $\Psh{T}$ via the Yoneda embedding $\yoneda : T \to \Psh{T}$, the last part of the above is asking for a natural transformation $\introN : \yoneda \to \Hom_C(1, \introS(-))$.
\end{theorem}

\begin{proof}
Let $T$ be a lexcategory, and let $C$ be some $T$-indexed lexcategory. By \magicref{Lemma1} (keeping in mind the contravariance of the functors defining indexed structures), a map from the self-indexing $T/-$ to $C$ as $T$-indexed lexcategories is the same as a lexfunctor $\introS$ from $T$ to the global aspect of $C$, along with a map from $T/-$ to $C$ as $T$-indexed objects of $T/\LexCat$ (where the map $\introS$ is used to treat $C$ as a $T$-indexed object of $T/\LexCat$).

Next we apply \magicref{SelfIndexingIsFree}. The map from $T/-$ to $C$ as $T$-indexed objects of $T/\LexCat$ is the same as choosing, in a natural way over all $t$ in $T$, some $t$-defined value in $\Hom_C(1, \introS(t))$. That is, maps from each $t \in \Ob(T)$ to $\Hom_C(1, \introS(t))$, comprising a natural transformation.
\end{proof}

\begin{remark}\label{IntrospGeneralDoctrine}
It wasn't fundamentally important that we were dealing with lexcategories here. The use of \magicref{Lemma1} as applied to $\op{C}$ only required a terminal object in $C$. And for the invocation of \magicref{SelfIndexingIsFree}, we only needed that there is some free construction of adjoining global elements. (Even the role terminality plays here is to some degree eliminable, though we have no interest for now in eliminating it). In particular, we get a completely analogous result when lexcategories are replaced throughout by any of the structures noted in \magicref{SelfIndexingIsFreeCorollary}, including for categories with finite products using the simple self-indexing.
\end{remark}

As a result of \cref{SNCorrespondence}, we can give an alternative definition equivalent to \cref{DefnPreIntrospIndexed}:

\begin{definition}\label{DefnPreIntrospSN}
A \defined{pre-introspective theory} is a lexcategory $T$, a $T$-indexed lexcategory $C$, a lexfunctor $\introS$ from $T$ to the global aspect of $C$, and a natural transformation $\introN$ from each $t \in \Ob(T)$ to $\Hom_C(1, \introS(t))$. (That is, $\introN : \yoneda \to \Hom_C(1, \introS(-))$, where $\yoneda$ and $\Hom_C(1, \introS(-))$ are parallel functors from $T$ to $\Psh{T}$.)
\end{definition}

Much as before, we may write out $\langle T, C, \introS, \introN \rangle$ to be fully explicit, but in typical abuse of language, will refer to a pre-introspective theory by simply naming $T$ or the pair $\langle T, C\rangle$. We will frequently use the same names $\introS$ and $\introN$ as though they apply simultaneously to all such structures (in the same way that notation like $+$ and $\times$ is overloaded as applicable to separate rings simultaneously).

The definition of an introspective theory remains exactly as before (\cref{DefnIntrospIndexed}) regardless of how one thinks of pre-introspective theories, but for reminder's sake:

\begin{definition} \label{DefnIntrospSN}
An \defined{introspective theory} is a pre-introspective theory $\langle T, C \rangle$ in which $C$ is \repsmall/.
\end{definition}

While it may sometimes be easier to prove theorems about (pre-)introspective theories by using \cref{DefnPreIntrospIndexed}, it will often be easier to show structures actually are (pre-)introspective theories by using \cref{DefnPreIntrospSN}. But this is not the only benefit of \cref{DefnPreIntrospSN}. The reduction of the full indexed lexfunctor $\introF$ to just its global aspect ($\introS$) and a natural transformation between 1-functors means much less data around to explicitly fuss about. In particular, when we wish to strictify this into a lex definition eventually at \magicref{StrictIntrospHomoDefn}, we will find the appropriate coherence conditions much easier to manage. It will also be easier to define the appropriate notion of homomorphisms between (pre-)introspective theories by thinking about \cref{DefnPreIntrospSN}.

\Cref{DefnPreIntrospSN} also allows us to quickly appreciate the significance of introspective theories from a functorial semantics point of view. An introspective theory is precisely an essentially algebraic theory (this is the role of $T$) extending the theory of lexcategories (this is the role of $C$), such that every model of the theory (which thus has an underlying lexcategory as its interpretation of $C$) is equipped with a designated homomorphism (this is the role of $\introN$) into an internal model of that same theory in its underlying lexcategory (this is the role of $\introS$). In short, every model has a homomorphism into a further internal model.

It will be useful for us also to consider sometimes the following concept, intermediate between pre-introspective theories and introspective theories:

\begin{definition}\label{DefnLocallyIntrosp}
A \defined{locally introspective theory} is a pre-introspective theory $\langle T, C \rangle$ in which $C$ is locally \repsmall/.
\end{definition}

Almost all the results we discuss for introspective theories admit straightforward generalization to locally introspective theories. The sole major exception is the derivation of \Loeb/'s theorem for introspective theories in \magicref{IntrospLoeb}. However, because that one result is so important to us, our main interest in this document is in discussing introspective theories, rather than locally introspective theories more generally.

\subsection{Archetypal examples}

Let us now finally give the example-oriented reader a nontrivial example of an introspective theory by which to orient themselves. (On the other hand, the reader who prefers to consider abstract definitions without immediately diving into worked out examples of a highly concrete flavor may skip any or all of this section at this introductory time if they find its details a distraction. To each reader at their own taste!).

\subsubsection{Example based on a traditional logical theory}
In this subsection, we will give two closely related examples. The first example we present is somewhat atypical of general introspective theories, but important nonetheless. It is very similar to the arithmetic universe constructions considered by Joyal in his account of \Godel/'s incompleteness theorem and by others following up on this (Joyal himself never published this work, but a detailed account has been given in \autocite{van2020g}, building off the formalization of the initial arithmetic universe given in \autocite{maietti10a}). 

Although very similar, the category we use in this first example is not exactly the same as the initial arithmetic universe considered in \autocite{van2020g} and \autocite{maietti10a}. The variant and presentation we give is intended to feel natural to an audience of traditional logicians. The connection of this construction to the initial arithmetic universe will be discussed in more detail later at \magicref{Sigma1ModelIAUConnection}.

After having given this first example, we will then tweak it slightly into another introspective theory which provides much better intuition for the general nature of introspective theories.

\newcommand{\Zfin}{\mathrm{Z}}
\newcommand{\ZfinSigma}{\mathrm{Z}_{\Sigma_1}}
\newcommand{\InnerZfin}{\mathrm{Z}'}
\newcommand{\InnerZfinSigma}{\mathrm{Z}_{\Sigma_1}'}
\begin{construction}\label{SigmaModelSimple}
\sTODOinline{Perhaps do all the following in terms of ZFC rather than ZF-Finite?}
Let us start with the first-order logical theory ZF-Finite: This is the theory ZF but with the axiom of infinity replaced by its negation\footnote{This theory happens to be bi-interpretable with Peano Arithmetic, but it will be more convenient for us to speak in terms of ZF-Finite so as not to fret about codings of a sort every modern mathematician readily takes for granted in a ZF-style context. Pedantically, we must also make sure to take the Axiom of Foundation in the definition of ZF-Finite to be suitably phrased, e.g. in terms of $\in$-induction, or else we will not have this bi-interpretability. \TODOinline{Give cite for this classic bi-interpretability result.}}. The universe this theory describes is the hereditarily finite sets $V_{\omega}$. Throughout this construction, whenever we speak of formulae, we mean formulae in the language of ZF-Finite, and whenever we speak of provability, we mean provable within ZF-Finite.

Certain formulae are $\Sigma_1$. These are the formulae which consist of an initial string of unbounded existential quantifiers (ranging over the entire universe), after which all other quantifiers are bounded (ranging only over the elements of some particular already introduced hereditarily finite set). 

Put another way, which may be more comfortable for some readers, the $\Sigma_1$ formulas $\phi$ are precisely those for which there is a computer program $P$ outputting a (possibly empty, possibly finite, possibly infinite) stream of tuples of hereditarily finite sets such that ZF-Finite proves that the tuples which $\phi$ holds of are precisely the ones output by $P$. That is, the $\Sigma_1$ formulas describe the computably enumerable relations.

(The equivalence between these two accounts of the $\Sigma_1$ formulas of ZF-Finite is well known, and we will not go over its details. At any rate, the reader may pick whichever account they like with which to think about the following.)

Now let us define a category whose objects are the $\Sigma_1$ formulae with one free variable. Such formulae amount to certain definable subsets of the universe $V_{\omega}$; that is, they describe classes of hereditarily finite sets. (Note that the classes these formulae describe may themselves be infinite! For example, the tautologically true formula describes the class of all hereditarily finite sets.)

Given two such objects $\phi(n)$ and $\psi(m)$, we take as morphisms between these any $\Sigma_1$ formula $F(n, m)$ which provably acts as the graph of a function between the corresponding classes. That is, such that both $\forall n, m . F(n, m) \implies (\phi(n) \wedge \psi(m))$ and $\forall n . \phi(n) \implies \exists! m . F(n, m)$ are provable.

Two such morphisms $F(n, m)$ and $G(n, m)$ are considered equal just in case $\forall n, m . F(n, m) \biimplies G(n, m)$ is provable. 

Finally, morphisms compose in the expected way for graphs of functions; that is, the composition of $F(n, p)$ with $G(p, m)$ is given by $(G \circ F)(n, m) = \exists p (F(n, p) \wedge G(p, m))$.

We omit here the straightforward details of verifying that this structure we have just described does indeed satisfy the rules to be a category. Indeed, it is furthermore a regular category (that is, it has finite limits and pullback-stable image factorization; it has finite products because of the definability of ordered pairs in ZF-Finite, and it furthermore has equalizers and image factorization using suitable instances of Separation in ZF-Finite). However, it is not an exact category (that is, not every equivalence relation in this category admits a corresponding quotient). Let $\ZfinSigma$ be its ex/reg completion.

(There is not in general any need for the categories involved in an introspective theory to be exact, or even regular. They need only have finite limits. However, for the particular construction we are outlining now, this ex/reg completion is the $\ZfinSigma$ we need to look at.)

More explicitly, we can describe $\ZfinSigma$ like so:

Its objects are the $\Sigma_1$ binary relations $\phi(n, m)$ which can be proven to be partial equivalence relations (i.e., symmetric and transitive), thus corresponding to certain subquotients of the universe of all hereditarily finite sets.

Given any two such formulae $\phi(n_1, n_2)$ and $\psi(m_1, m_2)$, a morphism in $\ZfinSigma$ from $\phi$ to $\psi$ is a $\Sigma_1$ formula $F(n, m)$ which provably corresponds to the graph of a function between the corresponding subquotients of the universe. That is, such that the universal closures of all the following are provable:

$F(n, m) \implies \phi(n, n) \wedge \psi(m, m)$

$\phi(n_1, n_2) \wedge \psi(m_1, m_2) \wedge F(n_1, m_1) \implies F(n_2, m_2)$

$\phi(n, n) \implies \exists m [F(n, m)]$

$F(n, m_1) \wedge F(n, m_2) \implies \psi(m_1, m_2)$.

Two such formulae $F(n, m)$ and $F'(n, m)$ are considered to be equal as morphisms from $\phi$ to $\psi$ if they are provably equivalent (that is, if both $\forall n, m . F(n, m) \implies F'(n, m)$ and $\forall n, m . F'(n, m) \implies F(n, m)$ are provable).

Given morphisms $F : \phi \to \psi$ and $G: \psi \to \chi$ of this sort, we again define their composition in the usual way of composing functions represented as graphs, as $(G \circ F)(n, m) = \exists p [F(n, p) \wedge G(p, m)]$.

This all describes the category $\ZfinSigma$, which one can verify is indeed a category and moreso, an exact category.

Note that our construction of $\ZfinSigma$ is such that the objects of $\ZfinSigma$, the morphisms of $\ZfinSigma$, the equality relation on morphisms of $\ZfinSigma$, the composition structure of $\ZfinSigma$, the finite limit structure of $\ZfinSigma$, etc, are all definable within the language of ZF-Finite; indeed, all definable by $\Sigma_1$ formulae. (In particular, keep in mind that provability in ZF-Finite is itself a $\Sigma_1$ property). Thus, there is a lexcategory $\InnerZfinSigma$ internal to $\ZfinSigma$ which corresponds to this very same construction of $\ZfinSigma$ we have just described. And we have a lexfunctor $\introS$ from $\ZfinSigma$ to the global aspect of $\InnerZfinSigma$ which sends each piece of the construction of $\ZfinSigma$ to the corresponding piece of the construction of $\InnerZfinSigma$. This is all straightforward.

As the last bit of introspective theory structure, we must build a natural transformation $\introN$ from the identity endofunctor to the endofunctor $\Hom_{\InnerZfinSigma}(1, \introS(-))$ on $\ZfinSigma$. The core idea behind this $\introN$ is simple. Essentially, to every hereditarily finite set $x$, we can assign it a code $\code{x}$, which is an explicit term in the language of ZF-Finite denoting that set. The easy way to do this is to recursively assign to each set $\{a, b, c, \ldots\}$ the term describing a finite set whose members are explicitly enumerated by the terms assigned to $a, b, c, \ldots$. We thus send a set such as $\{\{\}, \{\{\}\}\}$ to the term in the language of ZF-Finite which might be called \quote{$\{\{\}, \{\{\}\}\}$} within quotation marks, and so on.

This gives us a function $\code{-}$ from hereditarily finite sets to terms in the language of ZF-Finite which describe hereditarily finite sets. This function $\code{-}$ is definable by a $\Sigma_1$ formula and thus gives a morphism in $\ZfinSigma$. This serves as the component of $\introN$ at the object of $\ZfinSigma$ describing the collection of ALL hereditarily finite sets. 

(The categorically oriented reader may think of this recursive definition of $\code{-}$ as a catamorphism, where the collection of all hereditarily finite sets is understood as the initial algebra for the covariant finite powerset functor.)

All the other objects of $\ZfinSigma$ are subquotients of that object (and similarly for the objects of $\InnerZfinSigma$), and therefore the components of the natural transformation $\introN$ at these other objects can now be obtained uniquely so long as certain factorizations exist. That is to say, the component of $\introN$ at any object $\phi$ of $\ZfinSigma$ (that is, an object corresponding to a partial equivalence relation $\phi(n_1, n_2)$) will also be given by the action of $\code{-}$, but for this to indeed work to map $\phi$ into $\Hom_{\InnerZfinSigma}(1, \introS(\phi))$, we need to know that $\code{-}$ when acting on individuals which are related by the partial equivalence relation $\phi$ produces terms which provably describe individuals related by $\phi$.

This is where the $\Sigma_1$-ness of $\phi$ plays a vital role. We can prove that, for any $\Sigma_1$ property $\phi$, for all $x$, whenever $\phi$ holds of $x$, it furthermore provably holds of $x$ (in the sense that the particular term $\code{x}$, when substituted into the argument of the particular formula defining $\phi$, yields a sentence which is derivable in the formal system ZF-Finite).

Finally, let us observe the naturality of this $\introN$. Consider the general form of its naturality squares:

\[\begin{tikzcd}
	\phi & \psi \\
	{\Hom_{\InnerZfinSigma}(1, \introS(\phi))} & {\Hom_{\InnerZfinSigma}(1, \introS(\psi))}
	\arrow["{\code{-}}"', from=1-1, to=2-1]
	\arrow["m", from=1-1, to=1-2]
	\arrow["{\code{-}}", from=1-2, to=2-2]
	\arrow["{\introS(m) \circ -}"', from=2-1, to=2-2]
\end{tikzcd}\]

This says that, for any definable function $m$, it is provably the case that for every $x$, we have that applying the function $m$ to $x$ and then constructing the term encoding the result ($\code{m(x)}$) is a provably equivalent term to taking the term representing $x$ and substituting it into the argument of the formula defining $m$ (what might be called $m(\code{x})$ or perhaps $\code{m}(\code{x})$ or at any rate $\introS(m)(\code{x})$). To be clear, by the provable equivalence of terms here, we do not mean syntactic identity as symbol-strings; rather, we mean that there is a provable equality sentence whose left and right sides are comprised of these terms. That is, whatever the actual result of the function $m$ on the input $x$ is, we must have that this is also provably the same as applying $m$ to the input $x$. Here, again, the $\Sigma_1$-ness of the formula defining $m$ comes to our rescue, telling us that truth entails provability in the appropriate way.

Thus, we obtain an introspective theory $\langle \ZfinSigma, \InnerZfinSigma, \introS, \introN \rangle$. This concludes our first nontrivial example of an instrospective theory!
\end{construction}

\bigskip
However, $\langle \ZfinSigma, \InnerZfinSigma, \introS, \introN \rangle$ is not actually the most typical introspective theory! It has special properties which we should not expect of a general introspective theory. Its internal $\InnerZfinSigma$ acts as a perfect mirror image of $\ZfinSigma$, and can thus itself be equipped as an internal introspective theory. The internal $\InnerZfinSigma$ has in some informal sense no further objects (or morphisms, or equations) beyond the range of $\introS$. All of this is not typical for an introspective theory.

\begin{construction}\label{SigmaModelComplex}
Let us describe now a more archetypal introspective theory, to guide the reader's intuitions better for how general introspective theories act.

Throughout the construction of $\ZfinSigma$, we have imposed a $\Sigma_1$ constraint on formulae (both on the formulae defining objects and on the formulae defining morphisms). If we drop all such $\Sigma_1$ constraints and allow arbitrary formulae, we get by the same construction an analogous category $\Zfin$. $\ZfinSigma$ sits inside $\Zfin$ as a subcategory (but not a full subcategory! The inclusion from $\ZfinSigma$ into $\Zfin$ is faithful, but not full).

Just as the construction of $\ZfinSigma$ could itself be carried out in ZF-Finite to get a $\InnerZfinSigma$ internal to $\ZfinSigma$, so too can the construction of $\Zfin$ can be carried out in ZF-Finite, to get a $\InnerZfin$ internal to $\ZfinSigma$. Yes, this $\InnerZfin$ is internal to $\ZfinSigma$, not just internal to $\Zfin$! Even though $\Zfin$ includes as its objects and morphisms formulae which are not $\Sigma_1$, the description of $\Zfin$ (as a category whose objects are symbol-strings for which certain other symbol-strings exist, and whose morphisms are symbol-strings for which certain other symbol-strings exist, and so on) is $\Sigma_1$.

Finally, the inclusion of $\ZfinSigma$ into $\Zfin$ yields, analogously, an inclusion from $\InnerZfinSigma$ into $\InnerZfin$, internal to $\ZfinSigma$. This means the functor $\introS$ from $\ZfinSigma$ into the global aspect of $\InnerZfinSigma$ can just as well be thought of as having $\InnerZfin$ for its codomain, and similarly the natural transformation $\introN$ can just be well as thought of in this context. (This way of making one introspective theory from another is an instance of the general construction \magicref{IntrospInternalMap}.)

Summarizing, we get an introspective theory $\langle \ZfinSigma, \InnerZfin, \introS, \introN \rangle$, where $\ZfinSigma$ is the lexcategory of $\Sigma_1$-definable hereditarily finite sets and $\Sigma_1$-definable functions between them up to provable equivalence in ZF-Finite, $\InnerZfin$ is the lexcategory internal to $\ZfinSigma$ of arbitrary definable sets and arbitrary definable functions between them up to provable equivalence in ZF-Finite, $\introS$ assigns to each piece of $\ZfinSigma$ the corresponding (globally defined) piece of $\InnerZfin$, and $\introN$ is the $\Sigma_1$-definable function which sends any hereditarily finite set to the canonical term describing it, as well as witnessing the provable entailment from truth to provability for $\Sigma_1$ formulae.
\end{construction}

Phew! What a long walk it was to get to describing that example! All the better, then, that we have formalized introspective theories so abstractly, and can work with them without having to fuss about such concrete details as in that example. But this is indeed the archetypal example it will be best to keep in mind to guide the reader's intuition throughout all further discussion.

\begin{warningenv}
While we have above constructed introspective theories $\langle \ZfinSigma, \InnerZfinSigma \rangle$ and $\langle \ZfinSigma, \InnerZfin \rangle$, the reader should be cautioned that there is no natural introspective theory $\langle \Zfin, \InnerZfin \rangle$. As a check of their understanding, the reader is encouraged to think about why this is.
\end{warningenv}

\subsubsection{Examples based on presheaf categories}
\sTODOinline{Should we perhaps delete part or all of this section entirely and just cover its material in \magicref{GeneralPresheafLocalIntrosp} and \magicref{CardinalityConstrainedPresheafIntrosp} and ensuing discussion there? Or is it useful for readers to see all four of the constructions in this section up front as introductory examples?}

In this subsection, we will give some other instructive examples of introspective theories based on presheaf categories. These examples are of a very different flavor from those based on logical theories as in the previous section, thus helping to illustrate the generality of the notion of introspective theory.

The examples in the first half of this section are based on the topos of trees and the \quote{later} modality, as used in much work on step-indexing and guarded recursion. This may also be useful to build up intuition as we work towards the more complicated final examples in the latter half of this section.

The examples in the latter half of this section are closely related to the use of Kripke frames to interpret the K4 and GL modal logics.

In both examples in this section, we first construct a locally introspective theory using an unrestricted presheaf category. We then impose some cardinality constraints to cut these down into introspective theories.

All the constructions in this section are unified and vastly generalized in \magicref{GeneralPresheafLocalIntrosp} and \magicref{CardinalityConstrainedPresheafIntrosp}.

\paragraph{Presheaf example related to step-indexing in guarded recursion}

We present this example in terms of presheaves over the natural numbers (which comprise the so-called topos of trees), but analogous examples may be constructed for presheaves over arbitrary categories; see the generalization at \magicref{ModelsBasedOnPresheafCategories}. We focus on presheaves over the natural numbers in this introductory example as it is perhaps the simplest nontrivial presheaf category to consider, and also as the \quote{later} modality on the topos of trees which is at the core of this example is much studied in the literature on guarded recursion (e.g., as in \cite{birkedal2011first}).

\begin{construction}\label{StepIndexingLocallyIntrosp}
Let $\omega$ be the poset of natural numbers with their usual ordering, and consider the category of presheaves $\Psh{\omega}$ (often called the topos of trees). We will equip this as a locally introspective theory.

The functor $\mathrm{Succ} : n : \omega \mapsto n + 1 : \omega$ induces correspondingly a functor $\pullAlong{\mathrm{Succ}} : \Psh{\omega} \to \Psh{\omega}$. For convenience, we will use the name $\Prior$ to refer to this endofunctor on $\Psh{\omega}$. Thus, $\Prior(P)(n) = P(n + 1)$ for $n \in \omega$.

The map $n \leq n + 1$ from identity to $\mathrm{Succ}$ as endofunctors on $\omega$ induces a corresponding map from $\Prior$ to identity as endofunctors on $\Psh{\omega}$ (keeping in mind the contravariance of presheaves). We shall write $\prior : \Prior \to \id$ for this map.

Also, as with any functor between presheaf categories given by composition in this manner, $\Prior$ has a right adjoint, given by right Kan extension. [The right adjoint of $\Prior$ may be called $\Later$, or is often called $\later$ in guarded recursion literature. It can be described by $\Later(P)(0) = 1$ and $\Later(P)(n + 1) = P(n)$ for $n \in \omega$, with the obvious corresponding actions on restriction maps and on morphisms between presheaves. Note that we may pull the map $\prior : \Prior \to \id$ through the adjunction $\Prior \dashv \Later$ to obtain a corresponding map $\latermap : \id \to \Later$.]

Let $C$ be the $\Psh{\omega}$-indexed lexcategory given by $C(-) = \Psh{\omega}/\Prior(-)$. That is, $C$ is given by applying $\pullAlong{\Prior}$ to the self-indexing $\Psh{\omega}/-$. Note that $C$ is locally \repsmall/ by \magicref{RepLocallySmallRightAdjoint}, as $\Prior$ has a right adjoint and the self-indexing $\Psh{\omega}/-$ is locally \repsmall/ (because $\Psh{\omega}$ is locally cartesian closed, as it is a presheaf topos).

What remains to equip $\langle \Psh{\omega}, C \rangle$ as a locally introspective theory is to choose a suitable $\introF$ from $\Psh{\omega}/-$ to $C$. We do this via whiskering $\prior$ as in the following diagram:

% https://q.uiver.app/?q=WzAsMyxbMCwwLCJcXG9we1xcUHNoe1xcb21lZ2F9fSJdLFsyLDAsIlxcb3B7XFxQc2h7XFxvbWVnYX19Il0sWzQsMCwiXFxMZXhDYXQiXSxbMSwyLCJcXFBzaHtcXG9tZWdhfS8tIl0sWzAsMSwiXFxpZCIsMCx7ImxldmVsIjoyLCJzdHlsZSI6eyJoZWFkIjp7Im5hbWUiOiJub25lIn19fV0sWzAsMSwiXFxvcHtcXFByaW9yfSIsMix7ImN1cnZlIjo1fV0sWzQsNSwiXFxvcHtwcmlvcn0iLDAseyJzaG9ydGVuIjp7InNvdXJjZSI6MjAsInRhcmdldCI6MjB9fV1d
\[\begin{tikzcd}
	{\op{\Psh{\omega}}} && {\op{\Psh{\omega}}} && \LexCat
	\arrow["{\Psh{\omega}/-}", from=1-3, to=1-5]
	\arrow[""{name=0, anchor=center, inner sep=0}, "\id", Rightarrow, no head, from=1-1, to=1-3]
	\arrow[""{name=1, anchor=center, inner sep=0}, "{\op{\Prior}}"', curve={height=30pt}, from=1-1, to=1-3]
	\arrow["{\op{prior}}", shorten <=4pt, shorten >=4pt, Rightarrow, from=0, to=1]
\end{tikzcd}\]

[Pedantically, in this diagram, $\LexCat$ must be understood as including lexcategories of comparable size to $\Psh{\omega}$, so that the self-indexing of $\Psh{\omega}$ is valued in $\LexCat$.]

Again, keep in mind the contravariance of indexed structures here, so that $\prior : \Prior \to \id$ does indeed act as a map from any $\Psh{\omega}$-indexed structure $X$ into the corresponding $\pullAlong{\Prior}(X)$.

Thus, we have constructed a locally introspective theory $\langle \Psh{\omega}, C \rangle$ with $\introF$ given by whiskering $\prior$ as in the above diagram.

It may be illustrative to alternatively describe this $\introF$ in terms of its corresponding $\introS$ and $\introN$.

For $\introS$, let us first observe that $\Prior(1) = 1$; that is, $\Prior$ preserves the terminal object\footnote{This is closely related to the fact that $\omega$ has no maximal element, and would need modification were we carrying out the analogous construction for a poset which had maximal elements.}. Accordingly, $\Glob{C} = \Psh{\omega}/\Prior(1) = \Psh{\omega}$. And as the component $\prior_1 : \Prior(1) \to 1$ must be the identity on the terminal object, the map $\introS : \Psh{\omega} \to \Glob{C}$ corresponding to our choice of $\introF$ becomes the identity under this identification. 

Furthermore, the map $\Hom_C(1, \introS(-)) : \Psh{\omega} \to \Psh{\omega}$ can be seen to be the right adjoint to $\Prior$; thus, it is $\Later$. Finally, as for the $\introN$ corresponding to our $\introF$, this will be the map $\latermap : \id \to \Later$ given by pulling $\prior : \Prior \to \id$ through the adjunction $\Prior \dashv \Later$. \TODOinline{Flesh this out more?}
\end{construction}

This way of equipping $\Psh{\omega}$ as a locally introspective theory is illustrative. Unfortunately, this is not an introspective theory, as our $C = \Psh{\omega}/\Prior(-)$ is merely locally \repsmall/, not \repsmall/ simpliciter.

We do not have that $\Ob(C)$ is itself an object of $\Psh{\omega}$. Essentially, the obstruction is that $C(\yoneda(n)) = \Psh{\omega}/\Prior(\yoneda(n)) = \Psh{\omega}/\yoneda(n - 1)$ (for $n \geq 1$) has a proper class of objects, but the presheaves in $\Psh{\omega}$ are set-valued.

We might naively try to ameliorate this problem by replacing $\Psh{\omega} = \Set^{\omega}$ by $\left(\Set'\right)^{\omega}$ where $\Set'$ is some full subcategory of $\Set$, such as sets of cardinality below some particular cardinal. But it is soon seen that such a uniform cardinality constraint across all $n \in \omega$ will not be workable for fixing the issue.

Rather, what will fix the issue is to impose a variable cardinality constraint: We shall consider those presheaves whose values at each $n$ come from a particular full sublexcategory $\Set_n$ of $\Set$, where these restrictions get looser as $n$ gets larger.

\newcommand{\PshUnderN}[1]{\mathrm{Psh}'(\omega_{<#1})}
\newcommand{\PshUnder}{\mathrm{Psh}'(\omega)}

\begin{construction}\label{StepIndexingIntrosp}
Let $\Set_n$ for each  $n \in \omega$ be a \setsmall/ full sublexcategory of $\Set$. We shall think of each $\Set_n$ as a strict category. By $\PshUnder$, we mean the full sublexcategory of $\Psh{\omega}$ comprising presheaves $P$ such that $P(m) \in \Set_m$ for all $m \in \omega$.

By $\omega_{< n}$, we mean the sub-poset of $\omega$ restricted to those naturals which are less than $n$. By $\PshUnderN{n}$, we mean the full sublexcategory of $\Psh{\omega_{< n}}$ comprising presheaves such that $P(m) \in \Set_m$ for each $m < n$.

Observe that each $\PshUnderN{n}$ can be viewed as a \setsmall/ strict lexcategory. Its collection of objects and its collection of morphisms are readily seen to comprise bona fide sets. 

We also have obvious restriction maps from $\PshUnderN{n}$ to $\PshUnderN{m}$ for $m \leq n$ induced by the inclusion of $\omega_{< m}$ into $\omega_{< n}$, and any composition of such restriction maps yields the appropriate such restriction map.

Thus, we have an $\omega$-indexed \setsmall/ strict lexcategory $C'(n) = \PshUnderN{n}$. In other words, this $C'$ is a lexcategory internal to $\Psh{\omega}$.

Also note that once $\Set_0, \Set_1, \ldots, \Set_{n - 1}$ are determined, we have already determined what $\PshUnderN{n}$ is.

Thus, we may inductively choose $\Set_n$ for each $n$ such that both $\Ob(\PshUnderN{n})$ and $\Mor(\PshUnderN{n})$ are among the objects of $\Set_n$. 

[For example, we may satisfy this condition by choosing $\Set_n$ for each $n$ to be the von Neumann universe $V_{(n + 1) \times \omega}$ of sets of rank less than $(n + 1) \times \omega$. Many other possibilities are available, this is only one suggestion.]

When we choose $\Set_n$ satisfying this inductive condition, we have that $C'$ is not only internal to $\Psh{\omega}$, but indeed is internal to its full subcategory $\PshUnder$.

We now flesh $\langle \PshUnder, C'\rangle$ out into an introspective theory, by defining an appropriate $\introS$ and $\introN$. Much like before, $\Glob{C'}$ is readily identified with $\PshUnder$ and we take  $\introS$ to be this identification. As just as before, we find that under this identification, $\Hom_C(1, -)$ acts as $\Later$, so we may take $\introN$ to be $\latermap : \id \to \Later$. \TODOinline{Flesh this out more.}

This completes the description of $\langle \PshUnder, C' \rangle$ as an introspective theory  (relative to any suitable choice of the $\{\Set_n\}_{n \in \omega}$).
\end{construction}

\paragraph{Presheaf examples related to Kripke frames}

Here, we consider examples of locally introspective and introspective theories based on Kripke frames. Of note, our first construction of a locally introspective theory works for any transitive Kripke frame (corresponding to the K4 modal logic). When we attempt to make an introspective theory of this by imposing cardinality constraints, we will find we are only able to do this if the transitive Kripke frame is furthermore well-founded (corresponding to the GL modal logic).

\begin{construction}\label{KripkeLocallyIntrosp}
Let $<$ be a transitive relation on a discrete set $|P|$. The reflexive closure $\leq$ of $<$ equips $|P|$ as a preorder $P$. Let $Q$ be $P$ augmented with one further element $\infty$ which is greater than every element from $P$.

There is an inclusion functor $i : |P| \to Q$, and this induces correspondingly a functor $\pullAlong{i} : \Psh{Q} \to \Psh{|P|}$.

By $|P|_{< q}$ (where $q$ is any value in $Q$), we mean the discrete subset of $|P|$ comprising those values which, within $Q$, are less than $q$. Note that when $p \leq q$, there is a forgetful functor from $\Psh{|P|_{< q}}$ to $\Psh{|P|_{< p}}$ induced by the inclusion of $|P|_{< p}$ into $|P|_{< q}$. Any composition of such forgetful functors is another forgetful functor of the same form.

Thus, we obtain a $Q$-indexed lexcategory $C(q) = \Psh{|P|_{< q}}$. \TODOinline{Demonstrate this is in fact locally \repsmall/.} We will now equip $\langle \Psh{Q}, C \rangle$ as a locally introspective theory by providing a suitable $\introS$ and $\introN$.

Note that $\Glob{C} = \Psh{|P|_{< \infty}} = \Psh{|P|}$. (Here, the addition of $\infty$ into $Q$ plays an important role when $P$ contains maximal elements. If we took $C$ to be merely a $P$-indexed category, then we would find that $\Glob{C}$ ignored any maximal elements in $P$. This is the only reason for our introduction of $\infty$.)

Thus, $\pullAlong{i} : \Psh{Q} \to \Glob{C}$. We may refer to this also as $\introS$.

Furthermore, note that for $X \in \Psh{Q}$, we have that $\Hom_C(1, \introS(X))$ is the presheaf on $Q$ which assigns to $q \in Q$ the product of $X(p)$ over all $p < q$, with restriction maps given by forgetting components as appropriate. \TODOinline{Demonstrate this}.

Thus, we have a map $\introN : \id_{\Psh{Q}} \to \Hom_{C}(1, \introS(-))$, such that $\introN_X(q) : X(q) \to \Hom_C(1, \introS(X))(q)$ is given by the product of all the restriction maps out of $X(q)$ (these restriction maps being part of the structure of the presheaf $X$ itself). \TODOinline{Flesh this out.}
\end{construction}

\begin{observation}
The reader is advised to keep in mind that this last construction is very different from \magicref{StepIndexingLocallyIntrosp}, even if $P$ is chosen to be the poset $\omega$ of natural numbers. This distinction is emphasized again in the later discussion at \TODOinline{cite Modal Logic discussion of these examples}.
\end{observation}

This is an important archetypal example of a locally introspective theory. It corresponds closely to the interpretation of K4 modal logic using a transitive Kripke frame. However, just as at the beginning of our previous example, we have the issue that this is only a locally introspective theory and not an introspective theory. Once again, the various aspects of $C$ comprise a proper class of objects, too many for $\Ob(C)$ to be given by a set-valued presheaf, preventing $C$ from being \repsmall/. And as in our previous example, we will again address this by imposing variable cardinality constraints on our presheaves.

\begin{construction}\label{KripkeIntrosp}
We will from hereon out assume that the preorder $P$ is in fact well-founded (and thus so is $Q$). Suppose given \setsmall/ full sublexcategories $\Set_q$ of $\Set$ for each $q \in Q$. (It's not actually necessary that we restrict to such a subcategory at $q = \infty$, but for uniformity's sake, we do this for now.). We shall think of each $\Set_q$ as a strict category.

\newcommand{\PshUnderQ}[1]{\mathrm{Psh}'(|P|_{<#1})}
\newcommand{\PshUnderQInf}{\mathrm{Psh}'(Q)}

We define $\PshUnderQInf$ to be the full sublexcategory of $\Psh{Q}$ comprising presheaves $X$ for which $X(q) \in \Set_q$ for each $q \in Q$. And we analogously define $\PshUnderQ{q}$ to be the full sublexcategory of $\Psh{|P|_{< q}}$ comprising presheaves $X$ for which $X(p) \in \Set_p$ for each $p < q$.

There are restriction maps from $\PshUnderQ{q}$ to $\PshUnderQ{p}$ for $p \leq q$ induced by the inclusion of $|P|_{< p}$ into $|P|_{< q}$, and any composition of such restriction maps is such a restriction map. As a result, we have a $\Psh{Q}$-internal lexcategory $C'$ whose component at $q \in Q$ is given by $\PshUnderQ{q}$.

The \setsmall/ category $\PshUnderQ{q}$ only depends on the values of $\Set_p$ for $p < q$, and thus we may inductively choose $\Set_q$ in such a way that $\Ob(\PshUnderQ{q})$ as well as $\Mor(\PshUnderQ{q})$ are both objects of $\Set_q$ for each $q \in Q$. When we have done so, it follows that $C'$ is not merely internal to $\Psh{Q}$ but furthermore lives within $\PshUnderQInf$.

We observe that there is a forgetful lexfunctor $\introS : \PshUnderQInf \to \Glob{C'} = \PshUnderQ{\infty}$, induced by the inclusion of $|P|_{< \infty} = |P|$ into $Q$.

We observe as above that for $X \in \PshUnderQInf$, we have that $\Hom_{C'}(1, \introS(X))$ is the presheaf on $Q$ which assigns to $q \in Q$ the product of $X(p)$ over all $p < q$, with restriction maps given by forgetting components as appropriate.

Finally, we define $\introN$ in the same way as above, with $\introN_X(q) : X(q) \to \Hom_{C'}(1, \introS(X))(q)$ given by the product of all the restriction maps $: X(q) \to X(p)$ for $p < q$.

In this way, we have constructed an introspective theory $\langle \PshUnderQInf, C' \rangle$ (relative to any suitable choice of the $\{\Set_q\}_{q \in Q}$).
\end{construction}

This is an important archetypal example of an introspective theory. It corresponds closely to the interpretation of GL modal logic using a well-founded Kripke frame.

\subsection{Basic constructions}
Now let us discuss some general constructions for building new (pre-)introspective theories from old ones or from other data.

\begin{construction}\label{IntrospInternalMap}
If $\langle T, C, \introF \rangle$ is a pre-introspective theory, and any lexfunctor $G : C \to D$ is given for some other $T$-indexed lexcategory $D$, then $\langle T, D, G \circ \introF \rangle$ is itself a pre-introspective theory, like so: 

% https://q.uiver.app/?q=WzAsMixbMCwwLCJcXG9we1R9Il0sWzIsMCwiXFxMZXhDYXQiXSxbMCwxLCJDIiwxXSxbMCwxLCJULy0iLDAseyJvZmZzZXQiOi01fV0sWzAsMSwiRCIsMix7Im9mZnNldCI6NX1dLFszLDIsIlxcaW50cm9GIiwyLHsic2hvcnRlbiI6eyJzb3VyY2UiOjIwLCJ0YXJnZXQiOjQwfX1dLFsyLDQsIkciLDIseyJzaG9ydGVuIjp7InNvdXJjZSI6NDAsInRhcmdldCI6MjB9fV1d
\[\begin{tikzcd}
	{\op{T}} && \LexCat
	\arrow[""{name=0, anchor=center, inner sep=0}, "C"{description}, from=1-1, to=1-3]
	\arrow[""{name=1, anchor=center, inner sep=0}, "{T/-}", shift left=5, from=1-1, to=1-3]
	\arrow[""{name=2, anchor=center, inner sep=0}, "D"', shift right=5, from=1-1, to=1-3]
	\arrow["\introF"', shorten <=1pt, shorten >=3pt, Rightarrow, from=1, to=0]
	\arrow["G"', shorten <=3pt, shorten >=1pt, Rightarrow, from=0, to=2]
\end{tikzcd}\]

Of course, this yields an introspective or locally introspective theory just in case $D$ is \repsmall/ or locally \repsmall/, respectively.
\end{construction}

\begin{construction}\label{IntrospPullback}
If $\langle T, C, \introF \rangle$ is a pre-introspective theory, $U$ is any lexcategory, and $\Sigma: U \to T$ is any functor which preserves pullbacks (we do not require $\Sigma$ to preserve the terminal object), then $\langle U, \pullAlong{\Sigma} C \rangle$ can naturally be equipped as a pre-introspective theory, like so:
\end{construction}
\begin{proof}[Details]
% https://q.uiver.app/?q=WzAsMyxbMCwwLCJcXG9we1V9Il0sWzIsMCwiXFxvcHtUfSJdLFs0LDAsIlxcTGV4Q2F0Il0sWzEsMiwiVC8tIiwwLHsib2Zmc2V0IjotMn1dLFsxLDIsIkMiLDIseyJvZmZzZXQiOjJ9XSxbMCwxLCJcXG9we1xcU2lnbWF9Il0sWzAsMiwiVS8tIiwwLHsib2Zmc2V0IjotNSwiY3VydmUiOi0zfV0sWzMsNCwiXFxpbnRyb0YiLDIseyJzaG9ydGVuIjp7InNvdXJjZSI6MjAsInRhcmdldCI6MjB9fV0sWzYsMSwiXFxTaWdtYSIsMCx7InNob3J0ZW4iOnsic291cmNlIjoyMH19XV0=
\[\begin{tikzcd}
	{\op{U}} && {\op{T}} && \LexCat
	\arrow[""{name=0, anchor=center, inner sep=0}, "{T/-}", shift left=2, from=1-3, to=1-5]
	\arrow[""{name=1, anchor=center, inner sep=0}, "C"', shift right=2, from=1-3, to=1-5]
	\arrow["{\op{\Sigma}}", from=1-1, to=1-3]
	\arrow[""{name=2, anchor=center, inner sep=0}, "{U/-}", shift left=5, curve={height=-18pt}, from=1-1, to=1-5]
	\arrow["\introF"', shorten <=1pt, shorten >=1pt, Rightarrow, from=0, to=1]
	\arrow["\Sigma", shorten <=3pt, Rightarrow, from=2, to=1-3]
\end{tikzcd}\]

The 2-cell labelled $\Sigma$ above indicates the action of $\Sigma$ when acting as a lexfunctor from $U/u$ to $T/(\Sigma u)$ for each object $u$ in $U$. (Note that, as finite limits in slice categories are given by pullbacks in the underlying category, and as $\Sigma$ preserves pullbacks, we do indeed have that this functor from $U/u$ to $T/(\Sigma u)$ preserves finite limits.)

By \magicref{RepsmallRightAdjoint} or \magicref{RepLocallySmallRightAdjoint}, if $\Sigma$ has a right adjoint, we can further observe that if $C$ is \repsmall/ or locally \repsmall/, then so respectively will be $\pullAlong{\Sigma} C$.
\end{proof}

\sTODOinline{Note that functors between lexcategories preserving pullbacks and having right adjoints are commonly studied; this is the same as the notion of a geometric functor into a slice category. So there is a panoply of examples of this construction. But actually, the geometric embedding of sheaves into presheaves only directly allows us to take introspective theories on sheaf categories and obtain from them introspective theories on presheaf categories, via the above, since the direction of the left adjoint sheafification functor is from the presheaf category to the sheaf category. It feels like the opposite is what we would want. If we want to take an introspective theory on a presheaf category to an introspective theory on a sheaf category, what we should do instead is sheafify C to get C' along with a map from C to C', then take $\Sigma$ in the above to be the inclusion functor from sheaves into presheaves. This inclusion doesn't have to have a right adjoint, but C' will automatically be representable over the sheaf category (basically by definition), and thus we get an introspective theory on sheaves. This may also apply in some way to our cardinality-constrained presheaf categories but we'll have to think about the interaction of that with sheafification. Presumably, we'll need to choose ramps which are suitably closed so that the cardinality-constrained presheaves are closed under sheafification. Some version of this probably applies also to merely locally introspective theories. Let's move this note into the Examples chapter section on presheaves.}

A particular special case of the above which is often of importance is the following:

\openNamed{construction}{Slice Pre-Introspective Theories}\label{IntrospSlice}
If $\langle T, C, \introF \rangle$ is a pre-introspective theory, and $t$ is any object in $T$, then the slice category $T/t$ can be equipped in a natural way as a pre-introspective theory as well. If we start from an introspective or locally introspective theory, then so respectively will be the result of this construction.
\closeNamed{construction}
\begin{proof}[Details]
By the previous construction (\cref{IntrospPullback}), using the forgetful functor $\Sigma : T/t \to T$, which preserves pullbacks and has a right adjoint (given by pullback).

(Note that in this case, the corresponding 2-cell from $(T/t)/-$ to $T/\Sigma(-)$ is an equivalence, by how iterated slice categories amount to slice categories simpliciter.)
\end{proof}

When we abuse language and speak of $T/t$ as an introspective theory, the above construction is what we mean. \sTODOinline{Do we actually use this abuse of language?}

\sTODOinline{This freely augments $T$ with a single global point of $t$. There should be some generalization of the above that freely augments $T$ with as many global points of as many types as we like; that is, given a lexfunctor $M$ from $T$ to $\Set$ serving as a particular model of $T$, we should be able to freely augment $T$ to $T[M]$ such that all $T[M]$'s models extend $M$. I believe this amounts to Definition 7.14 of "A General Framework for the Semantics of Type
Theory" by Uemura, or section 5.4 of Uemura's PhD thesis ("Abstract and Concrete Type Theories").

Perhaps more generally, whenever we have a lexfunctor from $T$ to $S$, we can left or right Kan extend along this lexfunctor to turn a (pre)(locally)(actual)introspective theory structure on $T$ into such structure on $S$? Using the fact that left Kan extension along lexfunctors is lex, and left Kan extension along arbitrary functors takes representables to representables. Or the fact that right Kan extension along arbitrary functors is Lex. The slice category construction is then also the special case of this where we consider the lex map from $T$ into $T/t$ (which has an adjoint).}

\sTODOinline{This also means that the self-indexing is not just an indexed lexcategory, but an indexed pre-/locally/fully introspective theory, according as to the base.}

\begin{construction}\label{SubTPreIntrosp}
If $\langle T, C, \introF \rangle$ is a pre-introspective theory, and $S$ is a full sub-lexcategory of $T$ (thus, with a full and faithful inclusion lexfunctor $i : S \to T$), then $\langle S, \pullAlong{i} C \rangle$ can be equipped in a natural way as a pre-introspective theory as well.
\end{construction}
\begin{proof}[Details]
By \cref{IntrospPullback} again, taking $\Sigma$ to be the inclusion functor $i$.
\end{proof}

\begin{construction}\label{SubCPreIntrosp}
If $\langle T, C, \introF \rangle$ is a pre-introspective theory, and $D$ is a $T$-indexed full sub-lexcategory of $C$ containing the range of $\introF$ (thus, such that $\introF = i \circ \introF'$ for a uniquely determined $\introF' : T/- \to D$, where $i : D \to C$ is the inclusion), then $\langle T, D, \introF' \rangle$ is a pre-introspective theory.

[In this case, conversely, $\langle T, C, \introF \rangle$ is obtained from $\langle T, D, \introF' \rangle$ and $i : D \to C$ via \magicref{IntrospInternalMap}.]
\end{construction}

The last two constructions are often fruitfully combined: Given a pre-introspective theory $\langle T, C, \introF \rangle$, we may first pass from $T$ to a sub-lexcategory $S$ of $T$ and then, after having done so, find that $\introF$ when restricted to $S$ factors through a sub-lexcategory $D$ of $C$. In particular, the following scenario will be of note to us:

\begin{construction}\label{SubIntrospectionDefn}
Let $\langle T, C, \introF \rangle$ be a pre-introspective theory, let $S$ be a full sublexcategory of $T$ (with inclusion $i : S \to T$), and let $D$ be a $T$-indexed full sub-lexcategory of $C$ (with inclusion $j : D \to C$). Suppose furthermore that this $D$ is of the form $i(D')$ for some \repsmall/ $S$-indexed lexcategory $D'$. (It follows that this $D'$ is identified with $\pullAlong{i} D$.)

Finally, suppose also that $\introF$ restricted to $S$ has range restricted to $D$, in that there is a (uniquely determined) $\introF'$ making the following composite 2-cells equal:

% https://q.uiver.app/?q=WzAsNixbMCwwLCJcXG9we1N9Il0sWzIsMCwiXFxvcHtUfSJdLFs0LDAsIlxcTGV4Q2F0Il0sWzAsMiwiXFxvcHtTfSJdLFsyLDIsIlxcb3B7VH0iXSxbNCwyLCJcXExleENhdCJdLFswLDEsIlxcb3B7aX0iLDJdLFsxLDIsIlQvLSIsMCx7Im9mZnNldCI6LTJ9XSxbMSwyLCJDIiwyLHsib2Zmc2V0IjoyfV0sWzAsMiwiUy8tIiwwLHsiY3VydmUiOi01fV0sWzMsNCwiXFxvcHtpfSIsMl0sWzQsNSwiRCIsMCx7Im9mZnNldCI6LTJ9XSxbMyw1LCJTLy0iLDAseyJjdXJ2ZSI6LTV9XSxbNCw1LCJDIiwyLHsib2Zmc2V0IjoyfV0sWzcsOCwiXFxpbnRyb0YiLDIseyJzaG9ydGVuIjp7InNvdXJjZSI6MjAsInRhcmdldCI6MjB9fV0sWzksMSwiaSIsMix7ImxhYmVsX3Bvc2l0aW9uIjo3MCwic2hvcnRlbiI6eyJzb3VyY2UiOjIwfX1dLFsxMSwxMywiaiIsMix7InNob3J0ZW4iOnsic291cmNlIjoyMCwidGFyZ2V0IjoyMH19XSxbMTIsNCwiXFxpbnRyb0YnIiwyLHsibGFiZWxfcG9zaXRpb24iOjkwLCJzaG9ydGVuIjp7InNvdXJjZSI6MjB9fV1d
\[\begin{tikzcd}
	{\op{S}} && {\op{T}} && \LexCat \\
	\\
	{\op{S}} && {\op{T}} && \LexCat
	\arrow["{\op{i}}"', from=1-1, to=1-3]
	\arrow[""{name=0, anchor=center, inner sep=0}, "{T/-}", shift left=2, from=1-3, to=1-5]
	\arrow[""{name=1, anchor=center, inner sep=0}, "C"', shift right=2, from=1-3, to=1-5]
	\arrow[""{name=2, anchor=center, inner sep=0}, "{S/-}", curve={height=-30pt}, from=1-1, to=1-5]
	\arrow["{\op{i}}"', from=3-1, to=3-3]
	\arrow[""{name=3, anchor=center, inner sep=0}, "D", shift left=2, from=3-3, to=3-5]
	\arrow[""{name=4, anchor=center, inner sep=0}, "{S/-}", curve={height=-30pt}, from=3-1, to=3-5]
	\arrow[""{name=5, anchor=center, inner sep=0}, "C"', shift right=2, from=3-3, to=3-5]
	\arrow["\introF"', shorten <=1pt, shorten >=1pt, Rightarrow, from=0, to=1]
	\arrow["i"'{pos=0.7}, shorten <=3pt, Rightarrow, from=2, to=1-3]
	\arrow["j"', shorten <=1pt, shorten >=1pt, Rightarrow, from=3, to=5]
	\arrow["{\introF'}"'{pos=0.9}, shorten <=3pt, Rightarrow, from=4, to=3-3]
\end{tikzcd}\]

Then $\langle S, D', \introF' \rangle$ is an introspective theory, which we refer to as a \defined{sub-introspection} of $\langle T, C, \introF \rangle$.
\end{construction}

Observe that our constructions \magicref{StepIndexingIntrosp} and \magicref{KripkeIntrosp} were given as sub-introspections of \magicref{StepIndexingLocallyIntrosp} and \magicref{KripkeLocallyIntrosp}, respectively. \TODOinline{Make this clearer; it may not be clear that the internal categories used were in fact full subcategories of the restriction of the original self-indexing.}

\bigskip
The concepts of pre-introspective, locally introspective, or introspective theories are all nearly essentially algebraic concepts (\quote{nearly}, in that these involve categories up to equivalence rather than strict categories up to isomorphism). Thus these concepts automatically have available all the same properties as for any such nearly essentially algebraic concept. For example, we have free constructions, as are the subject of of our later chapter \magicref{GeminalChapter}. And we have Cartesian products in the straightforward way:

\begin{construction}
Let $K$ be any set, and suppose for each $k \in K$ we are given some pre-introspective theory $\langle T_k, C_k \rangle$. Then we may define the product of these pre-introspective theories in the obvious way. That is, we take the lexcategory $T = \prod_{k \in K} T_k$, and we define also a $T$-indexed lexcategory $C$, such that $C(t)$ for any object $t = \{ t_k \}_{k \in K}$ in $T$ is the product of $C_k(t_k)$ over each $k \in K$. Similarly, we define the reindexing functors in $C$ componentwise using the reindexing functors the various $C_k$, and we also define $\introF : T/- \to C$ componentwise using the $\introF$ for the various pre-introspective theories $\langle T_k, C_k \rangle$. It is readily seen that the result is furthermore locally introspective or introspective if each $\langle T_k, C_k \rangle$ is locally introspective or introspective, respectively.
\end{construction}

\sTODOinline{Make the useful observations that the theory of introspective theories is essentially lex, and that we can take therefore take products of introspective theories in the straightforward way. (We could also take limits of strict introspective theories more generally, but that involves talking about object equality, which we don't really want to do.). We can also take sub-introspective theories generated as the hull of subsets of their objects and morphisms, or other such free constructions.}

\bigskip
\TODOinline{Since we use the following only once, perhaps we should move it to its one use, at \magicref{Sigma1ModelIAUConnection}?}
\begin{construction}\label{LocalizeIntrosp}
If $\langle T, C, \introS, \introN \rangle$ is an introspective theory, and $f : T \to T[M^{-1}]$ is a lex localization in the sense of \magicref{DefnLexLocalization}, and every morphism in $M$ is sent to an isomorphism by $\introS$, then $f$ acts as an introspective theory homomorphism from $T$ to a uniquely determined introspective theory $\langle T[M^{-1}], f(C) \rangle$.

Furthermore, given any introspective theory homomorphism $h : \langle T, C \rangle \to \langle T_2, C_2 \rangle$ such that $h$ sends every morphism in $M$ to an isomorphism and $h[C] = C_2$, this $h$ factors uniquely through $f$ by an introspective theory homomorphism from $\langle T[M^{-1}], f(C) \rangle$ to $\langle T_2, C_2 \rangle$. In this sense, $\langle T[M^{-1}], f(C) \rangle$ is the localization qua introspective theory of $\langle T, C \rangle$ at $M$.

In particular, for any introspective theory, we can apply the above taking $M$ to be the set of all morphisms sent to isomorphisms by $\introS$. We may call the result the \defined{maximal localization} of our original introspective theory.
\end{construction}
\begin{proof}
If $f : T \to T[M^{-1}]$ is to act as an introspective theory homomorphism, it must be to some introspective theory $\langle T[M^{-1}], C', \introS', \introN' \rangle$. We will show that each of these components are uniquely determined by the requirements of \magicref{StrictIntrospHomoDefn} (\TODOinline{Or rather, its non-strict analogue. We need to move this theorem to after a suitable definition of introspective theory homomorphism has been given}).

The requirement on $C'$ in \magicref{StrictIntrospHomoDefn} directly determines it as $f(C)$.

The requirement on $\introS'$ is that $\introS' \circ f = \InducedHomo{f}{C} \circ \introS$. Note that the right hand side of this equation sends all morphisms in $M$ to isomorphisms (since $\introS$ already does so). Thus, by the defining property of localization, this uniquely determines $\introS'$ as a functor, and indeed it will be a lexfunctor by \magicref{LexLocalizationLemma}.

Finally, the requirement on $\introN'$ is that $\introN'$ whiskered along $f$ is $\introN$ whiskered along $f$ from the other side. By \magicref{LocalizingTransform}, this uniquely determines $\introN'$.

Next, we show the unique factorization property. \TODO.
\end{proof}

\subsection{The interaction of \texorpdfstring{$\introS$}{S} and \texorpdfstring{$\introN$}{N}}
We gather here two \wordsmall/ but useful lemmas for reasoning about (pre-)introspective theories, concerning the interaction of $\introS$ and $\introN$.

\openNamedManualIndexSort{lemma}{$\introS$ With $\introN$}{S With N}\label{SWithN}
Within a pre-introspective theory $\langle T, C \rangle$, let $F : X \to t$ be a morphism of $T$, and let $x$ be any generalized element of $X$. We have that $\introN_t(F(x)) = \introS(F) \circ_C \introN_X(x)$.
\closeNamed{lemma}
\begin{proof}
This is just the naturality square for $\introN$ with respect to $F$.

% https://q.uiver.app/?q=WzAsNCxbMCwwLCJYIl0sWzAsMSwiXFxIb21fQygxLCBcXGludHJvUyhYKSkiXSxbMSwxLCJcXEhvbV9DKDEsIFxcaW50cm9TKHQpKSJdLFsxLDAsInQiXSxbMCwxLCJcXGludHJvTl97WH0iLDJdLFsxLDIsIlxcaW50cm9TKEYpIFxcY2lyY19DIC0iLDJdLFswLDMsIkYiXSxbMywyLCJcXGludHJvTl97dH0iXV0=
\[\begin{tikzcd}
	X & t \\
	{\Hom_C(1, \introS(X))} & {\Hom_C(1, \introS(t))}
	\arrow["{\introN_{X}}"', from=1-1, to=2-1]
	\arrow["{\introS(F) \circ_C -}"', from=2-1, to=2-2]
	\arrow["F", from=1-1, to=1-2]
	\arrow["{\introN_{t}}", from=1-2, to=2-2]
\end{tikzcd}\]
\end{proof}

\openNamedManualIndexSort{lemma}{$\introS$ Matches $\introN$}{S Matches N}\label{SMatchesN}
Within a pre-introspective theory $\langle T, C \rangle$, let $t$ be some object of $T$ and let $\epsilon : 1 \to t$ in $T$ be taken as defining a global element $e$ of $t$. Then the global element $\introS(\epsilon)$ of $\Hom_C(\introS(1), \introS(t))$ is equal to the the global element $\introN_t(e)$ of $\Hom_C(1, \introS(t))$ under the canonical isomorphism identifying $\Hom_C(\introS(1), \introS(t))$ with $\Hom_C(1, \introS(t))$.

In short, $\introS$ and $\introN$ take global elements in $T$ to equal global elements of $C(1)$.
\closeNamed{lemma}
\begin{proof}
Consider the following commutative diagram  in $\Psh{T}$.

% https://q.uiver.app/?q=WzAsNyxbMiwwLCIxIl0sWzMsMCwidCJdLFszLDEsIlxcSG9tX0MoMSwgXFxpbnRyb1ModCkpIl0sWzIsMSwiXFxIb21fQygxLCBcXGludHJvUygxKSkiXSxbMywyLCJcXEhvbV9DKFxcaW50cm9TKDEpLCBcXGludHJvUyh0KSkiXSxbMiwyLCJcXEhvbV9DKFxcaW50cm9TKDEpLCBcXGludHJvUygxKSkiXSxbMCwyLCIxIl0sWzAsMywiXFxpbnRyb05fMSIsMl0sWzMsMiwiXFxpbnRyb1MoXFxlcHNpbG9uKSBcXGNpcmMgLSJdLFsxLDIsIlxcaW50cm9OX3QiXSxbMCwxLCIqIFxcbWFwc3RvIGUiXSxbMiw0LCItIFxcY2lyYyAhIl0sWzMsNSwiLSBcXGNpcmMgISIsMl0sWzUsNCwiXFxpbnRyb1MoXFxlcHNpbG9uKSBcXGNpcmMgLSJdLFs2LDAsIiIsMSx7ImN1cnZlIjotNSwibGV2ZWwiOjIsInN0eWxlIjp7ImhlYWQiOnsibmFtZSI6Im5vbmUifX19XSxbNiw1LCIqIFxcbWFwc3RvIFxcaWRfe1xcaW50cm9TKDEpfSJdLFs2LDQsIiogXFxtYXBzdG8gXFxpbnRyb1MoXFxlcHNpbG9uKSIsMix7ImN1cnZlIjo1fV1d
\[\begin{tikzcd}
	&& 1 & t \\
	&& {\Hom_C(1, \introS(1))} & {\Hom_C(1, \introS(t))} \\
	1 && {\Hom_C(\introS(1), \introS(1))} & {\Hom_C(\introS(1), \introS(t))}
	\arrow["{\introN_1}"', from=1-3, to=2-3]
	\arrow["{\introS(\epsilon) \circ -}", from=2-3, to=2-4]
	\arrow["{\introN_t}", from=1-4, to=2-4]
	\arrow["{* \mapsto e}", from=1-3, to=1-4]
	\arrow["{- \circ !}", from=2-4, to=3-4]
	\arrow["{- \circ !}"', from=2-3, to=3-3]
	\arrow["{\introS(\epsilon) \circ -}", from=3-3, to=3-4]
	\arrow[curve={height=-30pt}, Rightarrow, no head, from=3-1, to=1-3]
	\arrow["{* \mapsto \id_{\introS(1)}}", from=3-1, to=3-3]
	\arrow["{* \mapsto \introS(\epsilon)}"', curve={height=30pt}, from=3-1, to=3-4]
\end{tikzcd}\]

The top arrow is $\epsilon : 1 \to t$, thought of as sending the unique element of $1$ to $e$. The top rectangle is the naturality square for $\introN$ with respect to $\epsilon$.

The bottom rectangle is the associativity square for composition in $C$ (specifically, on one side composing with $\introS(\epsilon) : \introS(1) \to \introS(t)$ and on the other side composing with the unique morphism $! : \introS(1) \to 1$). Note that the right arrow of this associativity rectangle is the canonical isomorphism given by $\introS(1)$ being a terminal object of $C$.

The bottom wedge is the identity law for composition in $C$ (specifically, composing after the identity on $\introS(1)$).

Finally, the left wedge commutes because, as $\introS(1)$ is a terminal object of $C$, we have that $\Hom_C(\introS(1), \introS(1))$ is a terminal object of $\Psh{T}$; thus, any two parallel maps into it are equal. (Indeed, all arrows in the left wedge are unique isomorphisms between terminal objects.)

Now consider the composites around this commutative diagram along the two outermost paths. Along the bottom, the unique element of $1$ is sent to $\introS(\epsilon)$. Along the top and right, it is sent to $\introN_t(e)$ and then along the canonical isomorphism. This completes the proof.

(We would not ordinarily bother to distinguish between $1$ and $\introS(1)$ or in general explicitly write out the coherence isomorphisms for a product preserving functor, but in this one example it may be illuminating to see these distinctions and isomorphisms explicitly.)
\end{proof}

\sTODOinline{Note that in our Box notation, the above identifies $\introN$ and $\Box$, so far as their action on global elements of $T$ goes.}

\subsection{Recap}
We have defined the central notion of our interest, the concept of an introspective theory. We have proven that two different definitions of this concept are equivalent. We have also discussed a number of slight variations on this concept. We have seen how a canonical example of an introspective theory can be constructed by considering $\Sigma_1$ formulae in familiar theories such as ZF-Finite. Finally, we have discussed a number of other constructions which generate new introspective theories or pre-introspective theories from existing ones or from other categorical data (such as generating pre-introspective theories from cartesian closed categories).

\sTODOinline{Comment in passing on what in this document generalizes to finite product theories or the like.}

\fileend

\filestart

\section{Modal logic}
\subsection{Preview}
In this chapter, we will show how to interpret the $\Box$ operator of traditional modal logic in the context of introspective theories (or their generalizations). In particular, after defining the $\Box$ operator in this context, we observe in this chapter how it satisfies the rules of the modal logic K4.

\subsection{The box operator}
The following notation will be very convenient for us going forward. It is also suggestive of connections with modal logic we will eventually explore:

Let $\langle T, C \rangle$ be a locally introspective theory.

\TODOinline{Convention about using P(c) instead of P(t, c) to note t-indexed sets given by a doubly-indexed set}

We say a presheaf on $C$ is locally $T$-small if the map from its total space of elements to $\Ob(C)$ has $T$-small fibers. In other words, $P(c)$ is represented by an object of $T/t$ for each $t$-definable object $c$ of $C$. Put another way, such a presheaf is a $T$-indexed functor between the $T$-indexed categories $\op{C}$ and $T/-$.

Note that the category of locally $T$-small presheaves on $C$ is itself a $T$-indexed lexcategory. We will refer to this as $\Psh{C}$. (In more detail, the $T$-indexed category $C$ gives rise in an obvious way, by reversing arrows, to another $T$-indexed category $\op{C}$. We then have that the two $T$-indexed categories $\op{C}$ and $T/-$ are objects of $\Cat^{\op{T}}$. But $\Cat^{\op{T}}$ is a cartesian closed 2-category (in much the same way that $\Set^{\op{T}}$ is a cartesian closed category, at least when $\Set$ or $\Cat$ are interpreted expansively enough to include sets or categories of comparable size to $T$), and thus we can form within it the exponential object for $T/-$ raised to the power $\op{C}$. This exponential object is the $T$-indexed category we call $\Psh{C}$.)

Thus, we have three $T$-indexed lexcategories of note: $T$ itself (considered as a $T$-indexed category through the self-indexing $T/-$), $C$, and $\Psh{C}$.

Between these, we also have a cycle of $T$-indexed lexfunctors, like so:

\[\begin{tikzcd}
	&& {T/-} \\
	\\
	C &&&& {\Psh{C}}
	\arrow["\introF"', from=1-3, to=3-1]
	\arrow["{c \; \mapsto \Hom_{C}(-, c)}"', from=3-1, to=3-5]
	\arrow["{P \mapsto P(1)}"', from=3-5, to=1-3]
\end{tikzcd}\]

Here, the bottom arrow is the Yoneda embedding, sending each object of $C$ to the corresponding representable presheaf. The right arrow takes a presheaf on $C$ to its evaluation at the terminal object of $C$; that is, to its global elements. The left arrow is the $\introF$ which is part of the structure of an introspective theory.

\begin{definition}\label{BoxDefn}
In general, we will write $\Box$ for a roundtrip around this diagram, starting from any of its three nodes.

Thus, we will write $\Box$ for the $T$-indexed lexfunctor from $T$ to itself given by $t \mapsto \Hom_C(1, \introF(t))$.

We will ALSO write $\Box$ for the $T$-indexed lexfunctor from $C$ to itself given by $c \mapsto \introF(\Hom_C(1, c))$.

And we will ALSO write $\Box$ for the $T$-indexed lexfunctor from $\Psh{C}$ to $\Psh{C}$, which sends the presheaf $P$ to the presheaf represented by $\introF(P(1))$.

When we want to clarify precisely the domain we are operating on, we may write $\Box_T$, $\Box_C$, or $\Box_{\Psh{C}}$, as appropriate.

As the Yoneda embedding is naturally thought of as the inclusion of a full sub-lexcategory, identifying $C$ with the corresponding representable presheaves within $\Psh{C}$, we may also think of $\Box_{\Psh{C}}$ as a $T$-indexed lexfunctor from $\Psh{C}$ to $C$. That is, as the composition of merely the top two arrows above.
\end{definition}

The above was all discussed for $T$, $C$, and $\Psh{C}$ considered as $T$-indexed lexcategories, but this all (and the rest of this chapter as well) descends to corresponding structure on their global aspects as well. Keep in mind, in the global aspect context, $\introF$ is the same as $\introS$, so wherever in the above we discussed $\introF$, a corresponding statement holds as well for $\introS$, when considering just the global aspect.

\bigskip
\begin{observation}\label{BoxNotationSmallnessConcerns}
(Incidentally, the reason we restricted attention here to locally $T$-small presheaves, rather than arbitrary presheaves, is so that the map $P \mapsto P(1)$ can indeed be taken as always landing back within $T$ and not within $T$-indexed sets more generally.

Similarly, the reason we restricted attention to locally introspective theories (i.e., the case where $C$ is locally $T$-small) and not to pre-introspective theories more generally is so that the Yoneda embedding $c \mapsto \Hom_C(-, c)$ does indeed land within the locally $T$-small presheaves, rather than arbitrary presheaves.)
\end{observation}

\subsection{Box preservation}
Note that, having set up these various notions of $\Box$, we find that $\Box$ and each of the maps in the diagram \quote{commutes} in the appropriate sense; that is, they can be seen as preserving each other.

For example, $\introF$ preserves $\Box$, in that both $\introF(\Box(-))$ and $\Box(\introF(-))$ yield the same $T$-indexed functor from the self-indexing of $T$ to $C$. This is readily seen by unwinding their definitions: These are both $\introF(\Hom_C(1, \introF(-)))$.

We also have that taking $\Hom_C(1, -)$ preserves $\Box$ in the same way. $\Hom_C(1, \Box(-)) = \Box(\Hom_C(1, -)) = \Hom_C(1, \introF(\Hom_C(1, -)))$. More generally, consider the map $G$ which assigns to every locally $T$-small presheaf $P$ upon $C$ its object of global elements $P(1)$. Then $G$ preserves $\Box$ in the same way: $G(\Box P) = \Box(G(P)) = \Hom_C(1, \introF(P(1)))$.

Given object $c$ of $C$, and locally $T$-small presheaf $P$ on $C$, we will write $c \implies P$ to indicate the exponential presheaf $P^c$. That is, the presheaf $P(- \times c)$. We may also write $c \implies d$ where $d$ is an object of $C$ as well, to mean $c \implies P$ for the presheaf $P = \Hom_C(-, d)$ represented by $d$. Thus, $c \implies d$ is the presheaf $\Hom_C(- \times c, d)$.

Note that for any locally $T$-small presheaf $P$ on $C$ and object $c$ of $C$, we have that $P(c)$ can be identified with the presheaf $c \implies P$ evaluated at $1$, and thus $\introF(P(c))$ can be identified with $\Box(c \implies P)$. In particular, by considering the case when $P$ is represented by object $d$, we find that $\Box(c \implies d)$ is naturally identifiable with $\introF(\Hom_C(c, d))$.

\subsection{Modal logic and axiom 4}
The choice of this $\Box$ notation for these purposes is meant to convey an analogy with the $\Box$ operator of modal logic, and in particular, with the provability operator of provability logic. We will explore this more in later remarks.

The key point here is that the rules of the $\Box$ operator in a Kripke normal modal logic are essentially the rules of a lex endofunctor on a lexcategory, and any of our $\Box$ operators is certainly lex as a composite of lex functors.

Furthermore, each of our $\Box$ operators comes with a natural transformation from $\Box$ to $\Box \Box$ corresponding to the so-called 4 axiom $\Box A \vdash \Box \Box A$ in modal logic.

For the $\Box$ operator on $T$ this is clear, as the natural transformation $\introN$ from identity to $\Box$ encodes the even stronger property $t \vdash \Box t$. The 4 axiom is the special case where $t$ here is itself of the form $\Box A$.

\[\begin{tikzcd}
	&& {T/} \\
	\\
	C &&&& {\Psh{C}}
	\arrow["\introF"', from=1-3, to=3-1]
	\arrow["{P \mapsto P(1)}"', from=3-5, to=1-3]
	\arrow[""{name=0, anchor=center, inner sep=0}, "{c \mapsto \Hom_C(-, c)}"', from=3-1, to=3-5]
	\arrow["{\introN : \id_T \to \Box_T}"{description}, shorten >=7pt, Rightarrow, from=1-3, to=0]
\end{tikzcd}\]

We do not have such a strong natural transformation from identity to $\Box$ as acting on the other corners of the triangle ($C$ or $\Psh{C}$). However, by taking the natural transformation from $t$ to $\Box_T t$, whiskering it on both sides along the trips from any other corner of the triangle into and out of $T$, and then applying the commutativity of $\Box$ with each leg of the triangle, we get a natural transformation from $\Box$ to $\Box \Box$ at each other corner of the triangle as well.

(The general principle here is that, given morphisms $f$ and $g$ between the same pair of objects in opposite directions, composing with $f$ on one side and $g$ on the other turns $(gf)^n$ into $(fg)^{n + 1}$, and thus whiskering in this way turns a natural transformation $: (gf)^n \to (gf)^m$ into a natural transformation $: (fg)^{n + 1} \to (fg)^{m + 1}$)

% https://q.uiver.app/?q=WzAsOCxbMSwwLCJcXFBzaHtDfSJdLFsyLDAsIlQiXSxbMywxLCJDIl0sWzQsMSwiXFxQc2h7Q30iXSxbNSwwLCJUIl0sWzMsMF0sWzYsMCwiQyJdLFswLDAsIkMiXSxbMCwxXSxbMSwyXSxbMiwzXSxbMyw0XSxbMSw0LCJcXGlkX1QiLDAseyJsZXZlbCI6Miwic3R5bGUiOnsiaGVhZCI6eyJuYW1lIjoibm9uZSJ9fX1dLFs0LDZdLFsyLDYsIlxcQm94X0MiLDIseyJjdXJ2ZSI6NX1dLFs3LDBdLFs3LDYsIlxcQm94X0MiLDAseyJjdXJ2ZSI6LTV9XSxbNywyLCJcXEJveF9DIiwyXSxbMTIsMTAsIlxcaW50cm9OIiwyLHsic2hvcnRlbiI6eyJzb3VyY2UiOjIwLCJ0YXJnZXQiOjIwfX1dXQ==
\[\begin{tikzcd}
	C & {\Psh{C}} & T & {} && T & C \\
	&&& C & {\Psh{C}}
	\arrow[from=1-2, to=1-3]
	\arrow[from=1-3, to=2-4]
	\arrow[""{name=0, anchor=center, inner sep=0}, from=2-4, to=2-5]
	\arrow[from=2-5, to=1-6]
	\arrow[""{name=1, anchor=center, inner sep=0}, "{\id_T}", Rightarrow, no head, from=1-3, to=1-6]
	\arrow[from=1-6, to=1-7]
	\arrow["{\Box_C}"', curve={height=30pt}, from=2-4, to=1-7]
	\arrow[from=1-1, to=1-2]
	\arrow["{\Box_C}", curve={height=-30pt}, from=1-1, to=1-7]
	\arrow["{\Box_C}"', from=1-1, to=2-4]
	\arrow["\introN"', shorten <=4pt, shorten >=4pt, Rightarrow, from=1, to=0]
\end{tikzcd}\]

% https://q.uiver.app/?q=WzAsNyxbMCwwLCJcXFBzaHtDfSJdLFsxLDAsIlQiXSxbMiwxLCJDIl0sWzMsMSwiXFxQc2h7Q30iXSxbNCwwLCJUIl0sWzIsMF0sWzUsMCwiQyJdLFswLDFdLFsxLDJdLFsyLDNdLFszLDRdLFsxLDQsIlxcaWRfVCIsMCx7ImxldmVsIjoyLCJzdHlsZSI6eyJoZWFkIjp7Im5hbWUiOiJub25lIn19fV0sWzQsNl0sWzMsNiwiXFxCb3hfe1xcUHNoe0N9fSIsMl0sWzAsMywiXFxCb3hfe1xcUHNoe0N9fSIsMix7ImN1cnZlIjo1fV0sWzAsNiwiXFxCb3hfe1xcUHNoe0N9fSIsMCx7ImN1cnZlIjotNX1dLFsxMSw5LCJcXGludHJvTiIsMix7InNob3J0ZW4iOnsic291cmNlIjoyMCwidGFyZ2V0IjoyMH19XV0=
\[\begin{tikzcd}
	{\Psh{C}} & T & {} && T & C \\
	&& C & {\Psh{C}}
	\arrow[from=1-1, to=1-2]
	\arrow[from=1-2, to=2-3]
	\arrow[""{name=0, anchor=center, inner sep=0}, from=2-3, to=2-4]
	\arrow[from=2-4, to=1-5]
	\arrow[""{name=1, anchor=center, inner sep=0}, "{\id_T}", Rightarrow, no head, from=1-2, to=1-5]
	\arrow[from=1-5, to=1-6]
	\arrow["{\Box_{\Psh{C}}}"', from=2-4, to=1-6]
	\arrow["{\Box_{\Psh{C}}}"', curve={height=30pt}, from=1-1, to=2-4]
	\arrow["{\Box_{\Psh{C}}}", curve={height=-30pt}, from=1-1, to=1-6]
	\arrow["\introN"', shorten <=4pt, shorten >=4pt, Rightarrow, from=1, to=0]
\end{tikzcd}\]

(In the above two diagrams, all parallel paths commute except where an explicit natural transformation is noted, and all unmarked edges are the corresponding of the three edges along our triangle. In the diagram illustrating axiom 4 for $\Box_{\Psh{C}}$, we note that we can think of the codomain of $\Box_{\Psh{C}}$ as either $\Psh{C}$ or more narrowly its full subcategory $C$.)

Thus, our $\Box$ follows all the rules of the modal logic K4, in each of these contexts. Later, we shall see that the general logic followed by $\Box_C$ in a locally introspective theory is conversely no stronger than K4, while in an introspective theory, it is the modal logic GL. Indeed, in the very next chapter we will see how in an introspective theory we get the last ingredient for the modal logic GL, \Loeb/'s theorem.

\TODOinline{Citation to Boolos and other modal logic references at some point}

\TODOinline{Add commentary on how our box is actually a bifunctor instead of a single functor, as we do not presume cartesian closure. Add commentary on the case where we have cartesian closure, including how this reduces to a single box functor (either for the introspective theory or geminal category case), and including the relationship with double negation/Booleanness when Lob's theorem is introduced}

\subsection{Lex theories vs. finite product theories}
We have phrased the definitions and results in this chapter in terms of locally introspective theories (i.e., locally introspective finite limit theories) as these are our main objects of interest. But it is immediately seen that all the definitions and results in this chapter work just as well in the context of locally introspective finite product theories, replacing the self-indexing $T/-$ by the simple self-indexing $T//-$ and replacing all mention of \quote{lexfunctor}s with instead \quote{finite product preserving functor}s.

\subsection{Recap}
We have defined the $\Box$ operator in the context of locally introspective theories, and shown that it satisfies the rules of K4 modal logic.

In the next chapter, we will show that in an introspective theory, the $\Box$ furthermore satisfies the \Loeb/ property of the modal logic GL (aka, \quote{provability logic}).

Later, we shall also show the converses making these results tight: The rules which hold in general of the $\Box$ operator in a locally introspective theory are no stronger than (thus, precisely equal to) those of K4, while the rules which hold in general of the $\Box$ operator in an introspective theory are no stronger than (thus, precisely equal to) those of GL.

\subsection{As applied to our archetypal examples}
\TODO

\fileend

\filestart

\section{\Loeb/'s theorem}
\subsection{Preview}
In this chapter, we prove our most important theorem, justifying the significance of the simple concept of introspective theories. We show how every introspective theory automatically satisfies a general version of \Loeb/'s theorem, acting as the construction of general fixed points. We will also see how \Loeb/'s theorem is in full generality a phenomenon linked to presheaves, and not only constrained to representable presheaves.

The key results of this chapter are those covered in \magicref{PreIntrospDiagSection} and \magicref{IntrospDiagSection}, culminating in \magicref{IntrospLoeb}, our most important theorem. All material in those sections is original to this work.

The material on the \Loeb/ property in general categories in \magicref{LoebPropertySection} and \magicref{LoebPropertyReduxSection} includes some observations which can also be found (either explicitly or implicitly) in existing literature. We give our own exposition of this material, which felt important to include in a clean and complete exposition of the significance of our key result \magicref{IntrospLoeb}. In particular, we confirm how these general \Loeb/ property results continue to hold in our particular introspective theory context, even without common presumptions such as cartesian closure, and even with care paid to distinguish the roles of $T$, $C$, and $\Psh{C}$ in a general introspective theory $\langle T, C \rangle$.

The discussion in \magicref{LawvereFPTSection} concerns Lawvere's fixed point theorem, which of course is not original to us, but we also include some reframing and generalization of this which is due to us rather than Lawvere. The discussion in \magicref{LawvereFPTReduxSection} compares our reframing to some other reframings of Lawvere's fixed point theorem in the existing literature.

\subsection{The \Loeb/ property in abstract}\label{LoebPropertySection}
\begin{definition}\label{LoebPropertyDefn}
Let $D$ be any category with a terminal object and let $\Box : D \to D$ be a terminal-object-preserving endofunctor on $D$. We say $\Box$ has the \defined{\Loeb/ property} if, for every object $\Omega$ of $D$ and every morphism $\loebNeg : \Box \Omega \to \Omega$, there exists a morphism $\omega : 1 \to \Omega$ making the following square commute:
% https://q.uiver.app/?q=WzAsNCxbMCwxLCIxIl0sWzEsMCwiXFxCb3ggXFxPbWVnYSJdLFsxLDEsIlxcT21lZ2EiXSxbMCwwLCJcXEJveCAxIl0sWzEsMiwiXFxsb2ViTmVnIl0sWzAsMiwiXFxvbWVnYSIsMl0sWzMsMCwiIiwwLHsibGV2ZWwiOjIsInN0eWxlIjp7ImhlYWQiOnsibmFtZSI6Im5vbmUifX19XSxbMywxLCJcXEJveCBcXG9tZWdhIl1d
\[\begin{tikzcd}
	{\Box 1} & {\Box \Omega} \\
	1 & \Omega
	\arrow["\loebNeg", from=1-2, to=2-2]
	\arrow["\omega"', from=2-1, to=2-2]
	\arrow[Rightarrow, no head, from=1-1, to=2-1]
	\arrow["{\Box \omega}", from=1-1, to=1-2]
\end{tikzcd}\]

In other words, for every $\Omega \in D$ and $\loebNeg : \Box \Omega \to \Omega$, there is a fixed point of $\omega \mapsto \loebNeg \circ (\Box \omega) : \Hom_D(1, \Omega) \to \Hom_D(1, \Omega)$.

If such fixed points are furthermore always unique, we say $\Box$ has the \defined{\Loeb/ property with uniqueness}. (Note that the \Loeb/ property with uniqueness is the same as saying that the unique map $: \Box 1 \to 1$ is an initial $\Box$-algebra.)
\end{definition}

\begin{theorem}\label{LoebTransfer}
Let $D$ and $E$ be categories with terminal objects, and let $M : D \to E$ and $N : E \to D$ be functors preserving terminal objects. Suppose $NM : D \to D$ has the \Loeb/ property. Then so does $MN : E \to E$. Furthermore, if $NM$ has the \Loeb/ property with uniqueness, then so does $MN$.
\end{theorem}
\begin{proof}
This is by the general theorem \TODOinline{Note as a lemma in Preliminaries?} that fixed points of a composition of functions are in 1:1 correspondence with fixed points of any cyclic rearrangement of that composition. In particular, letting $comp_x(y) = x \circ y$, the fixed points of $\omega_E \mapsto comp_{\loebNeg}(M(N(\omega_E)))$ are in 1:1 correspondence with the fixed points of $\omega_D \mapsto N (comp_{{\loebNeg}}(M(\omega_D)))$, which is to say, of $\omega_D \mapsto comp_{N({\loebNeg})}(N (M (\omega_D)))$.

The latter fixed points must exist (or exist uniquely) if $NM : D \to D$ has the \Loeb/ property (or the \Loeb/ property with uniqueness, respectively), and thus in such cases so do the former fixed points, establishing the corresponding property for $MN : E \to E$.
\end{proof}

\begin{corollary}\label{LoebTransferIntrosp}
If $\langle T, C \rangle$ is a locally introspective theory where any of $\Box_T$, $\Box_C$, or $\Box_{\Psh{C}}$ has the \Loeb/ property, then all three do. And similarly for the \Loeb/ property with uniqueness. Thus, we may speak overall of $\langle T, C \rangle$ having the \Loeb/ property (without or with uniqueness) to describe this situation.
\end{corollary}
\begin{proof}
From \magicref{LoebTransfer}, when considering the definitions of $\Box$ given via the triangle at \magicref{BoxDefn}.
\end{proof}

(Later, via \magicref{IntrospAsGeminal}, we will see an alternative principled reason why we can automatically transfer properties which hold of $C$ in general introspective theories $\langle T, C \rangle$ to properties which hold of $T$ in general introspective theories $\langle T, C \rangle$.)

Recall that an introspective theory is a locally introspective theory $\langle T, C \rangle$ in which $C$ is \repsmall/. In this chapter, we will establish that every introspective theory has the \Loeb/ property with uniqueness (indeed, each $\Box$ endofunctor on each aspect of $T/-$, $C$, or $\Psh{C}$ has the \Loeb/ property with uniqueness \TODOinline{Perhaps explicitly observe somewhere in the previous chapter that our slice introspective theory construction interacts in the expected way with these aspects}).

But before we establish this version of \Loeb/'s theorem for introspective theories in particular, we will develop the theory of the \Loeb/ property and its consequences a little further in abstract.

\magicref{LoebTransfer} is a special case of a more general phenomenon, which will also be useful to us later. \TODOinline{What was I going to do with this more general phenomenon? I seem to have lost track}

\TODOinline{Put definition of coalgebra, algebra, coalgebra-to-algebra map in preliminaries}
\begin{definition}
Given an endofunctor $F$ on a category $D$, we say that a map $y : w \to m$ is an $F$-\defined{hylomorphism} from a coalgebra $W : w \to F(w)$ to an algebra $M : F(m) \to m$ just in case the following square commutes:

% https://q.uiver.app/?q=WzAsNCxbMCwxLCJ3Il0sWzAsMCwiRih3KSJdLFsxLDAsIkYobSkiXSxbMSwxLCJtIl0sWzAsMSwiVyJdLFsxLDIsIkYoeCkiXSxbMiwzLCJNIl0sWzAsMywieCIsMl1d
\[\begin{tikzcd}
	{F(w)} & {F(m)} \\
	w & m
	\arrow["W", from=2-1, to=1-1]
	\arrow["{F(x)}", from=1-1, to=1-2]
	\arrow["M", from=1-2, to=2-2]
	\arrow["x"', from=2-1, to=2-2]
\end{tikzcd}\]

In other words, just in case $x$ is a fixed point of $x \mapsto M \circ F(x) \circ W$.
\end{definition}

\begin{theorem}\label{HylomorphismTransfer}
Given categories $D$ and $E$, functors $A : D \to E$ and $B : E \to D$, and $BA$-coalgebra $W : w \to BA(w)$ and $BA$-algebra $M : BA(m) \to m$, the $BA$-hylomorphisms from $W$ to $M$ are in bijective correspondence (via the action of $A$) with the $AB$-hylomorphisms from $A(W) : A(w) \to ABA(w)$ to $A(M) : ABA(m) \to A(m)$.
\end{theorem}
\begin{proof}
This is by the general theorem that fixed points of a composition of functions are in bijective correspondence with fixed points of any cyclic rearrangement of that composition. In particular, letting $comp_{x, z}(-) = z \circ - \circ x$, the fixed points of $comp_{W, M}(BA(-))$ are in bijective correspondence with the fixed points of $A(comp_{W, M}(B(-)))$, which is to say, of $comp_{A(W), A(M)}(AB(-))$.
\end{proof}

Note that the square diagram in \magicref{LoebPropertyDefn} amounts to a $\Box$-hylomorphism. In this way, \magicref{LoebTransfer} is a special case of \magicref{HylomorphismTransfer}.

\TODOinline{Perhaps compare to the discussion in "Conjugate Hylomorphisms" by Hinze et al.}

\subsection{Lawvere's fixed point theorem}\label{LawvereFPTSection}
Let us refresh the reader on Lawvere's fixed point theorem \autocite{lawvere1969diagonal}, which captures the general structure of many diagonalization arguments and their relationship to cartesian closed structure. We shall first review a proof of Lawvere's fixed point theorem close in spirit to Lawvere's framing of his result.

Then we will note a slight generalization for which essentially the same argument applies. Then in the next section we will turn this generalization into a result in the context of general pre-introspective theories. Then we will specialize further down to introspective theories, and observe a wonderful \quote{bootstrapping} phenomenon which arises there, which shall ultimately provide us with the \Loeb/ property in that context, which is our main result.

\openNamed{theorem}{Lawvere's Fixed Point Theorem}\label{LawveresFixedPointTheorem}
Let $T$ be an arbitrary category. Let $X$ be an object of $T$ and let $\Omega$ be any $T$-indexed set. Suppose also given some map $\App' : X \to \Omega^X$ (equivalent to the data of a map $\App : X \times X \to \Omega$).

Let $\point$ be any object of $T$. By a \quote{point} of a $T$-indexed set, we shall mean an element of its aspect at $\point$ (equivalent to the data of a map into it from $\point$).

Suppose $\App$ has the surjectivity-like property that, for every map $F : X \to \Omega$, there is a point $f$ of $X$, such that for every point $x$ of $X$, we have that $\App(f, x) = F(x)$.

Then for any map $\loebNeg : \Omega \to \Omega$, there exists a point $\omega$ of $\Omega$ such that $\omega = \loebNeg(\omega)$. That is to say, $\loebNeg$ has a fixed point.
\closeNamed{theorem}
\begin{proof}
Let $F : X \to \Omega$ be the following composition:

% https://q.uiver.app/?q=WzAsNCxbMCwwLCJYIl0sWzIsMCwiWCBcXHRpbWVzIFgiXSxbMywwLCJcXE9tZWdhIl0sWzQsMCwiXFxPbWVnYSJdLFswLDEsIlxcbGFuZ2xlIFxcaWRfWCwgXFxpZF9YIFxccmFuZ2xlIl0sWzEsMiwiXFxBcHAiXSxbMiwzLCJcXGxvZWJOZWciXV0=
\[\begin{tikzcd}
	X && {X \times X} & \Omega & \Omega
	\arrow["{\langle \id_X, \id_X \rangle}", from=1-1, to=1-3]
	\arrow["\App", from=1-3, to=1-4]
	\arrow["\loebNeg", from=1-4, to=1-5]
\end{tikzcd}\]

That is, for any generalized element $x$ of $X$, we have that $F(x) = \loebNeg(\App(x, x))$.

We know there exists a point $f$ of $X$ which corresponds with $F$ in the manner of our surjectivity-like supposition on $\App$. Now consider the instance of this surjectivity-like supposition where $x = f$. This tells us that $\App(f, f) = F(f)$. But $F(f) = \loebNeg(\App(f, f))$.

Thus, taking $\omega = \App(f, f)$, we have that $\omega = \loebNeg(\omega)$ as desired.
\end{proof}

Let us make a few remarks on the scope of generality of this theorem.

Lawvere originally states this theorem specifically for the case where $T$ is a cartesian closed category, but later in \autocite{lawvere1969diagonal} notes that this implies the theorem just as well for the case where $T$ is merely a category with finite products, as any category can be embedded as a full subcategory of a cartesian closed category in a way which preserves any products or exponentials already present (via the Yoneda embedding). \autocite{lawvere1969diagonal} does not explicitly consider examples where the original category of interest $T$ lacks finite products, such that $X \times X$ is not an object of $T$, nor consider taking $\Omega$ to be merely a $T$-indexed set rather than an object of $T$, but of course these are covered in the same way by the same insight that we can work in $\Psh{T}$ instead of $T$.

Having observed that we can just as well frame the theorem with any of its objects drawn from $\Psh{T}$ rather than $T$, the reader might then well wonder why in our framing we have allowed some objects to be in $\Psh{T}$ but still constrained others (such as $X$) to come from $T$. We chose this particular framing partly as this is closest to the applications we have in mind, and also partly for what amount to stylistic reasons. In particular, having stated the theorem in this form, interpreting the surjectivity condition on $\App$ only requires quantification over the set of morphisms from object $X$ to presheaf $\Omega$ (i.e., the set $\Omega(X)$), instead of requiring quantification over the class of natural transformations from a presheaf $X$ to another presheaf $\Omega$ (which is potentially a proper class, if $T$ is proper-class-sized). But this is not really of much importance, and again the more general form of the theorem follows readily from the ostensibly less general one.

\autocite{lawvere1969diagonal} also only states this theorem in the particular case where $\point$ is a terminal object. In general, we can always pass from $T$ to a slice category $T/\point$, and in so doing we will turn what was $\point$-defined data in $T$ into globally defined data in $T/\point$. So constraining $\point$ to be a terminal object does not constrain the theorem excessively. However, it does constrain the theorem slightly, in that interpreting the surjectivity precondition in $T/\point$ in this way results in a stronger (that is, less often satisfied) surjectivity precondition than in the more flexible framing of the theorem we have given: The surjectivity condition in $T/\point$ would amount to requiring that for every $F : \point \times X \to \Omega$ in $\Psh{T}$, we could find a corresponding $f$. However, we have only required surjectivity with respect to the more constrained set of $F : X \to \Omega$ in $\Psh{T}$.

We do not actually need this extra flexibility. For our purposes, just like Lawvere's, it would suffice to always take $\point$ to be a terminal object. But we note the availability of this flexibility all the same (if only for the purpose of comparison at the end of this chapter to other variants on Lawvere's fixed point theorem recently noted in the literature, such as \magicref{MagmoidalFixedPointTheorem}).

Even this loosened surjectivity presumption is still far overkill as far as the needs of the argument go. All that really matters is for one specific definable value to be in the range of $\App'$. But in general practice and for our specific purposes, this is always established because of some such surjectivity condition anyway, so that seems the most useful framing in which to give the theorem.

Having said all that about the wide applicability of \magicref{LawveresFixedPointTheorem}, we actually will need to generalize it slightly further for our purposes. Having given the above discussion of the traditional theorem to prime the reader's intuitions through familiarity, we now put forward the following simple generalization:

\openNamed{theorem}{Self-Related Point Theorem}\label{SelfRelatedPointTheorem}
Let $T$ be an arbitrary category. Let $\point$ and $X$ be objects of $T$ and let $\Omega$ be any $T$-indexed set. Suppose also given some map $\App' : X \to \Omega^X$ (equivalent to the data of a map $\App : X \times X \to \Omega$).

As before, we shall use \quote{point of} as shorthand for \quote{element of the $\point$-aspect of}.

Suppose also given a binary relation $R$ on the points of $\Omega$. (We needn't presume $R$ to be symmetric or transitive or any such thing.). And suppose $\App$ has the surjectivity-like property that, for every morphism $F : X \to \Omega$, there is a point $f$ of $X$, such that for every point $x$ of $X$, we have $R(\App(f, x), F(x))$.

Then there exists a point $\omega$ of $\Omega$ such that $R(\omega, \omega)$. That is to say, $R$ has a self-related point.
\closeNamed{theorem}
\begin{proof}
Let $F : X \to \Omega$ be the following composition:

% https://q.uiver.app/?q=WzAsNCxbMCwwLCJYIl0sWzIsMCwiWCBcXHRpbWVzIFgiXSxbMywwLCJcXE9tZWdhIl0sWzQsMF0sWzAsMSwiXFxsYW5nbGUgXFxpZF9YLCBcXGlkX1ggXFxyYW5nbGUiXSxbMSwyLCJcXEFwcCJdXQ==
\[\begin{tikzcd}
	X && {X \times X} & \Omega & {}
	\arrow["{\langle \id_X, \id_X \rangle}", from=1-1, to=1-3]
	\arrow["\App", from=1-3, to=1-4]
\end{tikzcd}\]

That is, for any generalized element $x$ of $X$, we have that $F(x) = \App(x, x)$.

We know there exists a point $f$ of $X$ in accordance with our surjectivity-like supposition on $\App'$. Now consider the instance of the surjectivity-like supposition where $x = f$. This tells us that $R(\App(f, f), F(f))$. But $F(f) = \App(f, f)$.

Thus, we have found a point of $\Omega$ which is related to itself by $R$, as desired.
\end{proof}

It may not be obvious that this generalizes \magicref{LawveresFixedPointTheorem}. The following shows how this is so:

\openNamed{corollary}{Relatedly-Fixed Point Theorem}\label{RelatedlyFixedPointTheorem}
Consider the same setup as of \magicref{SelfRelatedPointTheorem}, and furthermore, suppose given $\loebNeg : \Omega \to \Omega$.

Then there exists a point $\omega$ of $\Omega$ such that $R(\omega, \loebNeg(\omega))$. We might describe this as \quote{$\omega$ is an $R$-fixed point of $\loebNeg$}.
\closeNamed{corollary}
\begin{proof}
Consider the binary relation $R_{\loebNeg}$ on points of $\Omega$ given by $R_{\loebNeg}(\omega_1, \omega_2) = R(\omega_1, \loebNeg(\omega_2))$.

We have been given the supposition that, for every morphism $F : X \to \Omega$, there is a point $f$ of $X$, such that for every point $x$ of $X$, we have $R(\App(f, x), F(x))$.

As this holds for arbitrary $F : X \to \Omega$, this also holds when an arbitrary $F$ is replaced by $\loebNeg \circ F : X \to \Omega$. That is to say, for every $F : X \to \Omega$, there is a point $f$ of $X$, such that for every point $x$ of $X$, we have $R(\App(f, x), (\loebNeg \circ F)(x))$, which is to say, $R_{\loebNeg}(\App(f, x), F(x))$.

But this is precisely the surjectivity supposition we need in order to invoke \magicref{SelfRelatedPointTheorem} with $R_{\loebNeg}$ in place of $R$. Doing so, we obtain a point $\omega$ of $\Omega$ such that $R_{\loebNeg}(\omega, \omega)$, which is to say $R(\omega, \loebNeg(\omega))$, as desired.
\end{proof}

Now we can see that \magicref{LawveresFixedPointTheorem} is of course the instance of \magicref{RelatedlyFixedPointTheorem} where the relation $R$ is taken to be equality. But \magicref{RelatedlyFixedPointTheorem} is strictly more general in allowing the use of an arbitrary relation.

(As for the relation between \magicref{RelatedlyFixedPointTheorem} and \magicref{SelfRelatedPointTheorem}, each is an instance of the other. We above obtained \magicref{RelatedlyFixedPointTheorem} as a corollary of \magicref{SelfRelatedPointTheorem}. But also conversely, \magicref{SelfRelatedPointTheorem} is the special case of \magicref{RelatedlyFixedPointTheorem} where $g$ is taken to be $\id_{\Omega}$.)

At any rate, we shall find the added flexibility of allowing a relation in place of equality to be valuable in the next sections, as we begin to specialize towards our application in introspective theories.

\subsection{Presheaf diagonalization for pre-introspective theories}\label{PreIntrospDiagSection}
\openNamed{theorem}{Pre-introspective Diagonalization}\label{PreIntrospDiag}
Let $T$ be a category, let $C$ be a $T$-indexed category, let $\introS$ be a functor from $T$ to the global aspect of $C$, let $\point_C$ be any object in the global aspect of $C$, and let $\introN$ be a map from $t$ to $\Hom_C(\point_C, \introS(t))$, natural in $t \in T$. Furthermore, let $\point_T$ be an object of $T$.

(For example, we have all the above structure if $\langle T, C, \introS, \introN \rangle$ is a pre-introspective theory, with $\point_C$ and $\point_T$ as terminal objects. This is the case we are interested in, but we note that the following theorem applies more broadly.)

Throughout the following, we use \quote{point of} as shorthand for \quote{element of the $\point_T$-aspect of}.

Furthermore, let $P$ be a $(T, C)$-indexed set \TODOinline{Make sure we discuss doubly-indexed sets in the Preliminaries}. We will write in the following $P(c)$ to mean the $T$-indexed set $t \mapsto P(t, c)$, for globally defined objects $c$ of $C$.

Suppose also given some object $\Omega$ in $T$ with a map $\quotient : \Omega \to P(\point_C)$ such that the induced function $\quotient \circ - : \Hom(X \times X, \Omega) \to \Hom(X \times X, P(\point_C))$ is surjective.

Suppose also given some object $X$ in $T$ and map $\alpha : X \to P(\introS(X))$. We also make a surjectivity-like assumption on $\alpha$. Specifically, we suppose that for every global element $p$ of $P(\introS(X))$, there is a point $x$ of $X$ such that $\alpha(x) = p$, as points of $P(\introS(X))$.

Finally, let $\loebNeg$ be a globally defined element of $P(\introS(\Omega))$.

Then we obtain a point $\omega$ of $\Omega$, such that $\quotient(\omega) = \pullAlong{\introN_{\Omega}(\omega)} \loebNeg$.
\closeNamed{theorem}
\begin{proof}
We shall show how this is an instance of \magicref{SelfRelatedPointTheorem}.

We define $\App : X \times X \to \Omega$ like so: Consider the two projection maps $\pi_1, \pi_2 : X \times X \to X$, as the two generic $(X \times X)$-defined elements of $X$. We thus obtain also $(X \times X)$-defined elements $\alpha(\pi_1)$ of $P(\introS(X))$ and $\introN_{X}(\pi_2)$ of $\Hom_C(\point_C, \introS(X))$. Combining these via the presheaf action of $P$, we get $\pullAlong{( \introN_{X}(\pi_2) )} ( \alpha(\pi_1) )$ as an $(X \times X)$-defined element of $P(\point_C)$. By the surjectivity presumption on $\quotient$, we find a preimage of this under the action of $\quotient : \Omega \to P(\point_C)$. We take this preimage to be our $\App : X \times X \to \Omega$. Thus, for any generalized elements $x_1, x_2$ of $X$ with the same domain, we have that $\quotient(\App(x_1, x_2)) = \pullAlong{( \introN_{X}(x_2) )} ( \alpha(x_1) )$.

We must now establish an appropriate surjectivity supposition on $\App$ for invoking \magicref{SelfRelatedPointTheorem}. 

Let an arbitrary $F : X \to \Omega$ be given. We then have that $\introS(F) : \introS(X) \to \introS(\Omega)$ in the global aspect of $C$. We can apply the action of $P$ along this morphism to $\loebNeg$ (a global element of $P(\introS(\Omega))$), thus obtaining a global element $\pullAlong{\introS(F)} \loebNeg$ of $P(\introS(X))$. By the surjectivity-like assumption on $\alpha$ we made, we now have a corresponding point $f$ of $X$, such that $\alpha(f) = \pullAlong{\introS(F)} \loebNeg$ (the right side here having been reinterpreted from a global element into a point).

It follows that for every point $x$ of $X$, we have that $\pullAlong{\introN_{X}(x)} \alpha(f) = \pullAlong{\introN_{X}(x)} \pullAlong{\introS(F)} \loebNeg$.

Note that by the definition of $\App$, we have that $\quotient(\App(f, x)) = \pullAlong{( \introN_{X}(x) )} \alpha(f)$.

Also note that by \magicref{SWithN}, we have that $\introS(F) \circ_C \introN_{X}(x) = \introN_{\Omega}(F(x))$. Thus, by the functoriality of $P$, we have that $\pullAlong{\introN_{X}(x)} \pullAlong{\introS(F)} \loebNeg =  \pullAlong{\introN_{\Omega}(F(x))} \loebNeg$.

Combining these last three paragraphs, we have that $\quotient(\App(f, x)) =  \pullAlong{\introN_{\Omega}(F(x))} \loebNeg$.

If we define the relation $R(\omega_1, \omega_2)$ as the equation $\quotient(\omega_1) = \pullAlong{\introN_{\Omega}(\omega_2)} \loebNeg$ accordingly, we have now established the surjectivity supposition required in order to invoke \magicref{SelfRelatedPointTheorem}. From this invocation, we get a point of $\Omega$ which is related by $R$ to itself, which is just what we desired, completing the proof.
\end{proof}
\begin{corollary}\label{PreIntrospDiagSpecialization}
In many cases we are interested in (though not all!), we furthermore take $P(\point_C)$ to be $T$-\repsmall/ and take $\Omega$ to be $P(\point_C)$, with $\quotient : \Omega \to P(\point_C)$ as the identity map between these. We then automatically have that the aspect of $\quotient$ at ${X \times X}$ is surjective as required.
\end{corollary}

\begin{theorem}\label{PreIntrospDiagFromIso}
Suppose given a locally introspective theory $\langle T, C, \introS, \introN \rangle$ and an object $P$ in the global aspect of $\Psh{C}$.

If there is any object $X$ of $T$ with an isomorphism from $X$ to $P(\introS(X))$, then, within the global aspect of $\Psh{C}$, for every $\loebNeg : \Box P \to P$, we obtain an $\omega : 1 \to P$, such that $\omega = g \circ \omega'$, where $\omega' = \Box_{\Psh{C}}(\omega) : 1 \to \Box P$. In other words, we obtain the instance of the \Loeb/ property constrained specifically to $P$.

We get the same result also if, within $\Glob{C}$, there is any object $Y$ along with an isomorphism from $Y$ to $\introS(P(Y))$.
\end{theorem}
\begin{proof}
Any isomorphism $\alpha : X \to P(\introS(X))$ (or even just a retraction) will automatically satisfy the surjectivity-like precondition allowing us to invoke \magicref{PreIntrospDiag} via \magicref{PreIntrospDiagSpecialization}, which takes $\Omega$ as $P(\point_C)$ and $\quotient$ as identity. Everything follows immediately from this, but has just been Yoneda-ized in its phrasing.

Specifically, keep in mind via the Yoneda lemma that the data of a map from $c \in C$ to $P \in \Psh{C}$ is the same as an element of $P(c)$. In this way, our $\loebNeg : \Box P \to P$ can be seen as indeed an element of $P(\Box P) = P(\introS(P(\point_C))) = P(\introS(\Omega))$, as required.

The invocation of \magicref{PreIntrospDiag} via \magicref{PreIntrospDiagSpecialization} will give us a global element $\omega$ of $\Omega = P(\point_C)$ such that $\omega = \pullAlong{\omega'}{\loebNeg}$, where $\omega' = \introN_{\Omega}(\omega)$ is a global element of $\Box P(\point_C)$. Again, by the Yoneda lemma, such an $\omega$ corresponds to a map from $1$ to $P$ in the global aspect of $\Psh{C}$, such an $\omega'$ corresponds to a map from $1$ to $\Box P$ (specifically, $\omega' = \Box \omega$, by \magicref{SMatchesN}), and our equation relating $\omega$ and $\omega'$ is that that $\omega = \loebNeg \circ \omega'$.

For the last remark about using a fixed point for $\introS(P(-))$ rather than a fixed point for $P(\introS(-))$, this is by the general 1:1 correspondence for fixed points of cyclic rearrangements of compositions. That is, let $F$ and $G$ be arbitrary covariant or contravariant functors, not necessarily of the same variance as each other. Note that fixed points up to isomorphism of $F \circ G$ are in correspondence with fixed points up to isomorphism of $G \circ F$, with the functors $G$ and $F$ carrying out the two directions of the correspondence. (For that matter, we can also observe that values $X$ which retract onto $F(G(X))$ induce values $Y$ which retract onto $G(F(Y))$, although this is no longer a 1 : 1 correspondence). In this particular case, this means that fixed points up to isomorphism of the contravariant endofunctor $P(\introS(-))$ on $T$ are in correspondence with fixed points up to isomorphism of the contravariant endofunctor $\introS(P(-))$ on the global aspect of $C$. \TODOinline{Make something in the preliminaries about fixed points of compositions and cyclic change of composition, then cite it here and elsewhere in this chapter}
\end{proof}

\sTODOinline{We note in passing that the above argument can be understood as working just the same in the context of a merely pre-introspective finite product theory. This just requires some care when interpreting the $\Box$ notation and discussing maps from objects of $C$ to objects of $\Psh{C}$, given the concerns from \magicref{BoxNotationSmallnessConcerns}. Without local introspectiveness, $C$ is no longer a full subcategory of $\Psh{C}$, and thus, $\Box_{\Psh{C}}$ is no longer an endofunctor on $\Psh{C}$, but can still be understood as a functor from $\Psh{C}$ to $C$. Furthermore, in such a context, we can still make sense of maps from objects of $C$ to objects of $\Psh{C}$ in the manner of the Yoneda lemma, or by seeing both $C$ and $\Psh{C}$ as full subcategories of the wider category of arbitrary $(T, C)$-indexed sets with no representability conditions.

Also nothing above depends on having equalizers, so it would work in a "pre-introspective finite product theory", but I got rid of this concept from the document for now.

Indeed, I'm getting rid of this whole note from the document for now.}

\subsection{Bootstrapping to \Loeb/'s theorem for introspective theories}\label{IntrospDiagSection}
We can see in the types of this last theorem the outline of \Loeb/'s theorem. But this last theorem contains the precondition of a certain isomorphism.

Incredibly, we can bootstrap away this isomorphism precondition, in the context of an introspective theory. That is, in the context of an introspective theory, we can use one particular instance of \magicref{PreIntrospDiag} itself to provide the very isomorphisms necessary in order to then re-invoke \magicref{PreIntrospDiag} via \magicref{PreIntrospDiagFromIso}.

Our plan is to consider the $(T, C)$-indexed set $P$ which assigns to every object $c$ of $C$ the set of isomorphism classes of $C/c$, with the action of $P$ on morphisms of $C$ correspondingly being given by pullback. That is, $P(t, c)$ is the set of $t$-defined objects of $C/c$ modulo $t$-defined isomorphisms of $C/c$. Note that this set of objects modulo isomorphism is well-defined even though $C$ is only a category and not a strict category! \TODOinline{Note that this $P$ is NOT in general $T$-\repsmall/, because $T$ does not in general have quotient objects.}

The reindexing action of this $P$ along morphisms in $T$ (reinterpreting $t_2$-defined data as $t_1$-defined data along any morphism $: t_1 \to t_2$ in $T$) is straightforward. As for reindexing along morphisms in $C$, for any fixed $t$ and any $t$-defined morphism $m : c_1 \to c_2$ of $C$, the action $P(t, m) : P(t, c_2) \to P(t, c_1)$ is given by pullback in the lexcategory $C$ along $m$; that is, this is given by $\pullAlong{m} : C/c_2 \to C/c_1$ considered as acting on isomorphism classes of objects. \TODOinline{Maybe move all this to the Preliminaries in the section on doubly-indexed sets}. Note that reindexing along morphisms in $C$ is indeed strictly functorial, because we are working with objects modulo isomorphism rather than with objects simpliciter.

We now choose any internal category $C_{strict}$ in $T$ which presents $C$ (by definition, such an internal category exists in an introspective theory; there may be multiple non-isomorphic such internal categories presenting $C$, but any will do for our purposes) and we take $\Omega$ to be $\Ob(C_{strict})$, with $\quotient : \Omega \to P(1)$ sending each object of each aspect of $C_{strict}$ to its isomorphism class within the corresponding aspect of $C$. Note that every component of $\quotient$ as a natural transformation between presheaves on $T$ is surjective (because $C$ is presented by $C_{strict}$, the isomorphism classes of $C$ and of $C_{strict}$ are the same, and there is clearly a surjection from the objects of $C_{strict}$ (at any aspect) onto the isomorphism classes of $C_{strict}$ (at the same aspect)). Thus in particular the component of $\quotient$ at the object $X \times X$ of $T$ is surjective. \TODOinline{Mention something about how this is a well-defined map of indexed sets; that is, $\quotient$ interacts appropriately with pullback}. 

We take $X$ to be the subobject of $\Mor(C_{strict})$ comprising those morphisms whose codomain is $\introS(\Mor(C_{strict}))$. That is, the object given by the following equalizer diagram.

% https://q.uiver.app/?q=WzAsNCxbMCwwLCJYIl0sWzEsMCwiXFxNb3IoQ197c3RyaWN0fSkiXSxbMywwLCJcXE9iKENfe3N0cmljdH0pIl0sWzIsMSwiMSJdLFswLDEsImkiLDIseyJzdHlsZSI6eyJ0YWlsIjp7Im5hbWUiOiJtb25vIn19fV0sWzEsMiwiXFxjb2QiXSxbMSwzLCIhIiwyXSxbMywyLCJcXGludHJvUycoXFxNb3IoQ197c3RyaWN0fSkpIiwyXV0=
\[\begin{tikzcd}
	X & {\Mor(C_{strict})} && {\Ob(C_{strict})} \\
	&& 1
	\arrow["i"', tail, from=1-1, to=1-2]
	\arrow["\cod", from=1-2, to=1-4]
	\arrow["{!}"', from=1-2, to=2-3]
	\arrow["{\introS'(\Mor(C_{strict}))}"', from=2-3, to=1-4]
\end{tikzcd}\]

In the above diagram, we have labelled an arrow with the name $\introS'(\Mor(C_{strict}))$. By this we mean some arbitrary globally defined object of $C_{strict}$ which presents the globally defined object $\introS(\Mor(C_{strict}))$ of $C$. We pedantically caution that there may actually be multiple non-equal global elements of $\Ob(C_{strict})$ which present objects isomorphic to $\introS(\Mor(C_{strict}))$. But any arbitrary choice of some such element will be fine to use as the arrow in this diagram for our purposes. (Indeed, it is readily seen that even two non-equal such choices will still lead to isomorphic $X$es. Or more precisely, isomorphic results as an object of $T$, though not isomorphic as a subobject of $\Mor(C_{strict})$, as the specific choice of inclusion map $i : X \to \Mor(C_{strict})$ will vary. But again, any so-arising choice will be fine for our purposes.).

Note that, by virtue of being an equalizer, the inclusion map $i : X \to \Mor(C_{strict})$ in $T$ is monic, and thus so also is $\introS(i) : \introS(X) \to \introS(\Mor(C_{strict}))$ in $C$. From this, we can define our $\alpha : X \to P(\introS(X))$ and establish its surjectivity condition. Specifically, observe that pullback along $\introS(i)$ gives us a functor $\pullAlong{\introS(i)} : C/\introS(\Mor(C_{strict})) \to C/\introS(X)$. If we focus on the action of $\pullAlong{\introS(i)}$ on objects, consider its input object as presented by an object of $C_{strict}/\introS'(\Mor(C_{strict}))$ (whose objects comprise $X$), and consider its output object modulo isomorphism, this yields $\pullAlong{\introS(i)} : X \to P(\introS(X))$, which we take as our definition of $\alpha$.

As for the surjectivity condition, let $F$ be an arbitrary global element of $P(\introS(X))$; that is, an arbitrary isomorphism class of objects of $C/\introS(X)$. The pushforward (i.e., composition) action of $\introS(i)$ gives us a functor from $C/\introS(X) \to C/\introS(\Mor(C_{strict}))$, taking $F$ to $\introS(i) \circ F$, an isomorphism class of objects of the global aspect of $C/\introS(\Mor(C_{strict}))$. This will be presented by at least one globally defined element $f$ of $X$ (keeping in mind the definition of $X$); there may be multiple non-equal such $f$ but any will do. Observe that $\alpha(f)$ is the isomorphism class of $C/\introS(X)$ corresponding to $\introS(i) \circ F$ pulled back along $\introS(i)$. This isomorphism class is the same as that of $F$ itself, because of the monicity of $\introS(i)$, like so:

% https://q.uiver.app/?q=WzAsNixbMSwxLCJcXGludHJvUyhYKSJdLFsxLDIsIlxcaW50cm9TKFxcTW9yKENfe3N0cmljdH0pKSJdLFsxLDAsIlxcYnVsbGV0Il0sWzAsMiwiXFxpbnRyb1MoWCkiXSxbMCwxLCJcXGludHJvUyhYKSJdLFswLDAsIlxcYnVsbGV0Il0sWzAsMSwiXFxpbnRyb1MoaSkiXSxbMiwwLCJGIl0sWzMsMSwiXFxpbnRyb1MoaSkiLDJdLFs0LDMsIlxcaWQiLDJdLFs0LDAsIlxcaWQiLDJdLFs0LDEsIiIsMix7InN0eWxlIjp7Im5hbWUiOiJjb3JuZXIifX1dLFs1LDQsIkYiLDJdLFs1LDIsIlxcaWQiXSxbNSwwLCIiLDAseyJzdHlsZSI6eyJuYW1lIjoiY29ybmVyIn19XV0=
\[\begin{tikzcd}
	\bullet & \bullet \\
	{\introS(X)} & {\introS(X)} \\
	{\introS(X)} & {\introS(\Mor(C_{strict}))}
	\arrow["{\introS(i)}", from=2-2, to=3-2]
	\arrow["F", from=1-2, to=2-2]
	\arrow["{\introS(i)}"', from=3-1, to=3-2]
	\arrow["\id"', from=2-1, to=3-1]
	\arrow["\id"', from=2-1, to=2-2]
	\arrow["\lrcorner"{anchor=center, pos=0.125}, draw=none, from=2-1, to=3-2]
	\arrow["F"', from=1-1, to=2-1]
	\arrow["\id", from=1-1, to=1-2]
	\arrow["\lrcorner"{anchor=center, pos=0.125}, draw=none, from=1-1, to=2-2]
\end{tikzcd}\]

Thus, $\alpha(f) = F$ as an element of $P(\introS(X))$, establishing the required surjectivity condition on $\alpha$.

Thus, all presumptions are satisfied for us to be able to apply \magicref{PreIntrospDiag} with these definitions, for an arbitrary globally defined element $g$ of $P(\introS(\Omega))$.

In particular, let an arbitrary $(T, C)$-indexed set $G$ be given (\TODOinline{satisfying the appropriate smallness conditions to be in $\Psh{C}$; local $T$-\repsmall/ness}). In fact, it suffices for $G$ to be merely $(T, \Ob(\core{C}))$-indexed, where $\Ob(\core{C})$ is the set of objects of $C$ modulo isomorphism.

This will be presented by a slice in $T/\Ob(C_{strict})$ (the slice whose fiber at any object $c_{strict}$ of $C_{strict}$ is the set $G(c)$, where $c$ is the object of $C$ presented by $c_{strict}$). By applying $\introS$ to this slice, we get a globally defined object of $C/\introS(\Ob(C_{strict}))$, which is to say, a global element of $P(\introS(\Omega))$. Take this to be our $\loebNeg$.

Invoking \magicref{PreIntrospDiag} (on the introspective theory $\langle T, C, \introS, \introN \rangle$, with $\point_C$ as the terminal object of $C$, $\point_T$ as the terminal object of $T$, and all other inputs ($P$, $\Omega$, $\quotient$, $X$, $\alpha$, and $\loebNeg$) as described with the same name above), we now get a globally defined element $\omega$ of $\Omega = \Ob(C_{strict})$ such that $\quotient(\omega) = \pullAlong{\introN_{\Omega}(\omega)} \loebNeg$. This equation is saying precisely that $\omega$ presents an object $Y$ of $C$ such that $Y$ is isomorphic to $\introS(G(Y))$. \TODOinline{Maybe more about how $\pullAlong{\introN_{\Omega}(\omega)} \loebNeg$ is $\introS(G(Y))$}

Thus, we have proven the following:
\begin{theorem}\label{IntrospTyConFixedPoints}
For any introspective theory $\langle T, C \rangle$, and any \TODOinline{locally $T$-\repsmall/} $(T, \Ob(\core{C}))$-indexed set $G$, there is some object $Y$ in the global aspect of $C$ along with a globally defined isomorphism from $Y$ to $\introS(G(Y))$.
\end{theorem}

Combining this with \magicref{PreIntrospDiagFromIso} to eliminate the latter's isomorphism precondition, we now reach the following conclusion:
\openNamed{theorem}{L\"ob's Theorem for Introspective Theories}\label{IntrospLoeb}
Suppose given an introspective theory $\langle T, C, \introS, \introN \rangle$ and any $P \in \Psh{C}$.

Then for every globally defined $\loebNeg : \Box P \to P$, we obtain a globally defined $\omega : 1 \to P$, such that $\loebNeg \circ (\Box \omega) = \omega$.

In other words, keeping in mind the equivalences and terminology of \magicref{LoebTransferIntrosp}, we have demonstrated that every introspective theory has the \Loeb/ property.
\closeNamed{theorem}

We can consider the particular case where $P$ is $C$-\repsmall/, just as $\Box P$ is. In other words, where $P(-) = \Hom_C(-, c)$ is the representable presheaf on $C$ represented by some object $c$ of $C$. All traditional accounts of \Loeb/'s theorem are along these lines. But note that we can also just as well consider this \magicref{IntrospLoeb} for non-representable presheaves $P$, a significant generalization of the traditional viewpoint.

Although we have just proven that this property holds for all introspective theories automatically, it does not hold automatically for merely locally introspective theories. \TODOinline{Cite a simple counterexample.} However, we shall see later on that there are some natural examples of locally introspective theories that still have this property. \TODOinline{Perhaps discuss the idea of a locally introspective theory where every object is contained in some "full sub-introspective theory" of it, and how this inherits the Loeb property.}

\subsection{Uniformity of the \Loeb/ property}
Keeping in mind also how we can construct slice introspective theories, a la \magicref{IntrospSlice}, we get also the following corollary:
\begin{corollary}\label{IntrospTransfersEverywhere}
For any introspective theory $\langle T, C \rangle$, the induced endolexfunctors $\Box$ on each slice category $T/t$ and on each $t$-defined aspect of $\Psh{C}$ all have the \Loeb/ property, for $t \in T$.
\end{corollary}

Beware, however! We do not in general have that slice categories of $C$ have the \Loeb/ property! \TODOinline{Expand on this. Link to our discussion of the interaction of cartesian closure and the \Loeb/ property. Beware that the $t$-defined aspect of $C$ is not the same as some slice category of the global aspect of $C$; that is, the concept of different aspects of $C$ qua $T$-indexed category is distinct from the concept of different aspects of $C$ qua its self-indexing}

Using this last observation about slice introspective theories, we see that the fixed points generated for $P \in \Psh{C}$ are generated uniformly by a fixed point combinator, in the following sense:
\begin{theorem}
For any introspective theory $\langle T, C \rangle$, any object $t$ in $T$, and any object $P$ in the $t$-defined aspect of $\Psh{C}$, there is a morphism $Loeb : P(\Box P) \to P(1_C)$ in the slice category $T/t$ such that \TODO.

(Note that applying $\introF$ to this morphism $Loeb : P(\Box P) \to P(1_C)$ gives us a morphism $: \Box((\Box P) \implies P) \to \Box P$ in the $t$-defined aspect of $C$, with the analogous fixed point property.)
\end{theorem}
\begin{proof}
Consider the \Loeb/ property as applied to the generic element of $P(\Box P)$ in the slice introspective theory $(T/t)/P(\Box P)$.
\end{proof}

\subsection{Uniqueness, initiality/terminality, and hylomorphisms}\label{LoebPropertyReduxSection}
In this section, we establish results on the uniqueness of the fixed points given by the \Loeb/ property for our introspective theories, as well as their initiality or terminality in certain contexts. We note that similar arguments to those given in this section have already been given in the literature on guarded recursion; for example, in \autocite{birkedal2011first} and in \autocite{birkedal2013universes}. We record these results here to explicitly confirm that they will hold in our particular setting (with $T$, $C$, and $\Psh{C}$ distinguished, with no assumption of cartesian closure, with no form of \Loeb/'s property directly assumed but rather this having been derived from other assumptions as in the prior sections, with no natural transformation from identity to $\Box$ on $C$ nor on $\Psh{C}$, etc.). We also observe that the uniqueness argument and the initiality/terminality argument are unified in a way that we do not believe has been stated before.

The key theorem for this entire section is the following:

\begin{theorem}\label{CoalgToAlg}
Let $D$ be any category with terminal object and let $\Box : D \to D$ be any terminal-object-preserving functor. Let $E$ be any \repsmall/ $D$-indexed category. (Note that $\Box$ acting on $E$ induces also another \repsmall/ $D$-indexed category $\Box E$ \TODOinline{Note in preliminaries how the choice of internal category presenting these indexed categories doesn't matter, as $\Box$ acts on the internal functors between these too}, as well as a functor from each $d$-defined aspect of $E$ to the $(\Box d)$-defined aspect of $\Box E$, for $d \in D$. In particular, as $\Box$ is terminal-object-preserving, $\Box$ acts as a functor from the global aspect of $E$ to the global aspect of $\Box E$.) 

Suppose also given a $D$-indexed functor $f : \Box E \to E$, and let the endofunctor $F$ on the global aspect of $E$ be given by first applying $\Box$ to arrive in the global aspect of $\Box E$, then applying $f$ to arrive back in the global aspect of $E$.

If $\Box$ has the \Loeb/ property, then there is an $F$-hylomorphism between any $F$-coalgebra $W : w \to F(w)$ and any $F$-algebra $M : F(m) \to m$ in the global aspect of $E$. And if $\Box$ furthermore has the \Loeb/ property with uniqueness, then this hylomorphism is unique.
\end{theorem}
\begin{proof}
A hylomorphism from $W$ to $M$ is a fixed point of $x \mapsto M \circ F(x) \circ W : \Hom_E(w, m) \to \Hom_E(w, m)$. But as $F(x) = f(\Box x)$, this is the same as a fixed point for $x \mapsto g(\Box x)$ where $g(-)$ is defined by $M \circ f(-) \circ W : \Box \Hom_E(w, m) \to \Hom_E(w, m)$. 

The hylomorphisms from $W$ to $M$ are thus the same as the fixed points given by the \Loeb/ property with respect to this $g$. This completes the proof.
\end{proof}

\begin{corollary}\label{LexLoebIsWithUniqueness}
If $\Box : D \to D$ is any endolexfunctor on a lexcategory with the \Loeb/ property, then it furthermore has the \Loeb/ property with uniqueness.
\end{corollary}
\begin{proof}
Let $E$ be an arbitrary object of $D$ (thus, a \repsmall/ $D$-indexed set) and let us construe this also as a \repsmall/ $D$-indexed discrete category. Let $f : \Box E \to E$ be an arbitrary map in $D$, and as above, let us take $F : \Hom_D(1, E) \to \Hom_D(1, E)$ to be given as the composition of $\Box : \Hom_D(1, E) \to \Hom_D(1, \Box E)$ with $f \circ - : \Hom_D(1, \Box E) \to \Hom_D(1, E)$.

As $E$ is a discrete category, observe that any $F$-coalgebra or $F$-algebra in the global aspect of $E$ amounts to a fixed point of $f \circ \Box(-) : \Hom_D(1, E) \to \Hom_D(1, E)$. The \Loeb/ property tells us such fixed points exist, while \magicref{CoalgToAlg} tells us there is a hylomorphism between any such fixed points. But as $E$ is a discrete category, such a hylomorphism amounts to just an equality between the two elements of $\Hom_D(1, E)$. Thus, any two such fixed points are equal, which is to say, we have the \Loeb/ property with uniqueness.
\end{proof}

Note that \magicref{LexLoebIsWithUniqueness} makes essential use of the structure available in a lexcategory, beyond what is merely available in a category with finite products. There are many examples of finite-product-preserving endofunctors on categories with finite products which have the \Loeb/ property but which do not have the \Loeb/ property with uniqueness (for example, when considering the identity endofunctor on the category whose objects are complete lattices and whose morphisms are arbitrary monotonic maps). It is perhaps easy to miss how the finite limit structure of $D$ has been used in the argument for \magicref{LexLoebIsWithUniqueness}. At one point within its invocation of \magicref{CoalgToAlg}, the argument considers the object $\Box \Hom_E(w, m)$ for parallel $w, m \in \Hom_D(1, E)$. As such, it depends upon the fact that $\Hom_E(w, m)$ is a \repsmall/ $D$-indexed set. This object $\Hom_E(w, m)$ of $D$ is given by an equalizer between parallel maps from $1$ to $E$ in $D$; this is where the fact that $D$ is a lexcategory is essential.

\begin{corollary}\label{InitialTerminalCoincidence}
Consider the same setup as of \magicref{CoalgToAlg}, and presume $\Box$ has the \Loeb/ property with uniqueness (as we have just seen follows from the \Loeb/ property simpliciter when $D$ has and $\Box$ preserves finite limits). Then any fixed point of $F$ (in the sense of an object $e$ of the global aspect of $E$ along with an isomorphism between $e$ and $F(e)$) is simultaneously an initial $F$-algebra and a terminal $F$-coalgebra. In particular, any two such fixed points are isomorphic, via a unique $F$-algebra isomorphism.
\end{corollary}
\begin{proof}
In that context, \magicref{CoalgToAlg} says that every $F$-coalgebra has a \emph{unique} hylomorphism into every $F$-algebra. In the particular case that the coalgebra is invertible, this can be read as a morphism between algebras, and establishes that the coalgebra's inverse is an initial algebra. Dually, for any invertible algebra, this establishes its inverse as a terminal coalgebra.
\end{proof}

\begin{TODOblock}

\TODOinline{Link this to \magicref{IntrospTyConFixedPoints}}

The above applies to every slice category of $T$, to every aspect of $C$, and to every aspect of $\Psh{C}$, by \magicref{IntrospTransfersEverywhere}. But it does not apply to slice categories of $C$ or slice categories of $\Psh{C}$!
\end{TODOblock}

\begin{observation}
Note that this uniqueness means that the apparent dependence on various arbitrary choices in our proof of the bootstrapping theorem doesn't matter in the end. We get the same result no matter what.
\end{observation}

\begin{remark}
The argument we have given for \magicref{CoalgToAlg} and thus for \magicref{InitialTerminalCoincidence} is essentially the same as that given for Lemma 7.6 in \autocite{birkedal2011first}. However, we have stated it in our slightly different framework of introspective theories, and we use it to derive uniqueness for the \Loeb/ property, whereas the argument is given in \autocite{birkedal2011first} in a context where the uniqueness of the \Loeb/ property has already been assumed.

Arguments establishing that the \Loeb/ property entails the \Loeb/ property with uniqueness in contexts with identity types have been noted in the literature on guarded recursion. For example, as Theorem V.8 in \autocite{birkedal2013universes} and as Theorem 9.5 in \autocite{birkedal2021Multimodal}. However, we are unaware of any prior observation in the literature that this uniqueness can also be understood as a special case of the existence of coalgebra-to-algebra hylomorphisms, unifying those arguments.

Finally, as opposed to the usual conventions of the guarded recursion literature, we stress that these arguments can be given in contexts where no natural transformation from identity to $\Box$ is presumed (indeed, this is important for us, as no such natural transformation exists in general for the $\Box_C$ or $\Box_{\Psh{C}}$ of an introspective theory $\langle T, C \rangle$), and also they do not depend on cartesian closure.
\end{remark}

\TODOinline{Show that the equalizer from X to []X to X equalized against $\id_X$ is $1$}

Note that the uniqueness property for $\Box_T$ can be captured via an equalizer diagram in $T$, like so: \TODOinline{And similarly an equalizer diagram in $T$ captures the uniqueness property for $\Box_{\Psh{C}}$; however, no such diagram in $\Psh{C}$ captures the uniqueness property}

Consider the equalizer $E$ of $a$ and $b$. This is a subobject of $1$. If there were a map from $1$ to $E$, then $a$ and $b$ would be equal. In just the same way, taking this equalizer diagram's image under the lexfunctor $\introS$, we see that any value in $\Hom_C(1, \introS(E))$ would lead to $\introS(a)$ and $\introS(b)$ being equal, thus corresponding to equal maps from $1$ to $\introS(P(1))$ in $C$. Transporting $f$ back along these two would yield equal values, therefore. But transporting $f$ back along these two yields $a$ and $b$ respectively, so we would have that $a$ and $b$ are equal. This argument yields a morphism from $\Hom_C(1, \introS(E))$ to $E$ in $T$. Applying $\introS$ to this, we have a morphism in $C$, which lives in $Q(\introS(Q(1)))$, where $Q$ is the presheaf on $C$ represented by $\introS(E)$. Applying \TODOinline{cite LocallyIntrospLoeb} to this, we get an element of $Q(1)$, which is to say, a global element of $\introS(E)$ in $C$. This makes $\introS(a)$ and $\introS(b)$ equal as global points of $\introS(P(1))$ in $C$. And by transporting $f$ back along these, we conclude as before that $a$ and $b$ are equal. \TODOinline{Word this all better}

\begin{theorem}\label{LocallyCartesianLoeb}
Let $T$ be any lexcategory, and equip it as an introspective theory $\langle T, C, \introF \rangle = \langle T, T/-, \id \rangle$ by taking $C$ to be $T$'s self-indexing and $\introF$ to be the identity (a la \magicref{TrivialPreIntrospIndexed}). Recall that this will be locally introspective (that is, the self-indexing will be locally \repsmall/) precisely when $T$ is locally cartesian closed.

This will furthermore be introspective (that is, the self-indexing will be \repsmall/) only when $T$ is the trivial terminal category.
\end{theorem}
\begin{proof}
For a lexcategory $T$ equipped as a pre-introspective theory in this way, the operation $\Box_T$ acts as the identity. Thus, every object of $T$ acts as a fixed point of $\Box_T$ up to isomorphism. If this is furthermore an introspective theory, then by our observation on uniqueness of such fixed points in an introspective theory, it follows that every object of $T$ is isomorphic, and thus every object of $T$ is a terminal object.
\end{proof}
This triviality result was demonstrated in \autocite{PittsTaylor1989} by an essentially identical argument to the argument we have given, when the abstractions in our argument are unwound to this special case.

But by generalizing the argument to introspective theories, we are able to expand from this negative result (there are no nontrivial locally cartesian closed categories whose self-indexing is \repsmall/) to a positive result (there are many nontrivial examples of introspective theories, which all end up satisfying the \Loeb/ property with uniqueness and all the further corollaries of this we've been discussing).

----

\sTODOinline{Demonstrate that we do NOT get Loeb's theorem internal to a geminal category G for arbitrary presheaves P on |G'|, thus demonstrating the necessity of the presheaf existing within an introspective theory. The presheaf needs to be parametrized by a parameter from an object of an enclosing introspective theory. So P(S(X)) |- []P(S(X)) is available.}

\subsection{Relating variations on Lawvere's fixed point theorem}\label{LawvereFPTReduxSection}
Although not important for our main narrative, we note here some further comments on the relation of Lawvere's fixed point theorem to generalizations of ours or others.

First, we observe that \magicref{LawveresFixedPointTheorem} can be straightforwardly re-obtained as a special case of our \magicref{PreIntrospDiag}.
\begin{proof}
First, we handle the special case of \magicref{LawveresFixedPointTheorem} where $T$ has finite limits and $\Omega$ is an object of $T$.

This is a special case of \magicref{PreIntrospDiag} where we furthermore take the pre-introspective theory $\langle T, C, \introF \rangle$ to be the trivial one where $C$ is the simple self-indexing $T/-$ and $\introF$ is the identity.

Furthermore, $P$ is taken to be the $(T, C)$-indexed set represented by $\Omega$; that is, such that $P(t, c) = \Hom_T(t \times c, \Omega)$. Note that $P(\introS(t))$ for objects $t$ of $T$ is therefore the $T$-indexed set $\Omega^t$. In particular, $P(1)$ is thus isomorphic to $\Omega$. As in \magicref{PreIntrospDiagSpecialization}, we can take $\quotient$ to be this isomorphism (one can think of it as an identity if one likes), and this will then automatically be surjective on its $X \times X$ aspect.

We take $\alpha : X \to P(\introS(X)) = \Omega^X$ to be given by the map $\App' : X \to \Omega^X$ presumed in \magicref{LawveresFixedPointTheorem}. The surjectivity presumption from \magicref{LawveresFixedPointTheorem} then becomes the surjectivity presumption of \magicref{PreIntrospDiag}. 

And to give a $g$ in the global aspect of $P(\introS(\Omega)) = \Omega^\Omega$ is precisely the data presumed by the name $g$ in \magicref{LawveresFixedPointTheorem}.

This matches all the presumptions of \magicref{PreIntrospDiag} up with corresponding presumptions from \magicref{LawveresFixedPointTheorem}, and the conclusion we then obtain from \magicref{PreIntrospDiag} is readily seen to be the same as the conclusion from \magicref{LawveresFixedPointTheorem}.

The above shows how to obtain \magicref{LawveresFixedPointTheorem} as an instance of \magicref{PreIntrospDiag} when $T$ is a lexcategory and $\Omega$ is an object of $T$. We then obtain \magicref{LawveresFixedPointTheorem} in full (that is, for arbitrary categories $T$ and $T$-indexed sets $\Omega$) from this special case, by first replacing $T$ with $\Psh{T}$, as noted at \TODOinline{Cite our previous discussion}.
\end{proof}

We also note in passing that another interesting generalization of \magicref{LawveresFixedPointTheorem} was recently remarked upon in \autocite{roberts2021substructural}. The following (or rather, its contrapositive) was given as Theorem 11 there. We shall present our own proof.

\openNamed{theorem}{Magmoidal Fixed Point Theorem}\label{MagmoidalFixedPointTheorem}
Let $T$ be an arbitrary category with objects $\point$ and $\Omega$, and let $B : T \times T \to T$ be a bifunctor on $T$ such that we have a transformation $\delta_t : t \to B(t, t)$ natural in $t$ from $T$. As ever, use \quote{point of} to mean \quote{element of the $\point$-aspect of}.

Suppose given an object $X$ of $T$ and an $\alpha : B(X, X) \to \Omega$ with the pointwise surjectivity property that for every $F : X \to \Omega$, there is a point $f$ of $X$, such that for every point $x$ of $X$, we have that the following diagram commutes:

% https://q.uiver.app/?q=WzAsNSxbMCwwLCJcXHBvaW50Il0sWzEsMCwiQihcXHBvaW50LCBcXHBvaW50KSJdLFsyLDAsIkIoWCwgWCkiXSxbMywwLCJcXE9tZWdhIl0sWzEsMSwiWCJdLFswLDEsIlxcZGVsdGFfe1xccG9pbnR9Il0sWzEsMiwiQihmLCB4KSJdLFsyLDMsIlxcYWxwaGEiXSxbMCw0LCJ4IiwyXSxbNCwzLCJGIiwyXV0=
\[\begin{tikzcd}
	\point & {B(\point, \point)} & {B(X, X)} & \Omega \\
	& X
	\arrow["{\delta_{\point}}", from=1-1, to=1-2]
	\arrow["{B(f, x)}", from=1-2, to=1-3]
	\arrow["\alpha", from=1-3, to=1-4]
	\arrow["x"', from=1-1, to=2-2]
	\arrow["F"', from=2-2, to=1-4]
\end{tikzcd}\]

Then for every $g : \Omega \to \Omega$, there is a point $\omega$ of $\Omega$ such that $\omega = g(\omega)$. That is to say, a fixed point of $g$.
\closeNamed{theorem}
\begin{proof}
Take $\App : X \times X \to \Omega$ to be defined like so: For each object $t$ of $T$, we define $\App_t : \Hom(t, X) \times \Hom(t, X) \to \Hom(t, \Omega)$ by giving $\App_t(m, n)$ as the following composition:

% https://q.uiver.app/?q=WzAsNCxbMSwwLCJCKHQsIHQpIl0sWzAsMCwidCJdLFszLDAsIkIoWCwgWCkiXSxbNCwwLCJcXE9tZWdhIl0sWzEsMCwiXFxkZWx0YV97dH0iXSxbMCwyLCJCKG0sIG4pIl0sWzIsMywiXFxhbHBoYSJdXQ==
\[\begin{tikzcd}
	t & {B(t, t)} && {B(X, X)} & \Omega
	\arrow["{\delta_{t}}", from=1-1, to=1-2]
	\arrow["{B(m, n)}", from=1-2, to=1-4]
	\arrow["\alpha", from=1-4, to=1-5]
\end{tikzcd}\]

That this definition of $\App_t$ is natural in $t$ follows from the naturality of $\delta$ and the functoriality of $B$. Specifically, naturality with respect to $h: s \to t$ is seen as follows:

% https://q.uiver.app/?q=WzAsNixbMSwwLCJCKHQsIHQpIl0sWzAsMCwidCJdLFszLDAsIkIoWCwgWCkiXSxbNCwwLCJcXE9tZWdhIl0sWzAsMSwicyJdLFsxLDEsIkIocywgcykiXSxbMSwwLCJcXGRlbHRhX3t0fSJdLFswLDIsIkIobSwgbikiXSxbMiwzLCJcXGFscGhhIl0sWzQsNSwiXFxkZWx0YV9zIiwyXSxbNCwxLCJoIl0sWzUsMCwiQihoLCBoKSIsMV0sWzUsMiwiQihtIGgsIG4gaCkiLDJdXQ==
\[\begin{tikzcd}
	t & {B(t, t)} && {B(X, X)} & \Omega \\
	s & {B(s, s)}
	\arrow["{\delta_{t}}", from=1-1, to=1-2]
	\arrow["{B(m, n)}", from=1-2, to=1-4]
	\arrow["\alpha", from=1-4, to=1-5]
	\arrow["{\delta_s}"', from=2-1, to=2-2]
	\arrow["h", from=2-1, to=1-1]
	\arrow["{B(h, h)}"{description}, from=2-2, to=1-2]
	\arrow["{B(m h, n h)}"', from=2-2, to=1-4]
\end{tikzcd}\]

The desired result now follows by \magicref{LawveresFixedPointTheorem}.
\end{proof}
\magicref{LawveresFixedPointTheorem} is of course the special case of \magicref{MagmoidalFixedPointTheorem} where $B$ is the familiar cartesian product and $\delta$ is the familiar diagonal transformation. Thus, in \autocite{roberts2021substructural}, \magicref{MagmoidalFixedPointTheorem} is considered as a generalization of Lawvere's fixed point theorem. But as we've just seen, \magicref{MagmoidalFixedPointTheorem} is also a special case of Lawvere's fixed point theorem, appropriately construed (as in our formulation of \magicref{LawveresFixedPointTheorem} which removes the $\point = 1$ constraint), despite the seeming mismatch between general bifunctors and specifically cartesian products. As noted before, there is no need for $X \times X$ to be $T$-\repsmall/, and if such closure of our underlying category is insisted upon, we can just as well always pass to $\Psh{T}$ first.

\TODOinline{Remarks on Cantor's theorem, Liar's paradox, and Y combinator as examples of Lawvere's fixed point theorem. Cantor's theorem is a contrapositive statement using surjection. Liar's paradox is a contrapositive statement using a retraction/isomorphism. Y combinator is a positive statement using a retraction, and also involves passing to a slice category. Note that the reason we presume in Cantor's theorem that negation on $\Omega$ has no fixed points is because of another instance of Lawvere's fixed point theorem, via Liar's paradox!}

\TODOinline{Point out the error in Yonofsky's discussion of Kleene's recursion theorem and how our more general formulation allows us to correct this.}

\subsection{Recap}
\TODO

\subsection{As applied to our archetypal examples}
\TODO

\fileend

\filestart

\section{Geminal categories: The free introspective theory}\label{GeminalChapter}
\subsection{Preview}
In this chapter, we give an explicit yet tractably compact description of the initial introspective theory (which we call the theory of \quote{geminal categories}).

We also show the remarkable result that any strict introspective theory can itself be equipped in a natural way as a model of this initial introspective theory; that is, any strict introspective theory can be seen as a geminal category.

This last statement is easy to misinterpret, so let me be a bit more clear as to what I mean by this. I do not mean the trivial statement that every introspective theory extends the initial introspective theory. Rather, I mean that the theory of strict introspective theories extends the initial introspective theory (even though the theory of strict introspective theories is not itself an introspective theory).

We will also discuss a partial converse of sorts, a way to extract an introspective theory from a geminal category, with the extracted introspective theory having a certain terminality property (that is, we construct a sort of co-free introspective theory induced by the given geminal category).

\subsection{Strict introspective theories}
It will be technically convenient for us to work in this chapter with a slightly less \quote{presentation-free} variant of our notion of introspective theories.

\begin{definition}
A \defined{strict introspective theory} is a strict lexcategory $T$, a lexcategory $C$ internal to $T$, a strict lexfunctor $\introS$ from $T$ to the global aspect of $C$, and a natural transformation $\introN$ from $\id_T$ to $\Hom_C(1, \introS(-))$.
\end{definition}

As usual, to name a strict introspective theory, we can enumerate the entire ordered tuple $\langle T, C, \introS, \introN \rangle$, or sometimes we just note $\langle T, C \rangle$ or $T$ explicitly and leave the rest implicit.

The definition of a strict introspective theory differs from the definition of an ordinary introspective theory (\magicref{DefnIntrospSN}) in the following ways: $T$ is made strict (thus, its internal structures can be considered up to equality instead of mere isomorphism), we demand the selection of $C$ as a particular $T$-internal lexcategory up to equality (instead of simply up to presenting equivalent indexed categories), and we take $\introS$ as a strict lexfunctor (thus, $\introS$ preserves chosen basis limits on-the-nose).

Clearly, any strict introspective theory presents some introspective theory. Conversely, we have the following:

\begin{theorem}\label{StrictifyingIntrosp}
Any introspective theory $\langle T, C, \introS, \introN \rangle$ is presented by some strict introspective theory.
\end{theorem}
\begin{proof}
Suppose given an introspective theory $\langle T, C, \introS, \introN \rangle$. By definition of the smallness of $C$, we can choose some lexcategory $C_{int}$ internal to $T$ (up to isomorphism of indexed strict lexcategories) which presents the $T$-indexed category $C$ (up to equivalence of indexed categories).

Now using \TODO, let $T_{strict}$ be some strict lexcategory which presents $T$ and which has the freeness property that \TODO. Because $T'$ presents $T$, we can choose some specific internal lexcategory $C_{strict}$ in $T_{strict}$ (up to equality!) which presents $C_{int}$. Because $C_{strict}$ presents $C_{int}$ which in turn presents $C$, $\introS$ can be viewed as a (non-strict) lexfunctor from $T$ to the global aspect of $C_{strict}$. Now using the freeness property of $T_{strict}$, we obtain a strict lexfunctor $\introS_{strict}$ from $T_{strict}$ to the global aspect of $C_{strict}$, such that $\introS_{strict}$ presents $\introS$.

Finally, we deal with $\introN$. Natural transformations are essentially unaffected by strictness considerations. That is, given parallel strict functors $A_{strict}$ and $B_{strict}$, natural transformations between these are in bijection with natural transformations between the non-strict functors these present. So our original $\introN$ corresponds to a unique natural transformation between the identity on $T_{strict}$ and $\Hom_{C_{strict}}(1, \introS_{strict}(-))$.

Thus, we have obtained a strict introspective theory $\langle T_{strict}, C_{strict}, \introS_{strict}, \introN \rangle$ presenting the introspective theory $\langle T, C, \introS, \introN \rangle$.
\end{proof}

Strict introspective theories are slightly more convenient than introspective theories for phrasing the results of this chapter, because strict introspective theories are themselves an essentially algebraic notion. That is, there is an essentially algebraic theory such that the models of this theory are the strict introspective theories. (This is in precisely the same way that the theory of strict categories is essentially algebraic, while the theory of categories construed up to equivalence is not quite essentially algebraic.)

As with any essentially algebraic theory, we get automatically a corresponding notion of homomorphism.

\begin{definition}\label{StrictIntrospHomoDefn}
A \defined{homomorphism} between strict introspective theories $\langle T_1, C_1, \introS, \introN \rangle$ and $\langle T_2, C_2, \introS, \introN \rangle$ is a strict lexfunctor $H : T_1 \to T_2$ such that $H[C_1] = C_2$, $H[\introN_t] = \introN_{H(t)}$ for each object $t$ of $T_1$, and the following diagram commutes:

% https://q.uiver.app/?q=WzAsNCxbMCwwLCJUXzEiXSxbMiwwLCJUXzIiXSxbMCwxLCJcXEdsb2J7Q18xfSJdLFsyLDEsIlxcR2xvYntDXzJ9Il0sWzAsMSwiSCJdLFswLDIsIlxcaW50cm9TIiwyXSxbMiwzLCJcXEluZHVjZWRIb21ve0h9e0NfMX0iLDJdLFsxLDMsIlxcaW50cm9TIl1d
\[\begin{tikzcd}
	{T_1} && {T_2} \\
	{\Glob{C_1}} && {\Glob{C_2}}
	\arrow["H", from=1-1, to=1-3]
	\arrow["\introS"', from=1-1, to=2-1]
	\arrow["{\InducedHomo{H}{C_1}}"', from=2-1, to=2-3]
	\arrow["\introS", from=1-3, to=2-3]
\end{tikzcd}\]

In the above diagram, the bottom arrow indicates the action of $H$ as a homomorphism from the global aspect of the $T_1$-internal structure $C_1$, to the global aspect of the corresponding $T_2$-internal structure $H[C_1] = C_2$.
\end{definition}

Such homomorphisms are closed under composition and thus we obtain the category of strict introspective theories.

As the category of models of an essentially algebraic theory, this category must have an initial object. That is, there is a strict introspective theory with a unique homomorphism into any other strict introspective theory. In this chapter, we will find a tractable explicit description of this initial strict introspective theory.

\subsection{Defining geminal categories}
We make heavy use in this section of the conventions from \magicref{MultiplyInternal}, with which the reader may wish to reacquaint themselves.

\begin{definition}[Geminal category]\label{VerboseGeminalCatDefn}
A \defined{geminal category} internal to lexcategory $C_0$ consists of several ingredients:

\begin{itemize}
    \item 
    The first ingredient is an infinite sequence $C_1, C_2, C_3, \ldots$, in which each $C_i$ (for $i \geq 1$) is the global aspect of a lexcategory $C'_i$ internal to $C_{i - 1}$.
\end{itemize}

Thus, each $C'_{i + n}$ is $n$-tuply internal to $C_i$.

(Throughout the following, it will be useful to keep in mind that we are using these general naming habits: Primed names are used for internal structures, while the corresponding unprimed names indicate the corresponding global aspects. Furthermore, names subscripted with index $i$ arise from structure internal to $C_{i - 1}$.)

\begin{itemize}
    \item
    The second ingredient comprising a geminal category is an infinite sequence of internal lexfunctors $F'_1, F'_2, F'_3, \ldots$, where each $F'_i : C'_i \to \Gamma[C'_{i + 1}]$ is internal to $C_{i - 1}$ (for $i \geq 1$).
\end{itemize}

Pictorially, this can be envisioned like so: 

% https://q.uiver.app/?q=WzAsMTEsWzEsMCwiQydfMSJdLFsyLDAsIlxcR2FtbWFbQydfMl0iXSxbMiwxLCJDJ18yIl0sWzMsMSwiXFxHYW1tYVtDJ18zXSJdLFswLDAsIkNfMDoiXSxbMCwxLCJDXzE6Il0sWzAsMiwiQ18yOiJdLFszLDIsIkMnXzMiXSxbNCwyLCJcXEdhbW1hW0MnXzRdIl0sWzAsMywiXFxsZG90cyJdLFs0LDMsIlxcbGRvdHMiXSxbMCwxLCJGJ18xIl0sWzIsMywiRidfMiJdLFs3LDgsIkYnXzMiXSxbNSw0LCJcXEdhbW1hIl0sWzYsNSwiXFxHYW1tYSJdLFs5LDYsIlxcR2FtbWEiXV0=
\[\begin{tikzcd}
	{C_0:} & {C'_1} & {\Gamma[C'_2]} \\
	{C_1:} && {C'_2} & {\Gamma[C'_3]} \\
	{C_2:} &&& {C'_3} & {\Gamma[C'_4]} \\
	\ldots &&&& \ldots
	\arrow["{F'_1}", from=1-2, to=1-3]
	\arrow["{F'_2}", from=2-3, to=2-4]
	\arrow["{F'_3}", from=3-4, to=3-5]
	\arrow["\Gamma", from=2-1, to=1-1]
	\arrow["\Gamma", from=3-1, to=2-1]
	\arrow["\Gamma", from=4-1, to=3-1]
\end{tikzcd}\]

Here, the first row is structure internal to $C_0$, the second row is structure internal to $C_1$ (thus, doubly internal to the ambient $C_0$), the third row is structure internal to $C_2$ (thus, triply internal to the ambient $C_0$), and so on.

In keeping with our naming convention, we shall also use $F_i : C_i \to C_{i + 1}$ to refer to the global aspect of $F'_i$. These global aspects line up more straightforwardly:

% https://q.uiver.app/?q=WzAsNCxbMCwwLCJDXzEiXSxbMSwwLCJDXzIiXSxbMiwwLCJDXzMiXSxbMywwLCJcXGxkb3RzIl0sWzAsMSwiRl8xIl0sWzEsMiwiRl8yIl0sWzIsMywiRl8zIl1d
\[\begin{tikzcd}
	{C_1} & {C_2} & {C_3} & \ldots
	\arrow["{F_1}", from=1-1, to=1-2]
	\arrow["{F_2}", from=1-2, to=1-3]
	\arrow["{F_3}", from=1-3, to=1-4]
\end{tikzcd}\]

Finally, the last ingredients we require are some equations:

\begin{itemize}
    \item 
     We require that $F_i[C'_j] = C'_{j + 1}$ and $F_i[F'_j] = F'_{j + 1}$ for $j > i \geq 1$.
     
    \item
    Furthermore, we require that the following diagram of lexfunctors internal to $C_{i - 1}$ commutes, for each $i \geq 1$. We call this equation $E_i$.
    
% https://q.uiver.app/?q=WzAsNCxbMCwwLCJDJ19pIl0sWzAsMiwiXFxHYW1tYSBbQydfe2kgKyAxfV0iXSxbMiwyLCJcXEdhbW1hIFtcXEdhbW1hW0MnX3tpICsgMn1dXSJdLFsyLDAsIlxcR2FtbWEgW0MnX3tpICsgMX1dIl0sWzAsMSwiRidfaSIsMl0sWzAsMywiRidfaSJdLFszLDIsIlxcR2FtbWEgW0YnX3tpICsgMX1dIl0sWzEsMiwiXFxJbmR1Y2VkSG9tb3tGJ19pfXtDJ197aSArIDF9fSIsMl1d
\[\begin{tikzcd}
	{C'_i} && {\Gamma [C'_{i + 1}]} \\
	\\
	{\Gamma [C'_{i + 1}]} && {\Gamma [\Gamma[C'_{i + 2}]]}
	\arrow["{F'_i}"', from=1-1, to=3-1]
	\arrow["{F'_i}", from=1-1, to=1-3]
	\arrow["{\Gamma [F'_{i + 1}]}", from=1-3, to=3-3]
	\arrow["{\InducedHomo{F'_i}{C'_{i + 1}}}"', from=3-1, to=3-3]
\end{tikzcd}\]

That is, we require that $\InducedHomo{F'_i}{C'_{i + 1}} \circ F'_i = \Gamma[F'_{i + 1}] \circ F'_i$. This could be naively glossed as \quote{$F'_i \circ F'_i = F'_{i + 1} \circ F'_i$}, in severe abuse of notation.
\end{itemize}

This concludes the definition of a geminal category internal to $C_0$.
\end{definition}

By a \defined{geminal category} simpliciter, we mean of course the case where $C_0 = \Set$. (Note that in this case, $C'_1$ can be identified with its global aspect $C_1$, in the same way that any structure internal to $\Set$ can be identified with its global aspect, as the global sections functor from $\Set$ to $\Set$ is the identity.). We wrote out here the definition for general $C_0$, instead of specifically for $C_0 = \Set$, in order to emphasize certain symmetries in this definition.

When being fully explicit, we reference a geminal category by enumerating $\langle C'_1, C'_2, C'_3, \ldots; F'_1, F'_2, F'_3, \ldots \rangle$. Given such a geminal category $K$, we may write $|K|$ to refer to its underlying lexcategory $C'_1$.

All aforementioned structure apart from $C_0$ itself has been given as $i$-tuply internal to $C_0$ for some $i > 0$. Thus, all of this structure is indeed given by diagrams within $C_0$.

Indeed, this definition of geminal category is manifestly essentially algebraic. That is, there is an essentially algebraic theory such that models of that theory internal to $C_0$ are the same thing as geminal categories internal to $C_0$.

Our ultimate goal will be to show that this theory of geminal categories is the initial introspective theory. This is the whole motivation for our study of geminal categories. But to show this result, we must develop some other machinery first.

\subsection{Geminal category homomorphisms}
As geminal categories are defined by an essentially algebraic theory, we automatically get a notion of homomorphism between geminal categories. It amounts to the following:

\begin{definition}\label{VerboseGeminalCatHomoDefn}
Given two geminal categories $\langle C'_1, C'_2, C'_3, \ldots; F'_1, F'_2, F'_3, \ldots \rangle$ and $\langle D'_1, D'_2, $ $D'_3, \ldots; \phi'_1, \phi'_2, \phi'_3, \ldots \rangle$, a \defined{homomorphism} from the former to the latter consists of a strict lexfunctor $H : C'_1 \to D'_1$ such that $H[C'_i] = D'_i$ and $H[F'_i] = \phi'_i$ for each $i > 1$, while also the following diagram commutes:

% https://q.uiver.app/?q=WzAsNCxbMCwwLCJDJ18xIl0sWzIsMCwiRCdfMSJdLFsyLDIsIlxcR2xvYntIW0MnXzJdfSA9IFxcR2xvYntEXzInfSJdLFswLDIsIlxcR2xvYntDXzInfSJdLFswLDEsIkgiXSxbMCwzLCJGJ18xIiwyXSxbMywyLCJcXEluZHVjZWRIb21ve0h9e0MnXzJ9IiwyXSxbMSwyLCJcXHBoaSdfMSJdXQ==
\[\begin{tikzcd}
	{C'_1} && {D'_1} \\
	\\
	{\Glob{C_2'}} && {\Glob{H[C'_2]} = \Glob{D_2'}}
	\arrow["H", from=1-1, to=1-3]
	\arrow["{F'_1}"', from=1-1, to=3-1]
	\arrow["{\InducedHomo{H}{C'_2}}"', from=3-1, to=3-3]
	\arrow["{\phi'_1}", from=1-3, to=3-3]
\end{tikzcd}\]
\end{definition}

\begin{theorem}\label{GeminalContainsGeminal}
Given any geminal category $K = \langle C'_1, C'_2, C'_3, \ldots; F'_1, F'_2, F'_3, \ldots \rangle$, we have also that $\langle C'_2, C'_3, C'_4, \ldots; F'_2, F'_3, F'_4 \ldots \rangle$ comprises a geminal category internal to $|K| = C'_1$. We refer to this internal geminal category as $\InteriorGeminal{K}$.

We furthermore have that $F'_1$ acts as a geminal category homomorphism from $K$ to the global aspect of $\InteriorGeminal{K}$. We refer to this homomorphism as $\IntoSelf{K} : K \to \Gamma[\InteriorGeminal{K}]$.
\end{theorem}
\begin{proof}
This is all direct by definition.

For the first part, each condition imposed upon each $C'_{i}$ or $F'_{i}$ in the definition of a geminal category comes with an analogous condition imposed upon $C'_{i + 1}$ or $F'_{i + 1}$. Thus, it is immediate that the given $\InteriorGeminal{K}$ satisfies the conditions to be a geminal category internal to $|K|$.

For the second part, the definition of a geminal category directly imposes upon $F'_1$ precisely the conditions which are necessary for $F'_1$ to comprise a geminal category homomorphism from $K$ to the global aspect of $\InteriorGeminal{K}$. In particular, equation $E_1$ from \magicref{VerboseGeminalCatDefn} is identical to the commutative diagram from \magicref{VerboseGeminalCatHomoDefn}, in this context.
\end{proof}

Via the yoga of functorial semantics, \magicref{GeminalContainsGeminal} states how the theory of geminal categories can be equipped as an introspective theory. In detail, this is given like so:

\begin{construction}\label{GLCatTheoryIsIntrosp}
Let $\GLCatTheory$ be the free strict lexcategory with an internal geminal category (that is, in the terminology of \magicref{QuasiTheoryTheory}, we take $\GLCatTheory$ to be the classifying strict lexcategory $\classifying{\theoryT}$, where $\theoryT$ is the theory of geminal categories). 

Thus, strict lexfunctors from $\GLCatTheory$ to any other strict lexcategory $D$ correspond to geminal categories internal to $D$, while natural transformations between such lexfunctors correspond to homomorphisms between these $D$-internal geminal categories.

Let $K$ denote the $\GLCatTheory$-internal geminal category corresponding to the identity functor on $\GLCatTheory$.

By \magicref{GeminalContainsGeminal} in the internal logic of $\GLCatTheory$, we obtain also a geminal category $\InteriorGeminal{K}$ internal to $|K|$, as well as a homomorphism $\IntoSelf{K} : K \to \Gamma[\InteriorGeminal{K}]$.

Thus, there is some strict lexfunctor $\introS$ from $\GLCatTheory$ to the global aspect of $|K|$, corresponding to $\InteriorGeminal{K}$. Furthermore, there is some natural transformation $\introN$ from the identity functor on $\GLCatTheory$ to $\Hom_{|K|}(1, \introS(-))$, corresponding to $\IntoSelf{K}$.

Putting this together, we have a strict introspective theory $\langle \GLCatTheory, |K|, \introS, \introN \rangle$.
\end{construction}

\sTODOinline{Note that there is a more limited analogue of the above, where we observe that the free X with an internal geminal Y is itself a geminal X, whenever Y extends X and X extends the notion of a lexcategory. The difficulty with turning this into an introspective theory is that the property we really depend on from the free lexcategory $L$ with an internal gadget of some sort is not just the 1-categorical property that it has a unique homomorphism to every other gadget, but the 2-categorical property that the category of homomorphisms from it to another lexcategory and natural transformations between those is equivalent to the 1-category of internal gadgets within that codomain lexcategory. This is true for lexcategories, but not necessarily for other doctrines, and I believe this is related to how $\Hom(1, -)$ is always a lexfunctor (thus turning internal models into genuine models,) but not always an X-functor. The subtle role of this in the above proof should be highlighted.}

\subsection{Compactly defined geminal categories}
The above all amounts to an infinitary presentation of the theory of geminal categories. For this reason, we call it the \quote{verbose presentation} of geminal categories. However, it turns out this same theory can be finitely axiomatized as well.

\begin{definition}[Geminal category, compact presentation]\label{CompactGeminalCatDefn}
A \defined{compactly presented geminal category} internal to lexcategory $C_0$ consists of the structure $C'_i$, $F'_i$, and equations $E_i$ of the verbose presentation, but only for $i \in \{1, 2\}$.

(Here, in interpreting the codomain of $F'_2$, we take $C'_3$ to be $F_1[C'_2]$, and in interpreting the equation $E_2$, we take $F'_3$ to be $F_1[F'_2]$)

That is, a compactly presented geminal category internal to $C_0$ consists of the following six pieces of data:

\begin{itemize}
    \item A lexcategory $C'_1$ internal to $C_0$, whose global aspect we call $C_1$.
    
    \item A lexcategory $C'_2$ internal to $C_1$, whose global aspect we call $C_2$.
    
    \item A lexfunctor $F'_1 : C'_1 \to \Gamma[C'_2]$, internal to $C_0$. We call the global aspect of this $F_1 : C_1 \to C_2$.
    
    \item A lexfunctor $F'_2 : C'_2 \to \Gamma[C'_3]$, internal to $C_1$.
    
    (Here, $C'_3$ is defined as $F_1[C'_2]$.)
    
    \item The equation $\InducedHomo{F'_1}{C'_{2}} \circ F'_1 = \Gamma[F'_2] \circ F'_1$, internal to $C_0$. We call this equation $E_1$.
    
    \item The equation $\InducedHomo{F'_2}{C'_{3}} \circ F'_2 = \Gamma[F'_3] \circ F'_2$, internal to $C_1$. We call this equation $E_2$.
    
    (Here, $F'_3$ is defined as $F_1[F'_2]$.)
\end{itemize}
\end{definition}

As usual, we reference a compactly presented geminal category by enumerating the ordered tuple $\langle C'_1, C'_2; F'_1, F'_2 \rangle$.

Clearly, the structure defining a compactly presented geminal category is part of the structure in our verbose definition of a geminal category. But in fact, these are equivalent definitions.

\begin{theorem}\label{GeminalCompactIsVerbose}
The structure of a compactly presented geminal category uniquely determines the further structure of a geminal category (as originally defined in \magicref{VerboseGeminalCatDefn})).
\end{theorem}
\begin{proof}
By definition, in a geminal category, we must have that $F_1[C'_j] = C'_{j + 1}$ and $F_1[F'_j] = F'_{j + 1}$ for each $j > 2$.

Accordingly, if we are given the structure in \magicref{CompactGeminalCatDefn}, and we are to extend it to all the further structure in \magicref{VerboseGeminalCatDefn}, it inductively must be the case that $C'_j = F_1^{j - 2}[C'_2]$ and $F'_j = F_1^{j - 2}[F'_2]$ for each $j > 2$. Adopt these definitions throughout the following accordingly.

The equations given to us directly in the compact presentation are the equations $E_1$ and $E_2$ of the verbose presentation. Furthermore, we obtain the equation $E_i$ for $i > 2$ by applying $F_1^{n - 1}$ to $E_2$.

What remains is only to see that each $F_i$ for $i > 1$ also takes $C'_j$ to $C'_{j + 1}$ and takes $F'_j$ to $F'_{j + 1}$, for $j > i \geq 1$.

We prove this by induction on $i$. For the base case of $i = 1$, we have ensured this by construction. As for the inductive step, suppose we know this already holds for $i$. Then $F_{i + 1}[C'_j] = F_{i + 1} [F_i [C'_{j - 1}]] = F_i [F_i [C'_{j - 1}]] = F_i [C'_j] = C'_{j + 1}$, where the second step is by $E_i$ \TODO and the other steps are by our induction hypothesis. And similarly with $F'$ in place of $C'$ throughout as well.
\end{proof}

\begin{corollary}\label{CompactGeminalCatHomoDefn}
In \magicref{VerboseGeminalCatHomoDefn}, the conditions $H[C'_i] = D'_i$ and $H[F'_i] = \phi'_i$ automatically follow for all $i > 2$ once they hold for $i = 2$.
\end{corollary}

Thus, we can go back and forth between thinking of geminal categories in either the verbose or compact presentation as we please, whichever is most convenient at any moment.

\subsection{Geminal categories from introspective theories}

\begin{construction}\label{IntrospAsGeminal}
From a strict introspective theory $\langle T, C, \introS, \introN \rangle$, we obtain a geminal category $\langle T, C; \introS, \introN_{C} \rangle$, whose underlying lexcategory is $T$. This is the canonical way to view an introspective theory as a geminal category.
\end{construction}
\begin{proof}
It is immediate in the definition of a strict introspective theory that $C$ is a lexcategory internal to $T$, and $\introS$ is a lexfunctor from $T$ to $\Glob{C}$. This gives us the first three out of the six ingredients of \magicref{CompactGeminalCatDefn}.

As for $\introN_{C}$, meaning the components of the natural transformation $\introN$ at the objects of $C$, this gives us a $T$-internal lexfunctor from $C$ to $\Hom_C(1, \introS[C]) = \Gamma[\introS[C]]$. This is the fourth ingredient of \magicref{CompactGeminalCatDefn}.

What remains are to verify equations $E_1$ and $E_2$. In this context, $E_1$ is a special case of \magicref{SMatchesN}, while $E_2$ is given by the naturality of $\introN$ with respect to the components of $\introN_C$ themselves.
\end{proof}

There is another closely related construction which is of even more importance:

\begin{construction}\label{IntrospContainsGeminal}
From a strict introspective theory $\langle T, C, \introS, \introN \rangle$, we obtain a $T$-internal geminal category $\langle C, \introS[C]; \introN_{C}, \introS[\introN_{C}] \rangle$, whose underlying lexcategory is $C$.
\end{construction}
\begin{proof}
This is the result of first obtaining the geminal category $\gamma = \langle T, C, \introS, \introN_C \rangle$ from \magicref{IntrospAsGeminal}, and then forming $\InteriorGeminal{\gamma}$.
\end{proof}

We now are ready to prove our main result about geminal categories.

\subsection{The free introspective theory}
\begin{theorem}\label{InitialIntrospectiveTheory}
The strict introspective theory given in \magicref{GLCatTheoryIsIntrosp} is the initial strict introspective theory.
\end{theorem}
\begin{proof}
We must show there is a unique homomorphism from the strict introspective theory $\langle \GLCatTheory, K \rangle$ of \magicref{GLCatTheoryIsIntrosp} to any other strict introspective theory $\langle T, D, \introS, \introN \rangle$.

Such a homomorphism is comprised of a strict lexfunctor $H : \GLCatTheory \to T$ satisfying certain conditions. By the nature of $\GLCatTheory$, this amounts to a geminal category $\langle D'_1, D'_2, D'_3, \ldots; F'_1, F'_2, F'_3, \ldots \rangle$ internal to $T$ satisfying certain conditions.

One particular geminal category internal to $T$ is the one that is given by $\gamma = \langle D, \introS[D]; \introN_D, \introS[\introN_D] \rangle$, as noted at \magicref{IntrospContainsGeminal}. In verbose terms, this geminal category is $\langle D, \introS[D], \introS[\introS[D]], \ldots;$ $ \introN_D, \introS[\introN_D], \introS[\introS[\introN_D]], \ldots \rangle$, with each successive component being $\introS$ applied to the previous component.

What remains is to show that the lexfunctor $H : \GLCatTheory \to T$ corresponding to this $\gamma$ uniquely satisfies the conditions of \magicref{StrictIntrospHomoDefn}.

The condition $H[C'_1] = D$ in \magicref{StrictIntrospHomoDefn} says in this context that we must use a geminal category whose underlying lexcategory is $D$.

The condition concerning $\introN$ in \magicref{StrictIntrospHomoDefn}, along with the definition of $\introN$ in \magicref{GLCatTheoryIsIntrosp}, says that we must use a geminal category whose first lexfunctor component is $\introN_{D}$.

Finally, the commutative diagram concerning $\introS$ in \magicref{StrictIntrospHomoDefn}, along with the definition of $\introS$ in \magicref{GLCatTheoryIsIntrosp}, says we must use a geminal category such that each successive component of this geminal category is $\introS$ applied to the previous component.

The conjunction of these conditions clearly is uniquely satisfied by $\gamma$. This completes the proof.
\end{proof}

\subsection{Geminal gadgets}
We have now successfully described the initial introspective theory. But we can also take our free construction results a bit further than this.

Specifically, every introspective theory is, among other things, an essentially algebraic theory extending the theory of strict lexcategories. That is, we have a functor from the category of introspective theories to the category of extensions of the essentially algebraic theory of strict lexcategories (essentially, this functor takes $\langle T, C, \introS, \introN \rangle$ to $\langle T, C \rangle$). This functor has a left adjoint.

Put in other words, for any essentially algebraic theory $Th$ such that models of $Th$ come with an underlying strict lexcategory, there is a free strict introspective theory $\langle T, C, \introS, \introN \rangle$ with a designated $T$-internal model of $Th$ with underlying lexcategory $C$.

For simplicity as a first introduction, everything done previously was the special case where $Th$ was simply the theory of strict lexcategories itself. But now we describe the more general results, which follow by almost exactly the same reasoning as used before:

\TODOinline{Now we can duplicate all of the above, except for the result that introspective theories are themselves geminal categories, for \quote{geminal gadgets} more generally. That is, the left adjoint mentioned above is given by a general construction perfectly analogous to the one we used to construct the concept of geminal categories above.}

\TODOinline{Perhaps also discuss the straightforward notion of a non-strict geminal category or gadget: One for which $C_1$ is a non-strict lexcategory or gadget, and $F_1$ needn't be strict either (preserves finite limits but not necessarily on the nose), but everything else remains strict. Every non-strict geminal gadget straightforwardly admits a presentation by a strict one, by using a presentation of $C_1$ with no nontrivial equations on objects in a suitable sense.}

\subsection{Modal logic in geminal categories}
\TODOinline{Rewrite this just in terms of the internal $\Box_C$ in the initial introspective theory, which then automatically applies to arbitrary geminal categories.}

The significance of our constellation of constructions turning introspective theories into geminal categories may be further illuminated by thinking about box notation for geminal categories.

\begin{definition}\label{BoxForGeminal}
Given a geminal category $\langle C'_1, C'_2; F'_1, F'_2 \rangle$, we may define a bifunctor on $C'_1$ like so: Given objects $c$ and $d$ in $C'_1$, we define $\Box(c \implies d)$ to mean $\Hom_{C'_2}(F'_1(c), F'_1(d))$. We then take $\Box d$ as shorthand for $\Box(1 \implies d)$.
\end{definition}

This is reminiscent of our previously introduced box notation for introspective theories (where we had both $\Box_T$ and $\Box_C$). And indeed, we will now link all of these box notations:

\begin{observation}
Given an introspective theory $\langle T, C, \introS, \introN \rangle$, if we think of this as a geminal category with underlying lexcategory $T$ via \magicref{IntrospAsGeminal} and then apply the box notation from \magicref{BoxForGeminal}, we find that for $t, s \in T$, we have $\Box(s \implies t) = \Hom_C(\introS(s), \introS(t))$, and in particular, $\Box t = \Hom_C(1, \introS(t))$. This exactly matches the definition of $\Box_T t$ used for introspective theories from \magicref{BoxDefn}.

Alternatively, if we use \magicref{IntrospContainsGeminal} to obtain a $T$-internal geminal category with underlying lexcategory $C$, and then apply the box notation from \magicref{BoxForGeminal}, we find that for $c, d \in C$, we have $\Box(c \implies d) = \Hom_{\introS[C]}(\introN_{\Ob(c)}(c), \introN_{\Ob(c)}(d))$. When $c$ and $d$ are globally defined, we can use \magicref{SMatchesN} to show this equal to $\Hom_{\introS[C]}(\introS(c), \introS(d)) =  \introS(\Hom_C(c, d))$. This exactly matches the definition of $\Box_C (c \implies d)$ used with introspective theories from \magicref{BoxDefn}.

\TODOinline{When $c$ and $d$ are not globally defined, then what?}
\end{observation}

\TODOinline{Guide readers by showing how we have the axioms of GL modal logic in a geminal category, but do NOT have A |- []A. Weave our two main classes of example into here.}

\subsection{Co-free introspective theories}

\begin{construction}\label{BoxCoalgebrasInGeminal}
Let $\langle C_1, C'_2; F_1, F'_2 \rangle$ be a geminal category. Because $\Box_{C_1}$ is an endolexfunctor, the category of $\Box_{C_1}$-coalgebras is a lexcategory, with its forgetful functor to $C_1$ creating finite limits (by the analogous reasoning to \magicref{CommaCategoryColimits} for inserters rather than comma categories; indeed, this category of coalgebras can be seen as a (non-full) sublexcategory of the comma category $\comma{\id}{\Box_{C_1}}$). As $C_1$ is in fact a strict lexcategory, we can thus equip the category of $\Box_{C_1}$-coalgebras as a strict lexcategory, with its forgetful functor to $C_1$ creating basic limits (note that it is fine here if $\Box_{C_1}$ does not preserve basic limits on the nose; all that mattered was that it preserves finite limits in the non-strict-sense).
\end{construction}

\begin{construction}
For any geminal category $C_1$, there is a terminal strict introspective theory equipped with a geminal category homomorphism to $C_1$. This co-free introspective theory admits a tractable explicit description, as a certain subcategory of the coalgebras for $\Box_{C_1}$, suitably equipped.
\end{construction}
\openDetails
Let $C_1 = \langle C_1, C'_2; F_1, F'_2 \rangle$ be a geminal category, let $T = \langle T, C, \introS, \introN \rangle$ be a strict introspective theory (which we can also construe as a geminal category via \magicref{IntrospAsGeminal}), and let $H : T \to C_1$ be a geminal category homomorphism.

By virtue of $H$ being a geminal category homomorphism, we have that $H \circ \Box_T = \Box_{C_1} \circ H$. In detail, this is seen via the following commutative diagram:

% https://q.uiver.app/?q=WzAsNixbMywwLCJDXzEiXSxbMywxLCJcXEdsb2J7QydfMn0iXSxbMywyLCJDXzEiXSxbMCwwLCJUIl0sWzAsMSwiXFxHbG9ie0N9Il0sWzAsMiwiVCJdLFswLDEsIkZfMSIsMl0sWzEsMiwiXFxHYW1tYV97QydfMn0iLDJdLFswLDIsIlxcQm94X3tDXzF9IiwwLHsiY3VydmUiOi01fV0sWzMsMCwiSCJdLFszLDQsIlxcaW50cm9TIl0sWzQsMSwiXFxJbmR1Y2VkSG9tb3tIfXtDfSIsMV0sWzQsNSwiXFxHYW1tYV9DIl0sWzUsMiwiSCIsMl0sWzMsNSwiXFxCb3hfVCIsMix7ImN1cnZlIjo1fV1d
\[\begin{tikzcd}
	T &&& {C_1} \\
	{\Glob{C}} &&& {\Glob{C'_2}} \\
	T &&& {C_1}
	\arrow["{F_1}"', from=1-4, to=2-4]
	\arrow["{\Gamma_{C'_2}}"', from=2-4, to=3-4]
	\arrow["{\Box_{C_1}}", curve={height=-30pt}, from=1-4, to=3-4]
	\arrow["H", from=1-1, to=1-4]
	\arrow["\introS", from=1-1, to=2-1]
	\arrow["{\InducedHomo{H}{C}}"{description}, from=2-1, to=2-4]
	\arrow["{\Gamma_C}", from=2-1, to=3-1]
	\arrow["H"', from=3-1, to=3-4]
	\arrow["{\Box_T}"', curve={height=30pt}, from=1-1, to=3-1]
\end{tikzcd}\]

In the above diagram, the left side is the definition of $\Box_T$ and the right side is the definition of $\Box_{C_1}$. The top rectangle is one of the conditions in \magicref{VerboseGeminalCatHomoDefn} and the bottom rectangle is by \magicref{InducedGlobalCommute}.

Thus, the whiskering of $\introN : \id_T \to \Box_T$ along $H$ yields a natural transformation from $H$ to $\Box_{C_1} \circ H$. Illustrated like so:

% https://q.uiver.app/?q=WzAsNixbNCwwLCJDXzEiXSxbNCwxLCJcXEdsb2J7QydfMn0iXSxbNCwyLCJDXzEiXSxbMCwwLCJUIl0sWzEsMSwiXFxHbG9ie0N9Il0sWzAsMiwiVCJdLFswLDEsIkZfMSIsMl0sWzEsMiwiXFxHYW1tYV97QydfMn0iLDJdLFswLDIsIlxcQm94X3tDXzF9IiwwLHsiY3VydmUiOi01fV0sWzMsMCwiSCJdLFszLDQsIlxcaW50cm9TIl0sWzQsMSwiXFxJbmR1Y2VkSG9tb3tIfXtDfSIsMV0sWzQsNSwiXFxHYW1tYV9DIl0sWzUsMiwiSCIsMl0sWzMsNSwiXFxpZCIsMix7ImxldmVsIjoyLCJzdHlsZSI6eyJoZWFkIjp7Im5hbWUiOiJub25lIn19fV0sWzE0LDQsIlxcaW50cm9OIiwyLHsic2hvcnRlbiI6eyJzb3VyY2UiOjIwfX1dXQ==
\[\begin{tikzcd}
	T &&&& {C_1} \\
	& {\Glob{C}} &&& {\Glob{C'_2}} \\
	T &&&& {C_1}
	\arrow["{F_1}"', from=1-5, to=2-5]
	\arrow["{\Gamma_{C'_2}}"', from=2-5, to=3-5]
	\arrow["{\Box_{C_1}}", curve={height=-30pt}, from=1-5, to=3-5]
	\arrow["H", from=1-1, to=1-5]
	\arrow["\introS", from=1-1, to=2-2]
	\arrow["{\InducedHomo{H}{C}}"{description}, from=2-2, to=2-5]
	\arrow["{\Gamma_C}", from=2-2, to=3-1]
	\arrow["H"', from=3-1, to=3-5]
	\arrow[""{name=0, anchor=center, inner sep=0}, "\id"', Rightarrow, no head, from=1-1, to=3-1]
	\arrow["\introN"', shorten <=4pt, Rightarrow, from=0, to=2-2]
\end{tikzcd}\]

This natural transformation from $H$ to $\Box_{C_1} \circ H$ acts as a functor $\beta$ from $T$ to the category of $\Box_{C_1}$-coalgebras, such that $\beta$ followed by the forgetful functor to $C_1$ yields $H$. As $H$ is a strict lexfunctor, this $\beta$ is also a strict lexfunctor, when the category of $\Box_{C_1}$-coalgebras is construed as a strict lexcategory in the manner of \magicref{BoxCoalgebrasInGeminal}.

Not only that, but every coalgebra in the range of $\beta$ has the following property:

% https://q.uiver.app/?q=WzAsNCxbMCwwLCJjIl0sWzAsMiwiXFxCb3ggYyJdLFsyLDAsIlxcQm94IGMiXSxbMiwyLCJcXEJveF4yIGMiXSxbMCwxLCJtIiwyXSxbMCwyLCJtIl0sWzIsMywiXFxCb3ggbSJdLFsxLDMsIjRfYyIsMl1d
\[\begin{tikzcd}
	c && {\Box c} \\
	\\
	{\Box c} && {\Box^2 c}
	\arrow["m"', from=1-1, to=3-1]
	\arrow["m", from=1-1, to=1-3]
	\arrow["{\Box m}", from=1-3, to=3-3]
	\arrow["{4_c}"', from=3-1, to=3-3]
\end{tikzcd}\]

\TODOinline{Explain above diagram}
\closeDetails

----

\TODOinline{Show that by restricting to coalgebras for $\Box_{C'_1}$ with the suitable property, we get an introspective theory, which is terminal among all introspective theories which map into this geminal category}

\begin{construction}
For any lexcategory $C_0$ with an internal geminal category $\gamma$, there is a terminal geminal category $G$ equipped with a lexfunctor $H : |G| \to C_0$ such that $H[\InteriorGeminal{G}] = \gamma$. This co-free $G$ admits a tractable explicit description as $C_0 \times |\gamma|$ suitably equipped.
\end{construction}
\openDetails
Let $\gamma = \langle C'_1, C'_2; F'_1, F'_2 \rangle$. Throughout the following, let unprimed names denote global aspects of primed names, in our usual fashion.

Let $G_0 = C_0 \times C_1$. We have that $\gamma \times \InteriorGeminal{\gamma}$ is a geminal category $\langle G'_1, G'_2; \phi'_1, \phi'_2 \rangle$ internal to $G_0$, with $G'_1 = C'_1 \times C'_2$. 

Let lexfunctor $\phi_0 : G_0 \to G_1$ be defined by $\phi_0(c_0, c_1) = (c_1, F_1(c_1))$. It's straightforward to then verify that $\langle G_0, G'_1; \phi_0, \phi'_1 \rangle$ is a geminal category (with the only nontrivial aspect being the equation $E_0$, as it were).

We also clearly have a projection lexfunctor from $G_0$ to $C_0$ which does indeed take the internal geminal category to $\gamma$.

\TODOinline{Show that this has the terminality property, and clean up this writing.}

Suppose given any arbitrary geminal category $K = \langle K_0, K'_1; P_0, P'_1 \rangle$ and lexfunctor $H_0 : K_0 \to C_0$ such that $H[\InteriorGeminal{K}] = \gamma$. Observe that we also have a lexfunctor $H_1 : K_0 \to C_1$ given by $H_1 = \InducedHomo{H_0}{K'_1} \circ P_0$. Thus, we obtain a unique lexfunctor $H : C_0 \to G_0$ whose two projections are $H_0$ and $H_1$. We shall show that $H$ is a geminal category homomorphism, and the unique one with the appropriate properties. \TODO
\closeDetails

\TODOinline{Observe that the above two co-free constructions can be combined, to find the terminal introspective theory with a suitable lexfunctor into a given lexcategory yielding a given internal geminal category}

\subsection{Don't read beyond here}
\TODOinline{Everything following in this chapter is about to get thrown out/rewritten}

\begin{TODOblock}
Note that the above means every geminal gadget is the carrier of a coalgebra for a particular endofunctor on geminal gadgets. Those gadgets for which this coalgebra is an isomorphism are of particular note (as related to GLS modal logic and not just GL modal logic). We can also iterate transfinitely and take a colimit (usually just an omega-colimit for finitary notions of gadget), i.e. apply Adamek's construction, to reflect arbitrary geminal gadgets into these GLS geminal gadgets. See the section on well-pointed endofunctors at "transfinite construction of free algebras" on nlab, or "A unified treatment of transfinite constructions for free algebras, free monoids, colimits, associated sheaves, and so on.". Keep in mind that the relevant endofunctor is indeed well-pointed, by the properties of geminal categories (actually, I no longer think this is well-pointed). Keep in mind, this infinitary colimit reflection can be carried out for geminal categories internal to $\Set$, given $\Set$'s sufficient infinitary colimit structure; we do not have a guarantee that a geminal category internal to an arbitrary lexcategory can be reflected into a GLS category.

Actually, none of this is specific to geminal gadgets (models of free introspective theories). It applies to models of any introspective theory. (Although maybe we don't actually have well-pointedness here in general?)

It may be curious to consider in what cases an introspective theory is itself GLS-geminal like so. Many natural examples do have this property (the $\Sigma_1$ introspective theory with internal $\Sigma_1$ C for a sound theory like PA, the topos of trees with step operator, the initial arithmetic universe, etc). These will generally also be "super-introspective theories" in the terminology of the TODO note at the end of this section.
\end{TODOblock}

There are analogues of all the above results where we use finite product theories instead, and talk about enriched structures rather than internal structures. This can be made to work because finite product structure suffices to discuss enriched category structure and even enriched category-with-finite-product structure once the structure of the objects themselves has been fixed. We decline for now to formalize this, as it is a bit off the path of our main interest in introspective theories. \TODOinline{Formalize this, as free locally introspective theories (consider both the finite limit and finite product cases) unify this with the above, show how Kripke-4 categories are the enriched analogue of a geminal cartesian closed category. Note how a geminal lex cartesian closed category always gives rise to such an enriched-geminal cartesian closed category. Discuss the weaker notion of enriched-geminal categories which do not presume cartesian closed structure as well, and note how any geminal category gives rise to such an enriched-geminal category.}

\begin{TODOblock}
Observe that geminal categories differ importantly from introspective theories because we do not have X |- []X for arbitrary objects in a geminal category, like we do in an introspective theory. Observe that we do have Loeb's theorem for representable presheaves in any geminal category, just from the fact that we have it as a claim about $C$ within any introspective theory, even for these objects which do not satisfy X |- []X. And similarly for certain functorial fixed points. (Write this intuition about how geminal categories differ from introspective theories most notably wrt to the X |- []X presumption into the introductory section.)

But a geminal category in itself does not give us the structure to talk about presheaves or functors of a sort not definable for, well, a generic geminal category, and so we do not get Loeb's theorem or fixed points for arbitrary presheaves or functors. (It is worth observing this specific failure with an example.)
\end{TODOblock}

\begin{TODOblock}
Write out that Geminal Ys are the same as Geminal (Geminal Y)s. More to the point, (geminal Ys)-extending-X, for a specific geminal Y structure called X, are the same as geminal (Geminal-Ys-extending-X).

Thus, every specific geminal structure (for one kind of gadget) is an initial geminal structure (for another kind of gadget).

One consequence of this is that every geminal structure indeed arises as the global aspect of some gadget-with-underlying-category C internal to some introspective theory (as it is easy, essentially tautological, to see that the free geminal gadget is the one which arises as the global sections of the introspective theory of geminal gadgets).
\end{TODOblock}

\begin{TODOblock}
Discuss \quote{super-introspective} theories: Introspective theories whose internal category is equipped not just as a geminal category but furthermore as an introspective theory, and satisfying the coherence axiom that for any X, when you internalize the map from X to []X, you get the internal introspective theory's map from (internalization of X) to [](internalization of X). Note that this is a kind of well-pointedness condition.

These are the same as geminal introspective theories such that the two underlying geminal category structures (from being a geminal X and from being an introspective theory) coincide.

An example is given by the first part of \cref{SigmaModelSimple}, before we extend the internal copy to be more than just $\Sigma_1$. When the internal copy is restricted to $\Sigma_1$ formulas, it is not merely geminal but furthermore canonically introspective, in a matching way with the outer introspective structure.
\end{TODOblock}

\TODOinline{Note that if the theory of gadgets is finitely presentable, then so is the theory of geminal gadgets}

\TODOinline{Note that the slice introspective theory construction gives us the free augmentation of an introspective theory with a global element of a particular object}

\TODOinline{Do we have slice geminal categories?}

\fileend

\section{Models}

\subsection{Automatic consistency results without models as such}
We already know that the theory of geminal categories is an introspective theory. And because every introspective theory is itself a geminal category, we know that the theory of introspective theories only prove $\Box A$ if it furthermore proves $A$.

Finally, we know that every lexcategory can be equipped as a geminal category in a trivial way, by taking its internal geminal category to be trivially $1$, even when the outer lexcategory needn't be trivial. From this, we can conclude that the theory of introspective theories is nontrivial in the sense that it does not prove its internal geminal category to be trivial. Thus, it does not prove $\Box A$ for all $A$. Furthermore, combining this with the previous paragraph, we have the stronger consistency result that for every $n$, the theory of introspective theories does not prove $\Box^n A$ for all $A$.

In this way, simply by consideration of the freeness properties already established in the chapter on geminal categories, we already know the theory of introspective theories to have highly nontrivial content, even without needing to find any models of it \quote{in the wild}.

\TODOinline{Discuss more why the stronger consistency result is really the relevant thing to think about.}

\subsection{Models based on sigma-1 or arbitrary extensions of PA, or ZFC, or etc}
\TODOinline{I will write this section in a sloppy way for now and then improve it later.}

This section reviews and builds upon the construction previously seen at \cref{SigmaModelSimple}.

\begin{construction}\label{Sigma1Model}

Consider a sigma-1 theory $\tau$ extending PA (or ZFC, or any such thing), in the sense of an extension whose axioms are computably enumerable. Actually, for now, let's just consider PA simpliciter.

\TODOinline{It probably isn't easy to pin down in a clean way exactly the minimal kind of system in which this goes through, but it could be useful to name some weak subsystems of arithmetic in which it goes through. In particular, we should not expect this to go through in Robinson's Arithmetic Q which lacks induction entirely, but we should expect it to still work in systems that just have induction for $\Sigma_1$ formulae).}

Consider the category $T$ whose objects are the sigma-1 formulas $\phi(n, m)$ in the language of PA which define binary relations on the natural numbers which PA proves to be partial equivalence relations (i.e., symmetric and transitive). Given any two such formulas $\phi(n, m)$ and $\psi(n, m)$, a morphism in $T$ from $\phi$ to $\psi$ is a sigma-1 formula $F(n, m)$ on the natural numbers which PA proves to correspond to the graph of a function between the subquotients of $\mathbb{N}$ corresponding to $\phi$ and to $\psi$, respectively. That is, such that PA proves the universal closures of the following:

$F(n, m) \implies \phi(n, n) \wedge \psi(m, m)$

$\phi(n_1, n_2) \wedge \psi(m_1, m_2) \wedge F(n_1, m_1) \implies F(n_2, m_2)$

$\phi(n, n) \implies \exists m [F(n, m)]$

$F(n, m_1) \wedge F(n, m_2) \implies \psi(m_1, m_2)$.

Two such formulas $F(n, m)$ and $F'(n, m)$ are considered to be equal as morphisms from $\phi$ to $\psi$ if PA proves them to be equivalent (that is, if PA proves $F(n, m) \implies F'(n, m)$ and $F'(n, m) \implies F(n, m)$).

Given morphisms $F : \phi \to \psi$ and $G: \psi \to \chi$ of this sort, we define their composition in the usual way of composing functions represented as graphs, as $(F \circ G)(n, m) = \exists p [G(n, p) \wedge F(p, m)]$.

This all describes the category $T$, which one can verify is indeed a category and moreso, a category with finite limits.

\TODOinline{Perhaps instead of imposing PERs from the beginning, we start only with the category of RE sets, and then take its ex/lex completion or some such thing. Like so:}

Consider the category $T'$ whose objects are the sigma-1 formulas $\phi(n)$ in the language of PA, and such that a morphism from $\phi(n)$ to $\psi(m)$ is a sigma-1 formula $F(n, m)$ such that $PA$ proves $\forall n, m . F(n, m) \implies (\phi(n) \wedge \psi(m))$ and $\forall n . \phi(n) \implies \exists! m . F(n, m)$. Two such morphisms $F(n, m)$ and $G(n, m)$ are considered equal just in case PA proves $\forall n, m . F(n, m) \biimplies G(n, m)$. Morphisms compose in the obvious way; that is, the composition of $F(n, p)$ with $G(p, m)$ is given by $(G \circ F)(n, m) = \exists p (F(n, p) \wedge G(p, m))$.

This category $T'$ is regular but not exact (that is, not every equivalence relation in $T'$ admits a corresponding quotient). Let $T$ be its ex/reg completion.

\TODOinline{Now, we describe the C inside T which is its internal copy, just by carrying out this exact same construction internal to T, and then we describe the indexed lexfunctor from T to C, which is a little more interesting or takes a little more care. Having this functor be indexed is where the sigma-1 restriction is important.}
\end{construction}

\TODOinline{Observe that we have somewhat distinct concepts of "T = PA Sigma-1, C = ZFC Sigma-1" vs "T = ZFC Sigma-1, C = ZFC Sigma-1", say. Also observe that as concerns ZFC, we can also consider for $C$ not just categories of definable subsets of naturals, but also of definable sets in general, or of definable classes.}

\subsection{Finitely axiomatizable lex theories}
A concept that will often be useful to us in the following.

\begin{definition}
A \defined{finitely axiomatizable lex theory} is a lex theory which, qua lexcategory, can be generated in finitely many steps of the following form, starting from the initial lexcategory: free augmentation with an object, free augmentation with a morphism between existing objects, or freely making two existing parallel morphisms equal. In other words, it can be presented by a finite lex \quote{sketch}. \TODOinline{Word this all better}
\end{definition}

\begin{TODOblock}
Discuss the concept of a lexcategory having initial internal models of ALL finitely axiomatizable lex theories.

As a bit of trivia, observe how this follows simply from having an internal free locally cartesian closed category on one object (verify the details on this; or perhaps from having internal free lex categories and the ability to freely augment internal lex categories with a new cell). Regardless of whether those details work out, conjecture that there are finitely many finitely axiomatizable lex theories such that having internal initial models of those implies having internal initial models of all finitely axiomatizable lex theories, so that the latter is itself a finitely axiomatizable condition.

Relate this also to the concept of arithmetic universes. Conjecturally, being an arithmetic universe is equivalent to something like having free internal models for sketches indexed by finite unions of internal objects (but there seems to be some hesitance in the literature to claim this? Understand that better). At any rate, an arithmetic universe should have internal initial models of all finitely axiomatizable lex theories.

This section basically only exists in order to claim that finitely axiomatizable lex theories which extend the theory of arithmetic universes are automatically examples of the next section.
\end{TODOblock}

\subsection{Theories with free internal models of themselves}
Fix some lex theory extending the theory of strict lexcategories, whose models we shall call \quote{gadgets}. Now suppose every gadget $G$ contains an initial internal gadget $G'$ (in the sense that $G'$ is a gadget internal to the underlying lexcategory $|G|$ of $G$, and for any other gadget $H$ internal to $|G|$, there is a unique $|G|$-internal gadget homomorphism from $G'$ to $H$).

In particular, then, this all applies to the actual initial gadget $G$. Internal to its underlying lexcategory $|G|$, we get a $G'$ as above. Because $G$ is initial, we automatically get a unique gadget-homomorphism $\introN_{G}$ from $G$ to $\Hom_{|G|}(1, G')$ as well. And because $G'$ is an initial $|G|$-internal gadget, we automatically get a unique $|G|$-internal homomorphism $\introN_{G'}$ from $G'$ to $\Hom_{|G'|}(1, G'')$ where $G'' = \introN_{G}[G']$.

This setup is thus a geminal gadget (with axioms 3 and 3' of a geminal gadget automatically satisfied by the uniqueness observations in the previous paragraph).

Indeed, this is the unique way to equip $G$ as a geminal gadget with internal gadget $G'$.

This immediately gives us models of our theory of geminal gadgets. For example, consider the theory of strict elementary toposes with natural numbers objects (let us call this an \defined{NNO-topos}, to make it less of a mouthful). This is indeed a lex theory in a straightforward way; indeed, a finitely axiomatizable lex theory. Furthermore, every NNO-topos has an internal initial model of every finitely axiomatizable lex theory. Thus, in particular, every NNO-topos has an internal initial NNO-topos, and thus, by the above, the initial NNO-topos is equipped as a geminal NNO-topos.

\TODOinline{It is important to observe that the initial NNO-topos is NOT an introspective theory. It is merely a geminal category.}

In the same way, the initial arithmetic universe is equipped as a geminal arithmetic universe. This is the structure discussed by Joyal and others after Joyal (e.g., Dijk and Oldenziel). \TODOinline{Give proper citations here}

\TODOinline{The initial arithmetic universe actually IS an introspective theory, via reasoning about Freyd covers that does not generalize to other examples of this section's phenomena. Discuss this further. This introspective theory extends the geminal arithmetic universe of the previous construction with a suitable natural transformation; furthermore, as its internalization functor preserves AU structure and thus takes its internal initial AU to ITS internal initial AU, the internal geminal AU in this introspective theory is also the internalization of the previous geminal gadget construction. This last part generalizes to any geminal gadget constructed in the previous way: Its internal geminal gadget will also have been constructed by the internalization of the previous construction.}

\begin{TODOblock}
Extend the above discussion of initiality to discuss corresponding introspective theories, beyond just the discussion of geminality. Although, e.g., the initial NNO-topos is not itself an introspective theory, there is nonetheless a corresponding introspective theory of note capturing the initiality properties. Furthermore, even any lex theory that does not automatically come with internal initial models of itself can be freely bumped up to do so, in a suitable sense, and we get a corresponding introspective theory as well.

Specifically, let us say a theory is sigma1esque if the identity lex endofunctor on it is initial with respect to all lex endofunctors on it. (Do we need full initiality? Perhaps weak initiality is all we care about.) Any lex theory $T$ has a free sigma1esque extension, in the sense of a sigma1esque lex theory $T'$ under $T$ with a unique map into any other sigma1esque lex theory under $T$. This $T'$ also comes with a unique map into the global points of any internal initial model of $T$ in any lex category.

If every model of $T$ has an internal initial model of $T$ (as would be the case if the generic model of $T$ has such an internal initial model AND $T$ is given by a finite lex sketch, so that this property is capturable by a lex statement), then the theory $T'$ also contains an internal initial model of $T$, and thus this theory $T'$ comes with a lexfunctor $\introS$ into that internal initial model of $T$. Furthermore, because $T'$ is sigma1esque, it then comes with a natural transformation $\introN$ from its identity functor to the global sections of that $\introS$. This equips said $T'$ as an introspective theory. And, as noted, this $T'$ is modelled by any initial model of $T$ internal to any category.

Thus, for example, if $T$ is the theory of elementary toposes with NNO, we obtain some introspective theory $T'$ extending $T$ which has the initial elementary topos with NNO as a model.
\end{TODOblock}

\begin{TODOblock}
The initial arithmetic universe $U$ probably satisfies a property slightly stronger than being sigma1esque. Given any lex functor $F$ from $U$ to other arithmetic universe $V$, it should be the case that there is a unique natural transformation from $\!_V$ to $F$, where $\!_V$ is the unique AU functor from $U$ to $V$. This is because the comma category $(\!_V / f)$ is an arithmetic universe and its projection to $U$ along both coordinates are AU maps (\TODO. The first half is Artin gluing same as for toposes, I believe. That the projection in the second coordinate is also an AU map is different than from toposes, though, on which the projection is not a logical functor). Those projections are both identity, therefore, and thus our unique AU map from $U$ into the comma category provides a natural transformation from $\!_V$ to $f$.

We now know that we have at least one such functor (the unique AU map from $A$ to $(A/f)$). But how do we see uniqueness (qua functor, not uniqueness qua AU map)? Perhaps a more careful analysis of the gluing construction will tell us that a functor into the comma category is an AU map just in case both its projections are? That is, the forgetful functor from the comma category to the product category creates (i.e., preserves AND reflects) AU structure. [Is this what we want? Is creating structure the same as saying that the forgetful functor composed with any other functor preserves and reflects the property of AU-structure-preservingness in the other functor? Yes, I think so.] This should be relatively straightforward to show for limits and colimits, by general comma category properties, and then also for list objects. \TODO
\end{TODOblock}

\subsection{Models based on well-founded trees/well-founded posets}
There are two flavors of models here: Those which give introspective theories (these come from well-founded trees using a certain size restriction; e.g., considering a model based on the von Neumann universe/cumulative hierarchy), and those which give only locally introspective theories with \Loeb's theorem fixed points (these come from arbitrary well-founded trees; these are related to the models used in guarded recursion theory, but our distinction between the roles of $T$ and $C$ has previously gone unnoticed and allows us to interpret these models as not proving $\lnot \lnot \Box 0$). We discuss the latter construction first, as it is simpler, and a step en route to grasping the former construction.

Previous iterations of this document at this point gave an overly complicated as a way of describing something simple (though still good to understand):

First of all, let $Disc$ be an arbitrary category. This gives rise also to the category $\Psh{Disc}$ of presheaves on $Disc$, which is automatically a lexcategory, and indeed locally cartesian closed. By the observation of \cref{TrivialPreIntrosp}, this yields a locally introspective theory $\langle \Psh{Disc}, \Psh{Disc}/-, \id \rangle$.

Now, let $f : Disc \to Struct$ be an arbitrary functor from $Disc$ into an arbitrary category $Struct$. This induces by composition a functor $f^* : \Psh{Struct} \to \Psh{Disc}$. This $f^*$ preserves pullbacks (as pullbacks are computed pointwise in presheaf categories. Indeed, $f^*$ furthermore preserves all limits, as it has a left adjoint given by left Kan extension). This $f^*$ also has a right adjoint (given by right Kan extension).

By now using \cref{IntrospPullback} with our functor $f^*$ as applied to our first locally introspective theory $\langle \Psh{Disc}, \Psh{Disc}/-, \id \rangle$, we get a second locally introspective theory $\langle \Psh{Struct}, \Psh{Disc}/- \circ f^*, \ldots \rangle$.

This is ALMOST the locally introspective theory we are interested in for Kripke semantics. But it needs to be massaged a bit more, in a manner requiring some further assumptions.

First, a lemmatic construction. Suppose given any arbitrary profunctor $H : X \profuncTo Y$. This $H$ induces by profunctor composition (with profunctors $:1 \profuncTo X$, which correspond to presheaves on $X$) correspondingly an ordinary functor $H \circ - : \Psh{X} \to \Psh{Y}$. Note that this functor $H \circ -$ has a right adjoint (right Kan lift of a profunctor along a profunctor).

If given two such $H_1, H_2$ and a transformation $n : H_1 \to H_2$, this extends also to a transformation $n \circ -$ from $H_1 \circ -$ to $H_2 \circ -$ as ordinary functors $: \Psh{X} \to \Psh{Y}$.

Let us now suppose that $Struct$ (from before) is in fact the free category adding identities to some semicategory $Struct^-$. Then we have a bifunctor $\Hom_{Struct^-} : \op{Struct} \times Struct \to \Set$, as the morphisms of $Struct^-$ are not only closed under composition with each other, but also (trivially) under composition with identities on either side, and thus closed under composition on either side with the morphisms of Struct. 

This bifunctor $\Hom_{Struct^-} : \op{Struct} \times Struct \to \Set$ comes with an inclusion transformation to the bifunctor $\Hom_{Struct} : \op{Struct} \times Struct$. These bifunctors can both be read as profunctors from Struct to Struct; the latter is in fact the identity bifunctor on Struct, and the former is what we will take to be our $H$ as above. The inclusion transformation thus will become an inclusion transformation $i$ from $H \circ -$ to identity as functors $: \Psh{Struct} \to \Psh{Struct}$.

These comprised the last ingredients we needed for proper Kripke semantics for irreflexive frames. Remember, we already had a locally introspective theory $\langle \Psh{Struct}, \Psh{Disc}/- \circ f^*\rangle$ from above. Let us call this $\langle \Psh{Struct}, C \rangle$ for convenience. We now modify it like so using: \cref{IntrospInternalMap}.

% https://q.uiver.app/?q=WzAsMyxbMCwwLCJcXG9we1xcUHNoe1N0cnVjdH19Il0sWzIsMCwiXFxMZXhDYXQiXSxbMSwyLCJcXG9we1xcUHNoe1N0cnVjdH19Il0sWzAsMSwiXFxQc2h7U3RydWN0fS8tIiwwLHsib2Zmc2V0IjotMn1dLFswLDEsIkMiLDIseyJvZmZzZXQiOjJ9XSxbMCwyLCJIIFxcY2lyYyAtIiwyXSxbMiwxLCJDIiwyXSxbMyw0LCIiLDAseyJzaG9ydGVuIjp7InNvdXJjZSI6MjAsInRhcmdldCI6MjB9fV0sWzQsMiwiQyBcXG9we2l9IiwxLHsic2hvcnRlbiI6eyJzb3VyY2UiOjIwfX1dXQ==
\[\begin{tikzcd}
	{\op{\Psh{Struct}}} && \LexCat \\
	\\
	& {\op{\Psh{Struct}}}
	\arrow[""{name=0, anchor=center, inner sep=0}, "{\Psh{Struct}/-}", shift left=2, from=1-1, to=1-3]
	\arrow[""{name=1, anchor=center, inner sep=0}, "C"', shift right=2, from=1-1, to=1-3]
	\arrow["{H \circ -}"', from=1-1, to=3-2]
	\arrow["C"', from=3-2, to=1-3]
	\arrow[shorten <=1pt, shorten >=1pt, Rightarrow, from=0, to=1]
	\arrow["{C \op{i}}"{description}, shorten <=7pt, Rightarrow, from=1, to=3-2]
\end{tikzcd}\]

Keeping in mind that $H \circ - : \Psh{Struct} \to \Psh{Struct}$ has a right adjoint, we may conclude that the result is a locally introspective theory. When we start off taking $Struct^-$ to be a Kripke frame presumed transitive but not reflexive, taking Disc to be the discrete category on the same objects as Struct-, and $f : Disc \to Struct$ to be the inclusion, then the result of the above process is the introspective theory which corresponds to Kripke semantics on $Struct^-$. \TODOinline{Write out in more detail what the construction comes down to and thus showing how it corresponds to traditional Kripke semantics.}

The result will be locally Loeb when the order on the objects of Struct- given by its morphisms is a converse well-founded order. \TODOinline{Expand on this}. We can then impose a suitable size constraint to get it to be fully introspective.

\TODOinline{Clarify the size constraint}.

\TODOinline{The above results immediately imply that the theorems of modal logic which hold for all locally introspective theories are no stronger than those which hold for all transitive Kripke frames, and the theorems which hold for all introspective theories or the theorems which hold in all locally Loeb theories are no stronger than those which hold for all transitive converse well-founded Kripke frames. From this, we can readily conclude that the theorems which hold in all locally introspective theories are K4 and the theorems which hold in all introspective theories or the theorems which hold in all locally Loeb theories are GL. Does the last two of these coinciding help us embed every locally introspective theory into an introspective theory, in the same way as we did for the unconstrained vs constrained presheaf models of GL Kripke frames?}

\TODOinline{LaTeXify the above better}

\TODOinline{Give topos of trees example as well. This is what happens when we take $f$ as the identity functor and $Struct = Disc$ as the free category on some semicategory (in particular, the semicategory of natural numbers with strict reverse ordering). Note that this is an example of an introspective theory in which the functor from the introspective theory to the global aspect of the geminal category is an equivalence of categories (probably an equivalence of geminal categories, even? Thus, what we were calling a GLS-category...). Our $\Box$ operator becomes, on this category, what Birkedal et al call the step operator.}

% Applications (e.g., naive set theory)

% Future work

\section{Appendix}

\printindex

\end{document}
