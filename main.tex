\documentclass{ucbthesis}
%\documentclass[12pt]{article}
% The above line really belongs in the prelude, but Overleaf gets a little weird about it not being in the main .tex file.

% Because ucbthesis uses chapters as the top-level units, not sections, but I developed this using sections as the top-level units (because of the article documentclass), I use the following to quickly transition:
\let\paragraph\subsubsection
\let\subsubsection\subsection
\let\subsection\section
\let\section\chapter

\pagestyle{headings}

\degreesemester{Summer}
\degreeyear{2023}
\degree{Doctor of Philosophy}
\chair{Professor Thomas Scanlon}
\othermembers{Professor Gabriel Goldberg \\ Professor Wesley Holliday \\ Professor Antonio Montalban \\ Professor Michael Shulman}
% For a co-chair who is subordinate to the \chair listed above
% \cochair{Professor Benedict Francis Pope}
% For two co-chairs of equal standing (do not use \chair with this one)
% \cochairs{Professor Richard Francis Sony}{Professor Benedict Francis Pope}
\numberofmembers{5}
% Previous degrees are no longer to be listed on the title page.
% \prevdegrees{B.A. (University of Northern South Dakota at Hoople) 1978 \\
%   M.S. (Ed's School of Quantum Mechanics and Muffler Repair) 1989}
\field{Logic and the Methodology of Science}
% Designated Emphasis -- this is optional, and rare
% \emphasis{Colloidal Telemetry}
% This is optional, and rare
% \jointinstitution{University of Western Maryland}
% This is optional (default is Berkeley)
% \campus{Berkeley}
% The document class really belongs in the prelude, but Overleaf gets a little weird about that not being in the main document.

% Just some made up sizing for now. Can tinker with this later.
% \usepackage[a4paper, margin=1in, headsep=5pt]{geometry}

% Copied from Reid
\usepackage{amssymb,amsmath, amsthm}
\usepackage[utf8]{inputenc} % allow utf-8 input
\usepackage[T1]{fontenc}    % use 8-bit T1 fonts
\usepackage{url}            % simple URL typesetting
\usepackage{booktabs}       % professional-quality tables
\usepackage{amsfonts}       % blackboard math symbols
\usepackage{nicefrac}       % compact symbols for 1/2, etc.
\usepackage{microtype}      % microtypography

% Put in by me, Sridhar
\usepackage{color}          % Color definitions (used for `blue`, `red`)
\usepackage{newpxtext}      % Use a nice text font
\usepackage{enumitem}       % Bulleted lists
  \setenumerate{parsep=10pt} % Futz with the exact value later
\usepackage{comment}        % Multiline comments
\raggedbottom               % No more underfull vbox errors
\usepackage{quiver}         % Commutative diagrams
\usepackage{indentfirst}    % Indent first paragraph in a section
\usepackage{csquotes}       % Quotation marks without needing `` and ''
\usepackage[style=alphabetic]{biblatex}
\bibliography{references}   % Uses the file `references.bib` for bibliography information.
\usepackage{etoolbox}       % Generic LaTeX tools, including nice conditionals. Maybe unnecessary.
\usepackage{underscore}     % Underscore subscripts in text mode. Just useful because I do this in a hurry in TODO notes often.

\renewcommand\qedsymbol{$\blacksquare$} % Change QED symbol to black square so it is not confused with the modal box

% The following are packages that might need to be loaded before hyperref
\usepackage{imakeidx}       % Indexing
  \makeindex[intoc]         % Include index in table of contents
  
% And now, hyperref!
\usepackage[colorlinks=true, linkcolor=blue, citecolor=teal]
  {hyperref}                % Hyper-linked references (note that certain packages need to be loaded before this instead of after this, such as indexing)
  
% And now... Packages that might need to be loaded after hyperref.
\usepackage[conf={restate}]
  {proof-at-the-end}        % Allows for moving proofs of theorems to later in the document, with auto-linking, restating, etc
\usepackage[capitalize, nameinlink]
  {cleveref}                % References named by type and number
\newcommand*{\parensref}[1]{\hyperref[{#1}]{\nameref*{#1} (\cref*{#1})}}              % References named by explicit name and parenthesized type and number
\newcommand*{\reverseparensref}[1]{\hyperref[{#1}]{\cref*{#1} (\nameref*{#1})}}   

% It's not ideal that I have to manually keep track of when to use \cref vs. when to use \nameref or \parensref.
% The following solution is based on https://tex.stackexchange.com/a/66096/231784.
%\newcommand{\magicref}[1]{%
%  \if\vcenter\getrefbykeydefault{#1}{name}{}\vcenter
%    \cref{#1}%
%  \else
%    \nameref{#1}%
%  \fi
%}
\newcommand{\magicref}[1]{%
  \if\vcenter\getrefbykeydefault{#1}{name}{}\vcenter
    \cref{#1}%
  \else
    \parensref{#1}%
  \fi
}
\newcommand{\magicreverseparensref}[1]{%
  \if\vcenter\getrefbykeydefault{#1}{name}{}\vcenter
    \cref{#1}%
  \else
    \reverseparensref{#1}%
  \fi
}

% Listing unused references in the compilation logs
% \usepackage[norefs, nocites]{refcheck}

% We have to use the workaround from https://tex.stackexchange.com/questions/87610/making-refcheck-work-with-cleveref:

%%% Infrastructure    
\makeatletter
\newcommand{\refcheckize}[1]{%
  \expandafter\let\csname @@\string#1\endcsname#1%
  \expandafter\DeclareRobustCommand\csname relax\string#1\endcsname[1]{%
    \csname @@\string#1\endcsname{##1}\wrtusdrf{##1}}%
  \expandafter\let\expandafter#1\csname relax\string#1\endcsname
}
\makeatother
%%%

%%% Now we add the reference commands we want refcheck to be aware of
% \refcheckize{\magicref}

% For subfiles, in a toggle-able way
\newif\ifsubfiles

% For TODO counters, in a toggle-able way
\newif\ifDisplayTODOCount

\newcommand{\filestart}{
    \ifsubfiles
    %\documentclass[./main.tex]{subfiles}
    % Commented out while we don't use subfiles as this appearance of the documentclass command is making Overleaf think of this prelude as an individually compilable file. See https://www.overleaf.com/learn/how-to/Set_Main_Document
    \begin{document}
    \fi
}

\newcommand{\fileend}{
    \ifsubfiles
    \end{document}
    \fi
}
\newcommand{\fileinclude}[1]{
    \setcounter{TODOcounter}{0}
    \ifsubfiles
    \newpage \subfile{#1}
    \else
    \include{#1}
    \fi
    \newcounter{#1Counter}
    \setcounter{#1Counter}{\value{TODOcounter}}
}

% Commands defined by Sridhar for this document

% Remember: \newtheorem{command name}{displayed name}[parent counter] makes a theorem whose counter is subordinate to counter, but \newtheorem{command name}[shared counter]{displayed name} makes a theorem which shares the counter. Also remember that there is automatically a counter available named "section".

% Playing around
\newtheoremstyle{envstyle}
  {\topsep}
  {\topsep}
  {}
  {}
  {\scshape \color{purple}}
  {}
  {.5em}
  {}

%\theoremstyle{plain}
\theoremstyle{envstyle}
\newtheorem{theorem}{Theorem}[section]
\newtheorem{lemma}[theorem]{Lemma}
\newtheorem{construction}[theorem]{Construction}
\newtheorem{example}[theorem]{Example}
\newtheorem{observation}[theorem]{Observation}
\newtheorem{corollary}[theorem]{Corollary}
\newtheorem{corollarytoproof}[theorem]{Corollary (to proof)}
\newtheorem{conjecture}[theorem]{Conjecture}

%\theoremstyle{definition}
\theoremstyle{envstyle}
\newtheorem{definition}[theorem]{Definition}
\newtheorem{convention}[theorem]{Convention}

\newcommand{\defined}[1]{\textbf{#1}\index{#1}} % For indicating the defined term in a definition
\newcommand{\definedManualIndexSort}[2]{\textbf{#1}\index{#2@#1}}

\newcommand{\openNamed}[2]{\begin{#1}[#2]\index{#2}}
\newcommand{\openNamedManualIndexSort}[3]{\begin{#1}[#2]\index{#3@#2}}
\newcommand{\closeNamed}[1]{\end{#1}}

%\theoremstyle{remark}
\theoremstyle{envstyle}
\newtheorem*{remark}{Remark}

% This doesn't quite work yet. The word "Proof" is bolded?
% \renewenvironment{proof}{\paragraph{\scshape \color{blue} Proof:}}{\hfill$\square$}

\newtheoremstyle{redstyle}
  {\topsep}
  {\topsep}
  {\color{red}}
  {}
  {\scshape \color{red}}
  {}
  {.5em}
  {}
  
\theoremstyle{redstyle}
\newtheorem*{TODOblock}{TO DO}

% Warning! LaTeX will swallow the following space stupidly in all of these commands. You may have to place a backslash-space after them instead.
\newcommand{\TODO}{{\color{red} TO DO}}
\newcommand{\TODOinline}[1]{\TODO\ {\color{red} #1}}
\newcommand{\Goedel}{G\"odel}
\newcommand{\Godel}{\Goedel}
\newcommand{\Loeb}{L\"ob}
\newcommand{\Lob}{\Loeb}

% Using \operatorname for constants is maybe a little wrong, as it suggests a closer semantic association to the right. For now, we just use \mathrm.
\newcommand{\const}[1]{\mathrm{#1}} % Typesetting named constants
\newcommand{\constcat}[1]{\const{#1}} % Typesetting named constant categories
\newcommand{\arrowcat}[1]{\operatorname{Arrow}(#1)} % Typesetting arrow categories
\newcommand{\Hom}{\operatorname{Hom}}
\newcommand{\Iso}{\operatorname{Iso}}
\newcommand{\Nat}{\operatorname{Nat}}
\newcommand{\dom}{\operatorname{dom}}
\newcommand{\cod}{\operatorname{cod}}
\newcommand{\op}[1]{#1^{\mathrm{op}}} % If we include the `physics` package, then this needs to be renewcommand, as \op is defined for \outerproduct by the `physics` package.
\newcommand{\Ob}{\operatorname{Ob}}
\newcommand{\Mor}{\operatorname{Mor}}
\newcommand{\Set}{\constcat{Set}}
\newcommand{\Cat}{\constcat{Cat}}
\newcommand{\LexCat}{\constcat{LexCat}}
\newcommand{\id}{\const{id}}
\newcommand{\unique}{!}
\newcommand{\iso}{\simeq}
\newcommand{\Psh}[1]{\operatorname{Psh}(#1)}
\newcommand{\Indexed}[1]{\operatorname{Indexed}(#1)}

\newcommand{\introF}{\mathcal{F}}
\newcommand{\introS}{\mathcal{S}}
\newcommand{\introN}{\mathcal{N}}
\newcommand{\glQuote}{\mathcal{J}}

\newcommand{\GLCatTheory}{\mathrm{Th}(\mathrm{GL})}

\newcommand{\outertheory}{X}
\newcommand{\innertheory}{Y}

\renewcommand{\implies}{\Rightarrow}
\newcommand{\biimplies}{\iff}

\newcommand{\code}[1]{\lceil #1 \rceil}

\newcommand{\pullAlong}[1]{#1^*}

% Should maybe use the word "representable" for this?
% Hack to define a command name whose name has to end with a 
% slash, thus preventing spacing problems. (Otherwise, space 
% automatically gets swallowed after invoking the command.)
% TODO: Use this same hack for Goedel and Loeb.
\def\repsmall/{repsmall}

\newcommand{\adjointTo}{\dashv}

\subfilesfalse

% \ifsubfiles
% \usepackage{subfiles}
% The above line is following the instructions from https://www.overleaf.com/learn/latex/Multi-file_LaTeX_projects
% \fi
% I've commented this out for now on the off-chance Overleaf is doing some weird parsing to detect if subfiles is used.

\title{Introspective Theories and Geminal Categories}

\author{Sridhar Ramesh}

\begin{document}
\maketitle

\copyrightpage

\begin{abstract}
In provability logic, a key principle is \Loeb/'s theorem, stating that if the provability of $P$ provably entails $P$, then $P$ itself is provable (in modal notation, $\Box P \vdash P$ has as a consequence $\vdash P$). This was first discovered in the follow-up work on \Goedel/'s incompleteness theorems, with \Goedel/'s results viewable as following from \Loeb/'s theorem. Later, it was also seen that the same formal pattern of \Loeb/'s theorem described certain fixed point constructions studied under the name of \quote{guarded recursion}.

The aim of these notes is to draw attention to a certain simple class of category-theoretic structures which serve as an abstract environment for deriving \Loeb/'s theorem and such fixed point constructions, allowing for a vastly generalized and unified understanding of the scope of applicability of such constructions. These are the structures we call \quote{introspective theories}.

This very minimal categorical definition nontrivially entails \Loeb/'s theorem and guarded recursion at both the term and type level. We also demonstrate how this abstraction offers a clean unification of the interpretation of the \Godel/-\Lob/ incompleteness theorems in traditional logic or via arithmetic universes a la Joyal, along with the interpretation by Birkedal et al of guarded recursion in presheaves over well-founded orders, along with the distinct classical interpretation of the GL modal logic in well-founded transitive Kripke frames.

We also explore free instances of our structure, which turn out to admit a tractable explicit description. The free introspective theory is what we call \quote{the theory of geminal categories}, and we explore also some further illuminating relationships between the concepts of introspective theories and geminal categories.

% It should perhaps be noted that the core of the work in this document was originally developed during my time in graduate school from 2006 to 2013, but I did not write it up suitably at that time.
\end{abstract}

\begin{frontmatter}

\begin{dedication}
\null\vfil
\begin{center}
To Amma, who showed me why adding \textit{this} many \textit{that}s is the same as adding \textit{that} many \textit{this}es,
\\\vspace{12pt}

and Appa, who showed me what those $\sin$ and $\cos$ buttons on calculators mean.

\end{center}
\vfil\null
\end{dedication}

\tableofcontents

\begin{acknowledgements}
\end{acknowledgements}

\end{frontmatter}

%\setcounter{section}{-1}
\setcounter{chapter}{-1}
\fileinclude{Introduction}

\fileinclude{Preliminaries}

\fileinclude{IntrospectiveTheory}

\fileinclude{Modal}

\fileinclude{Loeb}

\fileinclude{GLCategory}

\fileinclude{Models}

% \fileinclude{Future}

\printbibliography

\printindex

% \fileinclude{InProgress}

% {\color{red}
\section{TO DO Counters}
Introduction: \theIntroductionCounter

Preliminaries: \thePreliminariesCounter

Introspective Theories: \theIntrospectiveTheoryCounter

Modal logic: \theModalCounter

\Loeb/'s theorem: \theLoebCounter

Geminal categories: \theGLCategoryCounter

Examples in the wild: \theModelsCounter

% Future/miscellany: \theFutureCounter
}

\end{document}
