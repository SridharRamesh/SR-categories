\filestart

\section{Scratch work in progress}

\subsection{Unorganized draft category-theoretic lemmas}

\subsubsection{Initial algebra constructions}
\TODO

\subsubsection{Theories, models, functors, etc}
\begin{observation}\label{ModelTerminology}
We need to standardize our terminology on theories, models, and internal models. Is a lexfunctor from $T$ to $S$ an internal model of $T$ in $S$? Or is a lexfunctor from $T$ to the globalization of an internal category in $S$ an internal model of $T$ in $S$? Is a model always Set-valued? Etc. Let us do a Ctrl+F for "model" to make sure we are not confusing on this point. It may be best to speak scrupulously of lexcategories and lexfunctors at all times, and not of theories or models, except in the fixed phrase "introspective theory".
\end{observation}

\TODOinline{
Introduce terminology for "the walking X" meaning "the free lexcategory with an internal X". "Internal X" always means "Map into me from the lex theory of Xes", and NOT "Internal lexcategory in me, and map into THAT from the lex theory of Xes". The latter should be called doubly internal. The theory of Xes has an internal X. Hm. Maybe we need good terminology for "the theory of blahs" vs "a blah", when blah is itself a lexcategory-extending notion.

Let's say \included/ models of theory T within lexcategory S refers to lexfunctors from the lexcategory corresponding to T, to S. While \interior/ models of T within S refers to the case where S has a designated internal lexcategory C, and we also have a lexfunctor from T to the globalization of C. In other words, \interior/ is doubly internal.

The weird thing is that the identity lexfunctor on a category also corresponds to the generic internal model, even though the identity seems like staying on the same level, while internal sounds like going down one level. Hence why actually going down a level with \interior/ corresponds to doubly internal.}

\subsubsection{Constructing initializers}
So, for example, let us try constructing, for a lexcategory $T$, the initial lexcategory with an internal initial model of $T$. We find some quasi-equational theory $Th(T)$ presenting $T$. This theory is k-sized, so we work from now on in k-ary quasi-equational theories. We now make the k-ary quasi-equational theory of a strict lexcategory with an internal initial model of $Th(T)$, and apply the initial algebra theorem. We get an initial strict lexcategory with an internal initial model of $Th(T)$; call this $S$.

Suppose we have three quasi-equational theories:
Th(T)
Th(Init-T')
Th(D)

These present strict lexcategories
T'
Init-T'
D'

These present lexcategories
T
Init-T
D

Th(T) is chosen to present T, with T' arising from this.
Th(Init-T') is chosen to present Init-T' as the free strict lexcategory with an internal initial model of Th(T). Init-T then arises from this.
Th(D) is chosen to present D, with D' arising from this.

For any initial model of Th(T) in D', we get a unique map from Init-T' to D' which restricts to the corresponding map from T' to D'.

Given two strict maps f, g from Init-T' to D' which both take T' to initial models in D', we want a unique transform from f to g. Transforms from f to g correspond by lemma 55 (even without the strictness assumption on f, g) to transforms between the corresponding models of Th(Init-T') in D'. We want [TODO] to show that these correspond to transforms between the corresponding models of Th(T) in D'. If we can do that, then since those models are initial, we get a unique map between them, as desired.

\TODOinline{Perhaps we should mark with asterisks those theorems which are specific to $\Set$ and perhaps use Choice vs those theorems which are meant to internalize broadly}

\subsubsection{Cartesian closure and \Loeb/'s theorem}
\begin{theorem}
Suppose $\langle T, C \rangle$ is a locally introspective simple theory (i.e., locally introspective finite product theory). Furthermore, suppose we have an object $0$ of $T$ such that all exponentials $0^t$ exist in $T$ (as in when $T$ is cartesian closed), and furthermore, there is a morphism in $T$ of type $\Box ((\Box 0) \implies 0) \vdash \Box 0$ (a \Loeb/ morphism). Let us refer to $0^X$ as $\neg X$. We then get a morphism in $T$ of type $1 \vdash \neg \neg \Box 0$; equivalently, a morphism of type $\neg \Box 0 \vdash 0$.
\end{theorem}
\begin{proof}
In $T$, we have $(\neg \Box 0) \times (\Box 0) \vdash 0$. There is also a general principle that if $T$ contains a morphism $X \times Y \vdash Z$, then $T$ also contains a morphism $X \vdash \Box (Y \implies Z)$ (by using $\introS$ to get $\Box X \vdash \Box(Y \implies Z)$, and then using $\introN$ to get $X \vdash \Box X$ and composing these).

Applying that general principle to this particular morphism, we find in $T$ a morphism $\neg \Box 0 \vdash \Box (\Box 0) \implies 0$. The right-hand side here combines with our \Loeb/ morphism; the result of this composition is $\neg \Box 0 \vdash \Box 0$. And from here we quickly get $\neg \Box 0 \vdash 0$, aka, $1 \vdash \neg \neg \Box 0$.
\end{proof}

(In fact, for an introspective theory, we get this more easily: simply consider the aspect of $T$ defined over $(\Box 0) \implies 0$.)

The above shows the undesirability of using exponentiation in a cartesian closed $T$ to model ordinary implication, when using introspective theories to model ordinary provability logic, as this would imply a certain degree of incompatibility with Boolean logic. Specifically, in a Boolean context, when $0$ is actual falsehood so that $\neg \neg$ is supposed to be identity, the above would give us the rule $1 \vdash \Box 0$, asserting provable falsehood.

The appropriate way to model provability logic complete with an implication operator is not to take $T$ to be cartesian closed, but rather, to take only $C$ to be an indexed cartesian closed category. Note that even if $T$ happens to be cartesian closed as well, we will in general not have that $\introF$ must preserve exponentials; that is, exponentials computed via $T$'s cartesian closed structure may well differ from exponentials computed via $C$'s cartesian closed structure (especially \TODOinline{Only?} if $C$ has objects not in the range of $\introF$). It is the latter which are most relevant for ordinary provability logic. Thus, although $T$ being cartesian closed will give us $\introS(\neg_T \Box 0_T) \vdash_C \introS(0_T)$, it will not give us $\neg_C \introS(\Box 0_T) \vdash_C \introS(0_T)$. We will see this explicitly in model \TODO and model \TODO.

\subsubsection{Cartesian closed categories and locally introspective theories}
\begin{theorem}
We observe here that a locally introspective simple (i.e., finite product) theory in which $T$ and $C$ are cartesian closed and $\introF$ preserves cartesian closed structure and is essentially surjective on objects is the same as a cartesian closed theory $T$ along with a finite product preserving operator $\Box_T$ and natural transformation $\introN : \id_T \to \Box_T$.
\end{theorem}
\begin{proof}
As a cartesian closed category, $T$ is enriched over itself. A finite product preserving functor from $T_1$ to $T_2$ turns enrichments over $T_1$ into enrichments over $T_2$; using this, we can construe $T$ instead as a category $C$ enriched over $T$, where the objects of $C$ are those of $T$ but $\Hom_C(t, s) = \Box_T \Hom_T(t, s) = \Box_T s^t$.

A category enriched over $T$ (with terminal object) is the same as a locally \repsmall/ $T$-indexed category such that every object in $C(t)$ is the pullback of some object in $C(1)$. Thus, the above has produced $C$ as a locally \repsmall/ $T$-indexed category. Since both $\Box$ and $-^t$ are product preserving endofunctors of $T$, it's easy to see that $\Hom_C(t, s_1 \times \ldots \times s_n) = \Hom_C(t, s_1) \times \ldots \times \Hom_C(t, s_n)$ so that $C$ furthermore has finite products. (We can also see that $C$ has exponentials, from \TODO).

For $\introS$, we send each object to itself and send each morphism $m: t \to s$ in $T$ (equivalently, $: 1 \to s^t$) to the corresponding value in $: 1 \to \Box(t^s)$ given by applying our natural transformation from identity to $\Box$. It is readily verified this is functorial and product-preserving (\TODO).

The required $\introN : t \to \Hom_C(1, \introS(t)) = \Box_T t$ is given by the supplied natural transformation as well. This completes this direction of the proof.

Put more cleanly:

% https://q.uiver.app/?q=WzAsNCxbMSwwLCJcXG9we1R9Il0sWzIsMCwiRmluUHJvZENhdCJdLFsxLDEsIlxcb3B7VH0iXSxbMCwwLCJcXG9we1R9Il0sWzAsMSwiVC8vLSJdLFsyLDEsIlxcTGFuX3tcXEJveH0oVC8vLSkiLDJdLFswLDIsIlxcQm94IiwyXSxbMywyLCJcXGlkIiwyXSxbMywwLCJcXGlkIl0sWzQsMiwiIiwwLHsic2hvcnRlbiI6eyJzb3VyY2UiOjIwLCJ0YXJnZXQiOjIwfX1dLFswLDcsIiIsMix7InNob3J0ZW4iOnsidGFyZ2V0IjoyMH19XV0=
\[\begin{tikzcd}
	{\op{T}} & {\op{T}} & FinProdCat \\
	& {\op{T}}
	\arrow[""{name=0, anchor=center, inner sep=0}, "{T//-}", from=1-2, to=1-3]
	\arrow["{\Lan_{\Box}(T//-)}"', from=2-2, to=1-3]
	\arrow["\Box"', from=1-2, to=2-2]
	\arrow[""{name=1, anchor=center, inner sep=0}, "\id"', from=1-1, to=2-2]
	\arrow["\id", from=1-1, to=1-2]
	\arrow[shorten <=4pt, shorten >=4pt, Rightarrow, from=0, to=2-2]
	\arrow[shorten >=2pt, Rightarrow, from=1-2, to=1]
\end{tikzcd}\]

Plus the observation at https://sridharramesh.github.io/HowSridharThinks/math/IndexedEnrichedInternalCategories.html to ensure that the left Kan extension within $\Cat$ really does produce a category with finite products. Plus the verification that $\Hom_C(A, B) = \Box_T (B^A)$.

In the other direction (introspective theory with eso $\introF$ is cartesian closed $T$ plus operator), \TODO.
\end{proof}
Similarly, some analogue of the above for locally introspective finite limit theories? \TODO

\begin{TODOblock}
Similarly, some analogue of both of the above for geminal categories? The simplest observation is that, from any cartesian closed category C' with endo-fp-functor Box s.t. we have a natural transformation from Box to $\Box^2$ s.t. the two induced paths from Box to $\Box^3$ are equal, we can construct an introspective finite product theory T such that the objects of $T$ are finite products of objects of the form $\Hom_C(A, B)$ for objects $A$ and $B$ in $C(1)$, while the objects of $C$ at every aspect are the objects of $C'$. And of course $\introS(\Hom_C(A, B)) = \Box(B^A)$, while $\introN$ corresponds to our Box to $\Box^2$ operator. Maybe this is simpler put in terms of the observation as to what a geminal (simple/fp theory) cartesian closed category on given objects is.
\end{TODOblock}

\subsection{Cartesian closure and box}
We should be able to interpret $\Hom_C(S(a), S(b))$ as $\Box_T(a \implies b)$ for $a, b \in T$ with some analogy to how this works on $\Psh{C}$. (Search for where we say bifunctor in this document). Note that if $T$ is cartesian closed, this may be distinct from $\Box_T (B^A)$ using the cartesian closed structure of $T$ to define $B^A$.

If T is cartesian closed, we have $Hom_C(1, S(a -> b)) |- Hom_C(S(a), S(b))$, but not the other way around.

\subsection{Soundnesses}
\TODO

A global soundness is something like, given an introspective theory T, we can ask for a lexfunctor from T to Set which sends all of $\introN$'s components to isomorphisms.

This might be the same as asking of the lexfunctor (construed as a model M of T with an internal model M' and a homomorphism N from M to $Hom_M(1, M')$) to be such that this homomorphism $N : M \to \Hom_M(1, M')$ is an isomorphism.

-----

Note that we have a lexfunctor from T to (N -> T), because we have a functor from N to (T -lex-> T), which sends n to $Box^n$ and which sends the map from n to n + 1 to $\introN_{Box^n}$. (We want this map, not $\Box^n (\introN_{id})$)

Therefore, for any model M : T -Lex-> Set, we get also by composition a lexfunctor from T to (N -> Set). At each n, this specializes to the (globalization of the) geminal category n-tuply internal to the original model, and at each n to n + 1 map, this is the self-internalizing homomorphism of that geminal category (NOT the action of the original geminal category's self-internalizer on its internal categories).

But we also have a lexfunctor from (N -> Set) to Set given by taking colimits, since filtered colimits commute with finite limits.

Composing these, we get a lexfunctor from T to Set which is the colimit of all those models and homomorphisms. It should be possible to check directly that this colimit model is such that each $\introN$ component within it is an isomorphism. Thus, it has a soundness in the above sense. (I am no longer confident this works, I think it still falls apart on the failure of well-pointedness.)

----

The initial AU and the related Sigma1 models has such a super strong soundness: It is an introspective theory, with its C furthermore equipped as an internal introspective theory, and it is equivalent qua introspective theory to the globalization of this internal introspective theory (though of course it does not internally claim that C is equivalent to the globalization of C', thanks to G2IT).

\TODOinline{Returning to this months later, I think where it stood was like so: The above isn't exactly right and the problem is the well-pointedness failure. See https://ncatlab.org/nlab/show/transfinite+construction+of+free+algebras}

\subsection{Convenient terminology}
\begin{TODOblock}
Some note about how we will also make use in diagrams in this chapter liberally of the identification of $P(c)$ with $\Hom(c, P)$, when $P$ is a presheaf. If one likes, this can be seen as invoking the Yoneda lemma to draw diagrams in $\Psh{C}$ (although invoking the Yoneda lemma is actually a bit overkill for the mere fact that we can draw diagrams of presheaf elements and presheaf actions in this way, which is simply working within the appropriate collage, aka cocomma category).

We're a bit glib in all our writing when we swap between thinking of $x$ as an element in $X$, vs a map from $1$ to $X$, or similarly in swapping between a presheaf $\Omega^X$ and values in $\Omega(X)$ and morphisms from $X$ to $\Omega$, etc. Perhaps we should introduce some explicit notation for these kinds of conversions, to make everything perfectly clear.

Can move these notes to the Preliminaries.
\end{TODOblock}

\subsection{Fixed point interchange}
\begin{theorem}
Suppose $f : A \to B$ and $g : B \to A$. Then $f$ and $g$ comprise the two halves of a one-to-one correspondence between fixed points of $g \circ f$ and fixed points of $f \circ g$.
\end{theorem}

\subsection{Another box lemma}
\begin{theorem}\label{BoxMatchesN}
For a locally introspective theory (or finite product theory) $\langle T, C, \introS, \introN \rangle$, and a globally defined element $x$ of $P(1)$ for $P \in \Psh{C}$, we have that $\introN_{\Omega}(x)$ is the globally defined element of $(\Box P)(1)$ given by applying $\Box_{\Psh{C}}$ to $x$.
\end{theorem}
\begin{proof}
\TODO
\end{proof}

\subsection{Initiality with respect to Set itself}
\begin{observation}\label{InitialWrtSet}
In our metatheory, we have access to the following principle: If M is the term model of some finitely axiomatized lex theory of gadgets, and Set is also a gadget (but a large one), then M not only is initial with respect to set-sized gadgets but also there is a unique homomorphism from M to Set.
\end{observation}
\begin{proof}
The proof is by the exact same proof that shows M's initiality with respect to set-sized gadgets.

Alternatively, the proof can be carried out like so: Firstly, for existence of a map from M to Set, we take the finitely many finitary operations of our theory and note that the hull definable from these within Set describes a small (indeed, countable) subgadget of Set. M will have by initiality a homomorphism into this subgadget, and thus into Set itself. As for uniqueness, consider any two homomorphisms from M to Set. Again, their ranges will be small subgadgets of Set, and we can take the union of those ranges, close under the operations of our theory, and find some other small subgadget of Set containing them both. M will have a unique homomorphism into this enveloping subgadget, and thus the parallel homomorphisms of M must have been equal.
\end{proof}

Be cautioned, however, that the reasoning above only works in our metatheory, with typical principles available to us like the ability to reason about subcollections of Set. Analogous reasoning can fail internally; e.g., every topos T with NNO is such that its self-indexing T/- is an indexed topos with NNO, and such that it has an initial internal topos with NNO T' constructed as a term model, and yet there need be no topos-with-NNO homomorphism from T' to T/-. (Indeed, in the initial topos-with-NNO, there will not be such a homomorphism, by \Goedel/'s second incompleteness theorem considerations). Similarly for \quote{arithmetic universe} in place of \quote{topos with NNO}.

\TODOinline{Move the following somewhere where it belongs}
\begin{theorem}
The global sections functor $\Hom_{\IAU}(1, -)$ is the unique arithmetic functor from the initial arithmetic universe $\IAU$ to $\Set$.
\end{theorem}
\begin{proof}
A unique arithmetic functor $!$ from IAU to $\Set$ is known to exist by the initiality of IAU (keeping in mind \magicref{InitialWrtSet}). What remains is only to show that this $!$ is the same as the global sections functor. By \magicref{TermModelIsInitialForLex}, we know that the global sections functor is initial among lexfunctors from $\IAU$ to $\Set$. But by \magicref{AUStrongSigma1esque}, we know that $!$ is also initial among these. Thus, $!$ and the global sections functor are isomorphic (indeed, uniquely isomorphic), completing the proof.
\end{proof}

With this last theorem, we must be careful. As it invoked \magicref{InitialWrtSet}, its reasoning does not internalize. In particular, we do NOT know internal to $\IAU$ that the global sections functor from $\IAU'$ to the self-indexing $\IAU/-$ is arithmetic, or even that it preserves the initial object (this would violate \Goedel/'s second incompleteness theorem).

\subsection{Uniqueness of model constructions}
\begin{lemma}
Let $T$ be a lexcategory with a natural numbers object, such that every object of $T$ admits a monic map into the natural numbers object. Let $\Box$ be a lex endfunctor on $T$. There is at most one natural transformation from $\id_T$ to $\Box$.
\end{lemma}
\begin{proof}
Let $\nat$ be the natural numbers object of $T$, with zero map $z : 1 \to \nat$ and successor map $s : \nat \to \nat$. Note that $\Box z : 1 \to \Box \nat$ and $\Box s : \Box \nat \to \Box \nat$. Thus, by the universal property of $\nat$, there is a unique map $h : \nat \to \Box \nat$ such that $h \circ z = \Box z$ and $h \circ s = (\Box s) \circ h$.

Now we will show that each of the components of a natural transformation $\eta : \id_T \to \Box$ is uniquely determined (if such a natural transformation exists at all!).

First of all, by naturality of $\eta$ with respect to the morphisms $z$ and $s$, we see that any natural transformation $\eta : \id_T \to \Box$ would also be such that $\eta_{\nat} \circ z = \Box z$ and $\eta_{\nat} \circ s = (\Box s) \circ \eta_{\nat}$. Thus, we must have that $\eta_{\nat} = h$.

Furthermore, for any arbitrary object $t$ of $T$, consider some monic map $m : t \to \nat$. By naturality of $\eta$ with respect to $m$, we see that $(\Box m) \circ \eta_t = \eta_{\nat} \circ m$. Since $\Box m$ is monic (since $\Box$ as a lexfunctor preserves monicity), and since $\eta_{\nat}$ was already uniquely determined, this uniquely determines $\eta_t$ as well.
\end{proof}

\subsection{Geminal gadgets more generally}
A geminal gadget internal to lexcategory $C_0$ consists (among other things) of a sequence $G_1, C_1, G_2, C_2, G_3, C_3, \ldots$, where for $i \geq 0$, each $G_{i + 1}$ is a gadget internal to $C_i$ and each $C_{i + 1}$ is the global aspect of the underlying (strict) lexcategory of $G_{i + 1}$.

Note that $\Hom_{C_{i + 1}}(1, -)$ can be seen as a lexfunctor from $C_{i + 1}$ to $C_i$ (since $C_{i + 1}$ is the global aspect of a lexcategory internal to $C_i$). Call this global sections lexfunctor $\Gamma_i : C_{i + 1} \to C_i$.

We now also require a sequence of internal gadget-homomorphisms $F_1, F_2, F_3, \ldots$, where each $F_i : G_i \to \Gamma_i(G_{i + 1})$ is internal to $C_i$.

Finally, we impose some equations. We require that each $F_i$ takes $G_j$ to $G_{j + 1}$ and takes $F_j$ to $F_{j + 1}$ for $j > i$. Furthermore, we demand that each $F_i F_i = F_{i + 1} F_i$ in a suitable sense.

Note that all structure subscripted $j$ can be seen as $(j - i)$-tuply internal to $C_i$, for $j > i$. In particular, all of this structure is indeed internal to $C_0$.

Note furthermore that restricting attention to the sequence of $G_j$, $C_j$, and $F_j$ for $j > 1$ yields an instance of this structure internal to $C_1$.

Note even furthermore that $F_1$ acts as an internal homomorphism of this structure.

This all describes a geminal gadget internal to $C_0$.

\begin{theorem}
To uniquely generate all the rest of the aforementioned structure, it suffices only to be given $G_i$, $F_i$, and the equation $F_i F_i = F_{i + 1} F_i$ for $i \in \{1, 2\}$.
\end{theorem}
\begin{proof}
Define $G_n$ as $F_1^{n - 1}(G_2)$, and similarly define $F_n$ as $F_1^{n - 1}(F_2)$. \TODOinline{Clarify the notation, that $F_1(F_2)$ for example is not a composition but rather an application of a functor to the diagram specifying an internal functor. Can use square brackets instead.}

It is now automatically the case that $F_1$ takes $G_j$ to $G_{j + 1}$, and takes $F_j$ to $F_{j + 1}$.

Furthermore, we get the equation that $F_i F_i = F_{i + 1} F_i$, by applying $F_1^{n - 1}$ to the instance of this equation at $i = 2$.

What remains is only to see that each $F_i$ also takes $G_j$ to $G_{j + 1}$ and takes $F_j$ to $F_{j + 1}$, for $j > i$.

We prove this by induction on $i$. We have above established this for $i = 1$ as our base case. As for the inductive step, suppose we know this already holds for $i$. Then $F_{i + 1}[G_j] = F_{i + 1} [F_i [G_{j - 1}]] = (F_{i + 1} \circ F_i)[G_{j - 1}] = (F_i \circ F_i) = F_i [F_i [G_{j - 1}]] = F_i [G_j] = G_{j + 1}$. And similarly with $F$ in place of $G$ as well.
\end{proof}

\subsection{What follows is stuff to be removed, as it has been rewritten}

\subsection{Models based on \texorpdfstring{$\Sigma_1$}{Sigma-1} or arbitrary extensions of PA, or ZFC, or etc}
\TODOinline{I will write this section in a sloppy way for now and then improve it later.}

This section reviews and builds upon the construction previously seen at \cref{SigmaModelSimple}.

\begin{construction}\label{Sigma1Model}

Consider a sigma-1 theory $\tau$ extending PA (or ZFC, or any such thing), in the sense of an extension whose axioms are computably enumerable. Actually, for now, let's just consider PA simpliciter.

\TODOinline{It probably isn't easy to pin down in a clean way exactly the minimal kind of system in which this goes through, but it could be useful to name some weak subsystems of arithmetic in which it goes through. In particular, we should not expect this to go through in Robinson's Arithmetic Q which lacks induction entirely, but we should expect it to still work in systems that just have induction for $\Sigma_1$ formulae).}

Consider the category $T$ whose objects are the sigma-1 formulas $\phi(n, m)$ in the language of PA which define binary relations on the natural numbers which PA proves to be partial equivalence relations (i.e., symmetric and transitive). Given any two such formulas $\phi(n, m)$ and $\psi(n, m)$, a morphism in $T$ from $\phi$ to $\psi$ is a sigma-1 formula $F(n, m)$ on the natural numbers which PA proves to correspond to the graph of a function between the subquotients of $\mathbb{N}$ corresponding to $\phi$ and to $\psi$, respectively. That is, such that PA proves the universal closures of the following:

$F(n, m) \implies \phi(n, n) \wedge \psi(m, m)$

$\phi(n_1, n_2) \wedge \psi(m_1, m_2) \wedge F(n_1, m_1) \implies F(n_2, m_2)$

$\phi(n, n) \implies \exists m [F(n, m)]$

$F(n, m_1) \wedge F(n, m_2) \implies \psi(m_1, m_2)$.

Two such formulas $F(n, m)$ and $F'(n, m)$ are considered to be equal as morphisms from $\phi$ to $\psi$ if PA proves them to be equivalent (that is, if PA proves $F(n, m) \implies F'(n, m)$ and $F'(n, m) \implies F(n, m)$).

Given morphisms $F : \phi \to \psi$ and $G: \psi \to \chi$ of this sort, we define their composition in the usual way of composing functions represented as graphs, as $(F \circ G)(n, m) = \exists p [G(n, p) \wedge F(p, m)]$.

This all describes the category $T$, which one can verify is indeed a category and moreso, a category with finite limits.

\TODOinline{Perhaps instead of imposing PERs from the beginning, we start only with the category of RE sets, and then take its ex/lex completion or some such thing. Like so:}

Consider the category $T'$ whose objects are the sigma-1 formulas $\phi(n)$ in the language of PA, and such that a morphism from $\phi(n)$ to $\psi(m)$ is a sigma-1 formula $F(n, m)$ such that $PA$ proves $\forall n, m . F(n, m) \implies (\phi(n) \wedge \psi(m))$ and $\forall n . \phi(n) \implies \exists! m . F(n, m)$. Two such morphisms $F(n, m)$ and $G(n, m)$ are considered equal just in case PA proves $\forall n, m . F(n, m) \biimplies G(n, m)$. Morphisms compose in the obvious way; that is, the composition of $F(n, p)$ with $G(p, m)$ is given by $(G \circ F)(n, m) = \exists p (F(n, p) \wedge G(p, m))$.

This category $T'$ is regular but not exact (that is, not every equivalence relation in $T'$ admits a corresponding quotient). Let $T$ be its ex/reg completion.

\TODOinline{Now, we describe the C inside T which is its internal copy, just by carrying out this exact same construction internal to T, and then we describe the indexed lexfunctor from T to C, which is a little more interesting or takes a little more care. Having this functor be indexed is where the sigma-1 restriction is important.}
\end{construction}

\TODOinline{Observe that we have somewhat distinct concepts of "T = PA Sigma-1, C = ZFC Sigma-1" vs "T = ZFC Sigma-1, C = ZFC Sigma-1", say. Also observe that as concerns ZFC, we can also consider for $C$ not just categories of definable subsets of naturals, but also of definable sets in general, or of definable classes.}

\subsection{Finitely axiomatizable lex theories}
A concept that will often be useful to us in the following.

\begin{definition}
A \defined{finitely axiomatizable lex theory} is a lexcategory $W$ which can be generated in finitely many steps of the following form, starting from the initial lexcategory: free augmentation (qua lexcategory) with an object, free augmentation (qua lexcategory) with a morphism between two existing objects, or freely (qua lexcategory) making two existing parallel morphisms equal. In other words, it can be presented by a finite lex \quote{sketch}.
\end{definition}

If $W$ is a finitely axiomatizable lex theory, and $C$ is an internal lexcategory within a lexcategory $T$, then the set of lexfunctors from $W$ to $C$ is $T$-\repsmall/. This can be seen readily from the inductive definition of finite axiomatizability: 

\begin{TODOblock}
Discuss the concept of a lexcategory having initial internal models of ALL finitely axiomatizable lex theories.

As a bit of trivia, observe how this follows simply from having an internal free locally cartesian closed category on one object (verify the details on this; or perhaps from having internal free lex categories and the ability to freely augment internal lex categories with a new cell). Regardless of whether those details work out, conjecture that there are finitely many finitely axiomatizable lex theories such that having internal initial models of those implies having internal initial models of all finitely axiomatizable lex theories, so that the latter is itself a finitely axiomatizable condition.

Relate this also to the concept of arithmetic universes. Conjecturally, being an arithmetic universe is equivalent to something like having free internal models for sketches indexed by finite unions of internal objects (but there seems to be some hesitance in the literature to claim this? Understand that better). At any rate, an arithmetic universe should have internal initial models of all finitely axiomatizable lex theories.

This section basically only exists in order to claim that finitely axiomatizable lex theories which extend the theory of arithmetic universes are automatically examples of the next section.
\end{TODOblock}

\subsection{Self-initializing finitely axiomatizable theories}
\TODOinline{Get rid of this section}

\magicref{InitoGeminalYieldsGeminal} immediately gives us many nontrivial examples of geminal categories. For example, many theories $T$ simultaneously satisfy the following two properties:

A) $T$ is a finitely axiomatizable lex theory extending the theory of lex categories.

B) Every model of $T$ contains initial internal models of every finitely axiomatizable lex theory.

Any such $T$ will of course be \initogeminal/.

For example, consider the theory of strict elementary toposes with natural numbers objects (let us call this an \defined{NNO-topos}, to make it less of a mouthful). This is indeed a finitely axiomatizable lex theory extending the theory of lexcategories \TODOinline{Maybe make up a name for lex theories extending the theory of lexcategories, since we use them often, need them to define our concept of truly internal models, etc}. Furthermore, it satisfies the property B just noted:

\begin{theorem}\label{NNOToposIsInitoGeminal}
Every NNO-topos has an initial internal model of every finitely axiomatizable lex theory. Such initial internal models are furthermore preserved by functors preserving NNO-topos structure.
\end{theorem}
\begin{proof}
\TODOinline{Mention that terms can be partially defined (that is, not all terms denote), in the following}
This is simply by carrying out in its internal logic the ordinary mathematical construction establishing the existence of initial models of finitely axiomatizable lex theories. We do not give here a detailed proof, but sketch the key ideas:

We need to construct, internally to an arbitrary NNO-topos, the set of well-founded finitely branching labelled rooted trees corresponding to the term model (the labels on the nodes of the tree corresponding to the operators which build new terms or new equations from old ones in the algebraic theory). Once we have constructed these, we use effective regularity to quotient the trees corresponding to definable terms by the equivalence relation induced by trees corresponding to derivable equations. All difficulty is just in first constructing this object of well-founded labelled trees (a so-called W-type).

We first of all take a finite coproduct of $1$s to serve as the object of labels. This suffices as we only need finitely many labels for a finitely axiomatizable theory.

Next, we note that we can define the set of lists of $X$es in suitable fashion. For example, we can define lists of $X$es as suitable partial functions with domain $\nat$ and codomain $X$ (returning the $n$th element of a given list). This definition can then be interpreted in NNO-toposes using cartesian closure and subobject comprehension.

Finally, we can define the sets of arbitrary or well-founded countably branching trees similarly, as, e.g., suitable partial functions with domain the set of lists of naturals and codomain the set of labels (returning the label found by traversing a given sequence of branch indices down from the root). We can express within the internal logic of topos theory the conditions corresponding to being a well-founded tree formed by appropriate applications of the constructors of our algebraic theory (by a suitable quantification over the power object). We can thus take the appropriate subobject of the set of all such partial functions, to get the set of well-formed labelled trees we are interested in.

Finally, the well-foundedness of these trees lets us prove inductively the existence of partial functions satisfying any particular recursion conditions with any particular tree in their domain, and lets us prove that any two such partial functions agree wherever both are defined. An impredicative union of all such partial functions then yields a unique total function defined by such recursion. This gives us the unique homomorphisms from the term model to other internal models, establishing the term model as the initial internal model.

This is one simple approach available to us for constructing initial models in an NNO-topos. Other approaches are possible as well. \TODOinline{For example, by making the observation that arithmetic universes have the same property, and NNO-toposes are arithmetic universes.}

\TODOinline{Make further observation about k-ary theories, when we have k-ary coproducts. Note that it is key here how the theory actually recognizes internally anything which is externally a model of such an infinitary theory, as we can construct any external function on domain k as an internal map. Once we've noted the version of this for arbitrary k, we can invoke it later on when we wish to make any initial model in actual Set of an infinitary sort.}
\end{proof}

From \magicref{NNOToposIsInitoGeminal} and the finite axiomatizability of the theory of NNO-toposes, we have that in particular, the initial NNO-topos has an initial internal NNO-topos. That is to say, the theory of NNO-toposes is \initogeminal/. Thus, by \magicref{InitoGeminalYieldsGeminal}, the initial NNO-topos is equipped as a geminal NNO-topos (a fortiori, a geminal category).

\begin{warningenv}\label{InitoGeminalWarning}
It is important to observe that the initial NNO-topos is NOT an introspective theory! Using the name $G$ for the initial NNO-topos and $G'$ for its internal initial NNO-topos, we should not expect to have natural maps in $G$ from $t$ to $\Hom_{G'}(1, \introS(t)))$ (i.e., $\Box t$) for general $t$, as the $\introN$ of an introspective theory would provide. For example, we will not have notable maps of type $\Omega \to \Box \Omega$ or $\nat^{\nat} \to \Box(\nat^{\nat})$ (the presence of such a map would express the absurd logical assertion that every function from naturals to naturals (every such function at all) induces some corresponding definable morphism in the initial NNO-topos.). We have merely equipped it as a geminal category. We will in some cases have canonical such requoting maps (e.g., a map $: \nat \to \Box \nat$ will be available by initial algebra properties of $\nat$), but not in general.

So the construction in \magicref{InitoGeminalYieldsGeminal} does not give us new introspective theories. Rather, it takes the introspective theory of geminal gadgets (which we already constructed in \TODO) and constructs a model of it, for suitable notions of \quote{gadget}.
\end{warningenv}

Having established that NNO-toposes have initial internal models of all finitely axiomatizable theories, it follows that any finitely axiomatizable theory extending the theory of NNO-toposes is \initogeminal/.

Due to work by Maietti et al following in the footsteps of Joyal (\TODOinline{cite}), it is also known that any arithmetic universe contains an internal initial model of any finitely axiomatizable theory.

Thus, also, the theory of arithmetic universes is \initogeminal/, and thus the initial arithmetic universe can be equipped as a geminal arithmetic universe (a fortiori, a geminal category). Thus, we get \Godel/'s incompleteness results manifesting within the initial arithmetic universe. This is the structure discussed by Joyal in unpublished work on a category-theoretic account of \Godel/'s incompleteness theorem, and further discussed by others after Joyal (see in particular \autocite{van2020g}).

The fact that every NNO-topos contains an initial model of every finitely axiomatizable lex theory can of course be taken as a special case of the fact that every arithmetic universe has the same property, since NNO-toposes are straightforawrdly arithmetic universes. But the construction of initial internal models in an NNO-topos can also be carried out by much easier means than are available in an arbitrary arithmetic universe; e.g., as in the proof sketch we gave at \magicref{NNOToposIsInitoGeminal}, which made essential use of cartesian closure, quantification over power objects, and the like.

Of course, we could directly consider the theory of a lexcategory with an initial internal model for every finitely axiomatizable theory. This would be interno-geminal... if it were finitely axiomatizable. In the form we just stated this theory, it was axiomatized infinitely (there is a separate imposed basic constructor for every particular finitely axiomatizable theory). It is an open question to this author whether this theory admits some alternative finite selection of basic constructors allowing it to be finitely axiomatized.

\TODOinline{Still, stress that finite axiomatizability isn't the key thing. We have after all the theory of a topos with countable coproducts as a \initogeminal/ theory}.

It is a similarly open question whether there is an initial \initogeminal/ theory (in either the sense without or with the parenthetical condition noted at \magicref{InitoGeminalYieldsGeminal}). The theory of \initogeminal/ theories is not known to be equivalent to any lex theory (the condition for $T$ to be \initogeminal/ involves a higher-order quantification over all endolexfunctors of $T$), so we do not automatically have the existence of an initial such structure.

Just as with \magicref{InitoGeminalWarning}, we should remark that again, as of yet, we have only equipped the initial arithmetic universe as a geminal category, not an introspective theory. But it will turn out that, unlike the typical situation for a \initogeminal/ theory as with the initial NNO-topos, the theory of arithmetic universes is so special that we can in fact further equip it in a natural way as an introspective theory! We shall come back to this at the end of the next section, after developing some more tooling for constructing more sophisticated introspective theories from \initogeminal/ theories in general.

\TODOinline{Reorganize order of paragraphs here for clarity. Discuss toposes with k-ary coproducts as mentioned above.}

\TODOinline{Note that we have a soundness result for the geminal categories we get from the initial NNO-topos, the initial arithmetic universe, etc: Their internal views of the initial such-and-such do in fact match themselves; the uniquely determined structure-preserving (AU-preserving or NNO-topos-preserving) functor from these things to Set takes their internal initial such-and-such to themselves. We don't have this kind of soundness for all \initogeminal/ theories. For example, we might consider the theory of bloposes, where a blopos is an NNO topos in which the internal initial NNO topos is trivially 1. That is, a blopos is a topos that thinks the theory of toposes is inconsistent. Then the theory of bloposes incorrectly proves that the theory of bloposes is inconsistent, so the initial blopos is nontrivial, but its internal initial blopos is trivial. The key thing that makes bloposes different from toposes is that Set is itself a topos (and an AU and so on), but not a blopos. Our soundness result is only for those theories which Set itself models.}

\fileend