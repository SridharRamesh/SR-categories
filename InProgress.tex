\filestart

\section{Scratch work in progress}

\subsection{Unorganized draft category-theoretic lemmas}

\subsubsection{Initial algebra constructions}
\TODO

\subsubsection{Theories, models, functors, etc}
\begin{observation}\label{ModelTerminology}
We need to standardize our terminology on theories, models, and internal models. Is a lexfunctor from $T$ to $S$ an internal model of $T$ in $S$? Or is a lexfunctor from $T$ the globalization of an internal category in $S$ an internal model of $T$ in $S$? Is a model always Set-valued? Etc. Let us do a Ctrl+F for "model" to make sure we are not confusing on this point. It may be best to speak scrupulously of lexcategories and lexfunctors at all times, and not of theories or models, except in the fixed phrase "introspective theory".
\end{observation}

\TODOinline{
Introduce terminology for "the walking X" meaning "the free lexcategory with an internal X". "Internal X" always means "Map into me from the lex theory of Xes", and NOT "Internal lexcategory in me, and map into THAT from the lex theory of Xes". The latter should be called doubly internal. The theory of Xes has an internal X. Hm. Maybe we need good terminology for "the theory of blahs" vs "a blah", when blah is itself a lexcategory-extending notion.

Let's say \included/ models of theory T within lexcategory S refers to lexfunctors from the lexcategory corresponding to T, to S. While \interior/ models of T within S refers to the case where S has a designated internal lexcategory C, and we also have a lexfunctor from T to the globalization of C. In other words, \interior/ is doubly internal.

The weird thing is that the identity lexfunctor on a category also corresponds to the generic internal model, even though the identity seems like staying on the same level, while internal sounds like going down one level. Hence why actually going down a level with \interior/ corresponds to doubly internal.}

\subsubsection{Constructing initializers}
So, for example, let us try constructing, for a lexcategory $T$, the initial lexcategory with an internal initial model of $T$. We find some quasi-equational theory $Th(T)$ presenting $T$. This theory is k-sized, so we work from now on in k-ary quasi-equational theories. We now make the k-ary quasi-equational theory of a strict lexcategory with an internal initial model of $Th(T)$, and apply the initial algebra theorem. We get an initial strict lexcategory with an internal initial model of $Th(T)$; call this $S$.

Suppose we have three quasi-equational theories:
Th(T)
Th(Init-T')
Th(D)

These present strict lexcategories
T'
Init-T'
D'

These present lexcategories
T
Init-T
D

Th(T) is chosen to present T, with T' arising from this.
Th(Init-T') is chosen to present Init-T' as the free strict lexcategory with an internal initial model of Th(T). Init-T then arises from this.
Th(D) is chosen to present D, with D' arising from this.

For any initial model of Th(T) in D', we get a unique map from Init-T' to D' which restricts to the corresponding map from T' to D'.

Given two strict maps f, g from Init-T' to D' which both take T' to initial models in D', we want a unique transform from f to g. Transforms from f to g correspond by lemma 55 (even without the strictness assumption on f, g) to transforms between the corresponding models of Th(Init-T') in D'. We want [TODO] to show that these correspond to transforms between the corresponding models of Th(T) in D'. If we can do that, then since those models are initial, we get a unique map between them, as desired.

\TODOinline{Perhaps we should mark with asterisks those theorems which are specific to $\Set$ and perhaps use Choice vs those theorems which are meant to internalize broadly}

\subsubsection{Cartesian closure and \Loeb/'s theorem}
\begin{theorem}
Suppose $\langle T, C \rangle$ is a locally introspective simple theory (i.e., locally introspective finite product theory). Furthermore, suppose we have an object $0$ of $T$ such that all exponentials $0^t$ exist in $T$ (as in when $T$ is cartesian closed), and furthermore, there is a morphism in $T$ of type $\Box ((\Box 0) \implies 0) \vdash \Box 0$ (a \Loeb/ morphism). Let us refer to $0^X$ as $\neg X$. We then get a morphism in $T$ of type $1 \vdash \neg \neg \Box 0$; equivalently, a morphism of type $\neg \Box 0 \vdash 0$.
\end{theorem}
\begin{proof}
In $T$, we have $(\neg \Box 0) \times (\Box 0) \vdash 0$. There is also a general principle that if $T$ contains a morphism $X \times Y \vdash Z$, then $T$ also contains a morphism $X \vdash \Box (Y \implies Z)$ (by using $\introS$ to get $\Box X \vdash \Box(Y \implies Z)$, and then using $\introN$ to get $X \vdash \Box X$ and composing these).

Applying that general principle to this particular morphism, we find in $T$ a morphism $\neg \Box 0 \vdash \Box (\Box 0) \implies 0$. The right-hand side here combines with our \Loeb/ morphism; the result of this composition is $\neg \Box 0 \vdash \Box 0$. And from here we quickly get $\neg \Box 0 \vdash 0$, aka, $1 \vdash \neg \neg \Box 0$.
\end{proof}

(In fact, for an introspective theory, we get this more easily: simply consider the aspect of $T$ defined over $(\Box 0) \implies 0$.)

The above shows the undesirability of using exponentiation in a cartesian closed $T$ to model ordinary implication, when using introspective theories to model ordinary provability logic, as this would imply a certain degree of incompatibility with Boolean logic. Specifically, in a Boolean context, when $0$ is actual falsehood so that $\neg \neg$ is supposed to be identity, the above would give us the rule $1 \vdash \Box 0$, asserting provable falsehood.

The appropriate way to model provability logic complete with an implication operator is not to take $T$ to be cartesian closed, but rather, to take only $C$ to be an indexed cartesian closed category. Note that even if $T$ happens to be cartesian closed as well, we will in general not have that $\introF$ must preserve exponentials; that is, exponentials computed via $T$'s cartesian closed structure may well differ from exponentials computed via $C$'s cartesian closed structure (especially \TODOinline{Only?} if $C$ has objects not in the range of $\introF$). It is the latter which are most relevant for ordinary provability logic. Thus, although $T$ being cartesian closed will give us $\introS(\neg_T \Box 0_T) \vdash_C \introS(0_T)$, it will not give us $\neg_C \introS(\Box 0_T) \vdash_C \introS(0_T)$. We will see this explicitly in model \TODO and model \TODO.

\subsubsection{Cartesian closed categories and locally introspective theories}
\begin{theorem}
We observe here that a locally introspective simple (i.e., finite product) theory in which $T$ and $C$ are cartesian closed and $\introF$ preserves cartesian closed structure and is essentially surjective on objects is the same as a cartesian closed theory $T$ along with a finite product preserving operator $\Box_T$ and natural transformation $\introN : \id_T \to \Box_T$.
\end{theorem}
\begin{proof}
As a cartesian closed category, $T$ is enriched over itself. A finite product preserving functor from $T_1$ to $T_2$ turns enrichments over $T_1$ into enrichments over $T_2$; using this, we can construe $T$ instead as a category $C$ enriched over $T$, where the objects of $C$ are those of $T$ but $\Hom_C(t, s) = \Box_T \Hom_T(t, s) = \Box_T s^t$.

A category enriched over $T$ (with terminal object) is the same as a locally \repsmall/ $T$-indexed category such that every object in $C(t)$ is the pullback of some object in $C(1)$. Thus, the above has produced $C$ as a locally \repsmall/ $T$-indexed category. Since both $\Box$ and $-^t$ are product preserving endofunctors of $T$, it's easy to see that $\Hom_C(t, s_1 \times \ldots \times s_n) = \Hom_C(t, s_1) \times \ldots \times \Hom_C(t, s_n)$ so that $C$ furthermore has finite products. (We can also see that $C$ has exponentials, from \TODO).

For $\introS$, we send each object to itself and send each morphism $m: t \to s$ in $T$ (equivalently, $: 1 \to s^t$) to the corresponding value in $: 1 \to \Box(t^s)$ given by applying our natural transformation from identity to $\Box$. It is readily verified this is functorial and product-preserving (\TODO).

The required $\introN : t \to \Hom_C(1, \introS(t)) = \Box_T t$ is given by the supplied natural transformation as well. This completes this direction of the proof.

Put more cleanly:

% https://q.uiver.app/?q=WzAsNCxbMSwwLCJcXG9we1R9Il0sWzIsMCwiRmluUHJvZENhdCJdLFsxLDEsIlxcb3B7VH0iXSxbMCwwLCJcXG9we1R9Il0sWzAsMSwiVC8vLSJdLFsyLDEsIlxcTGFuX3tcXEJveH0oVC8vLSkiLDJdLFswLDIsIlxcQm94IiwyXSxbMywyLCJcXGlkIiwyXSxbMywwLCJcXGlkIl0sWzQsMiwiIiwwLHsic2hvcnRlbiI6eyJzb3VyY2UiOjIwLCJ0YXJnZXQiOjIwfX1dLFswLDcsIiIsMix7InNob3J0ZW4iOnsidGFyZ2V0IjoyMH19XV0=
\[\begin{tikzcd}
	{\op{T}} & {\op{T}} & FinProdCat \\
	& {\op{T}}
	\arrow[""{name=0, anchor=center, inner sep=0}, "{T//-}", from=1-2, to=1-3]
	\arrow["{\Lan_{\Box}(T//-)}"', from=2-2, to=1-3]
	\arrow["\Box"', from=1-2, to=2-2]
	\arrow[""{name=1, anchor=center, inner sep=0}, "\id"', from=1-1, to=2-2]
	\arrow["\id", from=1-1, to=1-2]
	\arrow[shorten <=4pt, shorten >=4pt, Rightarrow, from=0, to=2-2]
	\arrow[shorten >=2pt, Rightarrow, from=1-2, to=1]
\end{tikzcd}\]

Plus the observation at https://sridharramesh.github.io/HowSridharThinks/math/IndexedEnrichedInternalCategories.html to ensure that the left Kan extension within $\Cat$ really does produce a category with finite products. Plus the verification that $\Hom_C(A, B) = \Box_T (B^A)$.

In the other direction (introspective theory with eso $\introF$ is cartesian closed $T$ plus operator), \TODO.
\end{proof}
Similarly, some analogue of the above for locally introspective finite limit theories? \TODO

\begin{TODOblock}
Similarly, some analogue of both of the above for geminal categories? The simplest observation is that, from any cartesian closed category C' with endo-fp-functor Box s.t. we have a natural transformation from Box to $\Box^2$ s.t. the two induced paths from Box to $\Box^3$ are equal, we can construct an introspective finite product theory T such that the objects of $T$ are finite products of objects of the form $\Hom_C(A, B)$ for objects $A$ and $B$ in $C(1)$, while the objects of $C$ at every aspect are the objects of $C'$. And of course $\introS(\Hom_C(A, B)) = \Box(B^A)$, while $\introN$ corresponds to our Box to $\Box^2$ operator. Maybe this is simpler put in terms of the observation as to what a geminal (simple/fp theory) cartesian closed category on given objects is.
\end{TODOblock}

\subsection{Cartesian closure and box}
We should be able to interpret $\Hom_C(S(a), S(b))$ as $\Box_T(a \implies b)$ for $a, b \in T$ with some analogy to how this works on $\Psh{C}$. (Search for where we say bifunctor in this document). Note that if $T$ is cartesian closed, this may be distinct from $\Box_T (B^A)$ using the cartesian closed structure of $T$ to define $B^A$.

If T is cartesian closed, we have $Hom_C(1, S(a -> b)) |- Hom_C(S(a), S(b))$, but not the other way around.

\subsection{Soundnesses}
\TODO

A global soundness is something like, given an introspective theory T, we can ask for a lexfunctor from T to Set which sends all of $\introN$'s components to isomorphisms.

This might be the same as asking of the lexfunctor (construed as a model M of T with an internal model M' and a homomorphism N from M to $Hom_M(1, M')$) to be such that this homomorphism $N : M \to \Hom_M(1, M')$ is an isomorphism.

-----

Note that we have a lexfunctor from T to (N -> T), because we have a functor from N to (T -lex-> T), which sends n to $Box^n$ and which sends the map from n to n + 1 to $\introN_{Box^n}$. (We want this map, not $\Box^n (\introN_{id})$)

Therefore, for any model M : T -Lex-> Set, we get also by composition a lexfunctor from T to (N -> Set). At each n, this specializes to the (globalization of the) geminal category n-tuply internal to the original model, and at each n to n + 1 map, this is the self-internalizing homomorphism of that geminal category (NOT the action of the original geminal category's self-internalizer on its internal categories).

But we also have a lexfunctor from (N -> Set) to Set given by taking colimits, since filtered colimits commute with finite limits.

Composing these, we get a lexfunctor from T to Set which is the colimit of all those models and homomorphisms. It should be possible to check directly that this colimit model is such that each $\introN$ component within it is an isomorphism. Thus, it has a soundness in the above sense. (I am no longer confident this works, I think it still falls apart on the failure of well-pointedness.)

----

The initial AU and the related Sigma1 models has such a super strong soundness: It is an introspective theory, with its C furthermore equipped as an internal introspective theory, and it is equivalent qua introspective theory to the globalization of this internal introspective theory (though of course it does not internally claim that C is equivalent to the globalization of C', thanks to G2IT).

\TODOinline{Returning to this months later, I think where it stood was like so: The above isn't exactly right and the problem is the well-pointedness failure. See https://ncatlab.org/nlab/show/transfinite+construction+of+free+algebras}

\subsection{Convenient terminology}
\begin{TODOblock}
Some note about how we will also make use in diagrams in this chapter liberally of the identification of $P(c)$ with $\Hom(c, P)$, when $P$ is a presheaf. If one likes, this can be seen as invoking the Yoneda lemma to draw diagrams in $\Psh{C}$ (although invoking the Yoneda lemma is actually a bit overkill for the mere fact that we can draw diagrams of presheaf elements and presheaf actions in this way, which is simply working within the appropriate collage, aka cocomma category).

We're a bit glib in all our writing when we swap between thinking of $x$ as an element in $X$, vs a map from $1$ to $X$, or similarly in swapping between a presheaf $\Omega^X$ and values in $\Omega(X)$ and morphisms from $X$ to $\Omega$, etc. Perhaps we should introduce some explicit notation for these kinds of conversions, to make everything perfectly clear.

Can move these notes to the Preliminaries.
\end{TODOblock}

\subsection{Fixed point interchange}
\begin{theorem}
Suppose $f : A \to B$ and $g : B \to A$. Then $f$ and $g$ comprise the two halves of a one-to-one correspondence between fixed points of $g \circ f$ and fixed points of $f \circ g$.
\end{theorem}

\subsection{Another box lemma}
\begin{theorem}\label{BoxMatchesN}
For a locally introspective theory (or finite product theory) $\langle T, C, \introS, \introN \rangle$, and a globally defined element $x$ of $P(1)$ for $P \in \Psh{C}$, we have that $\introN_{\Omega}(x)$ is the globally defined element of $(\Box P)(1)$ given by applying $\Box_{\Psh{C}}$ to $x$.
\end{theorem}
\begin{proof}
\TODO
\end{proof}

\subsection{Initiality with respect to Set itself}
\begin{observation}\label{InitialWrtSet}
In our metatheory, we have access to the following principle: If M is the term model of some finitely axiomatized lex theory of gadgets, and Set is also a gadget (but a large one), then M not only is initial with respect to set-sized gadgets but also there is a unique homomorphism from M to Set.
\end{observation}
\begin{proof}
The proof is by the exact same proof that shows M's initiality with respect to set-sized gadgets.

Alternatively, the proof can be carried out like so: Firstly, for existence of a map from M to Set, we take the finitely many finitary operations of our theory and note that the hull definable from these within Set describes a small (indeed, countable) subgadget of Set. M will have by initiality a homomorphism into this subgadget, and thus into Set itself. As for uniqueness, consider any two homomorphisms from M to Set. Again, their ranges will be small subgadgets of Set, and we can take the union of those ranges, close under the operations of our theory, and find some other small subgadget of Set containing them both. M will have a unique homomorphism into this enveloping subgadget, and thus the parallel homomorphisms of M must have been equal.
\end{proof}

Be cautioned, however, that the reasoning above only works in our metatheory, with typical principles available to us like the ability to reason about subcollections of Set. Analogous reasoning can fail internally; e.g., every topos T with NNO is such that its self-indexing T/- is an indexed topos with NNO, and such that it has an initial internal topos with NNO T' constructed as a term model, and yet there need be no topos-with-NNO homomorphism from T' to T/-. (Indeed, in the initial topos-with-NNO, there will not be such a homomorphism, by \Goedel/'s second incompleteness theorem considerations). Similarly for \quote{arithmetic universe} in place of \quote{topos with NNO}.

\subsection{Uniqueness of model constructions}
\begin{lemma}
Let $T$ be a lexcategory with a natural numbers object, such that every object of $T$ admits a monic map into the natural numbers object. Let $\Box$ be a lex endfunctor on $T$. There is at most one natural transformation from $\id_T$ to $\Box$.
\end{lemma}
\begin{proof}
Let $\nat$ be the natural numbers object of $T$, with zero map $z : 1 \to \nat$ and successor map $s : \nat \to \nat$. Note that $\Box z : 1 \to \Box \nat$ and $\Box s : \Box \nat \to \Box \nat$. Thus, by the universal property of $\nat$, there is a unique map $h : \nat \to \Box \nat$ such that $h \circ z = \Box z$ and $h \circ s = (\Box s) \circ h$.

Now we will show that each of the components of a natural transformation $\eta : \id_T \to \Box$ is uniquely determined (if such a natural transformation exists at all!).

First of all, by naturality of $\eta$ with respect to the morphisms $z$ and $s$, we see that any natural transformation $\eta : \id_T \to \Box$ would also be such that $\eta_{\nat} \circ z = \Box z$ and $\eta_{\nat} \circ s = (\Box s) \circ \eta_{\nat}$. Thus, we must have that $\eta_{\nat} = h$.

Furthermore, for any arbitrary object $t$ of $T$, consider some monic map $m : t \to \nat$. By naturality of $\eta$ with respect to $m$, we see that $(\Box m) \circ \eta_t = \eta_{\nat} \circ m$. Since $\Box m$ is monic (since $\Box$ as a lexfunctor preserves monicity), and since $\eta_{\nat}$ was already uniquely determined, this uniquely determines $\eta_t$ as well.
\end{proof}

\subsection{Localization lemmas}
\begin{lemma}
If a lexcategory has a parameterized NNO (that is, an NNO in its simple self-indexing), then this is furthermore a pullback-stable NNO (that is, an NNO in its full self-indexing).
\end{lemma}

\begin{lemma}
For any category with finite products $C$, there is a tractable explicit description of its free lex completion $C'$. That is, a lexcategory $C'$ such that we have a finite-product-preserving functor $f : C \to C'$, such that for any other lexcategory $D$ and finite-product-preserving $g: C \to D$, we have that $g$ factors through $f$ by a unique lexfunctor.

The functor $f: C \to C'$ is furthermore full and faithful, so that $C$ can be seen as a full subcategory of $C'$. Furthermore, every object of $C'$ comprises an equalizer of two parallel morphisms of $C$ (i.e., every object of $C'$ is a regular subobject of an object of $C$). Furthermore, every morphism from an object of $C'$ to an object of $C$ is given by composing such an equalizer with a morphism of $C$ (i.e., the objects of $C$ comprise regular injective objects of $C'$).

If $C$ furthermore has a parameterized NNO, then this is furthermore a pullback-stable NNO in $C'$.
\end{lemma}
\begin{proof}
The construction is given in section 2 of \autocite{BungeSymmetricTopos}, following unpublished notes of A. Pitts. They do not make the final observation about NNOs, but this follows straightforwardly from the rest.
\end{proof}

Let $PrimRec$ denote the free category with finite products and a parameterized NNO, and let $PrimRecLex$ denote the free category with finite limits and a paremeterized NNO. By the above lemma, PrimRec is equivalent to the full subcategory of PrimRecLex comprising finite powers of the natural numbers object.

\begin{construction}
Suppose $T$ is a lexcategory, fix some lexfunctor $q$ on domain $T$, and let $M$ be the subset of arrows of $T$ which are sent to isomorphisms by $q$.

Then using the \quote{calculus of (right) fractions}, we have a tractable explicit construction of the \quote{localized} category $T[M^{-1}]$ which freely makes each arrow in $M$ an isomorphism. That is, we have a functor $f : T \to T[M^{-1}]$ such that $f(m)$ is an isomorphism for each $m \in M$, and furthermore, for any other category $U$ and functor $g : T \to U$ such that $g(m)$ is an isomorphism for each $m \in M$, we have that $g$ factors uniquely through $f$ (up to natural isomorphism) by another functor $g'$. In fact, this $T[M^{-1}]$ is a lexcategory and $f$ is a lexfunctor. When $g$ preserves finite limits, the corresponding $g'$ preserves finite limits as well. Thus, this is simultaneously a localization qua category and localization qua lexcategory.

The details of this calculus of fractions construction are given in \autocite{gabrielzisman1967}, with the above result as the combination of their Propositions 3.1, 3.2, and 3.4.

We furthermore have the observation that, for any parallel $g_1, g_2 : T \to U$, with $g_1', g_2' : T[M^{-1}] \to U$ being their corresponding factorizations through $f : T \to T[M^{-1}]$, then each natural isomorphism from $g_1$ to $g_2$ is the whiskering along $f$ of a unique natural isomorphism from $g_1'$ to $g_2'$ (that is, $f$ induces an isomorphism from $\Hom(g_1', g_2')$ to $\Hom(g_1, g_2)$; that is, the functor $f$ induces from $U^{T[M^{-1}]}$ to $U^T$ is full and faithful). This follows from considering the arrow category $U^{arrow}$, its relationship to natural transformations, and the fact that morphisms in $U^{arrow}$ are invertible just in case both their projections to $U$ are invertible. \TODOinline{Write this last paragraph better or find a cite for it; the result, though not the proof, is noted at the nLab page on localization}
\end{construction}

\begin{lemma}
Suppose $T$ is a lexcategory with a pullback-stable NNO $\nat$, and suppose $T'$ is the localization of $T$ qua lexcategory at some collection of monics. Then the image of $\nat$ in $T'$ is also a pullback-stable NNO. \TODOinline{Not sure this is true}
\end{lemma}

\begin{lemma}
Let $T$ be any arithmetic universe in which every object is a subquotient of $N$, and every subobject of $N$ is equal to one coming from the initial arithmetic universe. Then $T$ is some localization of the initial arithmetic universe.
\end{lemma}

\TODOinline{Perhaps forget all the above leading to localizing AUs. Instead, we can just directly localize introspective theories, as long as they have corresponding internal localizations as well. In fact, the internal localization doesn't need the universal property, it just needs to have made the appropriate maps into isomorphisms. The tricky part of this is showing that we still have our N. This is because objects don't change under localization (so we can reuse the components of the old N), and the new morphisms are all compositions of old morphisms or inverses of old morphisms (so we automatically get naturality with respect to them just from the old naturalities). It appears we could also freely add new equations between morphisms without issue.

A corollary of this is that any AU (call it T) internal to the initial AU gives rise to an introspective theory, as there will be an internal map from the internal initial AU to T, and there will also be the actual interpretation of T in Set, and this will admit a lexfunctor into the global aspect of the internal T. Furthermore, the actual interpretation of T in Set will admit an arithmetic functor from the initial AU, and the image of this, in some suitable sense, will be the sigma_1 aspect of T, which is suitably constrained so as to allow us to still have our N.}

Let IAU be the initial AU.

We have automatically two lexfunctors from IAU to Set: one is the unique AU-functor $!$ from IAU to Set, and the other is the global sections lexfunctor $\Hom_{IAU}(1, -)$. From the sigma1esque properties of IAU, we have that $!$ is initial in Lex(IAU, Set), while from the yoga of lexfunctors, we also have that $\Hom_{IAU}(1, -)$ is initial in Lex(IAU, Set). Thus, $!$ and $\Hom_{IAU}(1, -)$ are uniquely isomorphic.

Let $C$ be any AU internal to IAU, and let $\Glob{C}$ be its global aspect (equivalently, its image under $!$). Let $\Gamma_C : \Glob{C} \to IAU$ be the global sections functor $\Hom_C(1, -)$.

We can equip IAU as an introspective theory via \magicref{IAUAsIntrosp}. Using the unique internal arithmetic functor from IAU's internal initial IAU to $C$, along with \magicref{IntrospInternalMap}, we can re-equip IAU as an introspective theory we will call $\langle IAU, C, \introS, \introN \rangle$ in the following. (Actually, we needn't even use this particular approach to the construction here. We obtain the same $\langle IAU, C, \introS, \introN \rangle$ by taking $\introS : IAU \to \Glob{C}$ to be the unique such arithmetic functor (by initiality of IAU), and $\introN : \id_{IAU} \to \Gamma_C \circ \introS$ to be the unique such natural transformation (by sigma1esqueness).)

Let $M$ be the morphisms in IAU whose image under $\introS$ is invertible; we can form the localization (qua lexcategory) $f : IAU \to IAU[M^{-1}]$, along with lexfunctor $g : IAU[M^{-1}] \to \Glob{C}$ such that $g \circ f = \introS$. This $IAU[M^{-1}]$ is like the $\Sigma_1$ aspect of $\Glob{C}$.

By whiskering along $f$, we get $f \introN : f \to f \circ \Gamma_C \circ \introS = f \circ \Gamma_C \circ g \circ f$. By the properties of localization with respect to natural transformation, this corresponds to some unique natural transformation $H : \id_{IAU[M^{-1}]} \to f \circ \Gamma_C \circ g$ such that $f \introN = H f$. \TODOinline{We can perhaps tighten this up by showing sigma1esqueness of $IAU[M^{-1}]$, so that this natural transformation is uniquely determined}

Now, let's define $C'$ as $f[C]$. Let $\Gamma_C' : \Glob{C'} \to IAU[M^{-1}]$ be the corresponding global sections functor. Note that we have also $\InducedHomo{f}{C} : \Glob{C} \to \Glob{C'}$. Let $\introS' : IAU[M^{-1}] \to \Glob{C'}$ be given as the composition $\InducedHomo{f}{C} \circ g$. Finally, we need a natural transformation $\introN' : \id_{IAU[M^{-1}]} \to \Gamma_C' \circ \introS' = \Gamma_C' \circ \InducedHomo{f}{C} \circ g$. We obtain this $\introN'$ by taking $H$ as above and composing it with the whiskering along $g$ of the canonical natural isomorphism (given by \magicref{InducedGlobalCommute}) from $f \circ \Gamma_C$ to $\Gamma_C' \circ \InducedHomo{f}{C}$.

This completes the construction of $\langle IAU[M^{-1}], C', \introS', \introN' \rangle$ as an introspective theory.

***

One upshot of all of the above is that, given IAU and any AU $C$ internal to IAU, there is a unique corresponding introspective theory $\langle IAU, C$ for which the $\introS$ is an arithmetic functor. And given any introspective theory, there is a unique localization of it for which the $\introS$ is a conservative functor.

\subsection{Geminal gadgets more generally}
A geminal gadget internal to lexcategory $C_0$ consists (among other things) of a sequence $G_1, C_1, G_2, C_2, G_3, C_3, \ldots$, where for $i \geq 0$, each $G_{i + 1}$ is a gadget internal to $C_i$ and each $C_{i + 1}$ is the global aspect of the underlying (strict) lexcategory of $G_{i + 1}$.

Note that $\Hom_{C_{i + 1}}(1, -)$ can be seen as a lexfunctor from $C_{i + 1}$ to $C_i$ (since $C_{i + 1}$ is the global aspect of a lexcategory internal to $C_i$). Call this global sections lexfunctor $\Gamma_i : C_{i + 1} \to C_i$.

We now also require a sequence of internal gadget-homomorphisms $F_1, F_2, F_3, \ldots$, where each $F_i : G_i \to \Gamma_i(G_{i + 1})$ is internal to $C_i$.

Finally, we impose some equations. We require that each $F_i$ takes $G_j$ to $G_{j + 1}$ and takes $F_j$ to $F_{j + 1}$ for $j > i$. Furthermore, we demand that each $F_i F_i = F_{i + 1} F_i$ in a suitable sense.

Note that all structure subscripted $j$ can be seen as $(j - i)$-tuply internal to $C_i$, for $j > i$. In particular, all of this structure is indeed internal to $C_0$.

Note furthermore that restricting attention to the sequence of $G_j$, $C_j$, and $F_j$ for $j > 1$ yields an instance of this structure internal to $C_1$.

Note even furthermore that $F_1$ acts as an internal homomorphism of this structure.

This all describes a geminal gadget internal to $C_0$.

\begin{theorem}
To uniquely generate all the rest of the aforementioned structure, it suffices only to be given $G_i$, $F_i$, and the equation $F_i F_i = F_{i + 1} F_i$ for $i \in \{1, 2\}$.
\end{theorem}
\begin{proof}
Define $G_n$ as $F_1^{n - 1}(G_2)$, and similarly define $F_n$ as $F_1^{n - 1}(F_2)$. \TODOinline{Clarify the notation, that $F_1(F_2)$ for example is not a composition but rather an application of a functor to the diagram specifying an internal functor. Can use square brackets instead.}

It is now automatically the case that $F_1$ takes $G_j$ to $G_{j + 1}$, and takes $F_j$ to $F_{j + 1}$.

Furthermore, we get the equation that $F_i F_i = F_{i + 1} F_i$, by applying $F_1^{n - 1}$ to the instance of this equation at $i = 2$.

What remains is only to see that each $F_i$ also takes $G_j$ to $G_{j + 1}$ and takes $F_j$ to $F_{j + 1}$, for $j > i$.

We prove this by induction on $i$. We have above established this for $i = 1$ as our base case. As for the inductive step, suppose we know this already holds for $i$. Then $F_{i + 1}[G_j] = F_{i + 1} [F_i [G_{j - 1}]] = (F_{i + 1} \circ F_i)[G_{j - 1}] = (F_i \circ F_i) = F_i [F_i [G_{j - 1}]] = F_i [G_j] = G_{j + 1}$. And similarly with $F$ in place of $G$ as well.
\end{proof}

\fileend