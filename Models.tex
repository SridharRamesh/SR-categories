\section{Models}

\subsection{Automatic consistency results without models as such}
We already know that the theory of geminal categories is an introspective theory. And because every introspective theory is itself a geminal category, we know that the theory of introspective theories only prove $\Box A$ if it furthermore proves $A$.

Finally, we know that every lexcategory can be equipped as a geminal category in a trivial way, by taking its internal geminal category to be trivially $1$, even when the outer lexcategory needn't be trivial. From this, we can conclude that the theory of introspective theories is nontrivial in the sense that it does not prove its internal geminal category to be trivial. Thus, it does not prove $\Box A$ for all $A$. Furthermore, combining this with the previous paragraph, we have the stronger consistency result that for every $n$, the theory of introspective theories does not prove $\Box^n A$ for all $A$.

In this way, simply by consideration of the freeness properties already established in the chapter on geminal categories, we already know the theory of introspective theories to have highly nontrivial content, even without needing to find any models of it \quote{in the wild}.

\TODOinline{Discuss more why the stronger consistency result is really the relevant thing to think about.}

\subsection{Models based on sigma-1 or arbitrary extensions of PA, or ZFC, or etc}
\TODOinline{I will write this section in a sloppy way for now and then improve it later.}

This section reviews and builds upon the construction previously seen at \cref{SigmaModelSimple}.

\begin{construction}\label{Sigma1Model}

Consider a sigma-1 theory $\tau$ extending PA (or ZFC, or any such thing), in the sense of an extension whose axioms are computably enumerable. Actually, for now, let's just consider PA simpliciter.

\TODOinline{It probably isn't easy to pin down in a clean way exactly the minimal kind of system in which this goes through, but it could be useful to name some weak subsystems of arithmetic in which it goes through. In particular, we should not expect this to go through in Robinson's Arithmetic Q which lacks induction entirely, but we should expect it to still work in systems that just have induction for $\Sigma_1$ formulae).}

Consider the category $T$ whose objects are the sigma-1 formulas $\phi(n, m)$ in the language of PA which define binary relations on the natural numbers which PA proves to be partial equivalence relations (i.e., symmetric and transitive). Given any two such formulas $\phi(n, m)$ and $\psi(n, m)$, a morphism in $T$ from $\phi$ to $\psi$ is a sigma-1 formula $F(n, m)$ on the natural numbers which PA proves to correspond to the graph of a function between the subquotients of $\mathbb{N}$ corresponding to $\phi$ and to $\psi$, respectively. That is, such that PA proves the universal closures of the following:

$F(n, m) \implies \phi(n, n) \wedge \psi(m, m)$

$\phi(n_1, n_2) \wedge \psi(m_1, m_2) \wedge F(n_1, m_1) \implies F(n_2, m_2)$

$\phi(n, n) \implies \exists m [F(n, m)]$

$F(n, m_1) \wedge F(n, m_2) \implies \psi(m_1, m_2)$.

Two such formulas $F(n, m)$ and $F'(n, m)$ are considered to be equal as morphisms from $\phi$ to $\psi$ if PA proves them to be equivalent (that is, if PA proves $F(n, m) \implies F'(n, m)$ and $F'(n, m) \implies F(n, m)$).

Given morphisms $F : \phi \to \psi$ and $G: \psi \to \chi$ of this sort, we define their composition in the usual way of composing functions represented as graphs, as $(F \circ G)(n, m) = \exists p [G(n, p) \wedge F(p, m)]$.

This all describes the category $T$, which one can verify is indeed a category and moreso, a category with finite limits.

\TODOinline{Perhaps instead of imposing PERs from the beginning, we start only with the category of RE sets, and then take its ex/lex completion or some such thing. Like so:}

Consider the category $T'$ whose objects are the sigma-1 formulas $\phi(n)$ in the language of PA, and such that a morphism from $\phi(n)$ to $\psi(m)$ is a sigma-1 formula $F(n, m)$ such that $PA$ proves $\forall n, m . F(n, m) \implies (\phi(n) \wedge \psi(m))$ and $\forall n . \phi(n) \implies \exists! m . F(n, m)$. Two such morphisms $F(n, m)$ and $G(n, m)$ are considered equal just in case PA proves $\forall n, m . F(n, m) \biimplies G(n, m)$. Morphisms compose in the obvious way; that is, the composition of $F(n, p)$ with $G(p, m)$ is given by $(G \circ F)(n, m) = \exists p (F(n, p) \wedge G(p, m))$.

This category $T'$ is regular but not exact (that is, not every equivalence relation in $T'$ admits a corresponding quotient). Let $T$ be its ex/reg completion.

\TODOinline{Now, we describe the C inside T which is its internal copy, just by carrying out this exact same construction internal to T, and then we describe the indexed lexfunctor from T to C, which is a little more interesting or takes a little more care. Having this functor be indexed is where the sigma-1 restriction is important.}
\end{construction}

\TODOinline{Observe that we have somewhat distinct concepts of "T = PA Sigma-1, C = ZFC Sigma-1" vs "T = ZFC Sigma-1, C = ZFC Sigma-1", say. Also observe that as concerns ZFC, we can also consider for $C$ not just categories of definable subsets of naturals, but also of definable sets in general, or of definable classes.}

\subsection{Finitely axiomatizable lex theories}
A concept that will often be useful to us in the following.

\begin{definition}
A \defined{finitely axiomatizable lex theory} is a lex theory which, qua lexcategory, can be generated in finitely many steps of the following form, starting from the initial lexcategory: free augmentation with an object, free augmentation with a morphism between existing objects, or freely making two existing parallel morphisms equal. In other words, it can be presented by a finite lex \quote{sketch}. \TODOinline{Word this all better}
\end{definition}

\begin{TODOblock}
Discuss the concept of a lexcategory having initial internal models of ALL finitely axiomatizable lex theories.

As a bit of trivia, observe how this follows simply from having an internal free locally cartesian closed category on one object (verify the details on this; or perhaps from having internal free lex categories and the ability to freely augment internal lex categories with a new cell). Regardless of whether those details work out, conjecture that there are finitely many finitely axiomatizable lex theories such that having internal initial models of those implies having internal initial models of all finitely axiomatizable lex theories, so that the latter is itself a finitely axiomatizable condition.

Relate this also to the concept of arithmetic universes. Conjecturally, being an arithmetic universe is equivalent to something like having free internal models for sketches indexed by finite unions of internal objects (but there seems to be some hesitance in the literature to claim this? Understand that better). At any rate, an arithmetic universe should have internal initial models of all finitely axiomatizable lex theories.

This section basically only exists in order to claim that finitely axiomatizable lex theories which extend the theory of arithmetic universes are automatically examples of the next section.
\end{TODOblock}

\subsection{Theories with free internal models of themselves}
Fix some lex theory extending the theory of strict lexcategories, whose models we shall call \quote{gadgets}. Now suppose every gadget $G$ contains an initial internal gadget $G'$ (in the sense that $G'$ is a gadget internal to the underlying lexcategory $|G|$ of $G$, and for any other gadget $H$ internal to $|G|$, there is a unique $|G|$-internal gadget homomorphism from $G'$ to $H$).

In particular, then, this all applies to the actual initial gadget $G$. Internal to its underlying lexcategory $|G|$, we get a $G'$ as above. Because $G$ is initial, we automatically get a unique gadget-homomorphism $\introN_{G}$ from $G$ to $\Hom_{|G|}(1, G')$ as well. And because $G'$ is an initial $|G|$-internal gadget, we automatically get a unique $|G|$-internal homomorphism $\introN_{G'}$ from $G'$ to $\Hom_{|G'|}(1, G'')$ where $G'' = \introN_{G}[G']$.

This setup is thus a geminal gadget (with axioms 3 and 3' of a geminal gadget automatically satisfied by the uniqueness observations in the previous paragraph).

This immediately gives us models of our theory of geminal gadgets. For example, consider the theory of strict elementary toposes with natural numbers objects (let us call this an \defined{NNO-topos}, to make it less of a mouthful). This is indeed a lex theory in a straightforward way; indeed, a finitely axiomatizable lex theory. Furthermore, every NNO-topos has an internal initial model of every finitely axiomatizable lex theory. Thus, in particular, every NNO-topos has an internal initial NNO-topos, and thus, by the above, the initial NNO-topos is equipped as a geminal NNO-topos.

\TODOinline{It is important to observe that the initial NNO-topos is NOT an introspective theory. It is merely a geminal category.}

In the same way, the initial arithmetic universe is equipped as a geminal arithmetic universe. This is the structure discussed by Joyal and others after Joyal (e.g., Dijk and Oldenziel). \TODOinline{Give proper citations here}

\TODOinline{The initial arithmetic universe actually IS an introspective theory, via reasoning about Freyd covers that does not generalize to other examples of this section's phenomena. Discuss this further.}

\begin{TODOblock}
Extend the above discussion of initiality to discuss corresponding introspective theories, beyond just the discussion of geminality. Although, e.g., the initial NNO-topos is not itself an introspective theory, there is nonetheless a corresponding introspective theory of note capturing the initiality properties. Furthermore, even any lex theory that does not automatically come with internal initial models of itself can be freely bumped up to do so, in a suitable sense, and we get a corresponding introspective theory as well.

Specifically, let us say a theory is sigma1esque if the identity lex endofunctor on it is initial with respect to all lex endofunctors on it. Any lex theory $T$ has a free sigma1esque extension, in the sense of a sigma1esque lex theory $T'$ under $T$ with a unique map into any other sigma1esque lex theory under $T$. This $T'$ also comes with a unique map into the global points of any internal initial model of $T$ in any lex category.

If $T$ has an internal initial model of $T$, then the theory $T'$ also contains an internal initial model of $T$, and thus this theory $T'$ comes with a lexfunctor $\introS$ into that internal initial model of $T$. Furthermore, because $T'$ is sigma1esque, it then comes with a natural transformation $\introN$ from its identity functor to the global sections of that $\introS$. This equips said $T'$ as an introspective theory. And, as noted, this $T'$ is modelled by any initial model of $T$ internal to any category.

Thus, for example, if $T$ is the theory of elementary toposes with NNO, we obtain some introspective theory $T'$ extending $T$ which has the initial elementary topos with NNO as a model.
\end{TODOblock}

\subsection{Models based on well-founded trees/well-founded posets}
There are two flavors of models here: Those which give introspective theories (these come from well-founded trees using a certain size restriction; e.g., considering a model based on the von Neumann universe/cumulative hierarchy), and those which give only locally introspective theories with \Loeb's theorem fixed points (these come from arbitrary well-founded trees; these are related to the models used in guarded recursion theory, but our distinction between the roles of $T$ and $C$ has previously gone unnoticed and allows us to interpret these models as not proving $\lnot \lnot \Box 0$). \TODO

\begin{TODOblock}
Related to \cref{CartesianClosedLocallyIntrosp} is the following construction: Given finite product category $T$, cartesian closed category $D$, a finite product preserving functor $f : T \to D$, a finite product preserving functor $g: D \to T$, we can turn $D$ into a $T$-enriched category $C$ whose set of objects is $\Ob(D)$ and such that $\Hom_C(c, d) = g(c \Rightarrow d)$. As a $T$-enriched category, it can also be thought of as a $T$-indexed category with a constant set of objects. If we furthermore have a natural transformation $\introN$ from $t$ in $T$ to $g(f(t))$, this gives rise to a locally introspective finite product theory $\langle T, C, \introS, \introN \rangle$ where $\introS$ matches $f$ on objects and takes morphism $m : t \to s$ to $g(f(m))$, where $f(m)$ is treated as a global element of $f(t) \Rightarrow f(s)$ and thus $g(f(m))$ is treated as a global element of $g(f(t) \Rightarrow f(s)) = \Hom_D(f(t), f(s))$. Note that $\introN_t : t \to \Hom_D(1, f(t)) = g(f(t))$.

We can extend this construction further. Suppose we have lex category $T$ and lex category $D$, a lex functor $f: T \to D$, and a lex functor $g: D \to T$

(Some analogue of the above holds as well using lex in place of finite product structure throughout, and allowing for the set of objects to vary. But I don't have this on the tip of my tongue right now. The following example will be illustrative to think about for working it out.)

A particular case of the above arises with presheaf categories and Kan extensions. Let $F : A \to B$ be any functor between small categories. This induces by composition a corresponding functor $f : \Psh{B} \to \Psh{A}$. This $f$ preserves all limits and colimits (because they are computed component-wise in presheaf categories), and also has both left and right adjoints (these are the left and right Kan extension operations). In particular, let its right adjoint be $g$. Then $g$ and $f$ satisfy the conditions from above (note that we get our natural transformation from $t$ to $g(f(t))$ as the unit of this adjunction), and so we get a locally introspective theory.

A particular case of THIS arises when $B$ is a poset, $A$ is its underlying discrete set, and $F$ is the corresponding inclusion functor. The introspective theory we get is something like Kripke semantics for a K4 theory. But, hm...

What we want to note is that this construction can be generalized to where $B$ is not a poset, but just a transitive relation. That is, when $B$ is an arbitrary relation on underlying set $A$, we have both the category $\Psh{B}$ of presheaves on the reflexive transitive closure of $B$, as well as the category $\Set^A$, an obvious forgetful functor from the former to the latter, and also a functor from the latter to the former which takes an $A$-indexed set $S$ to the presheaf whose value at any object $x$ is the product (or limit? Well, if we use trees rather than arbitrary posets, there's nothing to worry about here) of the values of $S$ at all objects accessible from $x$. This is basically Kripke semantics. When $B$ is transitive, this should work to give an introspective theory as above. Something something. I need to think this through a little more. But the main idea is, our category $T$ is presheaves on the reflexive closure of $B$, our category $C$ is presheaves on the discrete set underlying $B$, there's an obvious forgetful functor in one direction and the functor in the other direction is given by defining the box operator analogously to Kripke semantics.

Anyway, when $B$ is also well-founded, we should get a local version of Loeb's theorem, in the usual way that well-founded transitive relations correspond under Kripke semantics to GL modal logic. We can impose a size constraint to get introspective theories rather than locally introspective theories. These will become our Kripke models of introspective theories/geminal categories based on well-founded relations, analogous to Birkedal's model of guarded recursion based on the topos of trees.

No, wait, maybe scratch all that about Kan extensions. I don't think that's the right thing to say here anymore. Anyway, look at my discussion of modal logic at https://nforum.ncatlab.org/discussion/6352/necessity-and-possibility/, comment 51114.

Add in thoughts about precategories. Let $F : A \to B$ be any functor between small precategories. This induces by composition a corresponding functor $f : \Psh{B} \to \Psh{A}$, where $\Psh{A}$ means the obvious thing even for precategories. Furthermore, we also have a "prebox" functor from $\Psh{A}$ to $\Psh{B}$ like so: The presheaf $p$ on $A$ is mapped to the presheaf on $B$ which sends each object $b$ to the set of natural transformations from $\Hom_B(b, F(-))$ to $p$, so to speak. That is, ways to choose, for each morphism $m$ in $B$ such that $\dom(m) = b$, an element of $p$ at (one? all? elements of) $F^{-1}(\cod(m))$, such that these choices are natural in the sense that pushing forward the element of $p$ along a morphism $n$ yields the same value as is associated to $F(n) \circ m$.

If $F$ is an identity functor between actual categories, this prebox functor is also the identity. HOWEVER, for identity functors between precategories, this need not hold. An illustrative example is taking the precategory to be the natural numbers, where the morphisms correspond to strict inequality. This corresponds to the notion of box used for step-indexing in the topos of trees.

[Note to self: Is Psh(C) cartesian closed if C is merely a precategory? Yes, because it's the same as Psh(C') where C' is the free category (with identity morphisms) extending C'.

The other thing is, if you take a category C' and view it as a precategory C, and then look at Psh(C) qua precategory vs Psh(C') qua category, these might not exactly line up, because presheaves on C don't have to send identity morphisms to identity functions. But Psh(C') is at any rate a full subcategory of Psh(C), and probably a full subcategory closed under finite limits and perhaps even under exponentials? Anyway, this isn't important for us, we only care about Psh here in the qua precategory sense.]

It is also illustrative, though, to consider the case where $A$ is discrete (no morphisms) and $B$ is a transitive relation on the same objects. This yields the standard appropriate semantics for box as concerns ordinary Kripke models of K4. When B is furthermore well-founded, we get GL. Actually, it is good to consider $B$ to the relation of interest on the same objects PLUS an added object with a unique map to each normal object. This $B$ yields an introspective theory such that the global sections functor takes its geminal category directly corresponding to $\Set^A$.
\end{TODOblock}