\filestart

\section{Examples in the wild}

\subsection{Preview}
In previous chapters, we have defined introspective theories and geminal categories. That is, we have axiomatized the theory of introspective theories and the theory of geminal categories. Now we look at some notable models of these axiomatic theories, which is to say, at some notable specific examples of introspective theories and of geminal categories. These examples are of a sort which might be considered to have been found \quote{in the wild}, instead of being freely syntactically constructed as the examples of the last chapter were.

There are two broad classes of models/examples of note in this chapter:

Firstly, there are those which are similar in flavor to the traditional instances of \Goedel/ian phenomena studied in logic. These are based on logical theories which have some internal ability to discuss themselves, such as Peano Arithmetic, or higher-order intuitionistic logic, or the like. Here, it has long been recognized that \Goedel/ian phenomena arise at the propositional level, but the full phenomenon of guarded recursion which we proved for introspective theories in \TODO in has not been noted in these contexts before.

The second class of models/examples we consider are more similar in flavor to the traditional interpretation of the modal logic GL in well-founded orders. Here, the existence of guarded recursion is straightforward, but it is the unification with our general theory which is of note. Among these models are examples like the topos of trees, the canonical model discussed in the literature on guarded recursion. We also demonstrate similar but distinct models which support an interpretation of Boolean provability logic, as opposed to the fundamentally intuitionistic logic of the topos of trees.

\subsection{Automatic consistency results without models as such}
We already know that the theory of geminal categories is an introspective theory. And because every introspective theory is itself a geminal category, we know that the theory of introspective theories only proves $\Box A$ if it furthermore proves $A$.

Finally, we know that every lexcategory can be equipped as a geminal category in a trivial way, by taking its internal geminal category to be trivially $1$, even when the outer lexcategory needn't be trivial. From this, we can conclude that the theory of introspective theories is nontrivial in the sense that it does not prove its internal geminal category to be trivial. Thus, it does not prove $\Box A$ for all $A$. Furthermore, combining this with the previous paragraph, we have the stronger consistency result that for every $n$, the theory of introspective theories does not prove $\Box^n A$ for all $A$.

In this way, simply by consideration of the freeness properties already established in the chapter on geminal categories, we already know the theory of introspective theories to have highly nontrivial content, even without needing to find any models of it \quote{in the wild}.

\TODOinline{Discuss more why the stronger consistency result is really the relevant thing to think about.}

\subsection{Self-initializing and super-initializing theories}

\subsubsection{The initial model as a geminal category}
\begin{construction}\label{InitoGeminalYieldsGeminal}
Suppose given some lexcategory $T$ (the theory of \quote{gadgets}), along with a lexcategory $C$ internal to $T$ (the underlying lexcategory of a gadget).

Furthermore, suppose given an initial gadget $G_1$ with an initial internal gadget $G_2$. That is, suppose given some lexcategory $V$ such that $\LexCat(T, V)$ has an initial object (our $G_1$) and such that $\LexCat(T, G_1[C])$ has an initial object (our $G_2$).

Because $G_1$ is initial, we automatically get a unique homomorphism $F_1 : G_1 \to \Glob{G_2}$. And because $G_2$ is an initial $G$-internal gadget, we automatically get a unique $G_1$-internal homomorphism $F_2 : G_2 \to \Gamma[F_1[G_2]]$.

This setup is thus a geminal gadget internal to $V$ (with the equations $E_1$ and $E_2$ of \magicref{CompactGeminalCatDefn} automatically satisfied by the uniqueness observations in the previous paragraph).

Indeed, this is the unique way to equip $\langle G_1, G_2 \rangle$ as a geminal gadget.
\end{construction}

In practice, when an initial gadget has an initial internal gadget like above, this is usually not just some accident (caused by a paucity of globally defined structures, say), but rather, is due to the theory of gadgets itself encoding the construction of an internal initial gadget:
\begin{definition}
Suppose, as above, given some lexcategory $T$ (the theory of \quote{gadgets}), along with a lexcategory $C$ internal to $T$ (the underlying lexcategory of a gadget).

If every gadget has an initial internal gadget, and every gadget homomorphism preserves these initial internal gadgets, then we say the theory of gadgets is \defined{self-initializing}.

In other words, $T$ is self-initializing if $\LexCat(T, \Glob{C})$ has an initial object, and this initiality is preserved by $\InducedHomo{f}{C}$ for every lexfunctor $f$ out of $T$.
\end{definition}

The above all admits a generalization worth noting:

\begin{construction}\label{SuperInitoGeminalYieldsGeminal}
Suppose given some lexfunctor $i : T \to T'$, along with a lexcategory $C$ internal to $T$. Call $T$ the theory of \quote{gadgets}, and $T'$ the theory of \quote{supergadgets}. Via $i$, every supergadget has an underlying gadget, and via $C$, every gadget has an underlying lexcategory.

Furthermore, suppose given an initial gadget $G_1$ with an initial internal supergadget $G_2$. That is, suppose given some lexcategory $V$ such that $\LexCat(T, V)$ has an initial object (our $G_1$) and such that $\LexCat(T', G_1[C])$ has an initial object (our $G_2$).

Because $G_1$ is initial, we automatically get a unique gadget homomorphism $F_1$ from $G_1$ to $\Glob{G_2}$. And because $G_2$ is an initial $G_1$-internal supergadget, we automatically get a unique $G_1$-internal supergadget homomorphism $F_2$ from $G_2$ to $\Gamma[F_1[G_2]]$.

This setup is thus a geminal gadget internal to $V$ (with the equations $E_1$ and $E_2$ of \magicref{CompactGeminalCatDefn} automatically satisfied by the uniqueness observations in the previous paragraph).

Indeed, this is the unique way to equip $\langle G_1, G_2 \rangle$ as a geminal gadget $\langle G_1, G_2; F_1, F_2 \rangle$ such that $F_2$ comes from a supergadget homomorphism.
\end{construction}

And again, in practice, when an initial gadget has an initial internal supergadget like above, this is usually not just some accident caused by a paucity of globally defined structures, but rather, is due to the theory of gadgets itself encoding the construction of an internal initial supergadget:

\begin{definition}
Suppose, as above, given some lexfunctor $i : T \to T'$, along with a lexcategory $C$ internal to $T$. We call $T$ the theory of \quote{gadgets}, and $T'$ the theory of \quote{supergadgets}. Via $i$, every supergadget has an underlying gadget, and via $C$, every gadget has an underlying lexcategory.

If every gadget has an initial internal supergadget, and every gadget homomorphism preserves these initial internal supergadgets, then we say the theory of gadgets (or more precisely, the extension of the theory of gadgets by the theory of supergadgets) is \defined{super-initializing}.

In other words, this situation is super-initializing if $\LexCat(T', \Glob{C})$ has an initial object, and this initiality is preserved by $\InducedHomo{f}{C}$ for every lexfunctor $f$ out of $T$.

Note in this case that $T'$ will itself be self-initializing, as every supergadget is a fortiori a gadget (thus having an initial internal supergadget), and every supergadget homomorphism is a fortiori a gadget homomorphism (thus preserving initial internal supergadgets).
\end{definition}

The self-initializing situation is of course the special case of the super-initializing situation where $T' = T$ and $i$ is the identity.

There are a number of self- and super-initializing theories in the wild, which thus immediately give us examples of geminal categories in the wild.

For example: \TODOinline{Frequently, a theory is self-initializing because it has internal initial models of all finitely axiomatizable theories, and is itself finitely axiomatizable. Such theories will also be super-initializing for any finitely axiomatizable extension.}

\TODOinline{Note that when a theory is super-initializing, then we have that in the geminal category $\gamma = \langle G_1 G_2; F_1, F_2 \rangle$, $G_3 = F_1[G_2]$ is also an initial supergadget internal to $G_2$, and $F_3 = F_1[F_2]$ is a supergadget homomorphism. Thus, $\InteriorGeminal{\gamma}$ is itself given by the same construction of a geminal category for the self-initializing theory of supergadgets.}

\begin{TODOblock}
When the \initogeminal/ theory T is such that furthermore Set is a model of T, and we have the strong property noted above that EVERY homomorphism out of the initial model of T preserves its interior initial model's initiality, then we furthermore get a soundness result here: The map from the initial model M of T (the one defined by global elements in the lexcategory corresponding to T) to Set takes M's interior initial model of T to the actual M in Set, and so on.

Thus, we get a homomorphism from the globalization of M's interior initial model of T to the actual M. Since M is initial, this homomorphism onto it is a retraction. Thus, the map from M into the globalization of its internal model has a left inverse.

Note the following two caveats:

1. This composition needn't be identity in the other order.

Proof: Let p be a proposition which is independent from PA, and consider a term t such as "3 if p is true, 5 if p is false", well-defined classically. The map from M' to M to M' will take the term t living in M' to its interpretation as an actual particular number (either 3 or 5) and then to that actual particular number as a canonical term (either "3" or "5"), but that canonical term will not be equal to the original term since t is neither provably equal to 3 nor provably equal to 5.

2. If Set is not a model of T to begin with, there needn't even be a homomorphism from M's internal model to M in the first place.

Proof: Let T be a theory corresponding to PA + ~Con(PA) or the like. Then the initial model M of T is nontrivial, i.e. doesn't prove 1 = 0 (since PA + ~Con(PA) is consistent, since PA doesn't prove Con(PA), by G2IT). However, M's internal initial model of T is trivial, i.e. proves 1 = 0 (since T proves ~Con(PA) which in turn entails ~Con(T)). We cannot have a homomorphism from the latter to the former.
\end{TODOblock}

We have discussed all this just in the context of geminal categories, but this extends to give analogous constructions of introspective theories as well. We discuss these next.

\subsubsection{Self-initializing finitely axiomatizable theories}
\TODOinline{Get rid of this section}

\magicref{InitoGeminalYieldsGeminal} immediately gives us many nontrivial examples of geminal categories. For example, many theories $T$ simultaneously satisfy the following two properties:

A) $T$ is a finitely axiomatizable lex theory extending the theory of lex categories.

B) Every model of $T$ contains initial internal models of every finitely axiomatizable lex theory.

Any such $T$ will of course be \initogeminal/.

For example, consider the theory of strict elementary toposes with natural numbers objects (let us call this an \defined{NNO-topos}, to make it less of a mouthful). This is indeed a finitely axiomatizable lex theory extending the theory of lexcategories \TODOinline{Maybe make up a name for lex theories extending the theory of lexcategories, since we use them often, need them to define our concept of truly internal models, etc}. Furthermore, it satisfies the property B just noted:

\begin{theorem}\label{NNOToposIsInitoGeminal}
Every NNO-topos has an initial internal model of every finitely axiomatizable lex theory. Such initial internal models are furthermore preserved by functors preserving NNO-topos structure.
\end{theorem}
\begin{proof}
\TODOinline{Mention that terms can be partially defined (that is, not all terms denote), in the following}
This is simply by carrying out in its internal logic the ordinary mathematical construction establishing the existence of initial models of finitely axiomatizable lex theories. We do not give here a detailed proof, but sketch the key ideas:

We need to construct, internally to an arbitrary NNO-topos, the set of well-founded finitely branching labelled rooted trees corresponding to the term model (the labels on the nodes of the tree corresponding to the operators which build new terms or new equations from old ones in the algebraic theory). Once we have constructed these, we use effective regularity to quotient the trees corresponding to definable terms by the equivalence relation induced by trees corresponding to derivable equations. All difficulty is just in first constructing this object of well-founded labelled trees (a so-called W-type).

We first of all take a finite coproduct of $1$s to serve as the object of labels. This suffices as we only need finitely many labels for a finitely axiomatizable theory.

Next, we note that we can define the set of lists of $X$es in suitable fashion. For example, we can define lists of $X$es as suitable partial functions with domain $\nat$ and codomain $X$ (returning the $n$th element of a given list). This definition can then be interpreted in NNO-toposes using cartesian closure and subobject comprehension.

Finally, we can define the sets of arbitrary or well-founded countably branching trees similarly, as, e.g., suitable partial functions with domain the set of lists of naturals and codomain the set of labels (returning the label found by traversing a given sequence of branch indices down from the root). We can express within the internal logic of topos theory the conditions corresponding to being a well-founded tree formed by appropriate applications of the constructors of our algebraic theory (by a suitable quantification over the power object). We can thus take the appropriate subobject of the set of all such partial functions, to get the set of well-formed labelled trees we are interested in.

Finally, the well-foundedness of these trees lets us prove inductively the existence of partial functions satisfying any particular recursion conditions with any particular tree in their domain, and lets us prove that any two such partial functions agree wherever both are defined. An impredicative union of all such partial functions then yields a unique total function defined by such recursion. This gives us the unique homomorphisms from the term model to other internal models, establishing the term model as the initial internal model.

This is one simple approach available to us for constructing initial models in an NNO-topos. Other approaches are possible as well. \TODOinline{For example, by making the observation that arithmetic universes have the same property, and NNO-toposes are arithmetic universes.}

\TODOinline{Make further observation about k-ary theories, when we have k-ary coproducts. Note that it is key here how the theory actually recognizes internally anything which is externally a model of such an infinitary theory, as we can construct any external function on domain k as an internal map. Once we've noted the version of this for arbitrary k, we can invoke it later on when we wish to make any initial model in actual Set of an infinitary sort.}
\end{proof}

From \magicref{NNOToposIsInitoGeminal} and the finite axiomatizability of the theory of NNO-toposes, we have that in particular, the initial NNO-topos has an initial internal NNO-topos. That is to say, the theory of NNO-toposes is \initogeminal/. Thus, by \magicref{InitoGeminalYieldsGeminal}, the initial NNO-topos is equipped as a geminal NNO-topos (a fortiori, a geminal category).

\begin{warningenv}\label{InitoGeminalWarning}
It is important to observe that the initial NNO-topos is NOT an introspective theory! Using the name $G$ for the initial NNO-topos and $G'$ for its internal initial NNO-topos, we should not expect to have natural maps in $G$ from $t$ to $\Hom_{G'}(1, \introS(t)))$ (i.e., $\Box t$) for general $t$, as the $\introN$ of an introspective theory would provide. For example, we will not have notable maps of type $\Omega \to \Box \Omega$ or $\nat^{\nat} \to \Box(\nat^{\nat})$ (the presence of such a map would express the absurd logical assertion that every function from naturals to naturals (every such function at all) induces some corresponding definable morphism in the initial NNO-topos.). We have merely equipped it as a geminal category. We will in some cases have canonical such requoting maps (e.g., a map $: \nat \to \Box \nat$ will be available by initial algebra properties of $\nat$), but not in general.

So the construction in \magicref{InitoGeminalYieldsGeminal} does not give us new introspective theories. Rather, it takes the introspective theory of geminal gadgets (which we already constructed in \TODO) and constructs a model of it, for suitable notions of \quote{gadget}.
\end{warningenv}

Having established that NNO-toposes have initial internal models of all finitely axiomatizable theories, it follows that any finitely axiomatizable theory extending the theory of NNO-toposes is \initogeminal/.

Due to work by Maietti et al following in the footsteps of Joyal (\TODOinline{cite}), it is also known that any arithmetic universe contains an internal initial model of any finitely axiomatizable theory.

Thus, also, the theory of arithmetic universes is \initogeminal/, and thus the initial arithmetic universe can be equipped as a geminal arithmetic universe (a fortiori, a geminal category). Thus, we get \Godel/'s incompleteness results manifesting within the initial arithmetic universe. This is the structure discussed by Joyal in unpublished work on a category-theoretic account of \Godel/'s incompleteness theorem, and further discussed by others after Joyal (see in particular \autocite{van2020g}).

The fact that every NNO-topos contains an initial model of every finitely axiomatizable lex theory can of course be taken as a special case of the fact that every arithmetic universe has the same property, since NNO-toposes are straightforawrdly arithmetic universes. But the construction of initial internal models in an NNO-topos can also be carried out by much easier means than are available in an arbitrary arithmetic universe; e.g., as in the proof sketch we gave at \magicref{NNOToposIsInitoGeminal}, which made essential use of cartesian closure, quantification over power objects, and the like.

Of course, we could directly consider the theory of a lexcategory with an initial internal model for every finitely axiomatizable theory. This would be interno-geminal... if it were finitely axiomatizable. In the form we just stated this theory, it was axiomatized infinitely (there is a separate imposed basic constructor for every particular finitely axiomatizable theory). It is an open question to this author whether this theory admits some alternative finite selection of basic constructors allowing it to be finitely axiomatized.

\TODOinline{Still, stress that finite axiomatizability isn't the key thing. We have after all the theory of a topos with countable coproducts as a \initogeminal/ theory}.

It is a similarly open question whether there is an initial \initogeminal/ theory (in either the sense without or with the parenthetical condition noted at \magicref{InitoGeminalYieldsGeminal}). The theory of \initogeminal/ theories is not known to be equivalent to any lex theory (the condition for $T$ to be \initogeminal/ involves a higher-order quantification over all endolexfunctors of $T$), so we do not automatically have the existence of an initial such structure.

Just as with \magicref{InitoGeminalWarning}, we should remark that again, as of yet, we have only equipped the initial arithmetic universe as a geminal category, not an introspective theory. But it will turn out that, unlike the typical situation for a \initogeminal/ theory as with the initial NNO-topos, the theory of arithmetic universes is so special that we can in fact further equip it in a natural way as an introspective theory! We shall come back to this at the end of the next section, after developing some more tooling for constructing more sophisticated introspective theories from \initogeminal/ theories in general.

\TODOinline{Reorganize order of paragraphs here for clarity. Discuss toposes with k-ary coproducts as mentioned above.}

\TODOinline{Note that we have a soundness result for the geminal categories we get from the initial NNO-topos, the initial arithmetic universe, etc: Their internal views of the initial such-and-such do in fact match themselves; the uniquely determined structure-preserving (AU-preserving or NNO-topos-preserving) functor from these things to Set takes their internal initial such-and-such to themselves. We don't have this kind of soundness for all \initogeminal/ theories. For example, we might consider the theory of bloposes, where a blopos is an NNO topos in which the internal initial NNO topos is trivially 1. That is, a blopos is a topos that thinks the theory of toposes is inconsistent. Then the theory of bloposes incorrectly proves that the theory of bloposes is inconsistent, so the initial blopos is nontrivial, but its internal initial blopos is trivial. The key thing that makes bloposes different from toposes is that Set is itself a topos (and an AU and so on), but not a blopos. Our soundness result is only for those theories which Set itself models.}

\subsubsection{The theory of initial models as an introspective theory}
Throughout the following, we use the \quote{initial model} terminology of \magicref{LexModelTerminology}. As a reminder, we say a lexfunctor $f : T \to S$ is an initial model of $T$ if it is initial within the category $\LexCat(T, S)$.

By the 2-category $\initMod{T}$, we mean $\LexCat$ with its objects restricted to just those lexcategories with initial models of $T$, and its $1$-cells restricted to just those lexfunctors which preserve initial models of $T$. (The $2$-cells remain unchanged.)

\begin{theorem}
For every (\setsmall/) lexcategory $T$, there is an initial object within $\initMod{T}$.
\end{theorem}
\begin{proof}
This is by some general initial object of small algebraish theories theorem  we should establish somewhere, that we will be using over and over. \TODO
\end{proof}

\begin{theorem}\label{InitialInitializerStrongSigmaesque}
If $S$ is the initial object of $\initMod{T}$, then for any other lexcategory $U$ in $\initMod{T}$, we get that $!$ is initial within $\LexCat(S, U)$, where $! : S \to U$ is the unique lexfunctor from $S$ to $U$ which preserves the initial model of $T$.

In particular, the identity on $S$ is initial among all endolexfunctors of $S$.
\end{theorem}
\begin{proof}
This is the instance of \magicref{CommaKanStrongSigmesque} where we take $D$ to be $\LexCat$, with $Special$ as the sub-2-category $\initMod{T}$. The precondition for invoking \magicref{CommaKanStrongSigmesque} in this context (i.e., the left comma-stability of $\initMod(T)$ within $\LexCat$) is given by \magicref{InitialModelCommaStable}.
\end{proof}

In this way, we get an introspective theory on this $S$ once it also has a lexfunctor in the globalization of some internal category in $T$ (or, for that matter, in itself...). \TODO

More generally, let $T$ and $T_2$ be lexcategories, and $i : T \to T_2$ be given, such that $T_2$ is thought of as a lex theory extending $T$. \TODO

\subsubsection{A self-initializing theory with uncountable flavor}
\TODOinline{Consider the free topos with countable (co)products, which has an internal free topos with countable (co)products as well, with internal and external views coinciding on what things have or preserve countable coproducts.}

\subsubsection{A self-initializing theory with countable but uncomputable flavor}

\TODO

\subsection{The initial arithmetic universe}
\begin{theorem}\label{AUStrongSigma1esque}
Every arithmetic functor out of the initial arithmetic universe $\IAU$ is an initial model of $\IAU$. That is, if $\omega$ is any arithmetic universe, $! : \IAU \to \omega$ is the unique arithmetic functor between these, and $f: \IAU \to \omega$ is an arbitrary lexfunctor, then there is a unique natural transformation from $!$ to $f$.

In particular, the identity on $\IAU$ is initial among all endolexfunctors of $\IAU$.
\end{theorem}
\begin{proof}
This is the instance of \magicref{CommaKanStrongSigmesque} where we take $D$ to be $\LexCat$, with the special $0$-cells as the arithmetic universes and the special $1$-cells as the arithmetic functors. The precondition for invoking \magicref{CommaKanStrongSigmesque} in this context (i.e., the left comma-stability of these special cells) is given by \magicref{CommaStableArithmetic}.
\end{proof}

\begin{construction}\label{IAUAsIntrosp}
Any arithmetic universe contains an initial internal model of any finitely axiomatizable lex theory (\TODOinline{cite Maietti}). Furthermore, the theory of (strict) arithmetic universes is itself finitely axiomatizable. Thus, the initial arithmetic universe IAU also contains an initial internal arithmetic universe IAU'. By initiality, there is an arithmetic functor $\introS$ from IAU to the global aspect of IAU'. These provide three of the four ingredients for an introspective theory. From \magicref{AUStrongSigma1esque}, the natural transformation $\introN$ is also uniquely determined. (This natural transformation is the same as the one constructed in Lemma 5.15 of \autocite{van2020g}). This completes our construction of $\langle IAU, IAU', \introS, \introN \rangle$ as an introspective theory.
\end{construction}

\TODOinline{Show that this introspective theory structure extends the geminal category structure on IAU already obtained in previous sections by mere initiality properties (this is clear already from the uniqueness result on \initogeminal/ theories' corresponding geminal gadgets). Conclude that IAU in fact comprises a \quote{super-duper introspective theory}.}

\TODOinline{Note that we could equally well carry out an introspective theory construction on IAU for ANY internal category admitting a lexfunctor from IAU. Thus, the PA, ZFC, etc type models. It is not at all important that we use IAU' as the internal initial arithmetic universe. It's not even important that IAU admits internal initial models of finitely axiomatizable theories, except insofar as this helps us build internal arithmetic universes or other internal categories admitting lexfunctors from IAU.}

\TODOinline{Discuss the extension of this to sigma_1 localizations of IAU. Compare this to our previous description of the sigma_1 ZF-finite/Peano Arithmetic/etc model}

Let $T_{IAU}$ be some quotient of IAU' internal to IAU. Then the arithmetic functor from IAU to Set takes $T_{IAU}$ to an actual arithmetic universe T, such that $T$ is equivalent to the global aspect of $T_{IAU}$ \TODOinline{proof? We must show that the global sections functor on IAU is an arithmetic functor}. Furthermore, the arithmetic functor from IAU to T takes $T_{IAU}$ to an arithmetic universe $T'$ internal to $T$.

\begin{theorem}
The global sections functor $\Hom_{\IAU}(1, -)$ is the unique arithmetic functor from the initial arithmetic universe $\IAU$ to $\Set$.
\end{theorem}
\begin{proof}
A unique arithmetic functor $!$ from IAU to $\Set$ is known to exist by the initiality of IAU (keeping in mind \magicref{InitialWrtSet}). What remains is only to show that this $!$ is the same as the global sections functor. By \magicref{TermModelIsInitialForLex}, we know that the global sections functor is initial among lexfunctors from $\IAU$ to $\Set$. But by \magicref{AUStrongSigma1esque}, we know that $!$ is also initial among these. Thus, $!$ and the global sections functor are isomorphic (indeed, uniquely isomorphic), completing the proof.
\end{proof}

With this last theorem, we must be careful. As it invoked \magicref{InitialWrtSet}, its reasoning does not internalize. In particular, we do NOT know internal to $\IAU$ that the global sections functor from $\IAU'$ to the self-indexing $\IAU/-$ is arithmetic, or even that it preserves the initial object (this would violate \Goedel/'s second incompleteness theorem).

\TODOinline{Note that basically nothing here is special about IAU. We could similarly construct introspective theories using initial objects of any structure left comma-stable over LexCat, given any structure of the same kind internal to the initial one. What's special about IAU is just that it happens to actually contain interesting structures internally (such as PA, ZFC, the initial internal AU, etc), whereas the initial lexcat, or initial regular category, or initial lexcat with finite pullback-stable colimits, or such things all don't have much interesting internally. An example that DOES have interesting internal structure, however, might be the free lexcategory with pullback-stable disjoint etc countable coproducts.}

\subsection{Models based on well-founded posets or semicategories}
There are two flavors of models here: Those which give introspective theories (these come from well-founded trees using a certain size restriction; e.g., considering a model based on the von Neumann universe/cumulative hierarchy), and those which give only locally introspective theories with \Loeb/'s theorem fixed points (these come from arbitrary well-founded trees; these are related to the models used in guarded recursion theory, but our distinction between the roles of $T$ and $C$ has previously gone unnoticed and allows us to interpret these models as not proving $\lnot \lnot \Box 0$). We discuss the latter construction first, as it is simpler, and a step en route to grasping the former construction.

Previous iterations of this document at this point gave an overly complicated way of describing something simple (though still good to understand):

First of all, let $Disc$ be an arbitrary (set-sized) category. This gives rise also to the category $\Psh{Disc}$ of presheaves on $Disc$, which is automatically a lexcategory, and indeed locally cartesian closed. By the observation of \cref{TrivialPreIntrosp}, this yields a locally introspective theory $\langle \Psh{Disc}, \Psh{Disc}/-, \id \rangle$.

Now, let $f : Disc \to Struct$ be an arbitrary functor from $Disc$ into an arbitrary (also set-sized) category $Struct$. This induces by composition a functor $f^* : \Psh{Struct} \to \Psh{Disc}$. This $f^*$ preserves pullbacks (as pullbacks are computed pointwise in presheaf categories. Indeed, $f^*$ furthermore preserves all limits, as it has a left adjoint given by left Kan extension). This $f^*$ also has a right adjoint (given by right Kan extension).

\TODOinline{The above tells us that any geometric morphism between locally cartesian closed categories induces in the same way a locally introspective theory.}

By now using \cref{IntrospPullback} with our functor $f^*$ as applied to our first locally introspective theory $\langle \Psh{Disc}, \Psh{Disc}/-, \id \rangle$, we get a second locally introspective theory $\langle \Psh{Struct}, \Psh{Disc}/- \circ f^*, \ldots \rangle$.

This is ALMOST the locally introspective theory we are interested in for Kripke semantics. But it needs to be massaged a bit more, in a manner requiring some further assumptions. \TODOinline{Note that if we stop right here, we get a natural notion of model corresponding to S4 Kripke frames.}

First, a lemmatic construction. Suppose given any arbitrary profunctor $H : X \profuncTo Y$. This $H$ induces by profunctor composition (with profunctors $:1 \profuncTo X$, which correspond to presheaves on $X$) correspondingly an ordinary functor $H \circ - : \Psh{X} \to \Psh{Y}$. Note that this functor $H \circ -$ has a right adjoint (right Kan lift of a profunctor along a profunctor).

(Alternatively, we can think of the above like so: Given (set-sized) categories $X$ and $Y$ and any arbitrary functor $H : X \to \Psh{Y}$, this extends uniquely to a (set-sized-)colimit preserving functor $: \Psh{X} \to \Psh{Y}$, as $\Psh{X}$ is the free cocompletion of $X$ (with respect to set sized colimits). This functor is the one we call $H \circ -$, and by the Special Adjoint Functor Theorem, it will have a right adjoint.)

\TODOinline{Wherever above I put a set-sized constraint on a category, it sounds like I am constraining the category to not be too large. But really what this amounts to is to say that the corresponding presheaf category we are considering must not be too small: they must include presheaves of sufficiently high cardinality relative to the original category.}

If given two such $H_1, H_2$ and a transformation $n : H_1 \to H_2$, this extends also to a transformation $n \circ -$ from $H_1 \circ -$ to $H_2 \circ -$ as ordinary functors $: \Psh{X} \to \Psh{Y}$.

Let us now suppose that $Struct$ (from before) is in fact the free category adding identities to some semicategory $Struct^-$. Then we have a bifunctor $\Hom_{Struct^-} : \op{Struct} \times Struct \to \Set$, as the morphisms of $Struct^-$ are not only closed under composition with each other, but also (trivially) under composition with identities on either side, and thus closed under composition on either side with the morphisms of $Struct$. 

This bifunctor $\Hom_{Struct^-} : \op{Struct} \times Struct \to \Set$ comes with an inclusion transformation to the bifunctor $\Hom_{Struct} : \op{Struct} \times Struct \to \Set$. These bifunctors can both be read as profunctors from Struct to Struct; the latter is in fact the identity bifunctor on Struct, and the former is what we will take to be our $H$ as above. The inclusion transformation thus will become an inclusion transformation $i$ from $H \circ -$ to identity as functors $: \Psh{Struct} \to \Psh{Struct}$.

These comprised the last ingredients we needed for proper Kripke semantics for irreflexive frames. Remember, we already had a locally introspective theory $\langle \Psh{Struct}, \Psh{Disc}/- \circ f^*\rangle$ from above. Let us call this $\langle \Psh{Struct}, C \rangle$ for convenience. We now modify it like so using: \cref{IntrospInternalMap}.

% https://q.uiver.app/?q=WzAsMyxbMCwwLCJcXG9we1xcUHNoe1N0cnVjdH19Il0sWzIsMCwiXFxMZXhDYXQiXSxbMSwyLCJcXG9we1xcUHNoe1N0cnVjdH19Il0sWzAsMSwiXFxQc2h7U3RydWN0fS8tIiwwLHsib2Zmc2V0IjotMn1dLFswLDEsIkMiLDIseyJvZmZzZXQiOjJ9XSxbMCwyLCJIIFxcY2lyYyAtIiwyXSxbMiwxLCJDIiwyXSxbMyw0LCIiLDAseyJzaG9ydGVuIjp7InNvdXJjZSI6MjAsInRhcmdldCI6MjB9fV0sWzQsMiwiQyBcXG9we2l9IiwxLHsic2hvcnRlbiI6eyJzb3VyY2UiOjIwfX1dXQ==
\[\begin{tikzcd}
	{\op{\Psh{Struct}}} && \LexCat \\
	\\
	& {\op{\Psh{Struct}}}
	\arrow[""{name=0, anchor=center, inner sep=0}, "{\Psh{Struct}/-}", shift left=2, from=1-1, to=1-3]
	\arrow[""{name=1, anchor=center, inner sep=0}, "C"', shift right=2, from=1-1, to=1-3]
	\arrow["{H \circ -}"', from=1-1, to=3-2]
	\arrow["C"', from=3-2, to=1-3]
	\arrow[shorten <=1pt, shorten >=1pt, Rightarrow, from=0, to=1]
	\arrow["{C \op{i}}"{description}, shorten <=7pt, Rightarrow, from=1, to=3-2]
\end{tikzcd}\]

Keeping in mind that $H \circ - : \Psh{Struct} \to \Psh{Struct}$ has a right adjoint, we may conclude that the result is a locally introspective theory. When we start off taking $Struct^-$ to be a Kripke frame presumed transitive but not reflexive, taking Disc to be the discrete category on the same objects as Struct-, and $f : Disc \to Struct$ to be the inclusion, then the result of the above process is the introspective theory which corresponds to Kripke semantics on $Struct^-$. \TODOinline{Write out in more detail what the construction comes down to and thus showing how it corresponds to traditional Kripke semantics.}

The result will be locally Loeb when the order on the objects of Struct- given by its morphisms is a converse well-founded order. \TODOinline{Expand on this}. We can then impose a suitable size constraint to get it to be fully introspective.

\begin{TODOblock}
Clarify the size constraint. Note that it is very common in mathematics to take the relative point of view on Set, in terms of Grothendieck universes or the like, so as to consider the topos $Set^K$ as built up from a bunch of full subtoposes defined by a global cardinality constraint: those presheaves whose cardinality at each object is constrained by an upper bound, and this upper bound is the same at each object. But there is no reason we must only consider such constant upper bounds. We can just as well consider all kinds of varying upper bounds. And by allowing the the upper bounds to vary in the appropriate way, growing sufficiently fast, we get that $Set^K$ is built up from a bunch of full subtoposes which are all introspective theories. It is like a shift of frame of reference, to allow the upper bounds to vary with suitable \quote{slope} instead of having to be constant. But it serves all the same purposes as the very standard move in mathematics, of taking a relative point of view on Set.

Note that while typical categorical arguments work within structural set theory, the above can be done most readily within a material set theory. Furthermore, while typical categorical arguments work within the internal logic of toposes with NNO or some such thing, the above requires us to move beyond this, and is done most readily using the Axiom of Replacement. Thus, ZF or IZF or the like. Specifically, take Set to be a material set theory and a strict lexcategory, and take a cardinality constraint at a node to be a set of sets (corresponding to a full subcategory of Set) satisfying the condition that this full subcategory is closed under finite limits. Then we furthermore impose the condition that the full subcategory at node X contains the small category of all discrete presheaves on < X and all natural transformations between them. Using transfinite induction, we can easily define a function from nodes to sets that has this property.

The reason we must use the Axiom of Replacement is essentially because the initial algebra/transfinite recursion properties of well-founded sets within a mere topos $T$ are only with respect to endomorphisms of the subobject functor (which is representable, and thus such endomorphisms are themselves represented by endomorphisms on $\Omega$, living internally to the category), and not with respect to natural transformations of the self-indexing more generally (which is not representable, and thus its endomorphisms are not given by some internal data). Even simple natural transformations of the self-indexing such as the powerset operation on indexed sets may not admit corresponding catamorphisms defined by induction (e.g., there is in general no slice above the natural numbers in which the fiber of n + 1 is the powerset of the fiber of n).
\end{TODOblock}

\begin{TODOblock}
The above results immediately imply that the theorems of modal logic which hold for all locally introspective theories are no stronger than those which hold for all transitive Kripke frames, and the theorems which hold for all introspective theories or the theorems which hold in all locally Loeb theories are no stronger than those which hold for all transitive converse well-founded Kripke frames. From this, we can readily conclude that the theorems which hold in all locally introspective theories are K4 and the theorems which hold in all introspective theories or the theorems which hold in all locally Loeb theories are GL. Does the last two of these coinciding help us embed every locally introspective theory into an introspective theory, in the same way as we did for the unconstrained vs constrained presheaf models of GL Kripke frames?
\end{TODOblock}

\TODOinline{LaTeXify the above better}

\begin{TODOblock}
Give topos of trees example as well. This is what happens when we take $f$ as the identity functor and $Struct = Disc$ as the free category on some semicategory (in particular, the semicategory of natural numbers with strict reverse ordering). Note that this is an example of an introspective theory in which the functor from the introspective theory to the global aspect of the geminal category is an equivalence of categories (probably an equivalence of geminal categories, even? Thus, what we were calling a GLS-category...). Our $\Box$ operator becomes, on this category, what Birkedal et al call the step operator. This has a left adjoint as well, what Birkedal et al call the constant set operator. It's likely that in general we have left adjoints for these models based on well-founded semicategories.

Actually, many of the things we cite to Birkedal are already anticipated in "Unifying Recursive and Co-recursive Definitions in Sheaf Categories" by Pietro Di Gianantonio Marino Miculan.
\end{TODOblock}

\subsection{Recap}
\TODO

\fileend