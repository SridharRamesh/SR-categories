\section{Models}

\subsection{Automatic consistency results without models as such}
We already know that the theory of geminal categories is an introspective theory. And because every introspective theory is itself a geminal category, we know that the theory of introspective theories only prove $\Box A$ if it furthermore proves $A$.

Finally, we know that every lexcategory can be equipped as a geminal category in a trivial way, by taking its internal geminal category to be trivially $1$, even when the outer lexcategory needn't be trivial. From this, we can conclude that the theory of introspective theories is nontrivial in the sense that it does not prove its internal geminal category to be trivial. Thus, it does not prove $\Box A$ for all $A$. Furthermore, combining this with the previous paragraph, we have the stronger consistency result that for every $n$, the theory of introspective theories does not prove $\Box^n A$ for all $A$.

In this way, simply by consideration of the freeness properties already established in the chapter on geminal categories, we already know the theory of introspective theories to have highly nontrivial content, even without needing to find any models of it \quote{in the wild}.

\TODOinline{Discuss more why the stronger consistency result is really the relevant thing to think about.}

\subsection{Models based on \texorpdfstring{$\Sigma_1$}{Sigma-1} or arbitrary extensions of PA, or ZFC, or etc}
\TODOinline{I will write this section in a sloppy way for now and then improve it later.}

This section reviews and builds upon the construction previously seen at \cref{SigmaModelSimple}.

\begin{construction}\label{Sigma1Model}

Consider a sigma-1 theory $\tau$ extending PA (or ZFC, or any such thing), in the sense of an extension whose axioms are computably enumerable. Actually, for now, let's just consider PA simpliciter.

\TODOinline{It probably isn't easy to pin down in a clean way exactly the minimal kind of system in which this goes through, but it could be useful to name some weak subsystems of arithmetic in which it goes through. In particular, we should not expect this to go through in Robinson's Arithmetic Q which lacks induction entirely, but we should expect it to still work in systems that just have induction for $\Sigma_1$ formulae).}

Consider the category $T$ whose objects are the sigma-1 formulas $\phi(n, m)$ in the language of PA which define binary relations on the natural numbers which PA proves to be partial equivalence relations (i.e., symmetric and transitive). Given any two such formulas $\phi(n, m)$ and $\psi(n, m)$, a morphism in $T$ from $\phi$ to $\psi$ is a sigma-1 formula $F(n, m)$ on the natural numbers which PA proves to correspond to the graph of a function between the subquotients of $\mathbb{N}$ corresponding to $\phi$ and to $\psi$, respectively. That is, such that PA proves the universal closures of the following:

$F(n, m) \implies \phi(n, n) \wedge \psi(m, m)$

$\phi(n_1, n_2) \wedge \psi(m_1, m_2) \wedge F(n_1, m_1) \implies F(n_2, m_2)$

$\phi(n, n) \implies \exists m [F(n, m)]$

$F(n, m_1) \wedge F(n, m_2) \implies \psi(m_1, m_2)$.

Two such formulas $F(n, m)$ and $F'(n, m)$ are considered to be equal as morphisms from $\phi$ to $\psi$ if PA proves them to be equivalent (that is, if PA proves $F(n, m) \implies F'(n, m)$ and $F'(n, m) \implies F(n, m)$).

Given morphisms $F : \phi \to \psi$ and $G: \psi \to \chi$ of this sort, we define their composition in the usual way of composing functions represented as graphs, as $(F \circ G)(n, m) = \exists p [G(n, p) \wedge F(p, m)]$.

This all describes the category $T$, which one can verify is indeed a category and moreso, a category with finite limits.

\TODOinline{Perhaps instead of imposing PERs from the beginning, we start only with the category of RE sets, and then take its ex/lex completion or some such thing. Like so:}

Consider the category $T'$ whose objects are the sigma-1 formulas $\phi(n)$ in the language of PA, and such that a morphism from $\phi(n)$ to $\psi(m)$ is a sigma-1 formula $F(n, m)$ such that $PA$ proves $\forall n, m . F(n, m) \implies (\phi(n) \wedge \psi(m))$ and $\forall n . \phi(n) \implies \exists! m . F(n, m)$. Two such morphisms $F(n, m)$ and $G(n, m)$ are considered equal just in case PA proves $\forall n, m . F(n, m) \biimplies G(n, m)$. Morphisms compose in the obvious way; that is, the composition of $F(n, p)$ with $G(p, m)$ is given by $(G \circ F)(n, m) = \exists p (F(n, p) \wedge G(p, m))$.

This category $T'$ is regular but not exact (that is, not every equivalence relation in $T'$ admits a corresponding quotient). Let $T$ be its ex/reg completion.

\TODOinline{Now, we describe the C inside T which is its internal copy, just by carrying out this exact same construction internal to T, and then we describe the indexed lexfunctor from T to C, which is a little more interesting or takes a little more care. Having this functor be indexed is where the sigma-1 restriction is important.}
\end{construction}

\TODOinline{Observe that we have somewhat distinct concepts of "T = PA Sigma-1, C = ZFC Sigma-1" vs "T = ZFC Sigma-1, C = ZFC Sigma-1", say. Also observe that as concerns ZFC, we can also consider for $C$ not just categories of definable subsets of naturals, but also of definable sets in general, or of definable classes.}

\subsection{Finitely axiomatizable lex theories}
A concept that will often be useful to us in the following.

\begin{definition}
A \defined{finitely axiomatizable lex theory} is a lex theory which, qua lexcategory, can be generated in finitely many steps of the following form, starting from the initial lexcategory: free augmentation with an object, free augmentation with a morphism between existing objects, or freely making two existing parallel morphisms equal. In other words, it can be presented by a finite lex \quote{sketch}. \TODOinline{Word this all better}
\end{definition}

\begin{TODOblock}
Discuss the concept of a lexcategory having initial internal models of ALL finitely axiomatizable lex theories.

As a bit of trivia, observe how this follows simply from having an internal free locally cartesian closed category on one object (verify the details on this; or perhaps from having internal free lex categories and the ability to freely augment internal lex categories with a new cell). Regardless of whether those details work out, conjecture that there are finitely many finitely axiomatizable lex theories such that having internal initial models of those implies having internal initial models of all finitely axiomatizable lex theories, so that the latter is itself a finitely axiomatizable condition.

Relate this also to the concept of arithmetic universes. Conjecturally, being an arithmetic universe is equivalent to something like having free internal models for sketches indexed by finite unions of internal objects (but there seems to be some hesitance in the literature to claim this? Understand that better). At any rate, an arithmetic universe should have internal initial models of all finitely axiomatizable lex theories.

This section basically only exists in order to claim that finitely axiomatizable lex theories which extend the theory of arithmetic universes are automatically examples of the next section.
\end{TODOblock}

\subsection{Theories with free internal models of themselves}
Fix some lex theory extending the theory of strict lexcategories, whose models we shall call \quote{gadgets}. Now suppose every gadget $G$ contains an initial internal gadget $G'$ (in the sense that $G'$ is a gadget internal to the underlying lexcategory $|G|$ of $G$, and for any other gadget $H$ internal to $|G|$, there is a unique $|G|$-internal gadget homomorphism from $G'$ to $H$).

In particular, then, this all applies to the actual initial gadget $G$. Internal to its underlying lexcategory $|G|$, we get a $G'$ as above. Because $G$ is initial, we automatically get a unique gadget-homomorphism $\introN_{G}$ from $G$ to $\Hom_{|G|}(1, G')$ as well. And because $G'$ is an initial $|G|$-internal gadget, we automatically get a unique $|G|$-internal homomorphism $\introN_{G'}$ from $G'$ to $\Hom_{|G'|}(1, G'')$ where $G'' = \introN_{G}[G']$.

This setup is thus a geminal gadget (with axioms 3 and 3' of a geminal gadget automatically satisfied by the uniqueness observations in the previous paragraph).

Indeed, this is the unique way to equip $G$ as a geminal gadget with internal gadget $G'$.

This immediately gives us models of our theory of geminal gadgets. For example, consider the theory of strict elementary toposes with natural numbers objects (let us call this an \defined{NNO-topos}, to make it less of a mouthful). This is indeed a lex theory in a straightforward way; indeed, a finitely axiomatizable lex theory. Furthermore, every NNO-topos has an internal initial model of every finitely axiomatizable lex theory. Thus, in particular, every NNO-topos has an internal initial NNO-topos, and thus, by the above, the initial NNO-topos is equipped as a geminal NNO-topos.

\TODOinline{It is important to observe that the initial NNO-topos is NOT an introspective theory. It is merely a geminal category.}

In the same way, the initial arithmetic universe is equipped as a geminal arithmetic universe. This is the structure discussed by Joyal and others after Joyal (e.g., Dijk and Oldenziel). \TODOinline{Give proper citations here}

\TODOinline{The initial arithmetic universe actually IS an introspective theory, via reasoning about Freyd covers that does not generalize to other examples of this section's phenomena. Discuss this further. This introspective theory extends the geminal arithmetic universe of the previous construction with a suitable natural transformation; furthermore, as its internalization functor preserves AU structure and thus takes its internal initial AU to ITS internal initial AU, the internal geminal AU in this introspective theory is also the internalization of the previous geminal gadget construction. This last part generalizes to any geminal gadget constructed in the previous way: Its internal geminal gadget will also have been constructed by the internalization of the previous construction.}

\begin{TODOblock}
Extend the above discussion of initiality to discuss corresponding introspective theories, beyond just the discussion of geminality. Although, e.g., the initial NNO-topos is not itself an introspective theory, there is nonetheless a corresponding introspective theory of note capturing the initiality properties. Furthermore, even any lex theory that does not automatically come with internal initial models of itself can be freely bumped up to do so, in a suitable sense, and we get a corresponding introspective theory as well.

Specifically, let us say a theory is sigma1esque if the identity lex endofunctor on it is initial with respect to all lex endofunctors on it. (Do we need full initiality? Perhaps weak initiality is all we care about.) Any lex theory $T$ has a free sigma1esque extension, in the sense of a sigma1esque lex theory $T'$ under $T$ with a unique map into any other sigma1esque lex theory under $T$. This $T'$ also comes with a unique map into the global points of any internal initial model of $T$ in any lex category.

If every model of $T$ has an internal initial model of $T$ (as would be the case if the generic model of $T$ has such an internal initial model AND $T$ is given by a finite lex sketch, so that this property is capturable by a lex statement), then the theory $T'$ also contains an internal initial model of $T$, and thus this theory $T'$ comes with a lexfunctor $\introS$ into that internal initial model of $T$. Furthermore, because $T'$ is sigma1esque, it then comes with a natural transformation $\introN$ from its identity functor to the global sections of that $\introS$. This equips said $T'$ as an introspective theory. And, as noted, this $T'$ is modelled by any initial model of $T$ internal to any category.

Thus, for example, if $T$ is the theory of elementary toposes with NNO, we obtain some introspective theory $T'$ extending $T$ which has the initial elementary topos with NNO as a model.
\end{TODOblock}

\begin{TODOblock}
The initial arithmetic universe $U$ probably satisfies a property slightly stronger than being sigma1esque. Given any lex functor $F$ from $U$ to other arithmetic universe $V$, it should be the case that there is a unique natural transformation from $!_V$ to $F$, where $!_V$ is the unique AU functor from $U$ to $V$. This is because the comma category $(!_V / f)$ is an arithmetic universe and its projection to $U$ along both coordinates are AU maps (\TODO. The first half is Artin gluing same as for toposes, I believe. That the projection in the second coordinate is also an AU map is different than from toposes, though, on which the projection is not a logical functor). Those projections are both identity, therefore, and thus our unique AU map from $U$ into the comma category provides a natural transformation from $!_V$ to $f$.

We now know that we have at least one such functor (the unique AU map from $A$ to $(A/f)$). But how do we see uniqueness (qua functor, not uniqueness qua AU map)? Perhaps a more careful analysis of the gluing construction will tell us that a functor into the comma category is an AU map just in case both its projections are? That is, the forgetful functor from the comma category to the product category creates (i.e., preserves AND reflects) AU structure. [Is this what we want? Is creating structure the same as saying that the forgetful functor composed with any other functor preserves and reflects the property of AU-structure-preservingness in the other functor? Yes, I think so.] This should be relatively straightforward to show for limits and colimits, by general comma category properties, and then also for list objects. \TODO
\end{TODOblock}

\subsection{Models based on well-founded trees/well-founded posets}
There are two flavors of models here: Those which give introspective theories (these come from well-founded trees using a certain size restriction; e.g., considering a model based on the von Neumann universe/cumulative hierarchy), and those which give only locally introspective theories with \Loeb's theorem fixed points (these come from arbitrary well-founded trees; these are related to the models used in guarded recursion theory, but our distinction between the roles of $T$ and $C$ has previously gone unnoticed and allows us to interpret these models as not proving $\lnot \lnot \Box 0$). We discuss the latter construction first, as it is simpler, and a step en route to grasping the former construction.

Previous iterations of this document at this point gave an overly complicated as a way of describing something simple (though still good to understand):

First of all, let $Disc$ be an arbitrary (set-sized) category. This gives rise also to the category $\Psh{Disc}$ of presheaves on $Disc$, which is automatically a lexcategory, and indeed locally cartesian closed. By the observation of \cref{TrivialPreIntrosp}, this yields a locally introspective theory $\langle \Psh{Disc}, \Psh{Disc}/-, \id \rangle$.

Now, let $f : Disc \to Struct$ be an arbitrary functor from $Disc$ into an arbitrary (also set-sized) category $Struct$. This induces by composition a functor $f^* : \Psh{Struct} \to \Psh{Disc}$. This $f^*$ preserves pullbacks (as pullbacks are computed pointwise in presheaf categories. Indeed, $f^*$ furthermore preserves all limits, as it has a left adjoint given by left Kan extension). This $f^*$ also has a right adjoint (given by right Kan extension).

By now using \cref{IntrospPullback} with our functor $f^*$ as applied to our first locally introspective theory $\langle \Psh{Disc}, \Psh{Disc}/-, \id \rangle$, we get a second locally introspective theory $\langle \Psh{Struct}, \Psh{Disc}/- \circ f^*, \ldots \rangle$.

This is ALMOST the locally introspective theory we are interested in for Kripke semantics. But it needs to be massaged a bit more, in a manner requiring some further assumptions. \TODOinline{Note that if we stop right here, we get a natural notion of model corresponding to S4 Kripke frames.}

First, a lemmatic construction. Suppose given any arbitrary profunctor $H : X \profuncTo Y$. This $H$ induces by profunctor composition (with profunctors $:1 \profuncTo X$, which correspond to presheaves on $X$) correspondingly an ordinary functor $H \circ - : \Psh{X} \to \Psh{Y}$. Note that this functor $H \circ -$ has a right adjoint (right Kan lift of a profunctor along a profunctor).

(Alternatively, we can think of the above like so: Given (set-sized) categories $X$ and $Y$ and any arbitrary functor $H : X \to \Psh{Y}$, this extends uniquely to a (set-sized-)colimit preserving functor $: \Psh{X} \to \Psh{Y}$, as $\Psh{X}$ is the free cocompletion of $X$ (with respect to set sized colimits). This functor is the one we call $H \circ -$, and by the Special Adjoint Functor Theorem, it will have a right adjoint.)

\TODOinline{Wherever above I put a set-sized constraint on a category, it sounds like I am constraining the category to not be too large. But really what this amounts to is to say that the corresponding presheaf category we are considering must not be too small: they must include presheaves of sufficiently high cardinality relative to the original category.}

If given two such $H_1, H_2$ and a transformation $n : H_1 \to H_2$, this extends also to a transformation $n \circ -$ from $H_1 \circ -$ to $H_2 \circ -$ as ordinary functors $: \Psh{X} \to \Psh{Y}$.

Let us now suppose that $Struct$ (from before) is in fact the free category adding identities to some semicategory $Struct^-$. Then we have a bifunctor $\Hom_{Struct^-} : \op{Struct} \times Struct \to \Set$, as the morphisms of $Struct^-$ are not only closed under composition with each other, but also (trivially) under composition with identities on either side, and thus closed under composition on either side with the morphisms of $Struct$. 

This bifunctor $\Hom_{Struct^-} : \op{Struct} \times Struct \to \Set$ comes with an inclusion transformation to the bifunctor $\Hom_{Struct} : \op{Struct} \times Struct \to \Set$. These bifunctors can both be read as profunctors from Struct to Struct; the latter is in fact the identity bifunctor on Struct, and the former is what we will take to be our $H$ as above. The inclusion transformation thus will become an inclusion transformation $i$ from $H \circ -$ to identity as functors $: \Psh{Struct} \to \Psh{Struct}$.

These comprised the last ingredients we needed for proper Kripke semantics for irreflexive frames. Remember, we already had a locally introspective theory $\langle \Psh{Struct}, \Psh{Disc}/- \circ f^*\rangle$ from above. Let us call this $\langle \Psh{Struct}, C \rangle$ for convenience. We now modify it like so using: \cref{IntrospInternalMap}.

% https://q.uiver.app/?q=WzAsMyxbMCwwLCJcXG9we1xcUHNoe1N0cnVjdH19Il0sWzIsMCwiXFxMZXhDYXQiXSxbMSwyLCJcXG9we1xcUHNoe1N0cnVjdH19Il0sWzAsMSwiXFxQc2h7U3RydWN0fS8tIiwwLHsib2Zmc2V0IjotMn1dLFswLDEsIkMiLDIseyJvZmZzZXQiOjJ9XSxbMCwyLCJIIFxcY2lyYyAtIiwyXSxbMiwxLCJDIiwyXSxbMyw0LCIiLDAseyJzaG9ydGVuIjp7InNvdXJjZSI6MjAsInRhcmdldCI6MjB9fV0sWzQsMiwiQyBcXG9we2l9IiwxLHsic2hvcnRlbiI6eyJzb3VyY2UiOjIwfX1dXQ==
\[\begin{tikzcd}
	{\op{\Psh{Struct}}} && \LexCat \\
	\\
	& {\op{\Psh{Struct}}}
	\arrow[""{name=0, anchor=center, inner sep=0}, "{\Psh{Struct}/-}", shift left=2, from=1-1, to=1-3]
	\arrow[""{name=1, anchor=center, inner sep=0}, "C"', shift right=2, from=1-1, to=1-3]
	\arrow["{H \circ -}"', from=1-1, to=3-2]
	\arrow["C"', from=3-2, to=1-3]
	\arrow[shorten <=1pt, shorten >=1pt, Rightarrow, from=0, to=1]
	\arrow["{C \op{i}}"{description}, shorten <=7pt, Rightarrow, from=1, to=3-2]
\end{tikzcd}\]

Keeping in mind that $H \circ - : \Psh{Struct} \to \Psh{Struct}$ has a right adjoint, we may conclude that the result is a locally introspective theory. When we start off taking $Struct^-$ to be a Kripke frame presumed transitive but not reflexive, taking Disc to be the discrete category on the same objects as Struct-, and $f : Disc \to Struct$ to be the inclusion, then the result of the above process is the introspective theory which corresponds to Kripke semantics on $Struct^-$. \TODOinline{Write out in more detail what the construction comes down to and thus showing how it corresponds to traditional Kripke semantics.}

The result will be locally Loeb when the order on the objects of Struct- given by its morphisms is a converse well-founded order. \TODOinline{Expand on this}. We can then impose a suitable size constraint to get it to be fully introspective.

\TODOinline{Clarify the size constraint. Note that it is very common in mathematics to take the relative point of view on Set, in terms of Grothendieck universes or the like, so as to consider the topos $Set^K$ as built up from a bunch of full subtoposes defined by a global cardinality constraint: those presheaves whose cardinality at each object is constrained by an upper bound, and this upper bound is the same at each object. But there is no reason we must only consider such constant upper bounds. We can just as well consider all kinds of varying upper bounds. And by allowing the the upper bounds to vary in the appropriate way, growing sufficiently fast, we get that $Set^K$ is built up from a bunch of full subtoposes which are all introspective theories. It is like a shift of frame of reference, to allow the upper bounds to vary with suitable \quote{slope} instead of having to be constant. But it serves all the same purposes as the very standard move in mathematics, of taking a relative point of view on Set.}.

\TODOinline{The above results immediately imply that the theorems of modal logic which hold for all locally introspective theories are no stronger than those which hold for all transitive Kripke frames, and the theorems which hold for all introspective theories or the theorems which hold in all locally Loeb theories are no stronger than those which hold for all transitive converse well-founded Kripke frames. From this, we can readily conclude that the theorems which hold in all locally introspective theories are K4 and the theorems which hold in all introspective theories or the theorems which hold in all locally Loeb theories are GL. Does the last two of these coinciding help us embed every locally introspective theory into an introspective theory, in the same way as we did for the unconstrained vs constrained presheaf models of GL Kripke frames?}

\TODOinline{LaTeXify the above better}

\TODOinline{Give topos of trees example as well. This is what happens when we take $f$ as the identity functor and $Struct = Disc$ as the free category on some semicategory (in particular, the semicategory of natural numbers with strict reverse ordering). Note that this is an example of an introspective theory in which the functor from the introspective theory to the global aspect of the geminal category is an equivalence of categories (probably an equivalence of geminal categories, even? Thus, what we were calling a GLS-category...). Our $\Box$ operator becomes, on this category, what Birkedal et al call the step operator.}