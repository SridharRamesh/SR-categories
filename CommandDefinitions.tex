% Commands defined by Sridhar for this document

% Remember: \newtheorem{command name}{displayed name}[counter] makes a theorem whose counter is subordinate to counter, but \newtheorem{command name}[counter]{displayed name} makes a theorem which shares the counter. Also remember that there is automatically a counter available named "section".

% Playing around
\newtheoremstyle{bluestyle}
  {\topsep}
  {\topsep}
  {}
  {}
  {\scshape \color{blue}}
  {}
  {.5em}
  {}

%\theoremstyle{plain}
\theoremstyle{bluestyle}
\newtheorem{theorem}{Theorem}[section]
\newtheorem{lemma}[theorem]{Lemma}
\newtheorem{construction}[theorem]{Construction}
\newtheorem{observation}[theorem]{Observation}
\newtheorem{corollary}[theorem]{Corollary}

%\theoremstyle{definition}
\theoremstyle{bluestyle}
\newtheorem{definition}[theorem]{Definition}

%\theoremstyle{remark}
\theoremstyle{bluestyle}
\newtheorem*{remark}{Remark}

\newcommand{\Nat}{\mathcal{N}}
\newcommand{\Loeb}{L\"ob}
\newcommand{\Hom}{\mathrm{Hom}}
\newcommand{\Set}{\mathrm{Set}}
\newcommand{\op}{\mathrm{op}}
\newcommand{\Ob}{\mathrm{Ob}}
\newcommand{\Mor}{\mathrm{Mor}}
\newcommand{\id}{\mathrm{id}}

\newcommand{\MultiplyInternalSet}{\mathrm{MultiplyInternalSet}}
\newcommand{\Glob}{\mathrm{Glob}}
\newcommand{\SRFunctor}{\mathcal{F}}
\newcommand{\SRTransform}{\mathcal{N}}