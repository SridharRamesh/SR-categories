% Commands defined by Sridhar for this document

% Remember: \newtheorem{command name}{displayed name}[parent counter] makes a theorem whose counter is subordinate to counter, but \newtheorem{command name}[shared counter]{displayed name} makes a theorem which shares the counter. Also remember that there is automatically a counter available named "section".

% Playing around
\newtheoremstyle{bluestyle}
  {\topsep}
  {\topsep}
  {}
  {}
  {\scshape \color{blue}}
  {}
  {.5em}
  {}

%\theoremstyle{plain}
\theoremstyle{bluestyle}
\newtheorem{theorem}{Theorem}[section]
\newtheorem{lemma}[theorem]{Lemma}
\newtheorem{construction}[theorem]{Construction}
\newtheorem{observation}[theorem]{Observation}
\newtheorem{corollary}[theorem]{Corollary}
\newtheorem{corollarytoproof}[theorem]{Corollary (to proof)}

%\theoremstyle{definition}
\theoremstyle{bluestyle}
\newtheorem{definition}[theorem]{Definition}

\newcommand{\defined}{\textbf} % For indicating the defined term in a definition

%\theoremstyle{remark}
\theoremstyle{bluestyle}
\newtheorem*{remark}{Remark}

% This doesn't quite work yet. The word "Proof" is bolded?
% \renewenvironment{proof}{\paragraph{\scshape \color{blue} Proof:}}{\hfill$\square$}

\newcommand{\Nat}{\mathcal{N}}
\newcommand{\Loeb}{L\"ob}
\newcommand{\Hom}{\mathrm{Hom}}
\newcommand{\Set}{\mathrm{Set}}
\newcommand{\op}{\mathrm{op}}
\newcommand{\Ob}{\mathrm{Ob}}
\newcommand{\Mor}{\mathrm{Mor}}
\newcommand{\id}{\mathrm{id}}

\newcommand{\LexCat}{\mathrm{LexCat}}

\newcommand{\MultiplyInternalSet}{\mathrm{MultiplyInternalSet}}
\newcommand{\Glob}{\mathrm{Glob}}
\newcommand{\SRFunctor}{\mathcal{F}}
\newcommand{\SRTransform}{\mathcal{N}}

\newcommand{\indexedToPresheaf}[1]{\hat{#1}}
\newcommand{\presheafToIndexed}[1]{#1_t}