\filestart

\section{Category-theoretic preliminaries}

\subsection{Higher categorical terminology conventions}
We assume familiarity with sets, functions, categories, functors, natural transformations, limits, presheaves, $\Set$, $\Cat$, all in the ordinary sense. At times, we may also call upon some comfort with concepts such as 2-categories or abstract Kan extensions. It will also be very useful to have some familiarity with functorial semantics and internal algebraic structures such as internal categories.

We will take all categories to be locally small (which is to say, we will take $\Set$ to be large enough to include the hom-set between any two objects of any category we work with). Generally speaking, we are interested in the categories we work with being small as well, except for those particular large categories such as $\Set$, $\Set^X$, $\Cat$, etc, but wherever this is important we will make some explicit note.

We write $m \circ n$ or just $mn$ for composition of morphisms $n : X \to Y$, $m : Y \to Z$ in a category. We may occasionally write $n ; m$ to mean $m \circ n$.

We use the term \defined{lexcategory} for a category with finite limits. We use the term \defined{lexfunctor} for a functor preserving finite limits, whose domain and codomain are both lexcategories. By $\LexCat$, we mean the 2-category of lexcategories, lexfunctors, and natural transformations. We will not generally be explicit about making distinctions between $f(a \times b)$ and $f(a) \times f(b)$, etc, when $f$ is a lexfunctor, but shall instead write with the ordinary fluency for working with limit-preserving functors.

We will speak frequently of category-valued presheaves (i.e., contravariant functors into the category of categories) and natural transformations between these. Technically, what we mean by these are not \quote{functors} and \quote{natural transformations} in the traditional sense, but what some call "pseudofunctors" and \quote{pseudonatural transformations}, or \quote{2-functors} and \quote{2-natural transformations}, as the category of categories should be viewed as a 2-category (by which we mean the non-strict concept some call \quote{bicategory}), lacking a notion of equality between its 1-cells and only having a notion of isomorphism between them instead. That is, wherever one might traditionally ask for an (automatically coherent) system of equalities, this is replaced by a coherent system of isomorphisms. We take the convention that this is what terminology such as \quote{functor} and \quote{natural transformation} already means, in such a context. But we will try our best to construct arguments in such a way as that this is not a bother that needs to be explicitly worried about.

Similarly, we do not worry about distinguishing between terms like \quote{isomorphic} and \quote{equivalent} in statements like \quote{category $C$ is isomorphic/equivalent to category $D$}, always meaning by such a statement an adjoint equivalence. Everything means the weakest thing it could mean, unless we explicitly say we are dealing with something stricter.

Similarly, if we ever describe diagrams involving functors between categories as commuting, we really mean that these diagrams commute up to natural isomorphism. If we make claims about uniqueness in such a context, we mean the space of choices with the relevant isomorphisms is contractible. And so on. Again, our convention is that this is what such terminology already means, in any categorical context where one has such concepts of isomorphism around, unless we have taken care to say we are working with stricter notions instead (see more on strictness below). Unless we have said we are talking about strict notions, we never distinguish between equivalent categorical structures.

(That all said, nothing we do is higher-dimensional than 2-categorical, so everything could in theory be strictified in some fashion, if so desired.)

\subsection{Indexed, \repsmall/, and internal structures}
We will now give a series of related definitions, concerning what are called indexed structures. The notions being described in this section are all relatively old hat, none of them are newly invented by us, but we wish to pin them down with particular names to establish a vocabulary for easily talking about the things we wish to talk about in the rest of this document.

As we give these definitions, we will also observe a basic stock of theorems about them which. Again, we make no claim to originality with these preliminaries. They simply may be useful to remind the reader of, or to give labels to in order to reference as we use them.

The reader who is already very familiar with these notions and just unfamiliar with our conventions of vocabulary is advised to just skim these preliminaries on initial read and then return as needed when faced with unfamiliar vocabulary. Frankly, the reader who is not very familiar with these notions is also given similar advice. It is probably best to read a bit of the preliminaries to get the lay of the land, then go off and read the actual content and come back as needed. But who knows? To each reader, their own reading style may be best. \TODOinline{Figure out exactly the roadmap or reading advice we want to give.}

The key notion upon which everything else builds is the following:

\begin{definition}
Let $T$ be an arbitrary category. By a $T$-\defined{indexed set}, we mean a presheaf on $T$; that is, a contravariant functor from $T$ to $\Set$. By a \defined{function} or \defined{map} or any such thing between $T$-indexed sets, we mean a natural transformation between the corresponding presheaves.
\end{definition}

The category of $T$-indexed sets and maps between them is thus the presheaf category $\Set^{\op{T}}$. We may also refer to this as $\Psh{T}$.

We may refer to the data of an indexed set at any object $t$ of the category over which it is indexed as its data \defined{defined over} $t$, or which is $t$-\defined{indexed}, or as its $t$-\defined{aspect}. We can refer to the $t$-aspect of an indexed set $P$ as the set $P(t)$ or $P_t$.

Note that data defined over $t$ is automatically transferred to corresponding data defined over $s$ by any morphism from $s$ to $t$ in $T$, by the action of the presheaf. More explicitly, given morphism $m : s \to t$ in $T$, we may write $P(m) : P(t) \to P(s)$ for the corresponding function in $\Set$, or $P_m$.

In contexts where it is clear what presheaf $P$ we have in mind, we may also write $\pullAlong{m}$ for $P(m)$. Also, in contexts where it would cause no confusion to speak in this way, given some $t$-indexed datum $d \in P(t)$ and a morphism $m : s \to t$, we use the same name $d$ also to refer to the corresponding $s$-indexed datum which more explicitly would be called $P(m)(d)$ or $\pullAlong{m} d$. It will be especially common for us to abuse language in this name-reusing way when $t$ is a terminal object.

In the particular case where $t$ is a terminal object, we may refer to the aspect at $t$ of an indexed set as its \defined{global aspect}. By the Yoneda lemma, this global aspect data $P(1)$ of a presheaf $P$ on category $T$ is the same as the data of a map from the terminal object $1$ to $P$, which is the same as the data of a map from the constantly $1$ presheaf to $P$. This is also the same as the data of the limit of $P$, thought of a $\op{T}$-indexed diagram. In this way, even if $T$ does not have a terminal object, we may still speak of the global aspect of $T$-indexed sets.

\begin{definition}
We say an indexed set is \defined{\repsmall/} (or $T$-\repsmall/, when we wish to emphasize which indexing category we are talking about) if the corresponding presheaf is representable. \TODOinline{I am in the middle of changing instances of "small" in this document to "\repsmall/". They have not all changed yet, so you will see the word \quote{small} still used in much of the document to mean \repsmall/.}
\end{definition}

It is perhaps a bit misleading to use \quote{small}-derived terminology here, as this notion is not closed under subobjects. Indeed, what might be considered the smallest indexed set, the one which constantly takes the value of the empty set, is never \repsmall/ in this technical representability sense. But the analogy to the familiar distinction between \quote{small} sets and non-\quote{small} proper classes is often a fruitful one, motivating this terminology. Cf. Definition 7.3.3 in \autocite{jacobs1999categorical}, which uses the word \quote{small} in essentially the same way. When the indexing category $T$ has finite limits (or even just splittings of idempotents), note that this notion of \repsmall/ is equivalent to the standard notion \quote{tiny}; see \url{https://ncatlab.org/nlab/show/tiny+object#in_presheaf_categories}.

The word \quote{small} of course has a conventional meaning of set-sized (as opposed to proper-class-sized). When we wish to be clear that this is what we mean and not risk confusion with \repsmall/, we will say explicitly \defined{\setsmall/}.

\begin{convention}
Via the Yoneda embedding (which we denote $\yoneda$), we identify $T$ itself as the full subcategory of \repsmall/ $T$-indexed sets within the category of all $T$-indexed sets. In this way, we may speak, for example, of functions from objects of $T$ to $T$-indexed sets. That is, when $t$ is an object of $T$, we will readily write $t$ also to mean the Yoneda embedding of $t$, when we wish to treat it as a \repsmall/ $T$-indexed set; we will usually not explicitly write $\yoneda(t)$. And conversely, given a \repsmall/ $T$-indexed set $P$, we freely write also $P$ to name an object in $T$ representing $P$, rather than explicitly writing $\yoneda^{-1}(P)$.
\end{convention}

\begin{theorem}
Note that \repsmall/ sets, construed as objects of $\Psh{T}$, are closed under any limits which exist in $T$. In particular, if $T$ is a lexcategory, \repsmall/ sets are closed under finite limits. (This is essentially the observation that the Yoneda embedding preserves limits.)
\end{theorem}

\begin{definition}
Note that given an arbitrary functor $f : S \to T$, this induces by composition a functor $\pullAlong{f} : \Psh{T} \to \Psh{S}$.

That is, from a $T$-indexed set $P$, we may construct the following $S$-indexed set $\pullAlong{f} P$:

% https://q.uiver.app/?q=WzAsMyxbMCwwLCJcXG9we1N9Il0sWzIsMCwiXFxvcHtUfSJdLFs0LDAsIlxcU2V0Il0sWzAsMSwiXFxvcHtmfSJdLFsxLDIsIlAiXV0=
\[\begin{tikzcd}
	{\op{S}} && {\op{T}} && \Set
	\arrow["{\op{f}}", from=1-1, to=1-3]
	\arrow["P", from=1-3, to=1-5]
\end{tikzcd}\]
\end{definition}

\begin{theorem}\label{YonedaPullalong}
Given $f : S \to T$ and an object $s$ in $S$ and a $T$-indexed set $P$, we have that $\Hom(s, \pullAlong{f}P) = \Hom(f(s), P)$, with this correspondence being natural in both $s$ and $P$.
\end{theorem}
\begin{proof}
Keep in mind that in these Hom-expressions, $s$ and $f(s)$ have implicitly been construed as $S$-indexed sets via the Yoneda embedding. That is, more explicitly, our claim is $\Hom(\yoneda(s), \pullAlong{f}P) = \Hom(\yoneda(f(s)), P)$. To establish this claim, we can apply the Yoneda lemma to both sides of the equation to reduce it to $(\pullAlong{f}P)(s) = P(f(s))$, which is the definition of $\pullAlong{f}$.

This completes the proof. (In fancy categorical jargon, we have demonstrated that $\yoneda \circ f : S \to \Psh{T}$ is the relative left adjoint of $\pullAlong{f} : \Psh{T} \to \Psh{S}$, relative to the Yoneda embedding $\yoneda : S \to \Psh{S}$.)
\end{proof}

\begin{corollary}\label{YonedaPullalongLemma}
Given $f$, $s$, and $P$ as in \magicref{YonedaPullalong}, we have that every morphism $m : s \to \pullAlong{f} P$ factors through a morphism in the range of $\pullAlong{f}$. That is, $m = \pullAlong{f}(m') \circ \eta$ for some $m' : f(s) \to P$ and $\eta : s \to \pullAlong{f}(f(s))$.
\end{corollary}
\begin{proof}
This is corollary to \magicref{YonedaPullalong} by the general yoga of relative adjoints.

Specifically, consider the following naturality diagram for the correspondence in \magicref{YonedaPullalong}, where $m'$ is the morphism in $\Hom(f(s), P)$ corresponding to $m \in \Hom(s, \pullAlong{f}P)$ and $\eta$ is the morphism in $\Hom(s, \pullAlong{f}f(s))$ corresponding to $\id_{f(s)} \in \Hom(f(s), f(s))$.

% https://q.uiver.app/?q=WzAsOCxbMCwwLCJcXEhvbShmKHMpLCBmKHMpKSJdLFszLDAsIlxcSG9tKGYocyksIFApIl0sWzAsMywiXFxIb20ocywgXFxwdWxsQWxvbmd7Zn1mKHMpKSJdLFszLDMsIlxcSG9tKHMsIFxccHVsbEFsb25ne2Z9UCkiXSxbMSwxLCJcXGlkX3tmKHMpfSJdLFsyLDEsIm0nIl0sWzEsMiwiXFxldGEiXSxbMiwyLCJcXHB1bGxBbG9uZ3tmfShtJykgXFxjaXJjIFxcZXRhID0gbSJdLFswLDEsIm0nIFxcY2lyYyAtIl0sWzIsMywiXFxwdWxsQWxvbmd7Zn0obScpIFxcY2lyYyAtIl0sWzIsMCwiIiwxLHsibGV2ZWwiOjIsInN0eWxlIjp7ImhlYWQiOnsibmFtZSI6Im5vbmUifX19XSxbMywxLCIiLDEseyJsZXZlbCI6Miwic3R5bGUiOnsiaGVhZCI6eyJuYW1lIjoibm9uZSJ9fX1dLFs0LDUsIiIsMSx7InN0eWxlIjp7InRhaWwiOnsibmFtZSI6Im1hcHMgdG8ifX19XSxbNCw2LCIiLDEseyJzdHlsZSI6eyJ0YWlsIjp7Im5hbWUiOiJtYXBzIHRvIn19fV0sWzYsNywiIiwxLHsic3R5bGUiOnsidGFpbCI6eyJuYW1lIjoibWFwcyB0byJ9fX1dLFs1LDcsIiIsMSx7InN0eWxlIjp7InRhaWwiOnsibmFtZSI6Im1hcHMgdG8ifX19XV0=
\[\begin{tikzcd}
	{\Hom(f(s), f(s))} &&& {\Hom(f(s), P)} \\
	& {\id_{f(s)}} & {m'} \\
	& \eta & {\pullAlong{f}(m') \circ \eta = m} \\
	{\Hom(s, \pullAlong{f}f(s))} &&& {\Hom(s, \pullAlong{f}P)}
	\arrow["{m' \circ -}", from=1-1, to=1-4]
	\arrow["{\pullAlong{f}(m') \circ -}", from=4-1, to=4-4]
	\arrow[Rightarrow, no head, from=4-1, to=1-1]
	\arrow[Rightarrow, no head, from=4-4, to=1-4]
	\arrow[maps to, from=2-2, to=2-3]
	\arrow[maps to, from=2-2, to=3-2]
	\arrow[maps to, from=3-2, to=3-3]
	\arrow[maps to, from=2-3, to=3-3]
\end{tikzcd}\]
\end{proof}

\begin{theorem}\label{AspectIsSliceGlobal}
Let $\Sigma$ be the forgetful functor from a slice category $T/t$ to its ambient category $T$. Then the $t$-aspect of a $T$-indexed set $P$ is in correspondence with the global aspect of $\pullAlong{\Sigma} P$.
\end{theorem}
\begin{proof}
This is corollary to \magicref{YonedaPullalong}, which tells us $\Hom_{\Psh{T/t}}(1_{T/t}, \pullAlong{\Sigma} P)$ is in correspondence with $\Hom_{\Psh{T}}(\Sigma 1_{T/t}, P)$, where $1_{T/t}$ is the terminal object in $T/t$. As this terminal object is given by the identity morphism into $t$, we have that $\Sigma 1_{T/t} = t$. Thus, this equation is telling us that the global aspect of $\pullAlong{\Sigma} P$ corresponds to the $t$-aspect of $P$, as desired.
\end{proof}

\begin{theorem}\label{KanExtensionOfAdjoint}
If $f \adjointTo g$, then $\pullAlong{f} \adjointTo \pullAlong{g}$. Thus $\pullAlong{f} = \Lan_{\op{g}}$, while $\pullAlong{g} = \Ran_{\op{f}}$.
\end{theorem}
\begin{proof}
\TODOinline{This is simply the fact that adjunction is preserved by 2-functors and reversed by each of co and op, thus preserved by $\Hom(\op{-}, C)$ for a fixed category $C$. We don't really need to remark on this theorem here in the final draft, but I'm including it for my own reference, so I can stop getting things backwards in my head.}
\end{proof}

\begin{theorem}\label{RepsmallRightAdjoint}
If $f$ has a right adjoint $g$, then $\pullAlong{f}$ takes \repsmall/ sets to \repsmall/ sets. Specifically, $\pullAlong{f}(t) = g(t)$.
\end{theorem}
\begin{proof}
$\pullAlong{f}$ takes any representable presheaf with representing object $t$ in $T$ to the representable presheaf $\Hom_T(f(-), t) = \Hom_S(-, g(t))$.
\end{proof}

\begin{theorem}\label{PullalongIsLex}
Any functor of the form $\pullAlong{f}$ preserves finite limits.
\end{theorem}
\begin{proof}
This can be seen in several ways. Perhaps most familiarly, this can be seen from the fact that (co)limits in a a functor category are computed pointwise where the pointwise (co)limits exist, and of course set-sized (co)limits all exist in $\Set$. Secondly, when the domain of $f$ is a small category, it can be seen from the fact that $\pullAlong{f}$ has left and right adjoints (the left and right Kan extensions along $\op{f}$), so that $\pullAlong{f}$ in fact preserves ALL (co)limits that happen to exist, regardless of size. We can also note that $\pullAlong{f}(P) = \Hom(f(-), P)$, which is manifestly limit preserving (though this argument does not generalize as easily to colimit-preservation).
\end{proof}

We also define more generally the concept of a function between indexed sets having \repsmall/ fibers:
\begin{definition}\label{RepsmallFibersDefn}
A function $f : A \to B$ between $T$-indexed sets has \defined{\repsmall/ fibers} if the pullback of $f$ along any map into $B$ from a \repsmall/ set is itself \repsmall/ (thus, lives within a slice category of $T$). That is, we say $f$ has \repsmall/ fibers just in case for every pullback diagram of the following sort within the category of $T$-indexed sets, if $t$ is \repsmall/, then so is $s$:

% https://q.uiver.app/?q=WzAsNCxbMSwwLCJBIl0sWzEsMSwiQiJdLFswLDAsInMiXSxbMCwxLCJ0Il0sWzAsMSwiZiJdLFsyLDBdLFsyLDNdLFszLDFdLFsyLDEsIiIsMix7InN0eWxlIjp7Im5hbWUiOiJjb3JuZXIifX1dXQ==
\[\begin{tikzcd}
	s & A \\
	t & B
	\arrow["f", from=1-2, to=2-2]
	\arrow[from=1-1, to=1-2]
	\arrow[from=1-1, to=2-1]
	\arrow[from=2-1, to=2-2]
	\arrow["\lrcorner"{anchor=center, pos=0.125}, draw=none, from=1-1, to=2-2]
\end{tikzcd}\]
\end{definition}

(Beware that when $T$ does not have finite limits, this definition does not have the properties which might be expected. For example, we might expect that any morphism between $T$-\repsmall/ sets should have $T$-\repsmall/ fibers, which would not be true if $T$ itself did not have pullbacks. If $T$ does not have binary products, it will not even be true that a map into the terminal object $1$ has \repsmall/ fibers whenever its domain is \repsmall/.)

\begin{theorem}\label{RepsmallSumOfRepsmallFibers}
If $f : A \to B$ has \repsmall/ fibers and $B$ is \repsmall/, then $A$ is \repsmall/ too.
\end{theorem}
\begin{proof}
Apply the definition of \repsmall/ fibers to the trivial case of pulling $f$ back along $\id_B$.
\end{proof}

The following two theorems follow from the composition of pullback squares into larger pullback squares (or pullback rectangles, one might say):

\begin{theorem}
Maps with \repsmall/ fibers are closed under composition.
\end{theorem}
\begin{proof}
% https://q.uiver.app/?q=WzAsNixbMSwwLCJBIl0sWzEsMSwiQiJdLFswLDAsInMiXSxbMCwxLCJ0Il0sWzEsMiwiQyJdLFswLDIsInUiXSxbMCwxLCJmIl0sWzIsMF0sWzIsM10sWzMsMV0sWzIsMSwiIiwyLHsic3R5bGUiOnsibmFtZSI6ImNvcm5lciJ9fV0sWzEsNCwiZyJdLFszLDVdLFs1LDRdLFszLDQsIiIsMSx7InN0eWxlIjp7Im5hbWUiOiJjb3JuZXIifX1dXQ==
\[\begin{tikzcd}
	s & A \\
	t & B \\
	u & C
	\arrow["f", from=1-2, to=2-2]
	\arrow[from=1-1, to=1-2]
	\arrow[from=1-1, to=2-1]
	\arrow[from=2-1, to=2-2]
	\arrow["\lrcorner"{anchor=center, pos=0.125}, draw=none, from=1-1, to=2-2]
	\arrow["g", from=2-2, to=3-2]
	\arrow[from=2-1, to=3-1]
	\arrow[from=3-1, to=3-2]
	\arrow["\lrcorner"{anchor=center, pos=0.125}, draw=none, from=2-1, to=3-2]
\end{tikzcd}\]

When we presume $g$ to have \repsmall/ fibers, we find that $t$ is \repsmall/. Then when we presume $f$ to have \repsmall/ fibers, we find that $s$ is \repsmall/. The composition of the individual pullback squares yields a pullback rectangle, which allows us to conclude that the composition $g \circ f$ has \repsmall/ fibers.

The above illustrates the argument for binary composition, by simply composing pullbacks. The argument for $n$-ary composition for any finite $n$ works inductively in the same way (note that the base $0$-ary case works in the same way as well; the pullback of an identity morphism is an identity morphism, and an identity morphism with small codomain has small domain).
\end{proof}

\begin{theorem}
Maps with \repsmall/ fibers are closed under pullback along arbitrary maps.
\end{theorem}
\begin{proof}
% https://q.uiver.app/?q=WzAsNixbMiwwLCJBIl0sWzIsMSwiQiJdLFsxLDAsIkQiXSxbMSwxLCJDIl0sWzAsMSwidCJdLFswLDAsInMiXSxbMCwxLCJmIl0sWzIsMF0sWzIsMywiZiciXSxbMywxXSxbMiwxLCIiLDIseyJzdHlsZSI6eyJuYW1lIjoiY29ybmVyIn19XSxbNCwzXSxbNSw0XSxbNSwyXSxbNSwzLCIiLDAseyJzdHlsZSI6eyJuYW1lIjoiY29ybmVyIn19XV0=
\[\begin{tikzcd}
	s & D & A \\
	t & C & B
	\arrow["f", from=1-3, to=2-3]
	\arrow[from=1-2, to=1-3]
	\arrow["{f'}", from=1-2, to=2-2]
	\arrow[from=2-2, to=2-3]
	\arrow["\lrcorner"{anchor=center, pos=0.125}, draw=none, from=1-2, to=2-3]
	\arrow[from=2-1, to=2-2]
	\arrow[from=1-1, to=2-1]
	\arrow[from=1-1, to=1-2]
	\arrow["\lrcorner"{anchor=center, pos=0.125}, draw=none, from=1-1, to=2-2]
\end{tikzcd}\]

Any pullback of $f'$ (along some arbitrary map) is a pullback of $f$ itself (along an extended map with the same domain). Thus, if $f$ has yields \repsmall/ objects whenever pulled back along a map with \repsmall/ domain, so does its pullback $f'$.
\end{proof}

\begin{theorem}\label{RepSmallRightAdjointFibers}
If $L : Q \to T$ is a functor with a right adjoint, on a category $Q$ with pullbacks, and $f$ is a map between $T$-indexed sets with $T$-\repsmall/ fibers, then $\pullAlong{L} f$ has $Q$-\repsmall/ fibers.
\end{theorem}
\begin{proof}
Let us say $f: A \to B$, and let an arbitrary map $m : q \to \pullAlong{L}(B)$ be given, where $q$ is an object of $Q$. We must show that the pullback of $\pullAlong{L} f$ along $m$ also lies within $Q$. For sake of a name, let us call the domain of this pullback $P$.

% https://q.uiver.app/?q=WzAsNCxbMSwwLCJcXHB1bGxBbG9uZ3tMfUEiXSxbMSwxLCJcXHB1bGxBbG9uZ3tMfUIiXSxbMCwxLCJxIl0sWzAsMCwiUCJdLFswLDEsIlxccHVsbEFsb25ne0x9ZiJdLFszLDJdLFsyLDEsIm0iLDJdLFszLDBdLFszLDEsIiIsMSx7InN0eWxlIjp7Im5hbWUiOiJjb3JuZXIifX1dXQ==
\[\begin{tikzcd}
	P & {\pullAlong{L}A} \\
	q & {\pullAlong{L}B}
	\arrow["{\pullAlong{L}f}", from=1-2, to=2-2]
	\arrow[from=1-1, to=2-1]
	\arrow["m"', from=2-1, to=2-2]
	\arrow[from=1-1, to=1-2]
	\arrow["\lrcorner"{anchor=center, pos=0.125}, draw=none, from=1-1, to=2-2]
\end{tikzcd}\]

First, observe via \magicref{YonedaPullalongLemma} that $m$ factors as $\pullAlong{L}(m') \circ \eta$ for some $m' : L(q) \to B$ and $\eta : q \to \pullAlong{L}L(q)$.

% https://q.uiver.app/?q=WzAsMyxbMiwwLCJcXHB1bGxBbG9uZ3tMfUIiXSxbMSwwLCJcXHB1bGxBbG9uZ3tMfUwocSkiXSxbMCwwLCJxIl0sWzEsMCwiXFxwdWxsQWxvbmd7TH0gbSciLDJdLFsyLDEsIlxcZXRhIiwyXSxbMiwwLCJtIiwxLHsib2Zmc2V0Ijo1LCJjdXJ2ZSI6Mn1dLFsxLDUsIiIsMSx7InNob3J0ZW4iOnsidGFyZ2V0IjoyMH0sInN0eWxlIjp7ImhlYWQiOnsibmFtZSI6Im5vbmUifX19XV0=
\[\begin{tikzcd}
	q & {\pullAlong{L}L(q)} & {\pullAlong{L}B}
	\arrow["{\pullAlong{L} m'}"', from=1-2, to=1-3]
	\arrow["\eta"', from=1-1, to=1-2]
	\arrow[""{name=0, anchor=center, inner sep=0}, "m"{description}, shift right=5, curve={height=12pt}, from=1-1, to=1-3]
	\arrow[shorten >=2pt, Rightarrow, no head, from=1-2, to=0]
\end{tikzcd}\]

Thus, the pullback yielding $P$ we are interested in can be decomposed as follows:

% https://q.uiver.app/?q=WzAsNixbMiwwLCJcXHB1bGxBbG9uZ3tMfUEiXSxbMiwxLCJcXHB1bGxBbG9uZ3tMfUIiXSxbMSwxLCJcXHB1bGxBbG9uZ3tMfUwocSkiXSxbMSwwLCJcXHB1bGxBbG9uZ3tMfShBIFxcdGltZXNfe0J9IEwocSkpIl0sWzAsMSwicSJdLFswLDAsIlAiXSxbMCwxLCJcXHB1bGxBbG9uZ3tMfWYiXSxbMiwxLCJcXHB1bGxBbG9uZ3tMfSBtJyIsMl0sWzMsMF0sWzQsMiwiXFxldGEiLDJdLFs1LDNdLFs1LDRdLFszLDJdLFs0LDEsIm0iLDEseyJvZmZzZXQiOjUsImN1cnZlIjoyfV0sWzIsMTMsIiIsMSx7InNob3J0ZW4iOnsidGFyZ2V0IjoyMH0sInN0eWxlIjp7ImhlYWQiOnsibmFtZSI6Im5vbmUifX19XSxbNSw5LCIiLDEseyJsZXZlbCI6MSwic3R5bGUiOnsibmFtZSI6ImNvcm5lciJ9fV0sWzMsNywiIiwxLHsibGV2ZWwiOjEsInN0eWxlIjp7Im5hbWUiOiJjb3JuZXIifX1dXQ==
\[\begin{tikzcd}
	P & {\pullAlong{L}(A \times_{B} L(q))} & {\pullAlong{L}A} \\
	q & {\pullAlong{L}L(q)} & {\pullAlong{L}B}
	\arrow["{\pullAlong{L}f}", from=1-3, to=2-3]
	\arrow[""{name=0, anchor=center, inner sep=0}, "{\pullAlong{L} m'}"', from=2-2, to=2-3]
	\arrow[from=1-2, to=1-3]
	\arrow[""{name=1, anchor=center, inner sep=0}, "\eta"', from=2-1, to=2-2]
	\arrow[from=1-1, to=1-2]
	\arrow[from=1-1, to=2-1]
	\arrow[from=1-2, to=2-2]
	\arrow[""{name=2, anchor=center, inner sep=0}, "m"{description}, shift right=5, curve={height=12pt}, from=2-1, to=2-3]
	\arrow[shorten >=2pt, Rightarrow, no head, from=2-2, to=2]
	\arrow["\lrcorner"{anchor=center, pos=0.125}, draw=none, from=1-1, to=1]
	\arrow["\lrcorner"{anchor=center, pos=0.125}, draw=none, from=1-2, to=0]
\end{tikzcd}\]

The right half of the above diagram is $\pullAlong{L}$ (known to preserve pullbacks by \magicref{PullalongIsLex}) applied to the following pullback diagram in $\Psh{T}$:

% https://q.uiver.app/?q=WzAsNCxbMSwwLCJBIl0sWzEsMSwiQiJdLFswLDEsIkwocSkiXSxbMCwwLCJBIFxcdGltZXNfe0J9IEwocSkiXSxbMCwxLCJmIl0sWzIsMSwibSciLDJdLFszLDJdLFszLDBdLFszLDUsIiIsMCx7ImxldmVsIjoxLCJzdHlsZSI6eyJuYW1lIjoiY29ybmVyIn19XV0=
\[\begin{tikzcd}
	{A \times_{B} L(q)} & A \\
	{L(q)} & B
	\arrow["f", from=1-2, to=2-2]
	\arrow[""{name=0, anchor=center, inner sep=0}, "{m'}"', from=2-1, to=2-2]
	\arrow[from=1-1, to=2-1]
	\arrow[from=1-1, to=1-2]
	\arrow["\lrcorner"{anchor=center, pos=0.125}, draw=none, from=1-1, to=0]
\end{tikzcd}\]

Note that, as $f$ has $T$-\repsmall/ fibers and $L(q)$ is an object of $T$ (i.e., $T$-\repsmall/), we find that $A \times_{B} L(q)$ is also $T$-\repsmall/.

By \magicref{RepsmallRightAdjoint}, it follows that $\pullAlong{L}(A \times_{B} L(q))$ is $Q$-\repsmall/, as is $\pullAlong{L}L(q)$.

Thus, the left half of our above diagram is a pullback of morphisms within $Q$:

% https://q.uiver.app/?q=WzAsNCxbMSwxLCJcXHB1bGxBbG9uZ3tMfUwocSkiXSxbMSwwLCJcXHB1bGxBbG9uZ3tMfShBIFxcdGltZXNfe0J9IEwocSkpIl0sWzAsMSwicSJdLFswLDAsIlAiXSxbMiwwLCJcXGV0YSIsMl0sWzMsMV0sWzMsMl0sWzEsMF0sWzMsNCwiIiwxLHsibGV2ZWwiOjEsInN0eWxlIjp7Im5hbWUiOiJjb3JuZXIifX1dXQ==
\[\begin{tikzcd}
	P & {\pullAlong{L}(A \times_{B} L(q))} \\
	q & {\pullAlong{L}L(q)}
	\arrow[""{name=0, anchor=center, inner sep=0}, "\eta"', from=2-1, to=2-2]
	\arrow[from=1-1, to=1-2]
	\arrow[from=1-1, to=2-1]
	\arrow[from=1-2, to=2-2]
	\arrow["\lrcorner"{anchor=center, pos=0.125}, draw=none, from=1-1, to=0]
\end{tikzcd}\]

As $Q$ is closed under pullbacks, it follows that $P$ is $Q$-\repsmall/, completing our proof.
\end{proof}

\begin{definition}\label{IndexedStructuresDefn}
We can talk about any kind of $T$-indexed structure or $T$-indexed maps between such structures, as the appropriate diagram of $T$-indexed sets and functions. For example, we can talk about $T$-indexed groups and group homomorphisms between them. When the $T$-indexed sets involved (the sorts within the structure, including the domains and codomains of all the maps defining the structure) are all \repsmall/, we say the entire structure is \defined{\repsmall/}, or equivalently, we say it is \defined{internal} to $T$\footnote{This \quote{$T$-internal gadgets} terminology makes most sense when $T$ is thought of as a kind of structure such that structure-preserving maps from $T$ to $S$ take $T$-internal gadgets to $S$-internal gadgets. Thus, if the definition of gadgets invokes maps whose domains are defined using finite limits, we will use this terminology of $T$-internal gadgets only in contexts where we are taking $T$ as a category with finite limits (for example, when speaking of internal categories). If the definiton of gadgets invokes maps whose domains are defined using finite products, we will use this terminology of $T$-internal gadgets only in contexts where we are taking $T$ as a category with finite products (for example, when speaking of internal groups). If the definition of gadgets invokes maps whose domains are defined using countably infinite products, then to speak of $T$-internal gadgets, $T$ must be carrying countably infinite product structure, etc.}. By the Yoneda lemma, this amounts to a diagram of objects and morphisms within $T$ itself.
\end{definition}

Observe that, as $\pullAlong{f}$ for an arbitrary functor $f : S \to T$ preserves finite limits (by \magicref{PullalongIsLex}), it not only takes $T$-indexed sets to $S$-indexed sets but also acts as a functor from $T$-indexed structures to $S$-indexed structures more generally, for any notion of structure definable using finite limits. For example, $\pullAlong{f}$ takes $T$-indexed groups to $S$-indexed groups, and so on. Furthermore, by \magicref{RepsmallRightAdjoint}, if $f$ has a right adjoint, then $\pullAlong{f}$ will take \repsmall/ structures to \repsmall/ structures.

\subsection{Indexed categories}
\begin{definition}
In the same vein as all this, by a $T$-\defined{indexed category}, we mean a category-valued presheaf on $T$; that is, a contravariant functor from $T$ to $\Cat$, and by an \defined{indexed functor} (or simply \defined{functor}) between $T$-indexed categories, we mean a natural transformation between such presheaves.\footnote{The machinery of indexed categories is equivalent to the machinery of fibered categories, a presentation some prefer, but we refrain from that presentation for now. Many of the features which make fibered categories most useful do not strongly apply to our ultimate interest largely in internal structures, while adding distracting complexity to the exposition. The current choice of presentation seemed the simpler one for our purposes, but the reader who disagrees may translate everything into the language of fibered categories if they prefer.}. (In keeping with our general convention, note that \quote{functor to $\Cat$} and \quote{natural transformation between functors to $\Cat$} here really refer to pseudofunctors and pseudonatural transformations, respectively, as $\Cat$ is a 2-category). We say this indexed category is an \defined{indexed lexcategory} (aka, \defined{has finite limits}) if this presheaf factors through the inclusion of $\LexCat$ into $\Cat$; that is, if it takes every object to a lexcategory and every morphism to a lexfunctor. We say an indexed functor between indexed lexcategories \defined{preserves finite limits} if it arises from a natural transformation between the corresponding $\LexCat$-valued presheaves. And in the same way as all this, we can speak of \defined{natural transformations} between functors between indexed categories, or any other familiar categorical structure or property.
\end{definition}

One might have thought our definition of $T$-indexed category-like structures would simply be a special case of our previous definition of $T$-indexed set-like structures as suitable diagrams within $\Psh{T}$ (that is, as suitable diagrams of $\Set$-valued functors). That is indeed the essence of this definition. However, the fact that we take indexed categories to be given by pseudofunctors into the 2-category $\Cat$, instead of treating $\Cat$ as a 1-category, provides a subtle but technically convenient generalization beyond directly demanding mere diagrams of $\Set$-valued functors.

Still, all the same notational conventions apply to indexed categories. E.g., given a $T$-indexed category $C$, we write $C(t)$ (or $C_t$) for the category which is the $t$-aspect of $C$ at an object $t$ of $T$, we write $C(m) : C(t) \to C(s)$ (or $C_m$) for the functor induced by a morphism $m: s \to t$ in $T$, we may write $\pullAlong{m}$ instead of $C(m)$ in contexts where it is clear that we are referring to the action of $C$, etc.

We now might like to speak about a category being \repsmall/, in the sense that its collection of objects and its collection of morphisms are both \repsmall/. This is the essence of the definition we will indeed adopt (at \magicref{RepsmallCategoryDefn}) but there is one pitfall to be aware of here, related to the just mentioned subtlety. We generally speak about categories in such a way as that they do not come with a particular notion of their set of objects, as such. That is, two categories may be equivalent (in the technical sense of \quote{equivalent} within the 2-category $\Cat$) though presented with different ostensible sets of objects. For example, a category presented as comprised of one terminal object, and a category presented as comprised of two isomorphic terminal objects, are equivalent categories; there is no pseudofunctor from the 2-category of categories, functors, and natural isomorphisms to $\Set$ which would send the first of these to a one-element set and the second to a two-element set. We are to treat them as the \quote{same} category. So to speak about a category as having a particular set of objects, we must imagine it as carrying more fine-grained equality structure on its objects than we normally do.

Though a category does not have a well-defined set of objects, it \emph{does} have a well-defined set of morphisms between any two given objects. Thus, there is no such difficulty in defining when an indexed category is locally small.
\begin{definition}\label{LocallyRepmallDefn}
Given a $T$-indexed category $C$, an object $t$ of $T$ and any two objects $a$ and $b$ in $C(t)$, we can define a $T$-indexed set whose aspect at objects $r$ of $T$ is the set $\{ (m, n) \mid m \in \Hom_T(r, t), n \in \Hom_{C(r)}(\pullAlong{m} a, \pullAlong{m} b) \}$, with the obvious corresponding action on morphisms of $T$. If the $T$-indexed set defined in this way is \repsmall/ for every object $t$ of $T$ and objects $a$ and $b$ in $C(t)$, then we say $C$ is \defined{locally \repsmall/}.
\end{definition}

Note that this is the same as saying that $\langle \cod, \dom \rangle : \Mor(C) \to \Ob(C) \times \Ob(C)$ has \repsmall/ fibers in the sense of \magicref{RepsmallFibersDefn}, except for that we do not need to think of $\Ob(C)$ as carrying an equality relation as such.

\subsection{Strict categories}
These bothers around the ill-defined set of objects of a general indexed category shall take us down some technical digressions for a bit, before we return to our big picture ideas. (Please keep in mind, the nuances of this section mostly do not matter for a big picture understanding. The main part of this document where such details might matter is in being rigorous in our chapter on geminal categories. We recommend that on a first read, the reader ignore all discussion of strictification or distinction between strict and non-strict concepts, in order to pick up the big picture ideas. The reader can then pay attention to these details on later more scrupulous re-reads as desired.)

\begin{definition}\label{StrictCategoryDefn}
Specifically, let us say a \defined{strict category} is a set of objects (including the ability to speak about equality of objects in a potentially finer-grained sense than isomorphism) and a set of morphisms, with the usual operations and satisfying the usual equations.\footnote{In certain contexts where quotient sets are not available, it is sometimes useful to allow strict categories to also come with a sort for 2-cell isomorphisms between their objects; that is, to take the collection of morphisms between any two objects in a strict category to comprise a setoid rather than a set. That is the more flexible and in some sense morally superior definition, but we will not need its extra complexity for now.} We may also speak of a \defined{strict functor}, meaning a homomorphism of such structure that preserves all of it on-the-nose. Strict categories and the strict functors between them comprise the 1-category $\StrictCat$. Similarly we can speak of natural transformations between strict functors, and so on for strict counterparts of any other categorical notion. Note that we can also straightforwardly talk about functors from strict categories to categories.
\end{definition}

Every strict category [or functor or etc], gives rise to a category [or functor or etc] in whatever ordinary sense one would like to think of these. We may say the strict category [or etc] presents the category [or etc] which results. Beware, non-isomorphic strict categories can both present the same (up to equivalence) category! That is, the relation \quote{strict structure $S$ presents non-strict-structure $C$} is invariant up to isomorphism of strict structures $S$ but invariant more broadly up to equivalence of non-strict structures $C$.

Just as every strict category presents a non-strict category, conversely, one would ordinarily say every category is presented by at least one strict category.\footnote{The situation is more nuanced for turning functors between arbitrary categories equivalent to given strict categories into strict functors between the given strict categories, but that will not concern us for now.} One might, if one likes, say that the only distinction between categories and strict categories is that we gather categories up into a 2-category and speak of categories up to equivalence in such, while we gather strict categories up into a 1-category and speak of strict categories up to isomorphism in such.

\begin{definition}
We now go further in defining a \defined{strict lexcategory}. Here, we mean more than just a strict category for which finite limits exist. We also mean that, when taking special \quote{basic limits}, the relevant limit is not merely defined up to isomorphism, but is given as a particular object (in keeping with the fact that objects can be distinguished more finely-grained than up to isomorphism, within a strict category). A \defined{strict lexfunctor} is accordingly one which preserves these chosen basic limits not merely up to isomorphism, but on-the-nose. Strict lexcategories and the strict lexfunctors between them comprise the 1-category $\StrictLexCat$. In the same way, we can also speak of a \defined{strict category with finite products}, or any similar such categorical structure.
\end{definition}

This business of \defined{basic limits} will require more explanation, another technical subtlety. What I mean by this is like so: Consider for example the concept of a category with a terminal object. And now consider the concept of a category with a pair of terminal objects, a terminal object A and a terminal object B. Ordinarily, we would like to say these are equivalent concepts or equivalent theories. They give rise to equivalent 2-categories (of categories with terminal objects, functors taking terminal objects to terminal objects, and natural transformations between these). However, the concept of a strict category with a single chosen terminal object, and the concept of a strict category with two chosen terminal objects A and B, are not equivalent concepts. We can ask questions in the one case that we cannot in the other; for example, in the latter case, we can distinguish between those models in which A and B are equal objects and those models in which A and B are not equal objects. This is reflected also in these giving rise to non-equivalent categories of models (of strict categories with the designated terminal objects, and functors preserving designated terminal objects on the nose). So when we go strictify the concept of a category with a terminal object, we really must make a choice as to how we choose to designate the terminal object; once or multiply.

This issue was illustrated above for terminal objects, but arises again, perhaps even more perniciously, for categories with finite products or finite limits or the like. Here, we find that the essentially algebraic theory of \quote{A strict category with a chosen terminal object and a binary operation sending any pair of objects to a chosen product} is not precisely the same as the essentially algebraic theory of \quote{A strict category with an $n$-ary operation on objects assigning chosen $n$-ary products, for each finite $n$}. Or the essentially algebraic theory of \quote{A strict category with a chosen terminal object and chosen (binary) pullbacks} is not precisely the same as the essentially algebraic theory of \quote{A strict category with a chosen terminal object, chosen binary products, and chosen (binary) equalizers}, particularly when we ask for homomorphisms between such structures which preserve their operations on-the-nose.

So in general, when we wish to talk about the appropriate notion of \quote{strict lexcategory} (or \quote{strict category with finite products} or \quote{strict cartesian closed category} or any such thing), we must make some decision as to how exactly to formalize this. We must make some choice of a basic stock of limit operations (or representing object operations more generally) of the desired sort, such that all the other desired limits (or representing objects) can be constructed from these basic operations. Different choices will yield slightly different strict concepts, albeit equivalent for all non-strict purposes.

None of the results in this work are ever particularly sensitive to what choice of basic such operations we take in strictifying a theory. We shall simply suppose some such choice has been made whenever needed, and refer to its operations as our basic limits. The one notable presumption we will make is that there are only finitely many basic limit operations involved in defining a strict lexcategory (or any such finitely axiomatizable thing); beyond that, any choice is fine. If the reader insists that we commit to a specific choice, let us for harmony with Palmgren and Vickers \TODOinline{Check this cite; they probably just use terminal object and pullback} say a strict lexcategory is defined by having a chosen terminal object and a chosen (binary) pullback operator.

\begin{definition}
Of course, we can speak of \defined{indexed strict categories} now (or indexed strict lexcategories, indexed strict categories with finite products, etc), straightforwardly via \magicref{IndexedStructuresDefn}, as the appropriate diagram of indexed sets and functions between them. And we can speak of such indexed strict categories as being \repsmall/, just in case their indexed sets of objects and of morphisms are both \repsmall/.
\end{definition}

\begin{definition}\label{RepsmallCategoryDefn}
We will now say an indexed category is \defined{\repsmall/} if it is equivalent to some indexed strict category which is \repsmall/. Note that we do not demand that, as part of its structure, any particular such strict category is selected; merely, that it is possible to do so. However, we may use the terminology \defined{internal category}, to mean the selection of a specific \repsmall/ indexed strict category; similarly, an \defined{internal lexcategory} will mean the selection of a specific \repsmall/ indexed strict lexcategory (including chosen basic limits), and so on.
\end{definition}

\begin{definition}\label{LocallyRepsmallStrictDefn}
We also say an indexed strict category is \defined{locally \repsmall/} if the map $\langle \dom, \cod \rangle$ from its set of morphisms to its set of pairs of objects has \repsmall/ fibers (in other words, though its set of objects may not be \repsmall/, everything that exists between any two particular objects is \repsmall/). 

We can repeat in this language the observation made at the end of \magicref{LocallyRepmallDefn}. Given an indexed category $C$ which is equivalent to some indexed strict category $C'$, we have that $C$ is locally \repsmall/ just in case $C'$ is locally \repsmall/. Note that, although an indexed category may be equivalent to non-isomorphic indexed strict categories, they will all agree on whether they are locally \repsmall/.

Note that a \repsmall/ strict category indexed over a category with finite limits is a fortiori locally \repsmall/, as expected, as the collection of morphisms between any particular pair of objects is given by an equalizer between sets already presumed \repsmall/ in a \repsmall/ strict category.
\end{definition}

We note without detailed proof (\TODOinline{Give or cite proofs}) some strictification results which will be useful to us later.

\begin{theorem}\label{StrictifyCategory}
Every category is equivalent to some strict category.
\end{theorem}
\begin{proof}
In traditional foundations, this is tautology, as categories are defined as presented by strict categories. In other foundations, any results of ours dependent on this should be interpreted or modified accordingly.
\end{proof}

\begin{theorem}\label{StrictifyLexcategory}
Every lexcategory is equivalent to some strict lexcategory.
\end{theorem}
\begin{proof}
One way to see this is by first noting that, for a given category $C$ with finite limits, we first of all have that $C$ is equivalent to some strict category $C'$ by \magicref{StrictifyCategory}. To further equip $C'$ as a strict lexcategory, we just need to make some choice of designated limits for each basic limit within $C'$. This is readily done using the Axiom of Choice.

In contexts without Choice, this theorem needs to be approached more carefully, with a more deliberate choice of $C'$ rather than an arbitrary one, but can still be carried out. In particular, we can create the free strict lexcategory $D$ extending $C'$ while preserving all basic limit cones (in the weaker, non-on-the-nose sense, as $C'$ does not itself have chosen limits). The map from $C'$ into this $D$ will be full, faithful, and essentially surjective on objects; thus, an equivalence of categories.
\end{proof}

\begin{theorem}\label{StrictifyLexfunctor}
Given a strict lexcategory $D$, a lexcategory $C$, and a lexfunctor $f : C \to D$, there is some strict lexcategory $C'$ and strict lexfunctor $f' : C' \to D$ such that $f$ and $f'$ are equivalent within the slice 2-category $\Cat/D$.
\end{theorem}

\begin{theorem}\label{StrictifyIndexedCategory}
Any indexed category is equivalent to some indexed strict category.
\end{theorem}

\begin{theorem}\label{StrictifyIndexedLexcategory}
Any indexed lexcategory is equivalent to some indexed strict lexcategory.
\end{theorem}

\begin{theorem}\label{StrictifyInternalCategoryToInternalLexcategory}
Any internal category which has finite limits (qua indexed category) can be further equipped as an internal lexcategory (without modifying the internal category structure).
\end{theorem}
\begin{proof}
Let the internal category $C$, internal to $T$, be given, and suppose its $t$-aspect has finite limits for each object $t$ of $T$. That is, the category whose objects are $\Hom(t, \Ob(t))$ and whose morphisms are $\Hom(t, \Mor(t))$, with suitable composition structure from the diagram internal to $T$ defining $C$, has finite limits.

Then in particular, for each basis finite limit shape, we can consider the case where $t$ is taken to be the set of diagrams of such shape within $C$ (for example, for binary products, we can consider $t = \Ob(C) \times \Ob(C)$, or for binary equalizers, we can consider $t$ taken to be the kernel pair (that is, pullback along itself) of $\langle \cod, \dom \rangle : \Mor(C) \to \Ob(C) \times \Ob(C)$). There will then be, within the $t$-aspect of $C$, a corresponding generic diagram of this shape, which will have some limit within $C$ as $C$ has finite limits. The selection of any particular such limit (that is, a particular value in $\Hom(t, \Ob(C))$ to serve as the apex of the limit cone, and particular further values in $\Hom(t, \Mor(C))$ to serve as the projection morphisms of the limit cone) gives us the morphisms within $T$ which serve as a limit-assigning operation on $C$ for this particular shape of basic limit. After making such a choice for each of the basic limit operations (of which we can presume there are only finitely many), we ultimately have equipped $C$ as an internal lexcategory.
\end{proof}
Note that it is NOT true that any indexed strict category which has finite limits (qua indexed category) can furthermore be equipped as an indexed strict lexcategory (without modification to the indexed strict category structure)! The former has reindexing functors which need only preserve finite limits in a non-strict-sense, while the latter's chosen basic limits must be such that all reindexing functors preserve basic limits on-the-nose. So it is rather remarkable that we get this for free once our indexed strict category is furthermore repsmall.

\subsection{Self-indexing}
\begin{definition}
Note that, from any lexcategory $T$ (or even just a category with pullbacks), we obtain a $T$-indexed lexcategory by considering the functor $T/-$ which assigns to each object $t$ of $T$ the slice category $T/t$, and whose action on morphisms is given by pullback. We refer to this as the \defined{self-indexing} of $T$.
\end{definition}

Note in the above that our flexibility in considering an indexed category as a pseudofunctor into $\Cat$, rather than a strict functor into $\StrictCat$, pays off in letting us not worry about how to choose specific pullback slices in a strictly functorial way. \TODOinline{That said, we will actually eventually use the construction under which this is made strictly functorial, at one point internally in our proof of Lob's theorem, so perhaps we should write out that construction in these Preliminaries as well.}

The self-indexing $T/-$ of a lexcategory $T$ is not in general \repsmall/, nor even locally \repsmall/. Given two globally defined objects $A$ and $B$ of the self-indexed category, their corresponding hom-set $\Hom_{T/-}(A, B)$ amounts to the presheaf $\Hom_{T}(A \times -, B)$ on $T$, which is to say, the exponential $B^A$ within $\Psh{T}$. This indexed set is \repsmall/ just in case an exponential object $B^A$ already exists within $T$. This extends in the same way to non-globally-defined objects of the self-indexed category (considered as globally defined over some slice category of $T$ instead, a la \magicref{AspectIsSliceGlobal}), and so the self-indexing of $T$ is locally \repsmall/ just in case $T$ is locally cartesian closed. Even if we do not have local cartesian closure in full, note that when $A = 1$, the exponential $B^A$ always is given by $B$ itself, so that hom-sets whose domain is $1$ are always \repsmall/ within the self-indexed category, with $\Hom_{T/-}(1, B)$ being the same as $B$ itself. In this way, the global sections presheaf upon the self-indexed category yields the canonical equivalence between the self-indexed category and the category of \repsmall/ sets. \TODOinline{Here, we are talking about a presheaf on an indexed category. Perhaps that should wait till after the section on doubly indexed sets}

\begin{definition}
In the same way, we can also speak of an \defined{indexed category with finite products}, and indeed, from any category with finite products $T$ (or even just a category with binary products), we obtain a $T$-indexed category with finite products by considering the functor $T//-$ which assigns to each object $t$ of $T$ the full subcategory of $T/t$ consisting of projection slices (slices given by the projection $: t \times s \to t$ for some object $s$ of $T$), and whose action on morphisms is again given by pullback (the pullback of a projection being another projection in a canonical way). We refer to this as the \defined{simple self-indexing} of $T$. Note that $T//t$ can also be thought of as the Kleisli category for the $t \times -$ comonad; that is, the objects of $T//t$ are the same as the objects of $T$, while a morphism $: s_1 \to s_2$ in $T//t$ is the same as a morphism $: t \times s_1 \to s_2$ in $T$, with suitable composition structure.

For a category with finite limits (or just pullbacks and binary products), the simple self-indexing can be thought of as a full subcategory of the self-indexing; specifically, the full subcategory whose objects in each aspect are restricted to those of $T$ itself.
\end{definition}

By analogous reasoning to before, the simple self-indexing $T//-$ of a category with finite products $T$ is locally \repsmall/ just in case $T$ is cartesian closed.

\begin{theoremEnd}[category=IntrospLemmas]{lemma}\label{Lemma1}
If $Y$ is a category with initial object $0$ and $X$ is a (2-)category, then to any functor $f : Y \to X$, we can associate a corresponding functor $f'$ from $Y$ to the slice category $f(0)/X$.

Furthermore, if $D$ and $C$ are parallel functors from $Y$ to $X$, then a natural transformation from $D$ to $C$ amounts to the same thing as a map $\introS$ from $D(0)$ to $C(0)$ along with a natural transformation from $D'$ to $\introS^{*} \circ C'$, where $\introS^{*} : C(0)/X \to D(0)/X$ is the functor between these slice categories given by composition with $\introS$.

(Dually, for contravariant functors $f : \op{Y} \to X$ (such as with indexed structures), acting on a category $Y$ with a terminal object $1$, we obtain a corresponding contravarint functor $f'$ from $Y$ to the co-slice category $X/f(1)$. And then the dual further result as well.)
\end{theoremEnd}
\begin{proof}
The first half of the lemma is thoroughly straightforward. $f : Y \to X$ induces a functor between the arrow category of $Y$ and the arrow category of $X$, and since $Y$ sits inside its arrow category as the slice category $0/Y$, this gives us a functor from $Y$ to the arrow category of $X$, which when followed by the domain projection back to $X$ is constantly $f(0)$. This is our $f'$.

The second half is also straightforward to mechanically verify when $X$ is a 1-category. This lemma should be understood as a triviality. But we will take some care to write out in detail an abstract demonstration that works just as well when $X$ is a 2-category (or indeed, when all categories involved are of whatever higher dimension), so that (in keeping with our linguistic convention) the functors involved are pseudofunctors, the natural transformations are pseudonatural transformations, etc, without having to get our hands dirty manually fussing about higher-dimensional coherence data.

See \hyperref[proof:prAtEnd\pratendcountercurrent]{full proof} on page~\pageref{proof:prAtEnd\pratendcountercurrent}.
\end{proof}
\begin{proofEnd}[no link to proof]
We pick up from where we left off previously in the proof of \cref{Lemma1}.

Throughout the remainder of this proof, all references to \quote{category}, \quote{functor}, etc, are in the sense of whatever dimension of higher-categories encapsulates both $Y$ and $X$.

Let $Z$ be the category obtained by augmenting $Y$ with a new object $0_Z$ and unique maps from $0_Z$ to each object of $Y$. We have an inclusion functor $i : Y \to Z$, and this inclusion is fully faithful, in the sense that the induced map $\Hom_Y(y_1, y_2) \to \Hom_Z(i(y_1), i(y_2))$ is an equivalence for all $y_1, y_2 \in \Ob(Y)$.

The unique maps from $0_Z$ to each object in the range of $i$ constitute a diagram of this form:

\[\begin{tikzcd}
	& 1 \\
	Y && Z
	\arrow["\unique", from=2-1, to=1-2]
	\arrow["{0_Z}", from=1-2, to=2-3]
	\arrow[""{name=0, anchor=center}, "i"', from=2-1, to=2-3]
	\arrow[Rightarrow, from=1-2, to=0]
\end{tikzcd}\]

What's more, because of how $Z$ was constructed by freely augmenting $Y$ with a new object and co-cone from it to the inclusion of $Y$, this diagram satisfies the universal property that for any other similar diagram
\[\begin{tikzcd}
	& 1 \\
	Y && Z'
	\arrow["\unique", from=2-1, to=1-2]
	\arrow[from=1-2, to=2-3]
	\arrow[""{name=0, anchor=center}, from=2-1, to=2-3]
	\arrow[Rightarrow, from=1-2, to=0]
\end{tikzcd}\]
there is a unique functor from $Z$ to $Z'$ commutatively relating the two diagrams. In jargon, this universal property is summarized by saying $Z$ (along with the data of $0_Z$ and $i$) is the co-comma of the unique functor from $Y$ to $1$ and the identity functor from $Y$ to $Y$.

Now, observe that $i$ has a left adjoint, the functor $q : Z \to Y$ such that $q \circ i$ is the identity on $Y$ and such that $q$ of the initiality co-cone for $0_Z$ in $Z$ is the initiality co-cone for $0$ in $Y$. That is, $q$ is the functor obtained by the co-comma property for $Z$ as applied to this diagram expressing the initiality co-cone of $0$ in $Y$:

\[\begin{tikzcd}
	& 1 \\
	Y && Y
	\arrow["\unique", from=2-1, to=1-2]
	\arrow["{0}", from=1-2, to=2-3]
	\arrow[""{name=0, anchor=center}, "\id"', from=2-1, to=2-3]
	\arrow[Rightarrow, from=1-2, to=0]
\end{tikzcd}\]

It is straightforward to verify that this $q$ is indeed left adjoint to $i$, as any data in $Z$ is either from the fully faithful inclusion of $Y$ or from the initiality co-cone for $0_Z$, and $\Hom_Y(q(i(y_1)), y_2) \iso \Hom_Y(y_1, y_2) \iso \Hom_Z(i(y_1), i(y_2))$ naturally in $y_1, y_2$ from $Y$, and $\Hom_Y(q(0_Z), y) = \Hom_Y(0, y) \iso 1 \iso \Hom_Z(0_Z, i(y))$ naturally in $y$ from $Y$.

Now consider any two parallel functors $D, C : Y \to X$. Because $q \circ i$ is the identity on $Y$, we have that $\Nat(D, C) \iso \Nat(D \circ q \circ i, C)$, where $\Nat$ denotes the space of natural transformations between these functors. But because $q \dashv i$, we in turn have that $\Nat(D \circ q \circ i, C) \iso \Nat(D \circ q, C \circ q)$.

Finally, let us consider what a natural transformation between $D \circ q$ and $C \circ q$ amounts to. This is the same as a functor from $Z$ to the arrow category of $X$ whose domain and codomain projections to $X$ yield $D \circ q$ and $C \circ q$. But by the co-comma property of $Z$, this functor out of $Z$ corresponds to data of the following form:

\[\begin{tikzcd}
	& 1 \\
	Y && {\arrowcat{X}}
	\arrow["\unique", from=2-1, to=1-2]
	\arrow[from=1-2, to=2-3]
	\arrow[""{name=0, anchor=center}, from=2-1, to=2-3]
	\arrow[Rightarrow, from=1-2, to=0]
\end{tikzcd}\]

such that the rightmost arrow of this diagram corresponds to some arrow $\introS$ in $X$ whose domain is $(D \circ q)(0_Z) = D(0)$ and whose codomain is $(C \circ q)(0_Z) = C(0)$, and such that the bottom arrow of this diagram corresponds to a natural transformation from $D \circ q \circ i \iso D$ to $C \circ q \circ i \iso C$. The 2-cell in the above diagram then corresponds to the remaining data necessary for us to construe this natural transformation from $D$ to $C$ as simply the codomain projection of a natural transformation between $D'$ and $\introS^{*} \circ C'$, the functors from $Y$ to $D(0)/X$ as mentioned in the statement of this lemma.

\TODOinline{Phew! That made a mountain out of a molehill. But perhaps people sometimes appreciate such written-out detail.}
\end{proofEnd}

In order to state the next lemma, some terminology:

\begin{definition}
If $T$ is a lexcategory, then for each object of $t$, we can construct the free lexcategory extending $T$ with a global element of $t$. Call this $T[1 \to t]$. Also, for any $f : s \to t$ in $T$, we can get a map from $T[1 \to t]$ to $T[1 \to s]$ by sending the generic global element of $t$ in $T[1 \to t]$ to the result of applying $f$ to the generic global element of $s$ in $T[1 \to s]$. This action is clearly functorial. Thus, $T[1 \to -]$ comprises a $T$-indexed object of $T/\LexCat$.

We can replace all references to finite limit structure above with finite product structure. In this case, let us use the name $T[[1 \to -]]$ for the resulting $T$-indexed object of $T/\FiniteProductCat$.
\end{definition}

By \magicref{Lemma1}, we can see $T/-$ as a contravariant functor from a lexcategory $T$ to $\LexCat/T$. And similarly for $T//-$ in terms of finite product structure.

\begin{theoremEnd}[category=IntrospLemmas]{lemma}\label{SelfIndexingIsFree}
$T[1 \to -]$ is equivalent to $T/-$, when the latter is viewed as a contravariant functor from a lexcategory $T$ to $\LexCat/T$ via \magicref{Lemma1}.

(And in just the same way, for a category with finite products $T$, we have that $T[[1 \to -]]$ is equivalent to $T//-$.)
\end{theoremEnd}
\begin{proof}
This is a standard observation (see 1.10.15 of Bart Jacobs' \quote{Categorical logic and type theory}, although this claims it without proof), and also simple enough to show. See \hyperref[proof:prAtEnd\pratendcountercurrent]{full proof} on page~\pageref{proof:prAtEnd\pratendcountercurrent}.
\end{proof}
\begin{proofEnd}[no link to proof]
To start off, we construe $T/-$ as not merely a $T$-indexed lexcategory, but furthermore a $T$-indexed lexcategory-under-$T$ (equivalently, $T/1$) by appeal to the first half of \cref{Lemma1} as applied to $T/- : \op{T} \to \LexCat$. Note that, for any particular object $t$ of $T$, this identifies $T/1$ inside $T/t$ via pullback along the unique map from $t$ to $1$; thus, $T$ is identified as the subcategory of \quote{constant} data within $T/t$. We may explicitly refer to this inclusion lexfunctor as $i_t : T \to T/t$. (Beware that in general, $i_t$ is faithful but not full!)

Now, what we want to show that given any fixed diagram in $\LexCat$ of this form
\[\begin{tikzcd}
	& T \\
	{T/t} && V
	\arrow["{i_t}"', from=1-2, to=2-1]
	\arrow["f", from=1-2, to=2-3]
\end{tikzcd}\]
there is a correspondence between commutative triangles extending this diagram with a lexfunctor from $T/t$ to $V$, and elements of $\Hom_V(1, f(t))$.

Well, suppose given $m : 1 \to f(t)$ in $V$. Consider now how the action of $f$ on arrows induces a functor $f' : T/t \to V/f(t)$. Because $f$ preserves finite limits, and finite limits in a category determine the finite limits in its slice categories, this functor $f'$ also preserves finite limits. Furthermore, by pullback along $m$, we get a lexfunctor $m^* : V/f(t) \to V$. Thus, altogether, we get a lexfunctor $m^* \circ f' : T/t \to V$. What's more, the following diagram commutes:

\[\begin{tikzcd}
	& T \\
	{T/t} && V \\
	& {V/f(t)}
	\arrow["{i_t}"', from=1-2, to=2-1]
	\arrow["f", from=1-2, to=2-3]
	\arrow["{f'}"', from=2-1, to=3-2]
	\arrow["{m^*}"', from=3-2, to=2-3]
\end{tikzcd}\]

To see that this diagram commutes, observe that it can be decomposed like so:

\[\begin{tikzcd}
	& T \\
	{T/t} & V & V \\
	& {V/f(t)}
	\arrow["{i_t}"', from=1-2, to=2-1]
	\arrow["f", from=1-2, to=2-3]
	\arrow["{f'}"', from=2-1, to=3-2]
	\arrow["{m^*}"', from=3-2, to=2-3]
	\arrow["f"', from=1-2, to=2-2]
	\arrow["{i_{f(t)}}"', from=2-2, to=3-2]
\end{tikzcd}\]

Here, $i_{f(t)}$ is the inclusion lexfunctor from $V$ into $V/f(t)$ in the analogous way to $i_t : T \to T/t$. The left half of this diagram commutes because, $f$ being a lexfunctor, the operations \quote{Pull back along the map from $t$ to $1$ and then apply $f$ to the resulting slice} and \quote{Apply $f$ and then pull back along the map from $f(t)$ to $1$} are the same. The right half of this diagram commutes because $m^* \circ i_{f(t)}$ is identity, because this amounts to pulling back along two morphisms in a row, ultimately pulling back along a path from $1$ to $1$, and any morphism from $1$ to $1$ is the identity. \TODOinline{Phrase this better?}

Thus, every element of $\Hom_V(1, f(t))$ induces a corresponding commutative triangle of lexfunctors from $i_t : T \to T/t$ to $f : T \to V$.

Conversely, any such commutative triangle induces an element of $\Hom_V(1, f(t))$. To see this, we just need to see that there is some global element of $i_t(t)$ already in $T/t$.

Note that $i_t(t)$ is one of the projections from $t \times t$ to $t$. Furthermore, the terminal object in $T/t$ is the identity slice from $t$ to $t$. So the following commutative triangle serves as a global element of $i_t(t)$ within $T/t$:

\[\begin{tikzcd}
	t && {t \times t} \\
	& t
	\arrow["id"', from=1-1, to=2-2]
	\arrow["{i_t(t)}", from=1-3, to=2-2]
	\arrow["{\langle \id, \id \rangle}", from=1-1, to=1-3]
\end{tikzcd}\]

Let us use the name $g_t$ for this global element of $i_t(t)$ within $T/t$.

Now, we need to show that these two processes are inverse. \TODO

Finally, we must show that the re-indexings given by morphisms in $T$ via the definition of $T[1 \to -]$ correspond to the same reindexings as defined via $T/-$. \TODO

\TODOinline{The details of this are probably in Uemura, and we can just cite that on top of Bart Jacobs and be done}
\end{proofEnd}

\begin{corollary}\label{SelfIndexingIsFreeCorollary}
The analogues of \magicref{SelfIndexingIsFree} automatically also follow for any categorical structure extending the structure of a lexcategory which is automatically transferred to slice categories and preserved by pullback; that is, any structure which automatically transfers from an instance that structure also to its self-indexing (e.g., for the concepts of locally cartesian closed categories or for elementary toposes or for regular categories). And in just the same way also for any categorical structure extending the structure of a category with finite limits, which automatically transfers from an instance of that structure to its simple self-indexing (e.g., for cartesian closed categories).
\end{corollary}

\subsection{Double or multiple indexing}
\TODOinline{Reader beware, this section is only needed for understanding some of the discussion at the beginning of the chapter on Geminal Categories. Apart from that, this section does not come up significantly. TODO: Emphasize the notion of multiply internal structures, the main thing this section is used to discuss.}

At this point, for any algebraic-categorical notion $S$ (e.g., the notion of a commutative ring, or the notion of a lexcategory), we also have a definition of the notion of a pair of a category and an instance of notion $S$ indexed over that category.

We can thus iterate this process. In particular, we can speak of a $T$-indexed (category $C$ and $C$-indexed set $P$). We can call this also a $(T, C)$-indexed set $P$. Let us observe in more detail what this amounts to.

What this means is that, in addition to having a category $T$ and a $T$-indexed category $C$, we also have for every object $t$ in $T$, some corresponding $C(t)$-indexed set. Thus, we obtain for each $t$-indexed object $c$ of $C$ a corresponding set we may denote $P(t)(c)$ or $P(t, c)$ or $P_t(c)$ (the $t$-defined $c$-defined elements of $P$). And for each morphism $n : c \to d$ in $C(t)$, we have a reindexing function $P(t, n) : P(t, d) \to P(t, c)$. These reindexings act functorially in that $P(t, n_1 \circ \ldots \circ n_k) = P(t, n_k) \circ \ldots \circ P(t, n_1)$ for any sequence of composable morphisms $n_1, \ldots, n_k$ in $C(t)$.

But furthermore, we must have functorial reindexing maps for $P$ along morphisms of $T$. This means, for any map $m : s \to t$ in $T$, we must have for every $t$-defined object $c$ of $C$ a reindexing function $P(m, c) : P(t, c) \to P(s, C(m)(c))$.  We may just write $P(m)$ to refer generically to any $P(m, c)$. These reindexings act functorially in that $P(m_1 \circ \ldots \circ m_k) = P(m_k) \circ \ldots \circ P(m_1)$ for any sequence of composable morphisms $m_1, \ldots, m_k$ in $T$.

Finally, the reindexings along morphisms of $T$ must preserve in a suitable sense the reindexings along morphisms of $C$. This means the following square of reindexings commutes, for any morphisms $m : s \to t$ in $T$ and $n : c \to d$ in $C(t)$:

% https://q.uiver.app/?q=WzAsNCxbMCwwLCJQKHQsIGQpIl0sWzAsMSwiUCh0LCBjKSJdLFsxLDAsIlAocywgQyhtKShkKSkiXSxbMSwxLCJQKHMsIEMobSkoYykpIl0sWzAsMSwiUCh0LCBuKSIsMl0sWzAsMiwiUChtKSJdLFsxLDMsIlAobSkiLDJdLFsyLDMsIlAocywgQyhtKShuKSkiXV0=
\[\begin{tikzcd}
	{P(t, d)} & {P(s, C(m)(d))} \\
	{P(t, c)} & {P(s, C(m)(c))}
	\arrow["{P(t, n)}"', from=1-1, to=2-1]
	\arrow["{P(m)}", from=1-1, to=1-2]
	\arrow["{P(m)}"', from=2-1, to=2-2]
	\arrow["{P(s, C(m)(n))}", from=1-2, to=2-2]
\end{tikzcd}\]

Using this coherence condition, any reindexing in $C$ followed by a reindexing in $T$ (the left-bottom path) can be turned into an equivalent reindexing in $T$ followed by a reindexing in $C$ (the top-right path). Thus, for any string of reindexings (alternating between reindexings in $C$ and reindexings in $T$), there is a unique reindexing in $T$ followed by a reindexing in $C$ which it is forced equivalent to by the coherence condition and functoriality.

Thus, we can resummarize all of these conditions like so: We create a category denoted $\Groth_{T} C$ (or just $\Groth C$) whose objects are pairs $(t, c)$ where $t$ is an object in $T$ and $c$ is an object in $C(t)$. A morphism in $\Groth C$ from $(s, c)$ to $(t, d)$ is given by a pair $(m, n)$ where $m : s \to t$ in $T$ and $n : c \to C(m)(d)$ in $C(s)$. This represents a reindexing along $m$ followed by a reindexing along $n$, and so by consideration of the previous paragraph, we get also the appropriate composition rule validating our desired coherence condition and automatically ensuring associativity. Specifically, the appropriate composition rule is that $(a, n)$ followed by $(m, b)$ composes to $((a ; m), (C(m)(n) ; b))$, as can be visualized from our above-noted coherence condition like so:

% https://q.uiver.app/?q=WzAsNixbMiwwLCJQKHQsIGQpIl0sWzIsMSwiUCh0LCBjKSJdLFs0LDAsIlAocywgQyhtKShkKSkiXSxbNCwxLCJQKHMsIEMobSkoYykpIl0sWzAsMCwiXFxidWxsZXQiXSxbNCwyLCJcXGJ1bGxldCJdLFswLDEsIlAodCwgbikiLDFdLFswLDIsIlAobSkiXSxbMSwzLCJQKG0pIl0sWzIsMywiUChzLCBDKG0pKG4pKSIsMV0sWzQsMCwiUChhKSJdLFszLDUsIlAoYikiXSxbNCwxXSxbMSw1XV0=
\[\begin{tikzcd}
	\bullet && {P(t, d)} && {P(s, C(m)(d))} \\
	&& {P(t, c)} && {P(s, C(m)(c))} \\
	&&&& \bullet
	\arrow["{P(t, n)}"{description}, from=1-3, to=2-3]
	\arrow["{P(m)}", from=1-3, to=1-5]
	\arrow["{P(m)}", from=2-3, to=2-5]
	\arrow["{P(s, C(m)(n))}"{description}, from=1-5, to=2-5]
	\arrow["{P(a)}", from=1-1, to=1-3]
	\arrow["{P(b)}", from=2-5, to=3-5]
	\arrow[from=1-1, to=2-3]
	\arrow[from=2-3, to=3-5]
\end{tikzcd}\]

Then, a $(T, C)$-indexed set is just the same as as a $(\Groth_T C)$-indexed set. This also gives us easily the right notion of maps between $(T, C)$-indexed sets. They are just maps between the corresponding $(\Groth_T C)$-indexed sets (i.e., natural transformations between presheaves on $\Groth_T C$). In this way, we can speak about $(\Groth_T C)$-indexed structures more generally than mere sets.

We will only rarely need to consider any of this multi-indexing, and to the extent we do, almost always will only consider $(T, C)$-indexed sets $P$ in cases where $C$ is in fact $T$-small, and furthermore $P$ is $T$-small. \TODOinline{Define what it means for $P$ to be $T$-small. Show how this leads to a simpler representation of $P$ as living internally to $T$}

The construction $\Groth_T C$ is called the Grothendieck construction. By projecting out first coordinates, we get a functor from $\Groth_T C$ to $T$; functors which arise in this way are called Grothendieck fibrations, or just fibrations. It turns out, given a fibration just as an abstract functor between categories, one can recover the indexed category which gave rise to it.

Thus, the data of an indexed category is equivalent to the data of a fibration. The entire machinery of indexed categories can therefore equivalently be presented in terms of fibrations. For this reason, fibrations are also called fibered categories. In particular, one can give a more intrinsic account of the conditions under which an arbitrary functor is a fibration. Furthermore, any natural transformation between $T$-indexed categories induces a corresponding map between the corresponding fibrations in $\Cat/T$, and again the natural transformation can be recovered from this map, and again a more intrinsic account can be given of which maps arise in this way. Some things are easier to describe in a fibration-based presentation. Other things are more difficult. For our purposes (using the general language of indexed structures ultimately for the goal of understanding specifically \repsmall/ or internal structures), we felt the indexed category presentation was the most apt. Thus, we will not describe the theory of fibered categories further. We use the Grothendieck construction only for the correspondence between $(\Groth_T C)$-indexed sets and $(T, C)$-indexed sets.

Of course, this construction can be iterated further now. A $(T, C)$-indexed category $D$ is a $(\Groth_T C)$-indexed category (i.e., a contravariant functor from $\Groth_T C$ to $\Cat$), and thus gives rise to another category $\Groth_{\Groth_T C} D$. Structures indexed over $\Groth_{\Groth_T C} D$ can be called $(T, C, D)$-indexed structures. And so on ad infinitum. Do not worry, we will not need to explicitly consider this to any further depth of indexing.



\TODOinline{Discuss the concept of being $T$-\repsmall/ vs. $C$-\repsmall/ when $(T, C)$-indexed}

Note also that any structure singly-indexed over $T$ can automatically be thought of as doubly-indexed over $T$ and $C$, where the indexing over $C$ is trivial. This is basically by the fact that the Grothendieck construction for $T$ and $C$ comes with a projection functor to $T$, so that all $T$-indexed structures thus induce, via this functor, a $T$-and-$C$-indexed structure. Thus, we can readily speak of maps between $T$-indexed structures and $(T, C)$-indexed structures, by treating the former as implicitly $(T, C)$-indexed themselves.

Indeed, more generally in the multiply indexed context, any structure indexed over some prefix of a string of categories is automatically indexed over the full string of categories. And in the same way, this allows us to speak of maps between structures indexed by different strings of categories. This is the main reason for us to bring all this up, just so that we can speak of maps between structures at different levels of indexing.

(Keep in mind also that an honest-to-goodness actual structure, living in $\Set$, is like the zero-ary case of indexing; it's indexed by the empty string of categories $()$, but can be seen in a trivial way as $T$-indexed for any category $T$).

Note that a map from a $T$-indexed structure $A$ to a $(T, C)$-indexed structure $B$ thus amounts to a map from $A$ to $\Hom_C(1, B)$, whenever $C$ has a terminal object. So all this high-faluting multiply indexed stuff just amounts to another way of thinking about maps into global aspects.

\subsection{Unorganized draft category-theoretic lemmas}

\TODOinline{Everything from here on out in the preliminaries is in a draft state, not yet readable}

\subsubsection{Laxly indexed strict categories}
\begin{definition}\label{LaxlyIndexedStrictCategory}
Given a strict category $C$, we define the notion of a \defined{laxly indexed strict category} on it. This is the concept of a lax functor from $\op{C}$ to the category of strict categories. In detail it is, like so: A laxly $C$-indexed strict category $L$ consists of the following data: A strict category $L(c)$ for each object $c$ of $C$. Furthermore, a strict functor $L(m) : L(d) \to L(c)$ for each morphism $m : c \to d$ in $C$. Furthermore, a natural transformation from $\id_{L(c)}$ to $L(\id_c)$ for each object $c$ in $C$. Furthermore, a natural transformation from $L(n) \circ L(m)$ to $L(m \circ n)$ for each composable pair of morphisms $n : b \to c$ and $m: c \to d$ in $C$. Furthermore, there are some coherence conditions to impose on these natural transformations, but they do not matter for our purposes so we do not bother stating them. These are the conditions which would ensure that, if these natural transformations were in fact invertible, this comprised a pseudofunctor.
\end{definition}

In fact, absolutely none of the conditions here which concern equalities between elements of any $\Mor(L(c))$ for any object $c$ in $C$ matter to us. It doesn't matter for our purposes that composition in any category $L(c)$ of a laxly indexed strict category be associative or satisfy the identity law or any of that. All that matters to us is that certain operations form elements of $\Mor(L(c))$ with the appropriate domains and codomains.

We only need this notion for one example in one situation. The one example we use is the following construction:

\begin{construction}\label{CoreOfSelfIndexing}
Given any strict lexcategory $C$, there is a straightforward way to create a laxly indexed strict category indexed by $C$ where $L(c)$ is the core of the slice category $C/c$ (that is, the subcategory of $C/c$ containing all its objects, but only those of its morphisms which are isomorphisms), while $L(m)$ is given by pullback along $m$ (the chosen pullbacks of our strict lexcategory $C$). Call this the \defined{core of the self-indexing} of $C$. The universal property of pullbacks ensures that this is pseudofunctorially indexed, and thus a fortiori laxly indexed. However, as chosen pullbacks satisfy no further demanded coherence beyond just being pullbacks, there is no reason to expect the natural transformations witnessing pseudofunctoriality here to in fact be strict identities (that is, chosen pullback along a composition of morphisms needn't strictly match the composition of chosen pullbacks along the individual morphisms).

Note that, when $C$ is internal to lexcategory $T$, this construction can be carried out internal to $T$ as well.
\end{construction}

There are methods by which one can work to strictify a pseudofunctor in such situations, but we do not need these methods for the one instance in which we deal with this. Just the bare minimum structure of a laxly indexed strict category, sans even the coherence conditions on its indexed morphisms, suffices for us.

A laxly $C$-indexed discrete strict category is the same as a $C$-indexed set, and thus we also get degenerate examples of laxly $C$-indexed strict categories from any $C$-indexed set.

\subsubsection{Comma categories}

\begin{theorem}[The Comma-Kan Lemma] \label{CommaKan}
Suppose, within some 2-category $C$, we have the following comma object and left Kan extensions:

% https://q.uiver.app/?q=WzAsNyxbNCwyLCJYIl0sWzQsNCwiXFxvbWVnYSJdLFsyLDQsIlkiXSxbMiwyLCIoZl9YL2ZfWSkiXSxbMSwxLCJBIl0sWzAsMCwiQiJdLFs1LDFdLFswLDEsImZfWCJdLFsyLDEsImZfWSIsMl0sWzMsMiwiXFxwaV9ZIiwyXSxbMywwLCJcXHBpX1giXSxbMCwyLCJcXGdhbW1hIiwyLHsibGV2ZWwiOjJ9XSxbNCwzLCJzIiwxXSxbNCw1LCJxIiwxXSxbNSwwLCJcXExhbl9xIChzOyBcXHBpX1gpIl0sWzUsMiwiXFxMYW5fcShzOyBcXHBpX1kpIiwyXSxbMywxNCwiXFxlcHNpbG9uX1giLDEseyJzaG9ydGVuIjp7InNvdXJjZSI6MjB9fV0sWzMsMTUsIlxcZXBzaWxvbl9ZIiwxLHsic2hvcnRlbiI6eyJzb3VyY2UiOjIwfX1dXQ==
\[\begin{tikzcd}
	B \\
	& A &&&& {} \\
	&& {(f_X/f_Y)} && X \\
	\\
	&& Y && \omega
	\arrow["{f_X}", from=3-5, to=5-5]
	\arrow["{f_Y}"', from=5-3, to=5-5]
	\arrow["{\pi_Y}"', from=3-3, to=5-3]
	\arrow["{\pi_X}", from=3-3, to=3-5]
	\arrow["\gamma"', Rightarrow, from=3-5, to=5-3]
	\arrow["s"{description}, from=2-2, to=3-3]
	\arrow["q"{description}, from=2-2, to=1-1]
	\arrow[""{name=0, anchor=center, inner sep=0}, "{\Lan_q (s; \pi_X)}", from=1-1, to=3-5]
	\arrow[""{name=1, anchor=center, inner sep=0}, "{\Lan_q(s; \pi_Y)}"', from=1-1, to=5-3]
	\arrow["{\epsilon_X}"{description}, shorten <=3pt, Rightarrow, from=3-3, to=0]
	\arrow["{\epsilon_Y}"{description}, shorten <=4pt, Rightarrow, from=3-3, to=1]
\end{tikzcd}\]

Furthermore, suppose $f_X$ preserves the Kan extension $\Lan_q (s; \pi_X)$. (We notably do NOT make any such assumption on $f_Y$).

Then $\Lan_q s : B \to (f_X / f_Y)$ exists and furthermore is preserved by both $\pi_X$ and $\pi_Y$.
\end{theorem}
\begin{proof}
We may compute as follows: By the universal property of the comma object $(f_X / f_Y)$, the set of 1-cells from $B$ to this comma object whose projections match our two Kan extensions is given by the set of 2-cells between the top and bottom path from $B$ to $\omega$ in the above diagram. Since $f_X$ preserves the top Kan extension, the top path is itself a left Kan extension, and using its universal property, we find that the 2-cells from the top path to the bottom path are the same as 2-cells between two different paths from $A$ to $\omega$; specifically, from $s; \pi_X; f_X$ to $q; \Lan_q(s; \pi_Y); f_Y$. Such a 2-cell is given by the composition of $\gamma$ and $\epsilon_Y$.

Thus, we do indeed get a 1-cell $m : B \to (f_X / f_Y)$ whose composition with each projection $\pi$ matches $\Lan_q(s; \pi)$. What remains is only to show that $m$ is indeed $\Lan_q s$.

Let an arbitrary $k : B \to (f_X/f_Y)$ be given. By the universal property of the comma category again, we have that 2-cells from $m$ to $k$ are in correspondence with choices of 2-cells from $m ; \pi$ to $k ; \pi$ for each projection $\pi$, such that both resulting composite 2-cells from $m; \pi_X; f_X$ to $k ; \pi_y; f_Y$ are equal. But each $m ; \pi = \Lan_q(s; \pi)$, so a 2-cell from this to $k ; \pi$ amounts to a 2-cell from $s; \pi$ to $q; k ; \pi$. A choice of such 2-cells satisfying the coherence condition is, again by the universal property of the comma category $(f_X / f_Y)$, the same thing as a 2-cell from $s$ to $q; k$. Thus, we have shown $\Hom(m, k) = \Hom(s, (q; k))$, which establishes $m$ as satisfying the universal property defining $\Lan_q s$. This completes the proof.
\end{proof}

Dually, we can turn all 2-cells around in this theorem, replacing the left Kan extensions with right Kan extensions and now having a condition that the functor on the codomain side of our comma category must preserve the corresponding Kan extension.

\begin{corollary}\label{CommaCategoryColimits}
Given a cospan of functors $f_X, f_Y$ from respective categories $X$ and $Y$, if $X$ and $Y$ both have colimits of a particular shape and $f_X$ preserves colimits of that shape, then $(f_X / f_Y)$ has colimits of that shape, preserved by both projections $\pi_X$ and $\pi_Y$.

Furthermore, an arbitrary functor into $(f_X / f_Y)$ then preserves colimits of that shape iff its composition with both projections preserves such colimits.

(Dually, we have the corresponding statements where all instances of \quote{colimit} are turned into \quote{limit} and the first statement's limit preservation condition is put on $f_Y$ rather than $f_X$.)
\end{corollary}
\begin{proof}
The first statement follows from \magicref{CommaKan} by taking $A$ to be the generic category of the indicated shape and taking $B$ to be the terminal category, considering how colimits correspond to left Kan extensions along such a functor.

The second statement then follows from the fact that in addition to the the forgetful functor from $(f_X / f_Y)$ into $X \times Y$ preserving colimits as just shown, such a forgetful functor from a comma category to a product category is always conservative (that is, if the image of a morphism under this functor is invertible, the morphism was already invertible in the comma category). A conservative functor which preserves colimits automatically has the desired property.
\end{proof}

\begin{corollary}\label{LexCatComma}
Comma objects exist in $\LexCat$, constructed in the same way as in $\Cat$ (thus, preserved by the forgetful functor into $\Cat$).
\end{corollary}
\begin{proof}
From \magicref{CommaCategoryColimits}, we see that when $f_X, f_Y$ are a co-span of finite limit preserving functors between categories which have finite limits, then the comma category $(f_X / f_Y)$ (the comma object in $\Cat$) is also a lexcategory and its projections are lexfunctors. Thus, this $(f_X / f_Y)$ and its projections exist within $\LexCat$. That these continue to comprise a comma object span within $\LexCat$ follows immediately from the fact that the forgetful functor $|-|$ from $\LexCat$ to $\Cat$ induces bijections between the sets of 2-cells $\Hom(f, g)$ and $\Hom(|f|, |g|)$ for any parallel 1-cells $f$ and $g$ in $\LexCat$ (that is, the 2-cells in $\LexCat$ between lexfunctors are just ordinary natural transformations, with no further property or structure).
\end{proof}

The conclusion of \magicref{CommaKan} will often be particularly useful to us in conjunction with the following lemma.

\begin{lemma}\label{CommaMap}
Let $C$ and $D$ be 2-categories, with $|-| : C \to D$ a 2-functor, and suppose for all objects $X$, $Y$, and $\omega$ of $C$ along with 1-cell $f_X : X \to \Omega$ in $C$ and 1-cell $f_Y : |Y| to |\omega|$ in $D$, we have a span in $C$ whose image under $|-|$ is the comma object $(|f_X| / f_Y)$ in $D$ along with its projections.

Furthermore, suppose $X$ is initial in $C$. Then the identity map on $|X|$ is initial within $D(X, X)$.
\end{lemma}
\begin{proof}
Take $Y = \omega = X$, let $f_X$ be the identity on $X$, and let $f_Y : |X| \to |X|$ in $D$ be arbitrary. \TODO
\end{proof}

For turning the initial AU into our PA sigma 1 models, etc, after we've established that AUs are closed under comma category constructions, let X = Y = omega =the initial AU in which some AU definable morphisms are invertible, let f_X = identity, and f_y = arbitrary lex. It follows that the comma category is also an AU, and because everything is component wise, the projections send the relevant AU-definable morphisms to invertible morphisms. Since projections from a comma object are jointly conservative, the comma category is also an AU where the AU definable morphisms of note are invertible. Thus, X maps into the comma category in a unique AU way; the projections following this mapping are identity, and we get a natural transformation from identity on X to f_Y.

\subsubsection{Theories, models, functors, etc}
\begin{TODOblock}\label{ModelTerminology}
We need to standardize our terminology on theories, models, and internal models. Is a lexfunctor from $T$ to $S$ an internal model of $T$ in $S$? Or is a lexfunctor from $T$ the globalization of an internal category in $S$ an internal model of $T$ in $S$? Is a model always Set-valued? Etc. Let us do a Ctrl+F for "model" to make sure we are not confusing on this point. It may be best to speak scrupulously of lexcategories and lexfunctors at all times, and not of theories or models, except in the fixed phrase "introspective theory".

We can use interior vs. included.
\end{TODOblock}

When we say "lex theory", this is a synonym for lexcategory (emphasizing the correspondence between lexcategories and essentially algebraic theories).

Given a lex theory $T$ thought of as \quote{the theory of gadgets}, we can identify any particular \quote{gadget} $G$ with a corresponding lexfunctor $g : T \to \Set$, or vice versa. Even though these are equivalent data, it may be linguistically natural to avoid identifying them. (For example, if $T$ is the free lexcategory with an internal category, then a functor $: T \to Set$ corresponds to a strict category, yet it could be confusing to refer to it simultaneously as a functor and as a category.). The linguistic device we use for disambiguating this when necessary is by calling the gadget $G$ the \defined{down-reification} of the functor $g : T \to \Set$, and conversely, calling the functor $g : T \to \Set$ the \defined{up-reification} of the gadget.

In the same way, we may say of a diagram $G$ (of a particular shape involving some particular finite limits) within a lexcategory $S$ as corresponding to a lexfunctor $g$ into $S$ from the free lexcategory containing a diagram of such a shape. Again, these are equivalent data but we may wish to linguistically differentiate between, say, \quote{a pullback square $G$ in $S$} (consisting of four objects and four morphisms) and \quote{a functor $g$ into $S$ from the countably infinite category which is the free lexcategory on a pullback square}. Again, in these scenarios, we say the lexfunctor is the up-reification of the diagram, or the diagram is the down-reification of the lexfunctor.

Given a category $T$ and a $T$-indexed (or $T$-internal) category $C$, we may speak of a diagram internal to $C$, as shorthand meaning a diagram internal to the global aspect of $C$. Similarly, in such a situation, we may speak of a functor into $C$ from another (non-indexed) category $S$, as shorthand meaning a functor from $S$ to the global aspect of $C$. We can make sense of this shorthand internally too: If $C$ is a $T$-indexed category, then by a category $C'$ internal to $C$, we mean that $C'$ is internal to the global aspect of $C$. Given another $T$-indexed category $D$, by a $T$-indexed functor from $D$ to $C'$, we mean a $T$-indexed map that assigns to each object of $D$ an object in the $T$-indexed category corresponding to the aspect of $C'$ at the terminal object of $C$. \TODOinline{Yeah, this belongs in the multi-indexing section. It's getting to be a mess. Let's try not to use this kind of shorthand, instead.}

\TODOinline{Introduce terminology for "the walking X" meaning "the free lexcategory with an internal X". "Internal X" always means "Map into me from the lex theory of Xes", and NOT "Internal lexcategory in me, and map into THAT from the lex theory of Xes". The latter should be called doubly internal. The theory of Xes has an internal X. Hm. Maybe we need good terminology for "the theory of blahs" vs "a blah", when blah is itself a lexcategory-extending notion.}

\subsubsection{Interpreting essentially algebraic theories into strict lexcategories}
\TODOinline{Clean this section up, motivate it, use it}

Consider the free/forgetful adjunction between strict categories and strict lexcategories. For any strict lexcategory, we can consider the unit of this adjunction at it. If this unit is injective on objects, we say this strict lexcategory \defined{imposes no equations on objects}. When we say \defined{strict lex theory}, we mean a strict lexcategory that imposes no equations on objects. That is, it does not have an object $X$ such that $X \times X = X$ strictly, or any such thing. These are the appropriate strict lexcategories to model essentially algebraic theory for interpretation into other strict lexcategories, in the following sense: 

If $T_{strict}$ is a strict lexcategory which imposes no equations on objects, and $S_{strict}$ is a strict lexcategory, and these present non-strict lexcategories $T$ and $S$, then the strict category $\StrictLexCat(T_{strict}, S_{strict})$ presents the category $\LexCat(T, S)$.
\begin{proof}
???? \TODO

There is a clear functor from the category presented by $\StrictLexCat(T_{strict}, S_{strict})$ to $\LexCat(T, S)$. We must show that this is essentially surjective on objects, full, and faithful.

So, for essential surjectivity on objects: Suppose given a lexfunctor from T to S. Perhaps this guarantees us an equivalent strict functor from T_{strict} to S_{strict}?

Then perhaps we can replace a strict functor which weakly preserves limits by an equivalent strict lexfunctor whenever the domain doesn't impose equations on objects? We take the strict functor from T_{strict} to S_{strict}, and turn it into a strict lexfunctor from Free(T_{Strict}) to S_{strict}. Ok, what we want is a strict lexfunctor from T_{Strict} to Free(T_{strict}).

If that's so, then the next step is fullness and faithfulness. Given two strict lexfunctors, suppose given an arbitrary natural transformation between them in LexCat. This is some selection of morphisms of S satisfying some conditions. This is unaffected by concerns about object duplication or basic limit preservation on the nose. So the conditions defining such a natural transformation in terms of selections of morphisms of S_{strict} are the same. Thus, fullness and faithfulness is guaranteed.
\end{proof}

Let's just make this true by definition instead. Let's say a strict lexcategory imposes no equations on objects if any strict functor out of it to another strict lexcategory which preserves limits weakly can be replaced with an equivalent functor which preserves limits on the nose.

Every strict lexcategory $T$ presents a category which can also be presented by a strict lexcategory which imposes no equations on objects.
\begin{proof}
??? \TODO

Take the free strict lexcategory on the underlying category of $T$ in which the basic limit comes in $T$ become limit cones (but not necessarily chosen limit cones; that is, we freely impose that the comparator map to the chosen limit is an isomorphism). This will be equivalent to T (proof??) and a map from it to someone else amounts to a strict lexfunctor out of $T$ which weakly preserves limits.
\end{proof}

Fix a lexcategory $T$, a set $D$, and a function $f : D \to \Ob(T)$ (since $T$ is non-strict, this really means a functor from the discrete category presented by $D$ to $\Ob(T)$, but that is fine). We say this function $f$ is dense if any full sublexcategory of $T$ containing (up to isomorphism) every object in the range of $f$ also contains (up to isomorphism) every object in $D$. Call this a lexcategory with strictly specified sorts.

The appropriate notion of a lex theory to have models in a strict lexcategory is a lexcategory with strictly specified sorts.

Consider the concept of a strict lexcategory $C$ and a lexfunctor from $T$ to the lexcategory presented by $C$. This is ALMOST an essentially algebraic concept, except for the fact that $T$ has not been given strictly specified sorts.

Let $T_{strict}$ be a strict lexcategory presenting the lexcategory $T$, and let $S_{strict}$ be a strict lexcategory presenting the lexcategory $S$.

The category $\StrictCat[T_{strict}, S_{strict}]$

Blah blah

Let us say the category of sketches is the minimal subcategory of the category of strict lexcategories closed under set-sized colimits indexed by posets and under freely adjoining objects, freely adjoining a morphism between existing objects, or freely making two parallel morphisms equal. A finite sketch is the same thing but with closure only under finite colimits (but also, we don't need the finite colimits here).

A sketch has the desired property of imposing no equations.

\TODOinline{Perhaps we should mark somehow those theorems which are specific to $\Set$ and perhaps use Choice vs those theorems which are meant to internalize}

\subsubsection{Existence of initial structures in $\Set$, $\LexCat$, etc}
\TODO

Some general theorem here which we will invoke in $\Set$, $\LexCat$ etc. Not to be invoked internal to toposes or anything, mind you, where different arguments are necessary than the following.

\begin{theorem}\label{Adamek}
The general theorem would presumably be the transfinite construction of initial algebras for pointed endofunctors preserving colimits indexed by ordinal $k$, on a category with colimits indexed by any ordinal $\leq k$. (For non-pointed endofunctors, we can do the same as long as we also have finite coproducts)
\end{theorem}

In our metatheory, we assume such a theorem and also we assume that initial small instances of some variety of category also always have unique homomorphsms into $\Set$ itself, if $\Set$ is that variety of category. (This is as opposed to the kind of initial algebras we get internal to a topos, arithmetic universe, etc, where $\Set$ is too large for the initiality to apply with respect to it)

Actually, we don't really need all that explicitly.

"Partial Horn logic and cartesian categories" by Palmgren and Vickers discusses the proof (internalizable to toposes, etc, and also adaptable to the infinitary case) that free models of essentially algebraic theories exist. Their construction makes it very clear that what is needed for the free model of a quasi-equational theory to exist is just an effective regular category with suitable W-types for modeling the terms (including partial terms) and equations of the theory. It is clear how quasi-equational theories generalize to arbitrary arity and the initial model construction as well.

This also discusses how to turn a quasi-equational theory T into the quasi-equational theory of a strict lexcategory with an internal model of T.

Applying the initial model to this gives us an initial strict lexcategory with an internal model of T; the strict lexcategory presented by T, we may say.

They also show, in essence, that the lexcategory this presents has the right universal property in $\LexCat$. And they show every lexcategory is presented by some quasi-equational theory (as is every strict lexcategory). So every result we might care about of this sort is implicit in there.

\subsubsection{Scribbles}
So, for example, let us try constructing, for a lexcategory $T$, the initial lexcategory with an internal initial model of $T$. We find some quasi-equational theory $Th(T)$ presenting $T$. This theory is k-sized, so we work from now on in k-ary quasi-equational theories. We now make the k-ary quasi-equational theory of a strict lexcategory with an internal initial model of $Th(T)$, and apply the initial algebra theorem. We get an initial strict lexcategory with an internal initial model of $Th(T)$; call this $S$.

Suppose we have three quasi-equational theories:
Th(T)
Th(Init-T')
Th(D)

These present strict lexcategories
T'
Init-T'
D'

These present lexcategories
T
Init-T
D

Th(T) is chosen to present T, with T' arising from this.
Th(Init-T') is chosen to present Init-T' as the free strict lexcategory with an internal initial model of Th(T). Init-T then arises from this.
Th(D) is chosen to present D, with D' arising from this.

For any initial model of Th(T) in D', we get a unique map from Init-T' to D' which restricts to the corresponding map from T' to D'.

Given two strict maps f, g from Init-T' to D' which both take T' to initial models in D', we want a unique transform from f to g. Transforms from f to g correspond by lemma 55 (even without the strictness assumption on f, g) to transforms between the corresponding models of Th(Init-T') in D'. We want [TODO] to show that these correspond to transforms between the corresponding models of Th(T) in D'. If we can do that, then since those models are initial, we get a unique map between them, as desired.
\fileend