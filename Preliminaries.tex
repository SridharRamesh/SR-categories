\section{Preliminaries}

\subsection{Terminology pedantry}
We speak in the following frequently of category-valued presheaves (i.e., contravariant functors into the category of categories) and natural transformations between these. Technically, what we mean by these are not ``functors" and ``natural transformations" in the traditional sense, but what some call "pseudofunctors" and ``pseudonatural transformations", or ``2-functors" and ``2-natural transformations", as the category of categories should be viewed as a 2-category, lacking a notion of equality and only having a notion of isomorphism between its 1-cells. That is, wherever one might traditionally ask for an (automatically coherent) system of equalities, this is replaced by a coherent system of isomorphisms. We take the convention that this is what terminology such as "functor" and "natural transformation" means. But we will construct all arguments in such a way as that this is not a bother that needs to be explicitly worried about. 

\subsection{Indexed, internal, and locally internal structures}
We assume familiarity with sets, functions, categories, functors, categories having finite limits (i.e., lexcategories), functors preserving finite limits (i.e., lexfunctors), natural transformations, presheaves, etc, all in the ordinary sense.

Let $T$ be an arbitrary category. By a \defined{$T$-indexed set}, we mean a presheaf on $T$; that is, a contravariant functor from $T$ to $\Set$. By a \defined{function} or \defined{map} or any such thing between $T$-indexed sets, we mean a natural transformation between the corresponding presheaves.

We say an indexed set is \defined{small} if the corresponding presheaf is representable. Via the Yoneda embedding, we identify $T$ itself as the full subcategory of small $T$-indexed sets within the category of all $T$-indexed sets, such that we may speak, for example, of functions from objects of $T$ to $T$-indexed sets.

When $T$ is a lexcategory, we may also define more generally the concept of a function $f : A \to B$ between $T$-indexed sets having \defined{small fibers}: This is defined as every fiber of $f$ over every object $t$ in $T$ being a representable presheaf over $T/t$.

\begin{TODOblock}
Make sure this is the right definition of small fibers and word it the best way. The definition can equally be given as $f$ being the pullback of the slice of global sections of the self-indexing above the self-indexing, though what we mean by pullback here is a pullback of a discrete fibration along a function, which is technically slightly distinct from a pullback of functions. Whatever. It is also useful in this regard to note how this discrete fibration of global sections over the self-indexing amounts to a full and faithful presheaf, by the Yoneda lemma, so that the self-indexing amounts to a full subcategory of Set, in this world.
\end{TODOblock}

We can talk about any kind of $T$-indexed structure or $T$-indexed maps between such structures, as the appropriate diagram of $T$-indexed sets and functions; e.g., $T$-indexed groups and group homomorphisms between them. When the $T$-indexed sets involved are all small, we say this structure is \defined{internal} to $T$. By the Yoneda lemma, this amounts to a diagram of objects and morphisms within $T$ itself.

In the same vein as all this, by a \defined{$T$-indexed category}, we mean a category-valued presheaf on $T$; that is, a contravariant functor from $T$ to $\Cat$, and by a \defined{functor} between $T$-indexed categories, we mean a natural transformation between such presheaves. We say this indexed category is an \defined{indexed lexcategory} (aka, \defined{has finite limits}) if this presheaf factors through the inclusion of $\LexCat$ into $\Cat$; that is, if it takes every object to a lexcategory and every morphism to a lexfunctor. We say a functor between indexed categories \defined{preserves finite limits} if it arises from a natural transformation between the corresponding $\LexCat$-valued presheaves. And in the same way as all this, we can speak of \defined{natural transformations} between functors between indexed categories, or any other familiar categorical structure or property.

Note that, from any lexcategory $T$, we obtain a $T$-indexed lexcategory by considering the functor $T/-$ which assigns to each object $t$ of $T$ the slice category $T/t$, and whose action on morphisms is given by pullback. We refer to this as the \defined{self-indexing} of $T$.

We now might like to speak about a category being small, in the sense that its collection of objects and collection of morphisms are small, but there is one nit to be aware of here. We generally speak about categories in such a way as that they do not have come with a particular notion of their set of objects, as such. That is, two categories may be considered equivalent though presented with different ostensible sets of objects. For example, a category comprised of one terminal object, or a category comprised of two terminal objects, are to be considered equivalent categories. So to speak about categories as having a set of objects, we must imagine the carry as carrying more fine-grained structure than we normally do.

Specifically, let us say a \defined{fine-grained category} is a set of objects, a set of morphisms, and a set of 2-cells imposing equalities between morphsims, satisfying \TODO. (This set of 2-cells is often ignored when talking about categories, but we allow the extra freedom of it, as it will be useful when working over categories which are not exact; it amounts to allowing the set of morphisms to exist not within the original category, but within its ex/lex completion). [We may also speak of a \defined{fine-grained functor}, meaning a homomorphism of such structure that preserves all of it on-the-nose, and so on for every other categorical notion.]

Every fine-grained category [or functor or etc], gives rise to a category [or functor or etc] in whatever ordinary sense one would like to think of these. And conversely, one would ordinarily say every category is equivalent to \emph{some} fine-grained category\footnote{The situation is more nuanced for turning functors between arbitrary categories equivalent to given fine-grained categories into fine-grained functors between the given fine-grained categories, but that will not concern us for now.}. Of course, we can speak of \defined{indexed fine-grained categories} now, as the appropriate diagram of indexed sets and functions between them, and can speak of such indexed fine-grained categories as being small and so on. We will also say an indexed fine-grained category is \defined{locally small} if both the map from its collection of morphisms to the pair of their domain and codomain, and the map from its collection of 2-cells to the pair of their domain and codomain, have small fibers (in other words, though its collection of objects may not be small, everything that exists between any two particular objects is small).

We will now say an indexed category is \defined{small} or \defined{locally small} if it is equivalent to some indexed fine-grained category which is small or locally small, respectively. Note that we do not demand that, as part of its structure, any particular such fine-grained category is selected; merely, that it is possible to do so.