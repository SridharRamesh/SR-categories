\documentclass[./main.tex]{subfiles}
\begin{document}

\section{\Loeb's theorem}

\subsection{Convenient terminology}
Let us introduce some evocative terminology. By a \defined{relation} between objects $A$ and $B$, we mean any pair of morphisms $r_A : R \to A$ and $r_B : R \to B$ with a common domain.

% https://q.uiver.app/?q=WzAsMyxbMSwwLCJSIl0sWzAsMSwiQSJdLFsyLDEsIkIiXSxbMCwxLCJyX0EiXSxbMCwyLCJyX0IiLDJdXQ==
\[\begin{tikzcd}
	& R \\
	A && B
	\arrow["{r_A}", from=1-2, to=2-1]
	\arrow["{r_B}"', from=1-2, to=2-3]
\end{tikzcd}\]

Equivalently, such a relation $\langle r_A, r_B \rangle$ can be thought of as a single morphism $r : R \to A \times B$. Note that we do not make any presumption that $r$ be monic, which is to say, we do not make any presumption that $r_a$ and $r_b$ be jointly monic.

For any generalized elements $a : D \to A$ and $b : D \to B$ of $A$ and $B$ respectively, we say they are \defined{related by} relation $\langle r_A, r_B \rangle: R \to A \times B$ if $\langle a, b \rangle : D \to A \times B$ factors through $\langle r_A, r_B \rangle$. That is, some diagram of the following form commutes:

% https://q.uiver.app/?q=WzAsNCxbMSwxLCJSIl0sWzAsMiwiQSJdLFsyLDIsIkIiXSxbMSwwLCJEIl0sWzAsMSwicl9BIl0sWzAsMiwicl9CIiwyXSxbMywwXSxbMywxLCJhIiwxXSxbMywyLCJiIiwxXV0=
\[\begin{tikzcd}
	& D \\
	& R \\
	A && B
	\arrow["{r_A}", from=2-2, to=3-1]
	\arrow["{r_B}"', from=2-2, to=3-3]
	\arrow[from=1-2, to=2-2]
	\arrow["a"{description}, from=1-2, to=3-1]
	\arrow["b"{description}, from=1-2, to=3-3]
\end{tikzcd}\]

For any map $q: Q \to A$, and any generalized element $b : D \to B$ of $B$, we will say \quote{$b$ is in the range of $q$ with respect to relation $r = \langle r_a, r_b \rangle$} if there is some $a : D \to A$ such that $a$ and $b$ are related by relation $r$ and $a$ factors through $q$. That is, some diagram of the following form commutes:

% https://q.uiver.app/?q=WzAsNSxbMSwxLCJSIl0sWzAsMiwiQSJdLFsyLDIsIkIiXSxbMSwwLCJEIl0sWzAsMCwiUSJdLFswLDEsInJfQSJdLFswLDIsInJfQiIsMl0sWzMsMF0sWzMsMSwiYSIsMV0sWzMsMiwiYiIsMV0sWzQsMSwicSIsMV0sWzMsNF1d
\[\begin{tikzcd}
	Q & D \\
	& R \\
	A && B
	\arrow["{r_A}", from=2-2, to=3-1]
	\arrow["{r_B}"', from=2-2, to=3-3]
	\arrow[from=1-2, to=2-2]
	\arrow["a"{description}, from=1-2, to=3-1]
	\arrow["b"{description}, from=1-2, to=3-3]
	\arrow["q"{description}, from=1-1, to=3-1]
	\arrow[from=1-2, to=1-1]
\end{tikzcd}\]

Our interest in this notion will often be in situations where $A$ and $B$ coincide, and often our relation $r$ will have the properties of an equivalence relation (such as symmetry). But we do not for now make any such assumptions.

Finally, let us say that any $q : Q \to A$ is \quote{\defined{surjective with respect to relation} $r$ over domain of definition $D$} if, for this particular $D$, every generalized element $b : D \to B$ is in the range of $q$ with respect to relation $r$.

If a relation $r$ is not specified when using this surjectivity terminology, it is presumed to be the equality relation; that is, the relation given by two identity morphisms on some same object $A$. In this case, the surjectivity of $q : Q \to A$ over domain of definition $D$ is equivalent to the corresponding function $\Hom(D, q) : \Hom(D, Q) \to \Hom(D, A)$ in $\Set$ being surjective in the ordinary sense.

This was all quite general, and we will use it now to observe a very general construction, which we will then specialize to introspective theories.

\begin{TODOblock}
Some note about how we will also make use in diagrams in this chapter liberally of the identification of $P(c)$ with $\Hom(c, P)$, when $P$ is a presheaf. If one likes, this can be seen as invoking the Yoneda lemma to draw diagrams in $\Psh{C}$ (although invoking the Yoneda lemma is actually a bit overkill for the mere fact that we can draw diagrams of presheaf elements and presheaf actions in this way, which is simply working within the appropriate collage, aka cocomma category).
\end{TODOblock}

\subsection{Lawvere's fixed point theorem}
Let us refresh the reader on Lawvere's fixed point theorem \autocite{lawvere1969diagonal}, which captures the general structure of many diagonalization arguments and their relationship to cartesian closed structure. We shall first present a proof of Lawvere's fixed point theorem in a more or less traditional way (very slightly generalized, but more or less traditional). Then in the next section we will vastly generalize the argument to produce a result applicable in the context of general pre-introspective theories. Then we will specialize back down to introspective theories, and observe a wonderful \quote{bootstrapping} phenomenon which arises there, which shall ultimately provide us with a form of \Loeb's theorem in that context.

\openNamed{theorem}{Lawvere's Fixed Point Theorem}\label{LawvereFPT}
Let $T$ be a cartesian closed category. Let $P$ be any object in $T$. Suppose also given some object $Y$ in $T$ and map $d : Y \to P^Y$ (equivalent to the data of a morphism $: Y \times Y \to P$).

If $d$ is surjective (in the sense of being relationally surjective with respect to the equality relation on $P^Y$, over the domain of definition $1$), then from any endomorphism $f: P \to P$, we obtain a global point $L(f) : 1 \to P$ such that $f \circ L(f) = L(f)$.

More generally, if $d$ is relationally surjective with respect to relation $R$ over domain of definition $1$, then from any endomorphism $f: P \to P$, we obtain a global point $L(f) : 1 \to P$ such that $f \circ L(f)$ is related by $R$ to $L(f)$.
\closeNamed{theorem}
\begin{proof}
\TODOinline{We're a bit glib in all our writing when we swap between thinking of $y$ as an element in $Y$, vs a map from $1$ to $Y$, or similarly in swapping between a presheaf $P^Y$ and values in $P(Y)$ and morphisms from $Y$ to $P$, etc. Perhaps we should introduce some explicit notation for these kinds of conversions, to make everything perfectly clear.}

Observe that we can define a map $W: Y \to P$ in the internal logic of this category like so: For any element $y$ in $Y$, $W(y)$ is $d(y) : P^Y$ applied to $y : Y$ itself, to yield an element of $P$. Thus, when we identify points in $Y$ with maps from $1$ to $Y$, we can illustrate $W(y) : P$ like so:

% https://q.uiver.app/?q=WzAsMyxbMCwwLCIxIl0sWzEsMCwiWSJdLFsyLDAsIlAiXSxbMCwxLCJ5IiwyXSxbMSwyLCJkKHkpIiwyXV0=
\[\begin{tikzcd}
	1 & Y & P
	\arrow["y"', from=1-1, to=1-2]
	\arrow["{d(y)}"', from=1-2, to=1-3]
\end{tikzcd}\]

As with all such definitions in internal logic, we may interpret this in terms of generalized elements (so that $y$ is a generalized element of $Y$, and the above diagram is interpreted in the corresponding aspect of the simple self-indexing $T//-$; in particular, we may do this with $y$ as the generic generalized element given by the identity on $Y$) to obtain a genuine morphism $W : Y \to P$.

Observe now that from any endomorphism $f : P \to P$, we obtain a global element $f: 1 \to P$ like so: By composing $f$ with $W$, we get a map $f \circ W : Y \to P$, which is to say, a global element of $P^Y$. By applying our relational surjectivity presumption on $d$, we obtain a global element $y$ of $Y$ such that this $f \circ W$ is related to $d(y)$.

% https://q.uiver.app/?q=WzAsMyxbMCwyLCJZIl0sWzMsMCwiUCJdLFs2LDIsIlAiXSxbMCwxLCJXIl0sWzEsMiwiZiJdLFswLDIsImQoeSkiLDJdLFsxLDUsIlxcdGV4dHtyZWxhdGVkIGJ5IH1SIiwyLHsic2hvcnRlbiI6eyJ0YXJnZXQiOjIwfSwic3R5bGUiOnsiYm9keSI6eyJuYW1lIjoic3F1aWdnbHkifX19XV0=
% Manually tweaked the shorten bit to be <=7pt, rather than >=7pt. This fixed some weirdness with the arrowhead.
\[\begin{tikzcd}
	&&& P \\
	\\
	Y &&&&&& P
	\arrow["W", from=3-1, to=1-4]
	\arrow["f", from=1-4, to=3-7]
	\arrow[""{name=0, anchor=center, inner sep=0}, "{d(y)}"', from=3-1, to=3-7]
	\arrow["{\text{related by }R}"', shorten <=7pt, Rightarrow, squiggly, from=1-4, to=0]
\end{tikzcd}\]

And applying $W$ to such a $y$, we end up with a global element $W(y)$ of $P$.

Now let us demonstrate that this $W(y)$ actually has the required related point property.

Observe that the relation between $f \circ W$ and $d(y)$ yields also a relation between $f(W(y))$ and $d(y)(y) = W(y)$, like so:

% https://q.uiver.app/?q=WzAsNCxbMSwyLCJZIl0sWzQsMCwiUCJdLFs3LDIsIlAiXSxbMCwyLCIxIl0sWzAsMSwiVyJdLFsxLDIsImYiXSxbMCwyLCJkKHkpIiwyXSxbMywwLCJ5IiwyXSxbMSw2LCJcXHRleHR7cmVsYXRlZCBieSB9UiIsMix7InNob3J0ZW4iOnsidGFyZ2V0IjoyMH0sInN0eWxlIjp7ImJvZHkiOnsibmFtZSI6InNxdWlnZ2x5In19fV1d
% Manually tweaked the shorten bit to be <=7pt, rather than >=7pt. This fixed some weirdness with the arrowhead.
\[\begin{tikzcd}
	&&&& P \\
	\\
	1 & Y &&&&&& P
	\arrow["W", from=3-2, to=1-5]
	\arrow["f", from=1-5, to=3-8]
	\arrow[""{name=0, anchor=center, inner sep=0}, "{d(y)}"', from=3-2, to=3-8]
	\arrow["y"', from=3-1, to=3-2]
	\arrow["{\text{related by }R}"', shorten <=7pt, Rightarrow, squiggly, from=1-5, to=0]
\end{tikzcd}\]

Keeping in mind that the bottom path here is the definition of $W(y)$, this shows that $f(W(y))$ is related to $W(y)$ as desired, completing the proof.

\end{proof}

\begin{corollary}\label{LawvereFPTGeneralDomain}
If the surjectivity assumption on $d$ above is modified to being relationally surjective over domain of definition $D$ rather than domain of definition $1$, we still obtain from any $D$-defined endomorpism $f$ of $Y$ (i.e., any morphism $f : D \times Y \to Y$) a $D$-defined point $L(f)$ of $Y$ (i.e., a morphism $L(f) : D \to Y$) such that, within $T//D$ (the $D$-defined aspect of the simple self-indexing of $T$), we have that $f \circ L(f)$ is related by $R$ to $L(f)$.
\end{corollary}
\begin{proof}
$D$-defined values in $T$ are the same as globally defined values in $T//D$, so we may simply apply the original result to $T//D$.
\end{proof}

\begin{corollary}
All the above results apply just as well for a category with finite products, even if it lacks cartesian closed structure (taking the data of $d$ to be equivalently given by the data of a morphism $: Y \times Y \to P$).
\end{corollary}
\begin{proof}
Any category with finite products embeds as a full sub-category-with-finite-products of a cartesian closed category. For example, we may use the Yoneda embedding into its topos of presheaves. (Indeed, via the Yoneda embedding, any category at all embeds as a full subcategory of a topos, in a manner preserving any limit or exponential structure already present). Thus, we may simply apply the original result to this expanded category.
\end{proof}

\subsection{Very general diagonalization}
Now that we've gone over the traditional Lawvere's fixed point theorem, let us generalize it.

\TODOinline{This revamp is in progress, a way of making the sections before and after this cleaner and easier to follow.}

\begin{theorem}[Name]\label{VeryGeneralSelfRelated}
Let $T$ be an arbitrary category, and let $P \in \Psh{T}$ and $P_2 \in \Psh{T \times T}$. Suppose given also a map $\omega : \pullAlong{\Delta}P_2 \to P$ where $\Delta : T \to T \times T$ is the diagonal functor. (That is, $\omega$ comprises a system of maps $\omega_t : P_2(t, t) \to P(t)$ for each object $t$ in $T$, natural in $t$).

Also, suppose given an object $point$ in $T$ and a binary relation $R$ on the set $P(point)$. (We do not presume $R$ to be an equivalence relation or symmetric or any such thing.)

Let us now suppose given an object $s$ of $T$ and an $App \in P_2(s, s)$ such that $App$ has the pointwise surjectivity-like property that, for every $F \in P(s)$, there is an $f \in \Hom(point, s)$ such that for every $x \in \Hom(point, s)$, we have that $\omega_{point}( App(f, x) ) \in P(point)$ is related by $R$ to $F(x) \in P(point)$.

Then there is an element of $P(point)$ which is related by $R$ to itself.

Specifically, take the supposition two sentences ago and consider the particular case where $F = \omega_s(App) \in P(s)$ and $x = f$. This tells us that $\omega_{point}( App(f, f) )$ is related by $R$ to $\omega_s(App)(f)$. But by naturality, $\omega_s(App)(f) = \omega_{point}( App(f, f) )$. Thus, this is a value which is related to itself.

% https://q.uiver.app/?q=WzAsOCxbMCwwLCJQXzIocywgcykiXSxbMCwzLCJQKHMpIl0sWzMsMCwiUF8yKHBvaW50LCBwb2ludCkiXSxbMywzLCJQKHBvaW50KSJdLFsxLDEsIkFwcCJdLFsyLDEsIkFwcChmLCBmKSJdLFsxLDIsIlxcb21lZ2FfcyhBcHApIl0sWzIsMiwiXFxvbWVnYV9zKEFwcCkoZikgPSBcXG9tZWdhX3twb2ludH0oQXBwKGYsIGYpKSJdLFswLDEsIlxcb21lZ2FfcyIsMl0sWzAsMiwiUChmLCBmKSJdLFsxLDMsIlAoZikiLDJdLFsyLDMsIlxcb21lZ2Ffe3BvaW50fSJdLFs0LDUsIiIsMix7InN0eWxlIjp7InRhaWwiOnsibmFtZSI6Im1hcHMgdG8ifX19XSxbNCw2LCIiLDAseyJzdHlsZSI6eyJ0YWlsIjp7Im5hbWUiOiJtYXBzIHRvIn19fV0sWzYsNywiIiwwLHsic3R5bGUiOnsidGFpbCI6eyJuYW1lIjoibWFwcyB0byJ9fX1dLFs1LDcsIiIsMix7InN0eWxlIjp7InRhaWwiOnsibmFtZSI6Im1hcHMgdG8ifX19XV0=
\[\begin{tikzcd}
	{P_2(s, s)} &&& {P_2(point, point)} \\
	& App & {App(f, f)} \\
	& {\omega_s(App)} & {\omega_s(App)(f) = \omega_{point}(App(f, f))} \\
	{P(s)} &&& {P(point)}
	\arrow["{\omega_s}"', from=1-1, to=4-1]
	\arrow["{P(f, f)}", from=1-1, to=1-4]
	\arrow["{P(f)}"', from=4-1, to=4-4]
	\arrow["{\omega_{point}}", from=1-4, to=4-4]
	\arrow[maps to, from=2-2, to=2-3]
	\arrow[maps to, from=2-2, to=3-2]
	\arrow[maps to, from=3-2, to=3-3]
	\arrow[maps to, from=2-3, to=3-3]
\end{tikzcd}\]
\end{theorem}

\begin{corollary}\label{VeryGeneralFixedModuloRelation}
Let all values be given as in \cref{VeryGeneralSelfRelated}. Furthermore, let the map $n : P \to P$ be given. Then there is an element $e$ of $P(point)$ such that $e$ is related by $R$ to $n(e)$.
\end{corollary}
\begin{proof}
Consider the relation $R'(e_1, e_2) = R(e_1, n(e_2))$. Now suppose given an arbitrary $F \in P(s)$ and let $F' = n(F) \in P(s)$. By the presumed pointwise surjectivity of $App$ with respect to $R$ as applied to $F'$, there is an $f \in \Hom(point, s)$ such that for every $x \in \Hom(point, s)$, we have that $\omega_{point}( App(f, x) ) \in P(point)$ is related by $R$ to $F'(x) = n(F(x)) \in P(point)$. In other words, for all $x$, we have that $\omega_{point}( App(f, x) ) \in P(point)$ is related by $R'$ to $F(x)$. This demonstrates the pointwise surjectivity of $App$ with respect to $R'$ for arbitrary $F$. Thus, we can invoke \cref{VeryGeneralSelfRelated} with $R'$ in place of $R$, to obtain some $e$ which is related by $R'$ to itself. Which is to say, $e$ is related by $R$ to $n(e)$, as desired.
\end{proof}

One corollary of the above is the generalization of Lawvere's fixed point theorem given by David Michael Roberts in \quote{Substructural fixed-point theorems and the diagonal argument: theme and variations}. \TODOinline{Cite properly}. We will not need this particular generalization, but we note it in passing.
\begin{corollary}
Let $T$ be an arbitrary category with objects $point$ and $p$, and let $B : T \times T \to T$ be a bifunctor on $T$ such that we have a transformation $\delta_t : t \to B(t, t)$ natural in objects $t$ of $T$.

Suppose given an object $s$ of $T$ and an $App : B(s, s) \to p$ with the pointwise surjectivity property that for every $F : s \to p$, there is an $f : point \to s$, such that for every $x : point \to s$, we have that $App \circ B(f, x) \circ \delta_{point} : point \to p$ equals $F \circ x : point to p$.

Then every $m : p \to p$ has a fixed point, in the sense of an $e : point \to p$ such that $m \circ e = e$.
\end{corollary}
\begin{proof}
This is the instance of \cref{VeryGeneralFixedModuloRelation} where we take $P_2(t) = \Hom(B(t, t), p)$, $P(t) = \Hom(t, p)$, $\omega_t(-) = - \circ \delta_t$, and take $R$ to be equality.
\end{proof}

The corollary we will need will be more suited to our use of introspective theories. It is like so: \TODO

\subsection{The general construction for pre-introspective theories}
Now that we've gone over the traditional Lawvere's fixed point theorem, let us generalize it to the level of pre-introspective theories.

\openNamed{construction}{General Diagonalization}\label{GeneralDiag}
\closeNamed{construction}
Let $\langle T, C, \introS, \introN, 1 \rangle$ be a pre-introspective unary theory. As a reminder to the reader, this means, let $T$ be a category, let $C$ be a $T$-indexed category, let $\introS$ be a functor from $T$ to the global aspect of $C$, let $1$ be an object in the global aspect of $C$ (not necessarily a terminal object), and let $\introN$ be a natural transformation from $t$ in $T$ to $\Hom_C(1, \introS(t))$.

(In particular, any introspective theory carries this structure, but we wish for a moment to allow ourselves to note that the following construction can be carried out in some greater generality.)

Consider now a $T$-indexed presheaf $P$ on $C$. Suppose also given some object $Y$ in $T$ and map $d : Y \to P(\introS(Y))$.

From these, we can define a map $W$ from $Y$ to $P(1)$ like so: For any $y$ in $Y$, $W(y)$ is $d(y) : P(\introS(Y))$ transported via the presheaf action of $P$ along the value $\introN_Y(y) : \Hom_C(1, \introS(Y))$, to yield a value inside $P(1)$. Thus, $W(y) : P(1)$ is this composition:

\[\begin{tikzcd}
	1 && {\introS(Y)} && P
	\arrow["{d(y)}"', from=1-3, to=1-5]
	\arrow["{\introN_Y(y)}"', from=1-1, to=1-3]
\end{tikzcd}\]

Now, let us suppose $P(1)$ is $T$-small, so that it will make sense to speak of $\introS(P(1))$ and $\introS(W)$.

Furthermore, let us make a relational surjectivity assumption on $d$. Specifically, suppose we are given a relation $R$ on $P$ (that is, another presheaf on $C$, and two parallel maps from it into $P$), and let us suppose $d$ is surjective with respect to relation $R$ (that is, with respect to the aspect of $R$ at $\introS(Y)$) over some domain of definition $D$.

Observe now that, from any $D$-defined generalized element $f$ of $P(\introS(P(1)))$, we can obtain a $D$-defined element of $P(1)$, like so: By transporting $f$ back along $\introS(W) : \Hom_C(\introS(Y), \introS(P(1)))$, we get an element $\introS(W)^*f$ of $P(\introS(Y))$. By applying our surjectivity presumption on $d$, we obtain a $D$-defined element $y$ of $Y$ such that this $\introS(W)^*f$ is related to $d(y)$.

\[\begin{tikzcd}
	&& {\introS(P(1))} \\
	\\
	{\introS(Y)} &&&& P
	\arrow[""{name=0, anchor=center, inner sep=0}, "{d(y)}"', from=3-1, to=3-5]
	\arrow["f", from=1-3, to=3-5]
	\arrow["{\introS(W)}", from=3-1, to=1-3]
	\arrow["{\text{related by } R}"', shorten <=7pt, Rightarrow, squiggly, from=1-3, to=0]
\end{tikzcd}\]

And applying $W$ to such a $y$, we end up with an element of $P(1)$.

This process of turning $f: P(\introS(P(1)))$ into $W(y): P(1)$ will ultimately represent \Loeb's theorem. In our box symbolism, this process turns $f : P(\Box P)$ into $W(y) : P(1)$, and when we apply $\introS$ to such a map, we get a map from $\Box((\Box P) \implies P)$ to $\Box P$. But this construction also captures diagonalization and associated fixed point arguments in general. Let us observe the fixed point properties of this construction in full generality for now, then we will specialize to introspective theories and investigate further.

\subsection{The fixed point/related point property}
Recall that $f$ is a generalized element of $P(\introS(P(1)))$ and $W(y)$ is a similarly generalized element of $P(1)$.
\begin{theorem}\label{GeneralDiagThm}
Transporting $f$ back along $\introN_{P(1)}(W(y)) : \Hom_C(1, \introS(P(1)))$ yields a value related (by our chosen relation $R$) to $W(y)$.
\end{theorem}
\begin{proof}
Recall that $W(y) : P(1)$ is $d(y)$ transported back along $\introN_Y(y) : \Hom_C(1, \introS(Y))$; that is, $W(y) = \introN_Y(y)^* d(y)$. By the naturality of our relation $R$ over $P$, the relationship between $\introS(W)^*f$ and $d(y)$ transports back to a relationship between $\introN_Y(y)^* \introS(W)^*f = (\introS(W) \circ \introN_Y(y))^* f$ and $\introN_Y(y)^* d(y) = W(y)$.

\[\begin{tikzcd}
	&&&& {\introS(P(1))} \\
	\\
	1 && {\introS(Y)} &&&& P
	\arrow[""{name=0, anchor=center, inner sep=0}, "{d(y)}"', from=3-3, to=3-7]
	\arrow["f", from=1-5, to=3-7]
	\arrow["{\introS(W)}", from=3-3, to=1-5]
	\arrow["{\introN_Y(y)}"', from=3-1, to=3-3]
	\arrow[""{name=1, anchor=center, inner sep=0}, "{\introN_{P(1)}(W(y))}", curve={height=-30pt}, from=3-1, to=1-5]
	\arrow["{\text{related by } R}"', shorten <=7pt, Rightarrow, squiggly, from=1-5, to=0]
	\arrow[shorten <=7pt, Rightarrow, no head, from=1, to=3-3]
\end{tikzcd}\]

\[\begin{tikzcd}
	Y &&& {P(1)} \\
	& y & {W(y)} \\
	& {\introN_Y(y)} & {S(W) \circ \introN_Y(y) = \introN_{P(1)}(W(y))} \\
	{\Hom_C(1, \introS(Y))} &&& {\Hom_C(1, \introS(P(1))}
	\arrow["{\introN_Y}"', from=1-1, to=4-1]
	\arrow["{\introN_{P(1)}}", from=1-4, to=4-4]
	\arrow["W", from=1-1, to=1-4]
	\arrow["{\introS(W) \circ -}"', from=4-1, to=4-4]
	\arrow[maps to, from=2-2, to=3-2]
	\arrow[maps to, from=2-2, to=2-3]
	\arrow[maps to, from=2-3, to=3-3]
	\arrow[maps to, from=3-2, to=3-3]
\end{tikzcd}\]

Finally, note that $\introN_Y(y) : \Hom_C(1, \introS(Y))$ composed with $\introS(W) : \Hom_C(\introS(Y),$ $ \introS(P(1)))$ is $\introN_{P(1)}(W(y))$ (by the naturality of $\introN$). This completes the proof.
\end{proof}

Note that $R$ here can be any kind of relation at all. When $R$ specifically represents an equivalence relation (e.g., as when the maps defining $R$ are taken to be the domain and codomain projections from morphisms to objects for some groupoid structure whose objects are given by $P$), this result is naturally thought of as a fixed point property. We will only be looking at such $R$ for now in this document.

Similarly, although we observed that the above construction works at the level of generality of pre-introspective unary theories, every example we consider in this document will be at least a pre-introspective finite product theory.

Before specializing all the way to introspective theories, let us observe some corollaries of this general result we can already see. First, a couple traditional special cases:

\openNamed{observation}{Lawvere's Fixed Point Theorem}\label{LawveresTheorem}
Let $T$ be a category with finite products. This is readily equipped as a pre-introspective finite product theory by \cref{TrivialPreIntrosp} (that is, by letting $C$ be given by the simple self-indexing, and letting $\introS$ and $\introN$ be the canonical isomorphisms of the appropriate type), and thus we can apply \cref{GeneralDiagThm} in this context.

Let $P$ be the presheaf represented by an object $\Omega$ of $T$, and let the relation $R$ on $P$ be actual equality (that is, the relation given by the diagonal map). 

Furthermore, suppose given an object $Y$ with a map $d : Y \to P(\introS(Y))$ (i.e., a map of $T$-indexed sets from $Y$ to $\Omega^Y$) which is surjective over domain of definition $D$.

Then $P(\introS(P(1))) \iso \Omega^{\Omega}$ while $P(1) \iso \Omega$. In this context, what \cref{GeneralDiagThm} tells us is that every $D$-defined morphism from $\Omega$ to $\Omega$ admits a $D$-defined element of $\Omega$ as a fixed point. (In particular this is usually applied with $D = 1$, and indeed, any other case can be reduced to this one by replacing $T$ with $T//D$.). \qed
\closeNamed{observation}

\openNamed{observation}{Cantor's Theorem}\label{CantorsTheorem}
One very familiar special case of \nameref{LawveresTheorem} is Cantor's theorem, in the form of the claim that there is no set $X$ with a surjection onto $\Omega^X$, where $\Omega$ is a set on which there is an endofunction with no fixed points (for example, taking $\Omega$ to be the set of truth values and the endofunction to be logical negation. Or taking $\Omega$ to be the two element set and the endofunction to be the one which swaps the two elements.).
\closeNamed{observation}

Returning to \nameref{GeneralDiag} and \cref{GeneralDiagThm} in general, we also have the following corollaries:

\begin{corollary}\label{RetractDiag}
Note that if the domain of definition $D$ over which $d$ is relationally surjective is $P(\introS(P(1)))$ itself, then we can take $f$ to be the generic generalized element of $P(\introS(P(1)))$ (the one given by the identity map on $P(\introS(P(1)))$). The $D$-defined $L(f)$ we obtain is then given by a morphism from $P(\introS(P(1)))$ to $P(1)$, which we can think of as representing $L$ in general, applicable to any other $f$. From hereon out, we focus on this situation.
\end{corollary}

\begin{corollary}\label{RetractInT}
If $\langle T, C, \introS, \introN, 1 \rangle$ and $P$ are given as in \nameref{GeneralDiag} such that there is an object $Y$ in $T$ with $P(\introS(Y))$ being a retract of $Y$, then there is a map $L : P(\introS(P(1))) \to P(1)$ in $T$ satisfying the fixed point property that for every generalized element $f$ of $P(\introS(P(1)))$, we have that $L(f)$ equals the result of transporting $f$ back via the action of $P$ along $\introN_{P(1)}(L(f))$.
\end{corollary}
\begin{proof}
Apply \cref{RetractDiag} with the relation $R$ being true equality; that is, given by two parallel identity maps from $P$ to itself. The retraction from $Y$ onto $P(\introS(Y))$ will play the role of $d$ as above, and the section from $P(\introS(Y))$ into $Y$ will witness the required surjectivity of $d$.
\end{proof}

\begin{corollary}\label{RetractInC}
If $\langle T, C, \introS, \introN, 1 \rangle$ and $P$ are given as in \nameref{GeneralDiag} such that there is an object $X$ in the global aspect of $C$ with $\introS(P(X))$ being a retract of $X$, then there is a map $L : P(\introS(P(1))) \to P(1)$ in $T$ satisfying the fixed point property that for every generalized element $f$ of $P(\introS(P(1)))$, we have that $L(f)$ equals the result of transporting $f$ back via the action of $P$ along $\introN_{P(1)}(L(f))$.
\end{corollary}
\begin{proof}
If $\introS(P(X))$ is a retract of $X$ in $C$, then by applying the contravariant functor $P$, we find that $P(\introS(P(X)))$ is a retract of $P(X)$ in $T$. Thus, taking $Y = P(X)$, we can apply \cref{RetractInT}.
\end{proof}

Combining \nameref{LawveresTheorem} and these last corollaries, we get the following:

\openNamed{observation}{Y Combinator}\label{YCombinator}
It is well-known that the equational theory of the untyped lambda calculus (in full generality not presuming the $\eta$-conversion rule, only $\alpha$- and $\beta$-conversion) is modelled by a cartesian closed category $T$ with an object $Y$ such that $Y^Y$ is a retract of $Y$. More generally, let us consider any cartesian closed category $T$ with objects $Y$ and $\Omega$ such that $\Omega^Y$ is a retract of $Y$.

As in \nameref{LawveresTheorem}, let $C$ be the $T$-indexed category defined by the simple self-indexing, with $\introS$ and $\introN$ being the canonical isomorphisms of the appropriate type, and let $P$ be the presheaf $\Hom_C(-, \Omega)$, which amounts to $\Omega^{-}$.

As we have presumed $\Omega^Y$ to be a retract of $Y$, we can apply either of our above corollaries (\cref{RetractInT} and \cref{RetractInC} amount to the same as each other in this case, as $T$ matches the global aspect of $C$), and find ourselves with a fixed point combinator from $P(\introS(P(1))) \iso \Omega^{\Omega}$ to $P(1) \iso \Omega$. 

In the particular case where $Y$ and $\Omega$ are the same object so we model the untyped lambda calculus in the ordinary way, the fixed point combinator we obtain is essentially the familiar \quote{Y combinator} (no pun intended on the $Y$). \qed
\closeNamed{observation}

In a sense, all we have shown so far are some very familiar results. Diagonalization is quite old hat. Its formalization in terms of cartesian closed structure to yield \nameref{LawveresTheorem}, with \nameref{CantorsTheorem} and \nameref{YCombinator} as special cases of this, is well known. Cf. Lawvere's and Yonofsky's existing papers on diagonal arguments, fixed point theorems, and cartesian closed categories. \TODOinline{Write whatever should be written here, make whatever citations. Point out the error in Yonofsky's discussion of Kleene's recursion theorem and how our more general formulation allows us to correct this.}

But the value of \nameref{GeneralDiag} is that we have now formalized it at a particularly general level of abstraction, even more general than \nameref{LawveresTheorem}, which will allow us to move beyond cartesian closed structure as such, into suitably modalized exponentials. Instead of only producing the traditional Y combinator, what we produce in general will be a suitably modalized Y combinator, as is necessary for the context we now turn to, of introspective theories.

In the context of an introspective theory, taking $P$ to be $T$-small in general\footnote{That is, presuming $P(c)$ is $T$-small for every generalized object $c$ of $C$; equivalently, that $P$ has a $T$-small category of elements.}, we shall be able to remove the precondition of the retraction from \cref{RetractInT} and \cref{RetractInC}, to get an unconditional version of \Loeb's theorem. Or rather, we shall be able to show that a suitable retraction always exists; indeed, a fortiori, a suitable isomorphism $Y \iso P(\introS(Y))$ always exists.

\subsection{Bootstrapping}
We will now show how to bootstrap away this precondition, in the context of an introspective theory:

Essentially, our plan is to apply \cref{RetractInT} taking $P$ to be the presheaf which assigns to every object $c$ of $C$, the underlying set\footnote{When we speak of \quote{the underlying set of objects} of $C/c$, we must have in mind some particular representation for $\Ob(C)$, but that is ok. An introspective theory is such that some representation exists, and any one we pick will suffice. Put another way, we can take ourselves to be working with an inner-strict introspective theory, as every introspective theory can be equipped as such.} of objects of $C/c$ (i.e., those morphisms in $C$ with codomain $c$), with pullback as action on morphisms.

There is one caveat! This presheaf isn't quite a presheaf, as it is naturally valued not in sets but in setoids\footnote{If we assumed $T$ to be an effective regular category, we could quotient any setoid internal to $T$ into a corresponding plain object of $T$, but we do not wish to require this assumption. Instead, we will take advantage of our ability to use relation $R$ in arbitrary $T$ to achieve the same end.}. Pullback isn't well-defined, it is only well-defined up to isomorphism, and if we pick chosen pullbacks in some arbitrary fashion, there is no guarantee that pullback will be strictly functorial. In other words, this is a 2-functor rather than a functor. We must be careful about this.

But supposing we take care of that, we will at any rate naturally want to impose upon this $P$ the relation given by those same isomorphisms. That is, we take $R(c)$ to be the isomorphisms of $C/c$, with its two projections to $\Ob(C/c)$ being the domain and codomain maps.

Indeed, armed with the idea of this relationship, it becomes easy to deal with the problem from two paragraphs ago: We can always replace a 2-functorial presheaf $P'$ and isomorphism-respecting relation $R'$ upon it, by a strictly functorial presheaf $P$, using the definition that $P(x)$ consists of an object $y$, a morphism $m$ from $x$ to $y$, and an element $e$ of $P(y)$, with the transport-back action of $P$ simply being composition upon its morphism component. This is automatically strictly functorial. There a clear inclusion $i$ of $P'(x)$ into $P(x)$ via sending each element $e$ of $P'(x)$ to the element $\langle x, \id_x, e \rangle$ of $P(x)$, and a corresponding retraction $f$ from $P(x)$ onto $P'(x)$, taking an element $\langle y, m : x \to y, e \rangle$ of $P'(y)$ to the transport $m^* e$ or $P(m) e$ of $e$ back along $m$ via the action of $P$.

We then impose upon this presheaf $P$ the relation $R$ under which two elements $(p, q)$ of $P$ are related by any relationship in $R'$ between elements $(p', q')$ of $P'$, along with an isomorphism between $p'$ and $f(p)$ and an isomorphism between $q'$ and $f(q)$. (When $R'$ itself is simply the isomorphism relationship, then $R$ also amounts to relating all and only those things which are isomorphic). It is straightforward to verify that this relation $R$ has a groupoid structure itself, and that its projection maps to $P$ are indeed natural transformations between presheaves. \TODOinline{Write this out in a detailed way.}

We now need a particular $Y$ in $T$ which acts as a retract-up-to-isomorphism in the necessary way to allow us to apply our \Loeb's theorem to this $P$ and $R$. Specifically, we can take this $Y$ to be $P'(\introS(\Mor(C)))$, the object of slices in $C$ above $\Mor(C)$. [Note that we are only able to consider such a $Y$ because $\Mor(C)$ is $T$-small, the defining property of an introspective theory]. The collection of such slices above $\Mor(C)$ injects into $\Mor(C)$ [this is an injection, i.e. monic, as the slices above any particular object can be defined as an equalizer subobject of $\Mor(C)$, and equalizer inclusions are always monic]. Applying $\introS$ gives us an injection in $C$ from $\introS(Y)$ to $\introS(\Mor(C))$ [$\introS$ preserves monicity as it is a lexfunctor]. Composition and pullback along this injection will be our maps in the two directions between $P'(\introS(Y))$ and $P'(\introS(\Mor(C))) = Y$. The fact that this is a retraction-up-to-isomorphism is the fact that composing and then pulling back along any monic map results in a slice isomorphic to the one started with. We have written this retraction-up-to-isomorphism now using $P'$ rather than our strictly functorial $P$, but as $P'$ is equivalent to $P$ in the sense of isomorphic-up-to-isomorphism, we get the same result for $P$ as well. \TODOinline{Write out details.}

The fixed point property we derive then tells us:

\begin{theorem}\label{ObjectToSetFixedPoint}
For any slice $Q$ above $\introS(\Ob(C))$ in $C$, there is some object $c$ of $C$ which is isomorphic to $Q$ pulled back along $\introN_{\Ob(C)}(c) : \Hom_C(1, \introS(\Ob(C)))$.
\end{theorem}

In particular, let $Q$ be $\introS$ applied to the slice above $\Ob(C)$ in $T$ corresponding to some arbitrary $T$-small presheaf $Q'$ on $C$. Then the above invocation of our \Loeb's theorem tells us there is some object $c$ of $C$ which is isomorphic to $\introS(Q'(c))$; what we might call a \quote{fixed point} of $Q'$ in abuse of language. As an isomorphism is automatically a retract, this in turn provides the precondition needed to run our categorical \Loeb's theorem on the presheaf $Q'$ itself (in the form of \cref{RetractInC}).

Thus, in the context of an introspective theory, we can run the \Loeb's fixed point theorem on ALL presheaves, unconditionally. That is, we have proven the following:

\openNamed{corollary}{Introspective \Loeb's theorem}\label{LoebInIntrosp}
If $\langle T, C, \introS, \introN \rangle$ is an introspective theory, and $P$ is any $T$-small presheaf upon $C$, then there is a map $L : P(\introS(P(1))) \to P(1)$ in $T$ satisfying the fixed point property that for every generalized element $f$ of $P(\introS(P(1)))$, we have that $L(f)$ equals the result of transporting $f$ back via the action of $P$ along $\introN_{P(1)}(L(f))$.

Using our $\Box$ symbolism and the Yoneda lemma identification of elements of $P(c)$ with maps from $c$ to $P$, we can phrase this fixed point property of $L(f)$ as the following morphism of $\Box$-algebras within (the appropriate aspect of) $\Psh{C}$:

\[\begin{tikzcd}
	{\Box 1} && {\Box P} \\
	\\
	1 && P
	\arrow["f", from=1-3, to=3-3]
	\arrow["{!}"', from=1-1, to=3-1]
	\arrow["{\Box L(f)}", from=1-1, to=1-3]
	\arrow["{L(f)}"', from=3-1, to=3-3]
\end{tikzcd}\]

Thus, the $\Box$-algebra with carrier $1$ is weakly initial within $\Psh{C}$, in the sense that it has some morphism into every other $\Box$-algebra.
\closeNamed{corollary}

Although we have just proven that this property holds for all introspective theories automatically, it does not hold automatically for merely locally introspective theories. However, we shall see later on that there are some natural examples of locally introspective theories that happen to have this property as well, and we shall see in a second a number of corollaries of this property that work just as well for locally introspective theories as for introspective theories. Thus, let us make this useful definition:

\begin{definition}
A \defined{locally \Loeb\ theory} is a locally introspective theory $T$ such that each of its slice locally introspective theories $T/t$ has the same properties of \cref{LoebInIntrosp}, save for being locally introspective rather than fully introspective.
\end{definition}

\begin{conjecture}
Every locally \Loeb\ theory can be embedded into an introspective theory.
\end{conjecture}

\begin{observation}
Although we above observed that in any introspective theory there is an object $Y$ such that the object of slices in $C$ above $\introS(Y)$ was merely a retract-up-to-isomorphism of $Y$, we can now reapply \cref{LoebInIntrosp} to construct an object $Y'$ such that the object of slices in $C$ above $\introS(Y')$ is in fact isomorphic-up-to-isomorphism to $Y'$. \TODOinline{Word this better and verify the details. Actually, I think this isn't true when we don't have regularity. I should remove this observation, after verifying that.}
\end{observation}

\subsection{Uniqueness and initiality/terminality for fixed points}
Throughout this section, we work in the context of an introspective theory $\langle T, C \rangle$. \TODOinline{Figure out how much of this works for a locally \Loeb\ theory, possibly tweaking the definition of a locally \Loeb\ theory as needed}

\begin{theorem}\label{UniqueFixedPoints}
The fixed points produced by \nameref{LoebInIntrosp} are unique, in the sense that for any $t$-defined point $f$ of $P(\introS(P(1)))$ and any two $t$-defined points $a$ and $b$ of $P(1)$ satisfying the appropriate fixed point property (that is, each is equal to $f$ transported back along $\introN$ of itself), we have that $a = b$.
\end{theorem}
\begin{proof}
By passing to the appropriate slice introspective theory $T/t$, we can assume without loss of generality that all relevant values in the following are globally defined; i.e., without loss of generality, we can presume $t = 1$.

Consider the equalizer $E$ of $a$ and $b$. This is a subobject of $1$. If there were a map from $1$ to $E$, then $a$ and $b$ would be equal. In just the same way, taking this equalizer diagram's image under the lexfunctor $\introS$, we see that any value in $\Hom_C(1, \introS(E))$ would lead to $\introS(a)$ and $\introS(b)$ being equal, thus corresponding to equal maps from $1$ to $\introS(P(1))$ in $C$. Transporting $f$ back along these two would yield equal values, therefore. But transporting $f$ back along these two yields $a$ and $b$ respectively, so we would have that $a$ and $b$ are equal. This argument yields a morphism from $\Hom_C(1, \introS(E))$ to $E$ in $T$. Applying $\introS$ to this, we have a morphism in $C$, which lives in $Q(\introS(Q(1)))$, where $Q$ is the presheaf on $C$ represented by $\introS(E)$. Applying \parensref{LoebInIntrosp} to this, we get an element of $Q(1)$, which is to say, a global element of $\introS(E)$ in $C$. This makes $\introS(a)$ and $\introS(b)$ equal as global points of $\introS(P(1))$ in $C$. And by transporting $f$ back along these, we conclude as before that $a$ and $b$ are equal. \TODOinline{Word this all better}
\end{proof}

\begin{observation}
Note that this argument made essential use of the structure (notably, equalizers) available in an introspective theory, and thus does not apply in such full generality as \parensref{GeneralDiag} does. For example, we saw in \parensref{YCombinator} that we get fixed points for arbitrary functions in the untyped lambda calculus, but as a cartesian closed category in general is not an introspective theory, we cannot conclude that fixed points of functions in the untyped lambda calculus are unique (and indeed, they will not be, as one such function is the identity function, whose fixed points are all values!).
\end{observation}

\begin{observation}
Note that this means that the apparent dependence on which particular object $Y$ and retraction from $P(\introS(Y))$ into $Y$ is used in the diagonalization theorem doesn't actually matter, in this context. We get the same result no matter what.
\end{observation}

A similar argument to \cref{UniqueFixedPoints} tells us that fixed points for endofunctors are unique, and even some initial algebra/terminal coalgebra properties for them:

\begin{theorem}\label{CoalgToAlgExist}
Suppose $Q$ is a $T$-small presheaf on the opposite category of $C$. (That is, a $T$-indexed covariant functor from $C$ to the self-indexing of $T$ whose category of elements is $T$-small). This induces a covariant endofunctor $Q(\introS(-))$ on $T$. Suppose $a : A \to Q(\introS(A))$ is a coalgebra for this functor within $T$ and $b : Q(\introS(B)) \to B$ is an algebra for this functor within $T$. Then there is some coalgebra-to-algebra morphism $g$ from $\introS(a)$ to $\introS(b)$ within $C$. That is, there is some $g : \introS(a) \to \introS(b)$ within $C$ such that $g = \introS(a) \circ \introS(Q(g)) \circ \introS(b)$.
\end{theorem}
\begin{proof}
Note that the action of $Q$ gives us a morphism from $\Hom_C(\introS(A), \introS(B)) \times Q(\introS(A))$ to $Q(\introS(B))$ within $T$. By pre- and post-composing this with $a$ and $b$, respectively, we get a morphism from $\Hom_C(\introS(A), \introS(B)) \times A$ to $B$ within $T$. By applying $\introS$ to this, we get a globally defined morphism within $C$ from $\introS(\Hom_C(\introS(A), \introS(B))) \times \introS(A)$ to $\introS(B)$. This morphism is of type $P(S(P(1)))$, where $P$ is the $T$-small presheaf upon $C$ defined by $P(-) = \Hom_C(- \times \introS(A), \introS(B))$.

Thus, we can apply our fixed point theorem \parensref{LoebInIntrosp} to this value. The result is a value $g$ in $P(1)$ (i.e., in $\Hom_C(\introS(A), \introS(B))$) which is equal to $\introS(a \circ Q(g) \circ b)$, as desired.

\TODOinline{Write diagrams to make this all clearer}
\end{proof}

\begin{theorem}\label{CoalgToAlgUnique}
The theorem \cref{CoalgToAlgExist} can be strengthened to not only produce a coalgebra-to-algebra morphism, but also to conclude that such a coalgebra-to-algebra morphism is unique.
\end{theorem}
\begin{proof}
After applying \cref{CoalgToAlgExist} with a suitable presheaf $Q$ to get existence, apply \cref{UniqueFixedPoints} with that same presheaf to get uniqueness.
\end{proof}

\begin{theorem}
Given $Q$ as in \cref{CoalgToAlgExist}, there exists a unique (up to isomorphism) fixed point of $Q(\introS(-))$, in the sense of a unique (up to isomorphism) object $A$ and isomorphism $a : A \to Q(\introS(A))$. This fixed point is both a terminal coalgebra and (its inverse is) an initial algebra.
\end{theorem}
\begin{proof}
We get the existence of an object $A$ and isomorphism $a : A \to Q(\introS(A))$ directly from \cref{ObjectToSetFixedPoint}.

Furthermore, by \cref{CoalgToAlgUnique}, any coalgebra has a unique map into any algebra, from which it follows that any inverse morphism of a coalgebra (or algebra) is in fact an initial algebra (or terminal coalgebra). Since initial (or terminal) objects are all isomorphic to each other, we immediately have the uniqueness up to isomorphism we seek.
\end{proof}

Essentially identical arguments work when dealing with presheaves on the opposite category of the core of $C$ (that is, its subcategory including only its isomorphisms):

\begin{theorem}\label{CoreCoalgToAlgExist}
Suppose $Q$ is a $T$-small presheaf on the opposite category of the core of $C$. This induces a covariant endofunctor $Q(\introS(-))$ on the core of $T$. Suppose $a : A \to Q(\introS(A))$ is a coalgebra for this functor within the core of $T$ and $b : Q(\introS(B)) \to B$ is an algebra for this functor within the core of $T$. Then there is some coalgebra-to-algebra isomorphism $g$ from $\introS(a)$ to $\introS(b)$ within $C$. That is, there is some $g : \introS(a) \to \introS(b)$ within $C$ such that $g = \introS(a) \circ \introS(Q(g)) \circ \introS(b)$.
\end{theorem}
\begin{proof}
Note that the action of $Q$ gives us a morphism from $\Iso_{C}(\introS(A), \introS(B)) \times Q(\introS(A))$ to $Q(\introS(B))$ within $T$. Actually, this satisfies a stronger property, of corresponding to a morphism from $\Iso_{C}(\introS(A), \introS(B))$ to $\Iso_T(Q(\introS(A)), Q(\introS(B)))$.

By pre- and post-composing this with $a$ and $b$, respectively, we get a morphism from $\Iso_C(\introS(A), \introS(B)) \times A$ to $B$ within $T$, that corresponds to a morphism from $\Iso_C(\introS(A), \introS(B))$ to $\Iso_T(A, B)$.

By applying $\introS$ to this, we get a globally defined morphism within $C$ from $\introS(\Iso_C($ $\introS(A), \introS(B))) \times \introS(A)$ to $\introS(B)$, that corresponds to a morphism from $\introS(\Iso_C(\introS(A), \introS(B)))$ to $\Iso_C(\introS(A), \introS(B))$.

This morphism is of type $P(S(P(1)))$, where $P$ is the $T$-small presheaf upon the core of $C$ defined by $P(c)$ amounting to $c$-defined isomorphisms from $\introS(A)$ to $\introS(B)$.

Thus, we can apply our fixed point theorem \parensref{LoebInIntrosp} to this value. The result is a value $g$ in $P(1)$ (i.e., in $\Iso_C(\introS(A), \introS(B))$) which is equal to $\introS(a \circ Q(g) \circ b)$, as desired.

\TODOinline{Write diagrams to make this all clearer}
\end{proof}

\begin{theorem}
The theorem \cref{CoreCoalgToAlgExist} can be strengthened to not only produce a coalgebra-to-algebra isomorphism, but also to conclude that such a coalgebra-to-algebra isomorphism is unique.
\end{theorem}
\begin{proof}
After applying \cref{CoreCoalgToAlgExist} with a suitable presheaf $Q$ to get existence, apply \cref{UniqueFixedPoints} with that same presheaf to get uniqueness.
\end{proof}

\TODOinline{Thus, not only is there an essentially unique fixed point for isovariant functors from C to T, but furthermore, this unique fixed point is rigid}

It is worth noting that the proofs of \cref{UniqueFixedPoints}, of \cref{CoalgToAlgExist}, and of \cref{CoreCoalgToAlgExist} can all be unified, seen as different instances of one general theorem. \TODOinline{Do this, and then having done so, the above cleans up a lot.}

\begin{TODOblock}
Discuss how we may conclude that our functorial fixed points are both initial algebras and terminal coalgebras. Note that Loeb's theorem itself is the initial algebra property for []1 |- 1.
\end{TODOblock}

\begin{TODOblock}
Discuss how we get \Loeb's theorem not just for globally defined objects of $C$, but for arbitrary objects of $C$, by working within the locally introspective theory $T/D$ for suitable base of definition $D$, or within $T/\Ob(C)$ generically in an introspective theory. And this transfers into $C$ itself (such that $C$ holds \Loeb's theorem to be true for arbitrary objects of $\introS(C) = C_1$) via $\introS$. Write up details for this, including making sure we have the right slice-introspective theory construction.
\end{TODOblock}

\begin{TODOblock}
Demonstrate that we do NOT get Loeb's theorem internal to a geminal category G for arbitrary presheaves P on |G'|, thus demonstrating the necessity of the presheaf existing within an introspective theory. The presheaf needs to be parametrized by a parameter from an object of an enclosing introspective theory. So P(S(X)) |- []P(S(X)) is available.
\end{TODOblock}

\end{document}