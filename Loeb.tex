\section{\Loeb's theorem}

\subsection{The basics of double indexing}
\begin{TODOblock}
The following is all preliminary scribblings, to be written up properly and moved to the Preliminaries later.
\end{TODOblock}

Suppose $T$ is some category and $C$ is some category indexed over $T$. What does it mean to speak of a $T$-indexed $C$-indexed structure? For example, a $T$-indexed $C$-indexed set. That is, a $T$-indexed $C$-indexed presheaf $P$.

What this means is that, for every object $t$ in $T$ and every $t$-defined object $c$ in $C$, we have some corresponding set of $P$ (the $t$-defined $c$-defined elements of $P$), and we also have a coherent system of pullback maps between these: Along any map $f : s \to t$ in $T$, we can pull back a $t$-defined $c$-defined element of $P$ to an $s$-defined $c$-defined element of $P$ (where the latter $c$ is implicitly the pullback along $f$ of the former $c$). And for any fixed $t$, given a $t$-defined map $g : d \to c$ in $C$, we can pull back a $t$-defined $c$-defined element of $P$ along this to a $t$-defined $d$-defined element of $P$. Both of these systems of pullbacks are functorial, and they also interact with each other \quote{commutatively} in the sense that pulling back along a map $g$ in $C$ and then pulling back along a map $f$ in $T$ is the same as pulling back along the map $f$ in $T$ and then pulling back along the map $g$ in $C$ (where, again, the latter, $g$ is implicitly the pullback along $f$ of the former $g$).

This all amounts to saying that $P$ is in fact indexed over the Grothendieck construction corresponding to $C$.

What about maps such doubly-indexed sets? Well, this is the same as the notion we get again by thinking of these as in fact singly-indexed over the Grothendieck construction. A natural transformation in that context.

Now that we understand doubly-indexed sets and maps between them, we can also understand doubly-indexed structures in general. Including doubly-indexed strict categories, and everything about doubly-indexed categorical structures works in basically the same fashion.

That all being said, we won't really need any of that just yet. All we need is the concept of doubly-indexed sets. Even maps between these aren't really going to concern us, just yet.

\subsection{The basics of \Loeb}
Let $\langle T, C, F \rangle = \langle T, C, S, N \rangle$ be a pre-introspective theory. Consider now a $T$-indexed presheaf $P$ on $C$ which is $T$-small, in the sense that for any $t$-defined object $c$ of $C$, the value of $P$ at object $c$ is given by some $t$-defined object of $T/-$. In other words, the corresponding discrete fibration to $P$ has $T$-small fibers.

Presume also given some globally defined\footnote{The globally defined presumption is not a major condition, I think; if it were merely $t$-defined, then we would simply pass to the slice introspective theory over $t$. But for convenience now, we take the definition as global. \TODOinline{Make sure the passage to a slice category here could work}} object $X$ in $C$ and an isomorphism $B$ in $C$ from $X$ to $S(P(X))$. Note that $S(P(X))$ is well-defined as $P(X)$ is an object in $T$, by our $T$-smallness presumption on $P$.

[Note that an $X$ in $C$ isomorphic to $S(P(X))$ is as good as a $Y$ in $T$ isomorphic to $P(S(Y))$, by $Y = P(X)$ and $X = S(Y)$; if speaking in terms of $Y$, let us call the isomorphism $B' = P(B)$ and $B = S(B')$]

Suppose now given any generalized element $y$ of $Y$. Then, we can take the value $B'(y)$ within $P(S(Y))$ and pull it back via the presheaf action of $P$ along the value $N_Y(y)$ within $\Hom_C(1, S(Y))$, to get a value inside $P(1)$. Thus, discharging our assumption of $y$, we have produced a morphism in $T$ from $Y$ to $P(1)$. Call this morphism $W$.

Suppose now given any generalized element $f$ of $P(S(P(1)))$.

By pulling $f$ back along $S(W)$, we get an element of $P(S(Y))$. By applying $B'^{-1}$ to this, we have an element of $Y$. And applying $W$ to this, we end up with an element of $P(1)$. Thus, discharging our assumption of $f$, we have produced a morphism in $T$ from $P(S(P(1)))$ to $P(1)$.

This final morphism represents \Loeb's theorem. \TODOinline{Demonstrate that it satisfies an important fixed point property!}

However, this was all dependent on our assumption of the fixed point $X$ (equivalently, $Y$) in the first place. We will now show how to bootstrap away this assumption, in the context of an introspective theory. \TODOinline{This will require first paying more attention that fixed point property!}

\begin{TODOblock}
Discuss uniqueness of the fixed point in general, and the simultaneous initial algebra and terminal coalgebra properties for functorial fixed points (even stating what these functorial fixed points amount to requires looking specifically at $P(c) = \Ob(C/c)$, which we will be using for our bootstrapping, but which may also drive us into talking about $P$ here as category-valued (at least, groupoid-valued) rather than set-valued).
\end{TODOblock}