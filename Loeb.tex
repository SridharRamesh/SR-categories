\filestart

\section{\Loeb/'s theorem}

\subsection{Lawvere's fixed point theorem}
Let us refresh the reader on Lawvere's fixed point theorem \autocite{lawvere1969diagonal}, which captures the general structure of many diagonalization arguments and their relationship to cartesian closed structure. We shall first review a proof of Lawvere's fixed point theorem close in spirit to Lawvere's framing of his result.

Then we will note a slight generalization for which essentially the same argument applies. Then in the next section we will turn this generalization into a result in the context of general pre-introspective theories. Then we will specialize further down to introspective theories, and observe a wonderful \quote{bootstrapping} phenomenon which arises there, which shall ultimately provide us with a form of \Loeb/'s theorem in that context, which is our main result.

\openNamed{theorem}{Lawvere's Fixed Point Theorem}\label{LawveresFixedPointTheorem}
Let $T$ be an arbitrary category. Let $X$ be an object of $T$ and let $\Omega$ be any $T$-indexed set. Suppose also given some map $\App' : X \to \Omega^X$ (equivalent to the data of a map $\App : X \times X \to \Omega$).

Let $\point$ be any object of $T$. By a \quote{point} of a $T$-indexed set, we shall mean an element of its aspect at $\point$ (equivalent to the data of a map into it from $\point$).\footnote{Lawvere takes $\point$ to always be a terminal object, and for our goals that will ultimately suffice as well. But we shall see in the argument and the post-argument discussion that there is no need for such a restriction, so we do not impose it here.}

Suppose $\App$ has the surjectivity-like property that, for every map $F : X \to \Omega$, there is a point $f$ of $X$, such that for every point $x$ of $X$, we have that $\App(f, x) = F(x)$.

Then for any map $g : \Omega \to \Omega$, there exists a point $\omega$ of $\Omega$ such that $\omega = g(\omega)$. That is to say, $g$ has a fixed point.
\closeNamed{theorem}
\begin{proof}
Let $F : X \to \Omega$ be the following composition:

% https://q.uiver.app/?q=WzAsNCxbMCwwLCJYIl0sWzIsMCwiWCBcXHRpbWVzIFgiXSxbMywwLCJcXE9tZWdhIl0sWzQsMCwiXFxPbWVnYSJdLFswLDEsIlxcbGFuZ2xlIFxcaWRfWCwgXFxpZF9YIFxccmFuZ2xlIl0sWzEsMiwiXFxBcHAiXSxbMiwzLCJnIl1d
\[\begin{tikzcd}
	X && {X \times X} & \Omega & \Omega
	\arrow["{\langle \id_X, \id_X \rangle}", from=1-1, to=1-3]
	\arrow["\App", from=1-3, to=1-4]
	\arrow["g", from=1-4, to=1-5]
\end{tikzcd}\]

That is, for any generalized element $x$ of $X$, we have that $F(x) = g(\App(x, x))$.

We know there exists a point $f$ of $X$ which corresponds with $F$ in the manner of our surjectivity-like supposition on $\App$. Now consider the instance of this surjectivity-like supposition where $x = f$. This tells us that $\App(f, f) = F(f)$. But $F(f) = g(\App(f, f))$.

Thus, taking $\omega = \App(f, f)$, we have that $\omega = g(\omega)$ as desired.
\end{proof}

Let us make a few remarks on the scope of generality of this theorem.

Lawvere originally states this theorem specifically for the case where $T$ is a cartesian closed category, but later in \autocite{lawvere1969diagonal} notes that this implies the theorem just as well for the case where $T$ is merely a category with finite products, as any category can be embedded as a full subcategory of a cartesian closed category in a way which preserves any products or exponentials already present (via the Yoneda embedding). \autocite{lawvere1969diagonal} does not explicitly consider examples where the original category of interest $T$ lacks finite products, such that $X \times X$ is not an object of $T$, nor consider taking $\Omega$ to be merely a $T$-indexed set rather than an object of $T$, but of course these are covered in the same way by the same insight that we can work in $\Psh{T}$ instead of $T$.

Having observed that we can just as well frame the theorem with any of its objects drawn from $\Psh{T}$ rather than $T$, the reader might then well wonder why in our framing we have allowed some objects to be in $\Psh{T}$ but still constrained others (such as $X$) to come from $T$. We chose this particular framing partly as this is closest to the applications we have in mind, and also partly for what amount to stylistic reasons. In particular, having stated the theorem in this form, interpreting the surjectivity condition on $\App$ only requires quantification over the set of morphisms from object $X$ to presheaf $\Omega$ (i.e., the set $\Omega(X)$), instead of requiring quantification over the (potentially large, if $T$ is large) class of natural transformations from a presheaf $X$ to another presheaf $\Omega$. But this is not really of much importance, and again the more general form of the theorem follows readily from the ostensibly less general one.

\autocite{lawvere1969diagonal} also only states this theorem in the particular case where $\point$ is a terminal object. In general, we can always pass from $T$ to a slice category $T/\point$, and in so doing we will turn what was $\point$-defined data in $T$ into globally defined data in $T/\point$. So constraining $\point$ to be a terminal object does not constrain the theorem excessively. However, it does constrain the theorem slightly, in that interpreting the surjectivity precondition in $T/\point$ in this way results in a stronger (that is, less often satisfied) surjectivity precondition than in the more flexible framing of the theorem we have given: The surjectivity condition in $T/\point$ would amount to requiring that for every $F : \point \times X \to \Omega$ in $\Psh{T}$, we could find a corresponding $f$. However, we have only required surjectivity with respect to the more constrained set of $F : X \to \Omega$ in $\Psh{T}$.

We do not actually need this extra flexibility. For our purposes, just like Lawvere's, it would suffice to always take $\point$ to be a terminal object. But we note the availability of this flexibility all the same (if only for the purpose of comparison at the end of this chapter to other variants on Lawvere's fixed point theorem recently noted in the literature, such as \magicref{MagmoidalFixedPointTheorem}).

Even this loosened surjectivity presumption is still far overkill as far as the needs of the argument go. All that really matters is for one specific definable value to be in the range of $\App'$. But in general practice and for our specific purposes, this is always established because of some such surjectivity condition anyway, so that seems the most useful framing in which to give the theorem.

Having said all that about the wide applicability of \magicref{LawveresFixedPointTheorem}, we actually will need to generalize it slightly further for our purposes. Having given the above discussion of the traditional theorem to prime the reader's intuitions through familiarity, we now put forward the following simple generalization:

\openNamed{theorem}{Self-Related Point Theorem}\label{SelfRelatedPointTheorem}
Let $T$ be an arbitrary category. Let $\point$ and $X$ be objects of $T$ and let $\Omega$ be any $T$-indexed set. Suppose also given some map $\App' : X \to \Omega^X$ (equivalent to the data of a map $\App : X \times X \to \Omega$).

As before, we shall use \quote{point of} as shorthand for \quote{element of the $\point$-aspect of}.

Suppose also given a binary relation $R$ on the points of $\Omega$. (We needn't presume $R$ to be symmetric or transitive or any such thing.). And suppose $\App$ has the surjectivity-like property that, for every morphism $F : X \to \Omega$, there is a point $f$ of $X$, such that for every point $x$ of $X$, we have $R(\App(f, x), F(x))$.

Then there exists a point $\omega$ of $\Omega$ such that $R(\omega, \omega)$. That is to say, $R$ has a self-related point.
\closeNamed{theorem}
\begin{proof}
Let $F : X \to \Omega$ be the following composition:

% https://q.uiver.app/?q=WzAsNCxbMCwwLCJYIl0sWzIsMCwiWCBcXHRpbWVzIFgiXSxbMywwLCJcXE9tZWdhIl0sWzQsMF0sWzAsMSwiXFxsYW5nbGUgXFxpZF9YLCBcXGlkX1ggXFxyYW5nbGUiXSxbMSwyLCJcXEFwcCJdXQ==
\[\begin{tikzcd}
	X && {X \times X} & \Omega & {}
	\arrow["{\langle \id_X, \id_X \rangle}", from=1-1, to=1-3]
	\arrow["\App", from=1-3, to=1-4]
\end{tikzcd}\]

That is, for any generalized element $x$ of $X$, we have that $F(x) = \App(x, x)$.

We know there exists a point $f$ of $X$ in accordance with our surjectivity-like supposition on $\App'$. Now consider the instance of the surjectivity-like supposition where $x = f$. This tells us that $R(\App(f, f), F(f))$. But $F(f) = \App(f, f)$.

Thus, we have found a point of $\Omega$ which is related to itself by $R$, as desired.
\end{proof}

It may not be obvious that this generalizes \magicref{LawveresFixedPointTheorem}. The following shows how this is so:

\openNamed{corollary}{Relatedly-Fixed Point Theorem}\label{RelatedlyFixedPointTheorem}
Consider the same setup as of \magicref{SelfRelatedPointTheorem}, and furthermore, suppose given $g : \Omega \to \Omega$.

Then there exists a point $\omega$ of $\Omega$ such that $R(\omega, g(\omega))$. We might describe this as \quote{$\omega$ is an $R$-fixed point of $g$}.
\closeNamed{corollary}
\begin{proof}
Consider the binary relation $R_g$ on points of $\Omega$ given by $R_g(\omega_1, \omega_2) = R(\omega_1, g(\omega_2))$.

We have been given the supposition that, for every morphism $F : X \to \Omega$, there is a point $f$ of $X$, such that for every point $x$ of $X$, we have $R(\App(f, x), F(x))$.

As this holds for arbitrary $F : X \to \Omega$, this also holds when an arbitrary $F$ is replaced by $g \circ F : X \to \Omega$. That is to say, for every $F : X \to \Omega$, there is a point $f$ of $X$, such that for every point $x$ of $X$, we have $R(\App(f, x), (g \circ F)(x))$, which is to say, $R_g(\App(f, x), F(x))$.

But this is precisely the surjectivity supposition we need in order to invoke \magicref{SelfRelatedPointTheorem} with $R_g$ in place of $R$. Doing so, we obtain a point $\omega$ of $\Omega$ such that $R_g(\omega, \omega)$, which is to say $R(\omega, g(\omega))$, as desired.
\end{proof}

Now we can see that \magicref{LawveresFixedPointTheorem} is of course the instance of \magicref{RelatedlyFixedPointTheorem} where the relation $R$ is taken to be equality. But \magicref{RelatedlyFixedPointTheorem} is strictly more general in allowing the use of an arbitrary relation.

(As for the relation between \magicref{RelatedlyFixedPointTheorem} and \magicref{SelfRelatedPointTheorem}, each is an instance of the other. We above obtained \magicref{RelatedlyFixedPointTheorem} as a corollary of \magicref{SelfRelatedPointTheorem}. But also conversely, \magicref{SelfRelatedPointTheorem} is the special case of \magicref{RelatedlyFixedPointTheorem} where $g$ is taken to be $\id_{\Omega}$.)

At any rate, we shall find the added flexibility of allowing a relation in place of equality to be valuable in the next sections, as we begin to specialize towards our application in introspective theories.

\subsection{Presheaf diagonalization for pre-introspective theories}
\openNamed{theorem}{Pre-introspective Diagonalization}\label{PreIntrospDiag}
Let $T$ be a category, let $C$ be a $T$-indexed category, let $\introS$ be a functor from $T$ to the global aspect of $C$, let $\point_C$ be any object in the global aspect of $C$, and let $\introN$ be a natural transformation from $t$ in $T$ to $\Hom_C(\point_C, \introS(t))$.

(In particular, any pre-introspective finite product theory carries all this structure, with $\point_C$ as the terminal object of $C$, which covers all the applications we will be interested in, but we observe that the following construction does not rely on such extra structure.)

Furthermore, let $\point_T$ be an object of $T$. We use \quote{point of} as shorthand for \quote{element of the $\point_T$-aspect of}.

(Again, our ultimate interest will only be in the case where $\point$ is a terminal object, but the following construction does not rely on this. We also do not need in the following construction any presumption that $\introS(\point_T)$ is isomorphic to $\point_C$.)

Furthermore, let $P$ be a $(T, C)$-indexed set. We will write in the following $P(c)$ to mean the $T$-indexed set $t \mapsto P(t, c)$, for globally defined objects $c$ of $C$.

Suppose also given some object $\Omega$ in $T$ with a map $\quotient : \Omega \to P(\point_C)$ such that the induced function $\quotient \circ - : \Hom(X \times X, \Omega) \to \Hom(X \times X, P(\point_C))$ is surjective.

Suppose also given some object $X$ in $T$ and map $\alpha : X \to P(\introS(X))$. We also make a surjectivity-like assumption on $\alpha$. Specifically, we suppose that for every global element $p$ of $P(\introS(X))$, there is a point $x$ of $X$ such that $\alpha(x) = p$.

(More precisely, by $p$ on the right side of this equation, we mean the reinterpretation of $p$ from a global element into a point. In other words, the induced function $\alpha \circ - : \Hom(\point_T, X) \to \Hom(\point_T, P(\introS(X)))$ has a range which includes the entire range of the function $- \; \circ \; ! : \Hom(1, P(\introS(X))) \to \Hom(\point_T, P(\introS(X)))$ induced by the unique map $! : \point_T \to 1$.)

Finally, let $g$ be a globally defined element of $P(\introS(\Omega))$.

Then we obtain a point $\omega$ of $\Omega$, such that $\quotient(\omega) = \pullAlong{\introN_{\Omega}(\omega)} g$.
\closeNamed{theorem}
\begin{proof}
We shall show how this is an instance of \magicref{SelfRelatedPointTheorem}.

We define $\App : X \times X \to \Omega$ like so: Consider the two projection maps $\pi_1, \pi_2 : X \times X \to X$, as the two generic $(X \times X)$-defined elements of $X$. We thus obtain also $(X \times X)$-defined elements $\alpha(\pi_1)$ of $P(\introS(X))$ and $\introN_{X}(\pi_2)$ of $\Hom_C(\point_C, \introS(X))$. Combining these via the presheaf action of $P$, we get $\pullAlong{( \introN_{X}(\pi_2) )} ( \alpha(\pi_1) )$ as an $(X \times X)$-defined element of $P(\point_C)$. By the surjectivity presumption on $\quotient$, we find a preimage of this under the action of $\quotient : \Omega \to P(\point_C)$. We take this preimage to be our $\App : X \times X \to \Omega$. Thus, for any generalized elements $x_1, x_2$ of $X$ with the same domain, we have that $\quotient(\App(x_1, x_2)) = \pullAlong{( \introN_{X}(x_2) )} ( \alpha(x_1) )$.

We must now establish an appropriate surjectivity supposition on $\App$ for invoking \magicref{SelfRelatedPointTheorem}. 

Let an arbitrary $F : X \to \Omega$ be given. We then have that $\introS(F) : \introS(X) \to \introS(\Omega)$ in the global aspect of $C$. We can apply the action of $P$ along this morphism to $g$ (a global element of $P(\introS(\Omega))$), thus obtaining a global element $\pullAlong{\introS(F)} g$ of $P(\introS(X))$. By the surjectivity-like assumption on $\alpha$ we made, we now have a corresponding point $f$ of $X$, such that $\alpha(f) = \pullAlong{\introS(F)} g$ (the right side here having been reinterpreted from a global element into a point).

It follows that for every point $x$ of $X$, we have that $\pullAlong{\introN_{X}(x)} \alpha(f) = \pullAlong{\introN_{X}(x)} \pullAlong{\introS(F)} g$.

Note that by the definition of $\App$, we have that $\quotient(\App(f, x)) = \pullAlong{( \introN_{X}(x) )} \alpha(f)$.

Also note that by \magicref{SWithN}, we have that $\introS(F) \circ_C \introN_{X}(x) = \introN_{\Omega}(F(x))$. Thus, by the functoriality of $P$, we have that $\pullAlong{\introN_{X}(x)} \pullAlong{\introS(F)} g =  \pullAlong{\introN_{\Omega}(F(x))} g$.

Combining these last three paragraphs, we have that $\quotient(\App(f, x)) =  \pullAlong{\introN_{\Omega}(F(x))} g$.

If we define the relation $R(\omega_1, \omega_2)$ as the equation $\quotient(\omega_1) = \pullAlong{\introN_{\Omega}(\omega_2)} g$ accordingly, we have now established the surjectivity supposition required in order to invoke \magicref{SelfRelatedPointTheorem}. From this invocation, we get a point of $\Omega$ which is related by $R$ to itself, which is just what we desired, completing the proof.
\end{proof}
\begin{corollary}\label{PreIntrospDiagSpecialization}
Although the above argument allowed for some greater generality in its stated presumptions, every instance we will be interested in is where $\langle T, C, \introS, \introN \rangle$ is given by a pre-introspective finite product theory, with $\point_C$ being the terminal object of $C$, and with $\point_T$ being the terminal object of $T$.

In many cases we are interested in (though not all!), we furthermore take $P(\point_C)$ to be $T$-\repsmall/ and take $\Omega$ to be $P(\point_C)$, with $\quotient : \Omega \to P(\point_C)$ as the identity map or isomorphism between these. We then automatically have that the aspect of $\quotient$ at ${X \times X}$ is surjective as required.
\end{corollary}

\begin{theorem}\label{PreIntrospDiagFromIso}
Suppose given a pre-introspective finite product theory $\langle T, C, \introS, \introN \rangle$ and a locally $T$-\repsmall/ presheaf $P$ \TODOinline{Standardize this terminology across this section, the modal logic section, and the preliminaries} on $C$.

If there is any object $X$ of $T$ with an isomorphism from $X$ to $P(\introS(X))$, then for every globally defined $g : \Box P \to P$, we obtain an $\omega : 1 \to P$, such that $g \circ \omega' = \omega$, where the globally defined $\omega' : 1 \to \Box P$ is given by $\introN_{P(1)}(\omega)$. \TODOinline{Clarify this notation; get rid of the ambiguous notation $P(1)$. We perhaps shouldn't use this notation here when we haven't presumed local introspection.}

We get the same result also if there is any object $Y$ in the global aspect of $C$ along with a globally defined isomorphism from $Y$ to $\introS(P(Y))$.
\end{theorem}
\begin{proof}
As a reminder, we are using here the $\Box$ notation of $\TODOinline{Cite chapter}$, where $\Box P$ for $P \in \Psh{C}$ means $\introS(P(\point_C))$, with $\point_C$ as the terminal object of $C$. We are also making use here of the Yoneda equivalence between $P(a)$ and $\Hom(a, P)$ for $a \in C$; thus, the data of $g$ is equivalently a global element of $P(\introS(P(\point_C)))$, and the data of $\omega$ is equivalently a global element of $P(\point_C)$.

Any isomorphism $\alpha : X \to P(\introS(X))$ (or even just a retraction) will automatically satisfy the surjectivitity-like precondition allowing us to invoke \magicref{PreIntrospDiag} via \magicref{PreIntrospDiagSpecialization}, which takes $\Omega$ as $P(\point_C)$ and $\quotient$ as identity. The result of that invocation is the result of this theorem.

For the final remark, let $F$ and $G$ be arbitrary covariant or contravariant functors, not necessarily of the same variance as each other. Note that fixed points up to isomorphism of $F \circ G$ are in correspondence with fixed points up to isomorphism of $G \circ F$, with the functors $G$ and $F$ carrying out the two directions of the correspondence. (For that matter, we can also observe that values $X$ which retract onto $F(G(X))$ are in correspondence with values $Y$ which retract onto $G(F(Y))$). In this particular case, this means that fixed points up to isomorphism of the contravariant endofunctor $P(\introS(-))$ on $T$ are in correspondence with fixed points up to isomorphism of the contravariant endofunctor $\introS(P(-))$ on the global aspect of $C$.
\end{proof}

\subsection{Bootstrapping to \Loeb/'s theorem for introspective theories}
We can see in the types of this last theorem the outline of \Loeb/'s theorem. But this last theorem contains the precondition of a certain isomorphism.

Incredibly, we can bootstrap away this isomorphism precondition, in the context of an introspective theory. That is, in the context of an introspective theory, we can use one particular instance of \magicref{PreIntrospDiag} itself to provide the very isomorphisms necessary in order to then re-invoke \magicref{PreIntrospDiag} via \magicref{PreIntrospDiagFromIso}.

Essentially, our plan is to consider the presheaf $P$ which assigns to every object $c$ of $C$ the set of isomorphism classes of $C/c$, with the action of $P$ on morphisms of $C$ correspondingly being given by pullback. That is, $P(t, c)$ is the set of $t$-defined objects of $C/c$ modulo $t$-defined isomorphisms of $C/c$. Note that this set of objects modulo isomorphism is well-defined even though $C$ is only a category and not a strict category!

The reindexing action of this $P$ along morphisms in $T$ (reinterpreting $t_2$-defined data as $t_1$-defined data along any morphism $: t_1 \to t_2$ in $T$) is straightforward. As for reindexing along morphisms in $C$, for any fixed $t$ and any $t$-defined morphism $m : c_1 \to c_2$ of $C$, the action $P(t, m) : P(t, c_2) \to P(t, c_1)$ is given by pullback in the lexcategory $C$ along $m$; that is, this is given by $\pullAlong{m} : C/c_2 \to C/c_1$ considered as acting on isomorphism classes of objects. \TODOinline{Maybe move all this to the Preliminaries in the section on doubly-indexed sets}. Note that reindexing along morphisms in $C$ is indeed strictly functorial, because we are working with objects modulo isomorphism rather than with objects simpliciter.

We now choose any internal category $C_{strict}$ in $T$ which presents $C$ (by definition, such an internal category exists in an introspective theory; there may be multiple non-isomorphic such internal categories presenting $C$, but any will do for our purposes) and we take $\Omega$ to be $\Ob(C_{strict})$, with $\quotient : \Omega \to P(1)$ sending each object of $C$ to its isomorphism class. \TODOinline{This should be a well-defined map of indexed sets, and all of its components should be surjective functions in Set; thus, in particular, the component at $X \times X$ will be surjective, as necessary.}. 

We take $X$ to be the subobject of $\Mor(C_{strict})$ comprising those morphisms whose codomain is $\introS(\Mor(C_{strict}))$. That is, the object given by the following equalizer diagram.

% https://q.uiver.app/?q=WzAsNCxbMCwwLCJYIl0sWzEsMCwiXFxNb3IoQ197c3RyaWN0fSkiXSxbMywwLCJcXE9iKENfe3N0cmljdH0pIl0sWzIsMSwiMSJdLFswLDEsImkiLDIseyJzdHlsZSI6eyJ0YWlsIjp7Im5hbWUiOiJtb25vIn19fV0sWzEsMiwiXFxjb2QiXSxbMSwzLCIhIiwyXSxbMywyLCJcXGludHJvUycoXFxNb3IoQ197c3RyaWN0fSkpIiwyXV0=
\[\begin{tikzcd}
	X & {\Mor(C_{strict})} && {\Ob(C_{strict})} \\
	&& 1
	\arrow["i"', tail, from=1-1, to=1-2]
	\arrow["\cod", from=1-2, to=1-4]
	\arrow["{!}"', from=1-2, to=2-3]
	\arrow["{\introS'(\Mor(C_{strict}))}"', from=2-3, to=1-4]
\end{tikzcd}\]

In the above diagram, we have labelled an arrow with the name $\introS'(\Mor(C_{strict}))$. By this we mean some arbitrary globally defined object of $C_{strict}$ which presents the globally defined object $\introS(\Mor(C_{strict}))$ of $C$. We pedantically caution that there may actually be multiple non-equal global elements of $\Ob(C_{strict})$ which present objects isomorphic to $\introS(\Mor(C_{strict}))$. But any arbitrary choice of some such element will be fine to use as the arrow in this diagram for our purposes. (Indeed, it is readily seen that even two non-equal such choices will still lead to isomorphic $X$es. Or more precisely, isomorphic results as an object of $T$, though not isomorphic as a subobject of $\Mor(C_{strict})$, as the specific choice of inclusion map $i : X \to \Mor(C_{strict})$ will vary. But again, any so-arising choice will be fine for our purposes.).

Note that, by virtue of being an equalizer, the inclusion map $i : X \to \Mor(C_{strict})$ in $T$ is monic, and thus so also is $\introS(i) : \introS(X) \to \introS(\Mor(C_{strict}))$ in $C$. From this, we can define our $\alpha : X \to P(\introS(X))$ and establish its surjectivity condition. Specifically, observe that pullback along $\introS(i)$ gives us a functor $\pullAlong{\introS(i)} : C/\introS(\Mor(C_{strict})) \to C/\introS(X)$. If we focus on the action of $\pullAlong{\introS(i)}$ on objects, consider its input object as presented by an object of $C_{strict}/\introS'(\Mor(C_{strict}))$ (whose objects comprise $X$), and consider its output object modulo isomorphism, this yields $\pullAlong{\introS(i)} : X \to P(\introS(X))$, which we take as our definition of $\alpha$.

As for the surjectivity condition, let $F$ be an arbitrary global element of $P(\introS(X))$; that is, an arbitrary isomorphism class of objects of $C/\introS(X)$. The pushforward (i.e., composition) action of $\introS(i)$ gives us a functor from $C/\introS(X) \to C/\introS(\Mor(C_{strict}))$, taking $F$ to $\introS(i) \circ F$, an isomorphism class of objects of the global aspect of $C/\introS(\Mor(C_{strict}))$. This will be presented by at least one globally defined element $f$ of $X$ (keeping in mind the definition of $X$); there may be multiple non-equal such $f$ but any will do. Observe that $\alpha(f)$ is the isomorphism class of $C/\introS(X)$ corresponding to $\introS(i) \circ F$ pulled back along $\introS(i)$. This isomorphism class is the same as that of $F$ itself, because of the monicity of $\introS(i)$, like so:

% https://q.uiver.app/?q=WzAsNixbMSwxLCJcXGludHJvUyhYKSJdLFsxLDIsIlxcaW50cm9TKFxcTW9yKENfe3N0cmljdH0pKSJdLFsxLDAsIlxcYnVsbGV0Il0sWzAsMiwiXFxpbnRyb1MoWCkiXSxbMCwxLCJcXGludHJvUyhYKSJdLFswLDAsIlxcYnVsbGV0Il0sWzAsMSwiXFxpbnRyb1MoaSkiXSxbMiwwLCJGIl0sWzMsMSwiXFxpbnRyb1MoaSkiLDJdLFs0LDMsIlxcaWQiLDJdLFs0LDAsIlxcaWQiLDJdLFs0LDEsIiIsMix7InN0eWxlIjp7Im5hbWUiOiJjb3JuZXIifX1dLFs1LDQsIkYiLDJdLFs1LDIsIlxcaWQiXSxbNSwwLCIiLDAseyJzdHlsZSI6eyJuYW1lIjoiY29ybmVyIn19XV0=
\[\begin{tikzcd}
	\bullet & \bullet \\
	{\introS(X)} & {\introS(X)} \\
	{\introS(X)} & {\introS(\Mor(C_{strict}))}
	\arrow["{\introS(i)}", from=2-2, to=3-2]
	\arrow["F", from=1-2, to=2-2]
	\arrow["{\introS(i)}"', from=3-1, to=3-2]
	\arrow["\id"', from=2-1, to=3-1]
	\arrow["\id"', from=2-1, to=2-2]
	\arrow["\lrcorner"{anchor=center, pos=0.125}, draw=none, from=2-1, to=3-2]
	\arrow["F"', from=1-1, to=2-1]
	\arrow["\id", from=1-1, to=1-2]
	\arrow["\lrcorner"{anchor=center, pos=0.125}, draw=none, from=1-1, to=2-2]
\end{tikzcd}\]

Thus, $\alpha(f) = F$ as an element of $P(\introS(X))$, establishing the required surjectivity condition on $\alpha$.

Thus, all presumptions are satisfied for us to be able to apply \magicref{PreIntrospDiag} with these definitions, for an arbitrary globally defined element $g$ of $P(\introS(\Omega))$.

In particular, let an arbitrary $(T, C)$-indexed set $G$ be given (\TODOinline{satisfying the appropriate smallness conditions to be in $\Psh{C}$; local $T$-\repsmall/ness}). In fact, it suffices for $G$ to be merely $(T, \Ob(\core{C}))$-indexed, where $\Ob(\core{C})$ is the set of objects of $C$ modulo isomorphism.

This will be presented by a slice in $T/\Ob(C_{strict})$ (the slice whose fiber at any object $c_{strict}$ of $C_{strict}$ is the set $G(c)$, where $c$ is the object of $C$ presented by $c_{strict}$). By applying $\introS$ to this slice, we get a globally defined object of $C/\introS(\Ob(C_{strict}))$, which is to say, a global element of $P(\introS(\Omega))$. Take this to be our $g$.

Invoking \magicref{PreIntrospDiag} (on the introspective theory $\langle T, C, \introS, \introN \rangle$, with $\point_C$ as the terminal object of $C$, $\point_T$ as the terminal object of $T$, and all other inputs ($P$, $\Omega$, $\quotient$, $X$, $\alpha$, and $g$) as described with the same name above), we now get a globally defined element $\omega$ of $\Omega = \Ob(C_{strict})$ such that $\quotient(\omega) = \pullAlong{\introN_{\Omega}(\omega)} g$. This equation is saying precisely that $\omega$ presents an object $Y$ of $C$ such that $Y$ is isomorphic to $\introS(G(Y))$. \TODOinline{Maybe more about how $\pullAlong{\introN_{\Omega}(\omega)} g$ is $\introS(G(Y))$}

Thus, we have proven the following:
\begin{theorem}\label{IntrospTyConFixedPoints}
For any introspective theory $\langle T, C \rangle$, and any \TODOinline{locally $T$-\repsmall/} $(T, \Ob(\core{C}))$-indexed set $G$, there is some object $Y$ in the global aspect of $C$ along with a globally defined isomorphism from $Y$ to $\introS(G(Y))$.
\end{theorem}

Combining this with \magicref{PreIntrospDiagFromIso} to eliminate the latter's isomorphism precondition, we now reach the following conclusion:
\openNamed{theorem}{L\"ob's Theorem for Introspective Theories}\label{IntrospLoeb}
Suppose given an introspective theory $\langle T, C, \introS, \introN \rangle$ and a $P \in \Psh{C}$.

Then for every globally defined $g : \Box P \to P$, we obtain a globally defined $\omega : 1 \to P$, such that $g \circ \omega' = \omega$, where the globally defined $\omega' : 1 \to \Box P$ is given by $\introN_{P(1)}(\omega)$. \TODOinline{Clarify this notation. Clarify the equivalence of $\Box$ and $\introN$ on global elements. Get rid of $P(1)$ ambiguity.}
\closeNamed{theorem}

We can consider the particular case where $P$ is $C$-\repsmall/, just as $\Box P$ is. In other words, where $P(-) = \Hom_C(-, c)$ is the representable presheaf on $C$ represented by some object $c$ of $C$. All traditional accounts of \Loeb/'s theorem are along these lines. But note that we can also just as well consider this \magicref{IntrospLoeb} for non-representable presheaves $P$, a significant generalization of the traditional viewpoint.

Using our $\Box$ symbolism and the Yoneda lemma identification of elements of $P(c)$ with maps from $c$ to $P$, we can phrase this fixed point property of $\omega$ as the following morphism of $\Box$-algebras within the global aspect of $\Psh{C}$:

% https://q.uiver.app/?q=WzAsNCxbMCwwLCJcXEJveCAxIl0sWzAsMSwiMSJdLFsxLDAsIlxcQm94IFAiXSxbMSwxLCJQIl0sWzAsMSwiISIsMl0sWzAsMiwiXFxCb3ggXFxvbWVnYSJdLFsyLDMsImciXSxbMSwzLCJcXG9tZWdhIiwyXV0=
\[\begin{tikzcd}
	{\Box 1} & {\Box P} \\
	1 & P
	\arrow["{!}"', from=1-1, to=2-1]
	\arrow["{\Box \omega}", from=1-1, to=1-2]
	\arrow["g", from=1-2, to=2-2]
	\arrow["\omega"', from=2-1, to=2-2]
\end{tikzcd}\]

Thus, the $\Box$-algebra with carrier $1$ is weakly initial within $\Psh{C}$, in the sense that it has some morphism into every other $\Box$-algebra.

Although we have just proven that this property holds for all introspective theories automatically, it does not hold automatically for merely locally introspective theories. However, we shall see later on that there are some natural examples of locally introspective theories that happen to still have this property. \TODOinline{Later, when we get to well-founded models, define locally Loeb theories in general, as locally introspective theories with the appropriate type-level (and therefore term-level) fixed points in all slice categories. Conjecture that any such thing embeds into a fully introspective theory, the same way as happens for the well-founded models. Note that all the results in this chapter apply to locally Loeb theories too.}

\subsection{Transfer and uniformity of the \Loeb/ property}
We proved above a version of \Loeb/'s theorem for objects of $\Psh{C}$ with respect to $\Box_{\Psh{C}}$, in the context of an introspective theory $\langle T, C \rangle$. Because of how $C$ is identified as a full subcategory of $\Psh{C}$, this automatically gives us the analogous theorem for objects of $C$ with respect to $\Box_C$ as a special case. But what about for objects of $T$ with respect to $\Box_T$? We shall show now that we automatically get the analogous theorem for this as well. In fact, all these theorems are equivalent. (Later, in \TODOinline{cite geminal categories chapter}, we will see another principled reason why we would automatically get this same transfer from properties of $C$ to properties of $T$.)

\begin{theorem}
Let $D$ and $E$ be categories with terminal objects, and let $M : D \to E$ and $N : E \to D$ be functors preserving terminal objects. Suppose $D$ has the property that for every object $d$ of $D$ and every morphism $g_D: NM(d) \to d$, there exists a morphism $\omega_D : 1 \to d$ making the following triangle commute:

% https://q.uiver.app/?q=WzAsMyxbMCwxLCIxIl0sWzEsMCwiTk0oZCkiXSxbMSwxLCJkIl0sWzAsMSwiTk0oXFxvbWVnYV9EKSJdLFsxLDIsImdfRCJdLFswLDIsIlxcb21lZ2FfRCIsMl1d
\[\begin{tikzcd}
	& {NM(d)} \\
	1 & d
	\arrow["{NM(\omega_D)}", from=2-1, to=1-2]
	\arrow["{g_D}", from=1-2, to=2-2]
	\arrow["{\omega_D}"', from=2-1, to=2-2]
\end{tikzcd}\]

Then $E$ has the analogous property that for every object $e$ of $E$ and every morphism $g_E: MN(e) \to e$, there exists a morphism $\omega_E: 1 \to e$ making the following triangle commute:

% https://q.uiver.app/?q=WzAsMyxbMCwxLCIxIl0sWzEsMCwiTU4oZSkiXSxbMSwxLCJlIl0sWzAsMSwiTU4oXFxvbWVnYV9FKSJdLFsxLDIsImdfRSJdLFswLDIsIlxcb21lZ2FfRSIsMl1d
\[\begin{tikzcd}
	& {MN(e)} \\
	1 & e
	\arrow["{MN(\omega_E)}", from=2-1, to=1-2]
	\arrow["{g_E}", from=1-2, to=2-2]
	\arrow["{\omega_E}"', from=2-1, to=2-2]
\end{tikzcd}\]
\end{theorem}
\begin{proof}
Let $g_E : MN(e) \to e$ be given. By our supposition on $D$, taking $d = N(e)$ and $g_D = N(g_E) : NMN(e) \to N(e)$, we get $\omega_D$ making the following diagram commute:

% https://q.uiver.app/?q=WzAsMyxbMCwxLCIxIl0sWzEsMCwiTk1OKGUpIl0sWzEsMSwiTihlKSJdLFswLDEsIk5NKFxcb21lZ2FfRCkiXSxbMSwyLCJOKGdfRSkiXSxbMCwyLCJcXG9tZWdhX0QiLDJdXQ==
\[\begin{tikzcd}
	& {NMN(e)} \\
	1 & {N(e)}
	\arrow["{NM(\omega_D)}", from=2-1, to=1-2]
	\arrow["{N(g_E)}", from=1-2, to=2-2]
	\arrow["{\omega_D}"', from=2-1, to=2-2]
\end{tikzcd}\]

Applying $M$ to this, we get the following diagram:
% https://q.uiver.app/?q=WzAsMyxbMCwxLCIxIl0sWzIsMCwiTU5NTihlKSJdLFsyLDEsIk1OKGUpIl0sWzAsMSwiTU5NKFxcb21lZ2FfRCkiXSxbMSwyLCJNTihnX0UpIl0sWzAsMiwiTShcXG9tZWdhX0QpIiwyXV0=
\[\begin{tikzcd}
	&& {MNMN(e)} \\
	1 && {MN(e)}
	\arrow["{MNM(\omega_D)}", from=2-1, to=1-3]
	\arrow["{MN(g_E)}", from=1-3, to=2-3]
	\arrow["{M(\omega_D)}"', from=2-1, to=2-3]
\end{tikzcd}\]

But now defining $\omega_E$ as $g_E \circ M(\omega_D)$ as shown and then observing that $MN(\omega_E)$ is the composite path from $1$ to $MN(e)$ around the top of this diagram, we see that this $\omega_E$ has the desired property, completing the proof.

% https://q.uiver.app/?q=WzAsNSxbMCwxLCIxIl0sWzIsMCwiTU5NTihlKSJdLFsyLDEsIk1OKGUpIl0sWzIsMiwiZSJdLFswLDIsIjEiXSxbMCwxLCJNTk0oXFxvbWVnYV9EKSJdLFsxLDIsIk1OKGdfRSkiXSxbMCwyLCJNKFxcb21lZ2FfRCkiLDJdLFsyLDMsImdfRSJdLFs0LDMsIlxcb21lZ2FfRSA9IGdfRSBcXGNpcmMgTShcXG9tZWdhX0QpIiwyXSxbMCw0LCIiLDIseyJsZXZlbCI6Miwic3R5bGUiOnsiaGVhZCI6eyJuYW1lIjoibm9uZSJ9fX1dXQ==
\[\begin{tikzcd}
	&& {MNMN(e)} \\
	1 && {MN(e)} \\
	1 && e
	\arrow["{MNM(\omega_D)}", from=2-1, to=1-3]
	\arrow["{MN(g_E)}", from=1-3, to=2-3]
	\arrow["{M(\omega_D)}"', from=2-1, to=2-3]
	\arrow["{g_E}", from=2-3, to=3-3]
	\arrow["{\omega_E = g_E \circ M(\omega_D)}"', from=3-1, to=3-3]
	\arrow[Rightarrow, no head, from=2-1, to=3-1]
\end{tikzcd}\]
\end{proof}

\TODOinline{Loeb's theorem for slice categories, including Loeb's theorem given by a generic slice thus qua combinator}

\subsection{Uniqueness and initiality/terminality}
Throughout this section, we work in the context of a locally \Loeb/ theory $\langle T, C \rangle$. We establish some results on the uniqueness of the fixed points given by \magicref{LocallyIntrospLoeb}, as well as their initiality or terminality in certain contexts. We note that similar arguments to those given in this section have already been given in the literature on guarded recursion; for example, in \autocite{birkedal2011first} and in \autocite{birkedal2013universes}. We record these results here to explicitly confirm that they hold in our particular setting (with $T$, $C$, and $\Psh{C}$ distinguished, with no assumption of cartesian closure, with no form of \Loeb/'s directly assumed but rather this having been derived from other assumptions, etc.).

\TODOinline{The rest of this section is old stuff to be cleaned up}

\begin{theorem}\label{UniqueFixedPoints}
The fixed points produced by \magicref{LocallyIntrospLoeb} are unique, in the sense that for any $t$-defined point $f$ of $P(\introS(P(1)))$ and any two $t$-defined points $a$ and $b$ of $P(1)$ satisfying the appropriate fixed point property (that is, each is equal to $f$ transported back along $\introN$ of itself), we have that $a = b$.
\end{theorem}
\begin{proof}
By passing to the appropriate slice introspective theory $T/t$, we can assume without loss of generality that all relevant values in the following are globally defined; i.e., without loss of generality, we can presume $t = 1$.

Consider the equalizer $E$ of $a$ and $b$. This is a subobject of $1$. If there were a map from $1$ to $E$, then $a$ and $b$ would be equal. In just the same way, taking this equalizer diagram's image under the lexfunctor $\introS$, we see that any value in $\Hom_C(1, \introS(E))$ would lead to $\introS(a)$ and $\introS(b)$ being equal, thus corresponding to equal maps from $1$ to $\introS(P(1))$ in $C$. Transporting $f$ back along these two would yield equal values, therefore. But transporting $f$ back along these two yields $a$ and $b$ respectively, so we would have that $a$ and $b$ are equal. This argument yields a morphism from $\Hom_C(1, \introS(E))$ to $E$ in $T$. Applying $\introS$ to this, we have a morphism in $C$, which lives in $Q(\introS(Q(1)))$, where $Q$ is the presheaf on $C$ represented by $\introS(E)$. Applying \magicref{LocallyIntrospLoeb} to this, we get an element of $Q(1)$, which is to say, a global element of $\introS(E)$ in $C$. This makes $\introS(a)$ and $\introS(b)$ equal as global points of $\introS(P(1))$ in $C$. And by transporting $f$ back along these, we conclude as before that $a$ and $b$ are equal. \TODOinline{Word this all better}
\end{proof}

\begin{observation}
Note that this argument made essential use of the structure (notably, equalizers) available in an introspective theory, and thus does not apply in such full generality as GeneralDig \TODO does. For example, we saw in \TODOinline{YCombinator} that we get fixed points for arbitrary functions in the untyped lambda calculus, but as a cartesian closed category in general is not an introspective theory, we cannot conclude that fixed points of functions in the untyped lambda calculus are unique (and indeed, they will not be, as one such function is the identity function, whose fixed points are all values!).
\end{observation}

\TODOinline{Show that the equalizer from X to []X to X equalized against $\id_X$ is $1$}

\begin{observation}
Note that this means that the apparent dependence on which particular object $Y$ and retraction from $P(\introS(Y))$ into $Y$ is used in the diagonalization theorem doesn't actually matter, in this context. We get the same result no matter what.
\end{observation}

A similar argument to \magicref{UniqueFixedPoints} tells us that fixed points for endofunctors are unique, and even some initial algebra/terminal coalgebra properties for them:

\begin{theorem}\label{CoalgToAlgExist}
Suppose $Q$ is a $T$-small presheaf on the opposite category of $C$. (That is, a $T$-indexed covariant functor from $C$ to the self-indexing of $T$ whose category of elements is $T$-small). This induces a covariant endofunctor $Q(\introS(-))$ on $T$. Suppose $a : A \to Q(\introS(A))$ is a coalgebra for this functor within $T$ and $b : Q(\introS(B)) \to B$ is an algebra for this functor within $T$. Then there is some coalgebra-to-algebra morphism $g$ from $\introS(a)$ to $\introS(b)$ within $C$. That is, there is some $g : \introS(a) \to \introS(b)$ within $C$ such that $g = \introS(a) \circ \introS(Q(g)) \circ \introS(b)$.
\end{theorem}
\begin{proof}
Note that the action of $Q$ gives us a morphism from $\Hom_C(\introS(A), \introS(B)) \times Q(\introS(A))$ to $Q(\introS(B))$ within $T$. By pre- and post-composing this with $a$ and $b$, respectively, we get a morphism from $\Hom_C(\introS(A), \introS(B)) \times A$ to $B$ within $T$. By applying $\introS$ to this, we get a globally defined morphism within $C$ from $\introS(\Hom_C(\introS(A), \introS(B))) \times \introS(A)$ to $\introS(B)$. This morphism is of type $P(S(P(1)))$, where $P$ is the $T$-small presheaf upon $C$ defined by $P(-) = \Hom_C(- \times \introS(A), \introS(B))$.

Thus, we can apply our fixed point theorem \magicref{LocallyIntrospLoeb} to this value. The result is a value $g$ in $P(1)$ (i.e., in $\Hom_C(\introS(A), \introS(B))$) which is equal to $\introS(a \circ Q(g) \circ b)$, as desired.

\TODOinline{Write diagrams to make this all clearer}
\end{proof}

\begin{theorem}\label{CoalgToAlgUnique}
The theorem \magicref{CoalgToAlgExist} can be strengthened to not only produce a coalgebra-to-algebra morphism, but also to conclude that such a coalgebra-to-algebra morphism is unique.
\end{theorem}
\begin{proof}
After applying \magicref{CoalgToAlgExist} with a suitable presheaf $Q$ to get existence, apply \magicref{UniqueFixedPoints} with that same presheaf to get uniqueness.
\end{proof}

\begin{theorem}
Given $Q$ as in \magicref{CoalgToAlgExist}, there exists a unique (up to isomorphism) fixed point of $Q(\introS(-))$, in the sense of a unique (up to isomorphism) object $A$ and isomorphism $a : A \to Q(\introS(A))$. This fixed point is both a terminal coalgebra and (its inverse is) an initial algebra.
\end{theorem}
\begin{proof}
We get the existence of an object $A$ and isomorphism $a : A \to Q(\introS(A))$ directly from the definition of a locally \Loeb/ theory.

Furthermore, by \magicref{CoalgToAlgUnique}, any coalgebra has a unique map into any algebra, from which it follows that any inverse morphism of a coalgebra (or algebra) is in fact an initial algebra (or terminal coalgebra). Since initial (or terminal) objects are all isomorphic to each other, we immediately have the uniqueness up to isomorphism we seek.
\end{proof}

Essentially identical arguments work when dealing with presheaves on the opposite category of the core of $C$ (that is, its subcategory including only its isomorphisms):

\begin{theorem}\label{CoreCoalgToAlgExist}
Suppose $Q$ is a $T$-small presheaf on the opposite category of the core of $C$. This induces a covariant endofunctor $Q(\introS(-))$ on the core of $T$. Suppose $a : A \to Q(\introS(A))$ is a coalgebra for this functor within the core of $T$ and $b : Q(\introS(B)) \to B$ is an algebra for this functor within the core of $T$. Then there is some coalgebra-to-algebra isomorphism $g$ from $\introS(a)$ to $\introS(b)$ within $C$. That is, there is some $g : \introS(a) \to \introS(b)$ within $C$ such that $g = \introS(a) \circ \introS(Q(g)) \circ \introS(b)$.
\end{theorem}
\begin{proof}
Note that the action of $Q$ gives us a morphism from $\Iso_{C}(\introS(A), \introS(B)) \times Q(\introS(A))$ to $Q(\introS(B))$ within $T$. Actually, this satisfies a stronger property, of corresponding to a morphism from $\Iso_{C}(\introS(A), \introS(B))$ to $\Iso_T(Q(\introS(A)), Q(\introS(B)))$.

By pre- and post-composing this with $a$ and $b$, respectively, we get a morphism from $\Iso_C(\introS(A), \introS(B)) \times A$ to $B$ within $T$, that corresponds to a morphism from $\Iso_C(\introS(A), \introS(B))$ to $\Iso_T(A, B)$.

By applying $\introS$ to this, we get a globally defined morphism within $C$ from $\introS(\Iso_C($ $\introS(A), \introS(B))) \times \introS(A)$ to $\introS(B)$, that corresponds to a morphism from $\introS(\Iso_C(\introS(A), \introS(B)))$ to $\Iso_C(\introS(A), \introS(B))$.

This morphism is of type $P(S(P(1)))$, where $P$ is the $T$-small presheaf upon the core of $C$ defined by $P(c)$ amounting to $c$-defined isomorphisms from $\introS(A)$ to $\introS(B)$.

Thus, we can apply our fixed point theorem \magicref{LocallyIntrospLoeb} to this value. The result is a value $g$ in $P(1)$ (i.e., in $\Iso_C(\introS(A), \introS(B))$) which is equal to $\introS(a \circ Q(g) \circ b)$, as desired.

\TODOinline{Write diagrams to make this all clearer}
\end{proof}

\begin{theorem}
The theorem \magicref{CoreCoalgToAlgExist} can be strengthened to not only produce a coalgebra-to-algebra isomorphism, but also to conclude that such a coalgebra-to-algebra isomorphism is unique.
\end{theorem}
\begin{proof}
After applying \magicref{CoreCoalgToAlgExist} with a suitable presheaf $Q$ to get existence, apply \magicref{UniqueFixedPoints} with that same presheaf to get uniqueness.
\end{proof}

\TODOinline{Thus, not only is there an essentially unique fixed point for isovariant functors from C to T, but furthermore, this unique fixed point is rigid}

It is worth noting that the proofs of \magicref{UniqueFixedPoints}, of \magicref{CoalgToAlgExist}, and of \magicref{CoreCoalgToAlgExist} can all be unified, seen as different instances of one general theorem. \TODOinline{Do this, and then having done so, the above cleans up a lot.}

\begin{TODOblock}
Discuss how we may conclude that our functorial fixed points are both initial algebras and terminal coalgebras. Note that Loeb's theorem itself is the initial algebra property for []1 |- 1.
\end{TODOblock}

\begin{TODOblock}
Discuss how we get \Loeb/'s theorem not just for globally defined objects of $C$, but for arbitrary objects of $C$, by working within the locally introspective theory $T/D$ for suitable base of definition $D$, or within $T/\Ob(C)$ generically in an introspective theory. And this transfers into $C$ itself (such that $C$ holds \Loeb/'s theorem to be true for arbitrary objects of $\introS(C) = C_1$) via $\introS$. Write up details for this, including making sure we have the right slice-introspective theory construction.
\end{TODOblock}

\begin{TODOblock}
Demonstrate that we do NOT get Loeb's theorem internal to a geminal category G for arbitrary presheaves P on |G'|, thus demonstrating the necessity of the presheaf existing within an introspective theory. The presheaf needs to be parametrized by a parameter from an object of an enclosing introspective theory. So P(S(X)) |- []P(S(X)) is available.
\end{TODOblock}

\begin{TODOblock}
Related work to our discussion of of uniqueness and initiality/terminality for fixed points within an introspective theory.

Discuss the relationship to models of guarded recursion, and the Birkedal et al paper on \quote{First steps in synthetic...}, where the initiality/terminality argument for contractive functors is presented. Basically, Birkedal's definition 7.2 and lemma 7.6 do the work for us in establishing initality/terminality, but we can use contractive internal categories, rather than locally contractive enriched categories. We could consider indexed categories too? Also, the establishment of the existence of fixed points from fixed points on the universe is done in "Intensional Type Theory with Guarded Recursive Types qua Fixed Points on Universes", and the uniqueness of fixed points following from Loeb's theorem is given there too. The uniqueness of the Loeb combinator is also given in "MULTIMODAL DEPENDENT TYPE THEORY" by Gratzer, Kavvos, Birkedal, et al, as theorem 9.5. So essentially everything in this subsection is anticipated by those. Except that we can comment on the implications when the distinction between T and C is drawn, and when there is no cartesian closure presumption, and when we use non-representable presheaves, etc. Note that C doesn't correspond to the universe U, but rather to []U (or some other category into which []U has a lexfunctor; in a super-introspective theory, perhaps C is particularly close to just []U itself). Our T is like the U-small sets. The assumption we make that others do not is that []U is itself U-small. By dropping the cartesian closure assumption, we are also able to encompass the initial arithmetic universe as a model.

Roughly speaking, the difference between our theory and the theory such as in "Intensional Type Theory with Guarded Recursive Types qua Fixed Points on Universes" is all about cartesian closure and the smallness of []U. A super-introspective theory which happens to be locally cartesian closed (on both the levels of T and C, though the [] operator needn't preserve exponentials) corresponds to a model of guarded recursive types with universes where []U and [](sigma u, v : U. El(u) -> El(v)) are U-small. We do not bother making this correspondence formal, but leave it for future work. The dropping of cartesian closure from the presumptions allows us to encompass the initial arithmetic universe and its ilk as models, while adding the presumption of []U etc as U-small allows us to derive \Loeb/'s theorem from this instead of having to presume it by fiat. (More generally, it allows us to show how \Loeb/'s theorem applies not just to objects of C but also to non-representable presheaves on C in the form P(S(P(1))) |- P(1)).
\end{TODOblock}

\TODOinline{Observe that we can, with care, recover Loeb's theorem for T from our Loeb's theorem for C}

\subsection{Relating variations on Lawvere's fixed point theorem}
Although not important for our main narrative, we note here some further comments on the relation of Lawvere's fixed point theorem to generalizations of ours or others.

First, we observe that \magicref{LawveresFixedPointTheorem} can be straightforwardly re-obtained as a special case of our \magicref{PreIntrospDiag}.
\begin{proof}
First, we handle the special case of \magicref{LawveresFixedPointTheorem} where $T$ has finite products and $\Omega$ is an object of $T$.

This is a special case of \magicref{PreIntrospDiag} where we furthermore take the pre-introspective finite product theory $\langle T, C \rangle$ to be the trivial one noted at \TODOinline{Cite earlier construction}, where $C$ is the simple self-indexing $T//-$, and $\introF$ is the identity.

Furthermore, $P$ is taken to be the $(T, C)$-indexed set represented by $\Omega$; that is, such that $P(t, c) = \Hom_T(t \times c, \Omega)$. Note that $P(\introS(t))$ for objects $t$ of $T$ is therefore the $T$-indexed set $\Omega^t$. In particular, $P(1)$ is thus isomorphic to $\Omega$; as in \magicref{PreIntrospDiagSpecialization} we can take $\quotient$ to be this isomorphism (one can think of it as an identity if one likes), and this will then automatically be surjective on its $X \times X$ aspect.

We take $\alpha : X \to P(\introS(X)) = \Omega^X$ to be given by the map $\App' : X \to \Omega^X$ presumed in \magicref{LawveresFixedPointTheorem}. The surjectivity presumption from \magicref{LawveresFixedPointTheorem} then becomes the surjectivity presumption of \magicref{PreIntrospDiag}. 

And to give a $g$ in the global aspect of $P(\introS(\Omega)) = \Omega^\Omega$ is precisely the data presumed by the name $g$ in \magicref{LawveresFixedPointTheorem}.

This matches all the presumptions of \magicref{PreIntrospDiag} up with corresponding presumptions from \magicref{LawveresFixedPointTheorem}, and the conclusion we then obtain from \magicref{PreIntrospDiag} is readily seen to be the same as the conclusion from \magicref{LawveresFixedPointTheorem}.

The above shows how to obtain \magicref{LawveresFixedPointTheorem} as an instance of \magicref{PreIntrospDiag} when $T$ is a finite product category and $\Omega$ is an object of $T$. We then obtain \magicref{LawveresFixedPointTheorem} in full (that is, for arbitrary categories $T$ and $T$-indexed sets $\Omega$) from this special case, by first replacing $T$ with $\Psh{T}$, as noted at \TODOinline{Cite our previous discussion}.
\end{proof}

We also note in passing that another interesting generalization of \magicref{LawveresFixedPointTheorem} was recently remarked upon in \autocite{roberts2021substructural}. The following (or rather, its contrapositive) was given as Theorem 11 there. We shall present our own proof.

\openNamed{theorem}{The Magmoidal Fixed Point Theorem}\label{MagmoidalFixedPointTheorem}
Let $T$ be an arbitrary category with objects $\point$ and $\Omega$, and let $B : T \times T \to T$ be a bifunctor on $T$ such that we have a transformation $\delta_t : t \to B(t, t)$ natural in $t$ from $T$. As ever, use \quote{point of} to mean \quote{element of the $\point$-aspect of}.

Suppose given an object $X$ of $T$ and an $\alpha : B(X, X) \to \Omega$ with the pointwise surjectivity property that for every $F : X \to \Omega$, there is a point $f$ of $X$, such that for every point $x$ of $X$, we have that the following diagram commutes:

% https://q.uiver.app/?q=WzAsNSxbMCwwLCJcXHBvaW50Il0sWzEsMCwiQihcXHBvaW50LCBcXHBvaW50KSJdLFsyLDAsIkIoWCwgWCkiXSxbMywwLCJcXE9tZWdhIl0sWzEsMSwiWCJdLFswLDEsIlxcZGVsdGFfe1xccG9pbnR9Il0sWzEsMiwiQihmLCB4KSJdLFsyLDMsIlxcYWxwaGEiXSxbMCw0LCJ4IiwyXSxbNCwzLCJGIiwyXV0=
\[\begin{tikzcd}
	\point & {B(\point, \point)} & {B(X, X)} & \Omega \\
	& X
	\arrow["{\delta_{\point}}", from=1-1, to=1-2]
	\arrow["{B(f, x)}", from=1-2, to=1-3]
	\arrow["\alpha", from=1-3, to=1-4]
	\arrow["x"', from=1-1, to=2-2]
	\arrow["F"', from=2-2, to=1-4]
\end{tikzcd}\]

Then for every $g : \Omega \to \Omega$, there is a point $\omega$ of $\Omega$ such that $\omega = g(\omega)$. That is to say, a fixed point of $g$.
\closeNamed{theorem}
\begin{proof}
Take $\App : X \times X \to \Omega$ to be defined like so: For each object $t$ of $T$, we define $\App_t : \Hom(t, X) \times \Hom(t, X) \to \Hom(t, \Omega)$ by giving $\App_t(m, n)$ as the following composition:

% https://q.uiver.app/?q=WzAsNCxbMSwwLCJCKHQsIHQpIl0sWzAsMCwidCJdLFszLDAsIkIoWCwgWCkiXSxbNCwwLCJcXE9tZWdhIl0sWzEsMCwiXFxkZWx0YV97dH0iXSxbMCwyLCJCKG0sIG4pIl0sWzIsMywiXFxhbHBoYSJdXQ==
\[\begin{tikzcd}
	t & {B(t, t)} && {B(X, X)} & \Omega
	\arrow["{\delta_{t}}", from=1-1, to=1-2]
	\arrow["{B(m, n)}", from=1-2, to=1-4]
	\arrow["\alpha", from=1-4, to=1-5]
\end{tikzcd}\]

That this definition of $\App_t$ is natural in $t$ follows from the naturality of $\delta$ and the functoriality of $B$. Specifically, naturality with respect to $h: s \to t$ is seen as follows:

% https://q.uiver.app/?q=WzAsNixbMSwwLCJCKHQsIHQpIl0sWzAsMCwidCJdLFszLDAsIkIoWCwgWCkiXSxbNCwwLCJcXE9tZWdhIl0sWzAsMSwicyJdLFsxLDEsIkIocywgcykiXSxbMSwwLCJcXGRlbHRhX3t0fSJdLFswLDIsIkIobSwgbikiXSxbMiwzLCJcXGFscGhhIl0sWzQsNSwiXFxkZWx0YV9zIiwyXSxbNCwxLCJoIl0sWzUsMCwiQihoLCBoKSIsMV0sWzUsMiwiQihtIGgsIG4gaCkiLDJdXQ==
\[\begin{tikzcd}
	t & {B(t, t)} && {B(X, X)} & \Omega \\
	s & {B(s, s)}
	\arrow["{\delta_{t}}", from=1-1, to=1-2]
	\arrow["{B(m, n)}", from=1-2, to=1-4]
	\arrow["\alpha", from=1-4, to=1-5]
	\arrow["{\delta_s}"', from=2-1, to=2-2]
	\arrow["h", from=2-1, to=1-1]
	\arrow["{B(h, h)}"{description}, from=2-2, to=1-2]
	\arrow["{B(m h, n h)}"', from=2-2, to=1-4]
\end{tikzcd}\]

The desired result now follows by \magicref{LawveresFixedPointTheorem}.
\end{proof}
\magicref{LawveresFixedPointTheorem} is of course the special case of \magicref{MagmoidalFixedPointTheorem} where $B$ is the familiar cartesian product and $\delta$ is the familiar diagonal transformation. Thus, in \autocite{roberts2021substructural}, \magicref{MagmoidalFixedPointTheorem} is considered as a generalization of Lawvere's fixed point theorem. But as we've just seen, \magicref{MagmoidalFixedPointTheorem} is also a special case of Lawvere's fixed point theorem, appropriately construed (as in our formulation of \magicref{LawveresFixedPointTheorem} which removes the $\point = 1$ constraint), despite the seeming mismatch between general bifunctors and specifically cartesian products. As noted before, there is no need for $X \times X$ to be $T$-\repsmall/, and if such closure of our underlying category is insisted upon, we can just as well always pass to $\Psh{T}$ first.

\TODOinline{Remarks on Cantor's theorem, Liar's paradox, and Y combinator as examples of Lawvere's fixed point theorem. Cantor's theorem is a contrapositive statement using surjection. Liar's paradox is a contrapositive statement using a retraction/isomorphism. Y combinator is a positive statement using a retraction, and also involves passing to a slice category. Y combinator is the best set up for what we do in the next section. Note that the reason we presume in Cantor's theorem that Omega has no fixed points is because of another instance of Lawvere's fixed point theorem, via Liar's paradox!}

\TODOinline{Point out the error in Yonofsky's discussion of Kleene's recursion theorem and how our more general formulation allows us to correct this.}

\fileend