\section{\Loeb's theorem}

\subsection{The basics of double indexing}
\begin{TODOblock}
The following is all preliminary scribblings, to be written up properly and moved to the Preliminaries later.
\end{TODOblock}

Suppose $T$ is some category and $C$ is some category indexed over $T$. What does it mean to speak of a $T$-indexed $C$-indexed structure? For example, a $T$-indexed $C$-indexed set. That is, a $T$-indexed $C$-indexed presheaf $P$.

What this means is that, for every object $t$ in $T$ and every $t$-defined object $c$ in $C$, we have some corresponding set of $P$ (the $t$-defined $c$-defined elements of $P$), and we also have a coherent system of pullback maps between these: Along any map $f : s \to t$ in $T$, we can pull back a $t$-defined $c$-defined element of $P$ to an $s$-defined $c$-defined element of $P$ (where the latter $c$ is implicitly the pullback along $f$ of the former $c$). And for any fixed $t$, given a $t$-defined map $g : d \to c$ in $C$, we can pull back a $t$-defined $c$-defined element of $P$ along this to a $t$-defined $d$-defined element of $P$. Both of these systems of pullbacks are functorial, and they also interact with each other \quote{commutatively} in the sense that pulling back along a map $g$ in $C$ and then pulling back along a map $f$ in $T$ is the same as pulling back along the map $f$ in $T$ and then pulling back along the map $g$ in $C$ (where, again, the latter, $g$ is implicitly the pullback along $f$ of the former $g$).

This all amounts to saying that $P$ is in fact indexed over the Grothendieck construction corresponding to $C$.

What about maps such doubly-indexed sets? Well, this is the same as the notion we get again by thinking of these as in fact singly-indexed over the Grothendieck construction. A natural transformation in that context.

Now that we understand doubly-indexed sets and maps between them, we can also understand doubly-indexed structures in general. Including doubly-indexed strict categories, and everything about doubly-indexed categorical structures works in basically the same fashion.

That all being said, we won't really need any of that just yet. All we need is the concept of doubly-indexed sets. Even maps between these aren't really going to concern us, just yet.

\subsection{The basics of \Loeb}
Let $\langle T, C, F \rangle = \langle T, C, S, N \rangle$ be a pre-introspective theory. Consider now a $T$-indexed presheaf $P$ on $C$ which is $T$-small, in the sense that for any $t$-defined object $c$ of $C$, the value of $P$ at object $c$ is given by some $t$-defined object of $T/-$. In other words, the corresponding discrete fibration to $P$ has $T$-small fibers.

Presume also given some globally defined\footnote{The globally defined presumption is not a major condition, I think; if it were merely $t$-defined, then we would simply pass to the slice introspective theory over $t$. But for convenience now, we take the definition as global. \TODOinline{Make sure the passage to a slice category here could work}} object $X$ in $C$ and an isomorphism $B$ in $C$ from $X$ to $S(P(X))$. Note that $S(P(X))$ is well-defined as $P(X)$ is an object in $T$, by our $T$-smallness presumption on $P$.

[Note that an $X$ in $C$ isomorphic to $S(P(X))$ is as good as a $Y$ in $T$ isomorphic to $P(S(Y))$, by $Y = P(X)$ and $X = S(Y)$; if speaking in terms of $Y$, let us call the isomorphism $B'$, satisfying $B' = P(B)$ and $B = S(B')$]

Suppose now given any generalized element $y$ of $Y$. Then, we can take the value $B'(y)$ within $P(S(Y))$ and pull it back via the presheaf action of $P$ along the value $N_Y(y)$ within $\Hom_C(1, S(Y))$, to get a value inside $P(1)$. Thus, discharging our assumption of $y$, we have produced a morphism in $T$ from $Y$ to $P(1)$. Call this morphism $W$.

Suppose now given any generalized element $f$ of $P(S(P(1)))$.

By pulling $f :P(S(P(1)))$ back along $S(W) : \Hom_C(S(Y), S(P(1)))$, we get an element of $P(S(Y))$. By applying $B'^{-1}$ to this, we have an element of $Y$. And applying $W$ to this, we end up with an element of $P(1)$. Thus, discharging our assumption of $f$, we have produced a morphism in $T$ from $P(S(P(1)))$ to $P(1)$.

This final morphism represents \Loeb's theorem. For sake of a name, let's call it $R$ for now.

\subsection{The fixed point property}
Recall that $f$ is a generalized element of $P(S(P(1)))$ and $R(f)$ is a similarly generalized element of $P(1)$. We will show that $R(f)$ is equal to pulling $f$ back along the map from $1$ to $S(P(1))$ given by applying $N_{P(1)}$ to $R(f)$.

Setting $y = B'^{-1}(S(W)^* f) : Y$, recall that $R(f)$ is $W(y) : P(1)$. Which is $B'(y) = B'(B'^{-1}(S(W)^* f)) = S(W)^* f : P(S(Y))$ pulled back along $N_Y(y) : \Hom_C(1, S(Y))$. But $N_Y(y) : \Hom_C(1, S(Y))$ composed with $S(W) : \Hom_C(S(Y), S(P(1)))$ is $N_{P(1)}(W(y))$ (by the naturality of $N$). So we find that $R(f)$ is $f$ pulled back along $N_{P(1)}(W(y))$. As $W(y) = R(f)$, this completes the proof.

\subsection{Continuing}

However, this was all dependent on our assumption of the fixed point $X$ (equivalently, $Y$) in the first place. We will now show how to bootstrap away this assumption, in the context of an introspective theory. \TODO

The gist is this: First, we observe that although we stated the above in terms of an isomorphism between $Y$ and $P(S(Y))$, the argument didn't actually make full use of an isomorphism. The argument only required one direction of cancelling out; that $P(S(Y))$ is a retract of $Y$. That is, there is $B'^{-1} : P(S(Y)) \to Y$ and $B' : Y \to P(S(Y))$ such that $B'$ following $B'^{-1}$ cancels out to the identity. The composition in the other direction did not matter.

In fact, it suffices to have either $P(S(Y))$ a retract of $Y$ in $T$, OR $Y$ a retract of $P(S(Y))$ in $T$, OR $S(P(X))$ a retract of $X$ in $C$, OR $X$ a retract of $S(P(X))$ in $C$. Since we have one covariant functor $S$ from $T$ to $C$, and one contravariant functor $P$ from $C$ back to $T$, by starting with any of these four instances and repeatedly applying $S$ or $P$, we obtain instances of each of the others as well. \TODOinline{This is straightforward abstract nonsense, but explain this more clearly all the same.}

But the terminal object $1$ is a retract of anything it maps into. So if $P(S(1))$ has a global element in $T$, or $S(P(1))$ has a global element in $C$, then we are set.

Now suppose we wish to apply our \Loeb's theorem taking $P$ to be the presheaf which assigns to every object $c$ of $C$, the underlying set of objects of $C/c$, with pullback as action on morphisms. Observe that $P(S(1))$ in $T$ is isomorphic to $\Ob(C)$, which has global elements (e.g., the terminal object of $C$, or $S$ applied to any object of $T$). So we can run our \Loeb's theorem on this presheaf.

With one caveat! This presheaf isn't quite a presheaf, as it is valued not in sets but in setoids. Pullback isn't well-defined, it is only well-defined up to isomorphism. We must be careful about this. It turns out this will be no bother for our \Loeb's theorem proof as above. \TODOinline{Show how our \Loeb's theorem allows us to work with setoids fine in this way.}

But supposing we take care of that, the fixed point property we derive then tells us that, for any slice $Q$ above $S(\Ob(C))$ in $C$, there is some object $c$ of $C$ which is isomorphic to $Q$ pulled back along $N_{\Ob(C)}(c) : \Hom_C(1, S(\Ob(C)))$.

In particular, let $Q$ be the slice corresponding to some arbitrarily presheaf $Q'$ on $C$. Then this tells us there is some object $c$ of $C$ which is isomorphic to $Q'(S(c))$. This is the condition we need to run our categorical \Loeb's theorem on the presheaf $Q'$ itself.

Thus, in the context of an introspective theory, we can run the categorical \Loeb's theorem on ALL presheaves, unconditionally.

\begin{TODOblock}
Discuss uniqueness of the fixed point in general, and the simultaneous initial algebra and terminal coalgebra properties for functorial fixed points (even stating what these functorial fixed points amount to requires looking specifically at $P(c) = \Ob(C/c)$, which we will be using for our bootstrapping, but which may also drive us into talking about $P$ here as category-valued (at least, groupoid-valued) rather than set-valued).
\end{TODOblock}