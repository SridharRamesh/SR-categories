\filestart

\section{Introduction}
The aim of these notes is to identify and draw attention to a certain surprisingly simple and category-theoretically natural mathematical structure which both serves as an abstract environment for the reasoning used in establishing \Goedel/'s incompleteness theorems and \Loeb/'s theorem in their traditional instances (as in \autocite{goedel1931formal} and \autocite{loeb1955solution}), and furthermore allows these and the further theorems and fixed-point results of the \Goedel/-\Loeb/ modal logic of provability (as in \autocite{boolos1995logic}) to be vastly generalized.

Some such \Loeb/-style fixed point phenomena have been explored in the literature before, but our abstraction is of note as a particularly simple and general one. This abstraction for the first time formally unifies three distinct threads of work in the literature, having as special cases the interpretation of the \Goedel/-\Loeb/ incompleteness theorems via the initial arithmetic universe a la Joyal (as discussed in \autocite{van2020g}), the interpretation of \Loeb/'s theorem as a guarded fixed point combinator and associated work on guarded (co)inductive types via step-indexing in contexts such as the topos of trees (as in \autocite{birkedal2011first}), and the classical interpretation of the GL modal logic in well-founded transitive Kripke frames.

Our interest is in a \emph{minimal} categorical structure which naturally reflects the abstract structure of the G\"odelian argument. We emphasize that (as opposed to much of the literature on categorical abstractions of guarded recursion), our abstraction does not have \Loeb/'s theorem built into it directly as an assumption, but rather allows \Loeb/'s theorem to be derived from much more basic presumptions. Our abstraction is indeed so simple that it does not even make such common presumptions as cartesian closure, regularity, or coproducts, all of which turn out not to be needed for the derivation of \Loeb/'s theorem. (Indeed, not presuming cartesian closure is vital for allowing our abstraction to cover the initial arithmetic universe!)

The core idea is the identification of those essentially algebraic theories satisfying the property that every model of these theories contains also, as part of its structure, a homomorphism into an internal model of the same theory. We call these \quote{introspective theories}. This document is devoted to initiating the study of introspective theories.

We give two category-theoretic formalizations of the concept of an introspective theory (one directly corresponding to the above description (\magicref{DefnIntrospSN}), the other less so (\magicref{DefnIntrospIndexed})) and prove them equivalent (\magicref{SNCorrespondence}). We then derive a form of \Loeb/'s theorem, in terms of the existence of suitably guarded fixed points, for arbitrary introspective theories (\magicref{IntrospLoeb}). In this demonstration of \Loeb/'s theorem, the relationship between \Loeb/'s theorem and presheaves is also highlighted, including the applicability of \Loeb/'s theorem to non-representable presheaves, which has previously gone unremarked upon.

This derivation of \Loeb/'s theorem for introspective theories is our most important key result. The separate demonstrations of how each of the three traditional instances of \Loeb/'s theorem noted above correspond to certain constructions of introspective theories comprise other key results.

(Specifically, these three traditional instances are seen as instantiations of our abstract theory like so: An introspective theory corresponding to Joyal's work with the initial arithmetic universe is discussed in \magicref{IAUSection}. An introspective theory corresponding to step-indexing in the topos of trees is discussed in \magicref{StepIndexingSection}. Introspective theories corresponding to the classical interpretation of GL modal logic in well-founded transitive Kripke frames are discussed in \magicref{KripkeFrameSection}. These last two constructions are themselves unified and generalized much further in \magicref{ModelsBasedOnPresheafCategories}.)

I believe this is the first formal demonstration of how traditional logical contexts such as the syntactic category of Peano Arithmetic (discussed as an introspective theory at \magicref{ZFFiniteSection}) support guarded recursion not just at the level of propositions (where this amounts to \Loeb/'s theorem in its traditional sense), but also for general terms of arbitrary type, and also for types themselves. Similarly for contexts such as the initial arithmetic universe or the initial topos with natural numbers object (discussed in \magicref{SelfInitializingSection}).

In addition to such traditional finitary logical theories, we give a similar demonstration of the initial topos with countable products as inducing another model of our formal abstraction (in \magicref{ToposWithCountableProductsSection}). As this structure contains both uncomputable and uncountable data, yet is constructed in a very similar way to the traditional logical incarnations of the \Godel/-\Lob/ phenomenon, this should vividly dispel the oft-repeated canard that the \Godel/-\Lob/ phenomenon in logic is fundamentally about or constrained to computability. (As amounts to the same thing, this illustrates that the phenomenon is not constrained to structures internalizable in the initial arithmetic universe).

The concept of an introspective theory is itself essentially algebraic in nature, and thus admits free instances as well, and we give a tractable explicit description of the initial introspective theory in \magicref{GeminalChapter}. This explicit description of the initial introspective theory is another key result of ours. We also observe a remarkable surprising relationship between the initial introspective theory and the theory of introspective theories (\magicref{EveryIntrospModelsInitialIntrospRemark}), and some dual co-free constructions of introspective theories (\magicref{CofreeGeminalSection}).

Though introspective theories are our fundamental objects of interest, along the way, we consider also relaxations of the definition of introspective theories to encompass more general structures (in particular, the relaxation we call \quote{locally introspective theories}, defined at \magicref{DefnLocallyIntrosp}) which, while not supporting the derivation of the \Godel/-\Lob/ phenomena, allow us to state other theorems and constructions in their natural generality and note broader connections with other mathematics.

\sTODOinline{It should be noted that the core ideas of this work were originally developed during my\footnote{It seems suddenly silly to use the otherwise conventional \quote{our}, \quote{we}, etc, for these particular sentences.} time in graduate school from 2006 to 2013 at the University of California, Berkeley, under the advising of Dana Scott, for which I am forever grateful. It was originally developed as the research which was to become my doctoral dissertation, but I did not write it up suitably at that time. Life took its turns and twists, and thankfully I have been able to write this material up now.}

\subsection{Reading roadmap}
The Preliminaries from \magicref{PreliminariesTerminologyConventions} through \magicref{PreliminariesSelfIndexing} cover conventions and material which are used throughout the entire document, which the reader will certainly want to familiarize themselves with. The remainder of the Preliminaries can be read on an as needed basis.

The first chapters \magicref{DefnChapter} and \magicref{ModalChapter} establish the basic concepts of introspective theories, which all later chapters depend on. However, the later chapters \magicref{LoebChapter}, \magicref{GeminalChapter}, and \magicref{ExamplesChapter} can be read essentially independently of each other, in any order or fashion the reader likes. The only dependence between these is that the concept of geminal categories from \magicref{GeminalChapter} is invoked in one isolated section of \magicref{ExamplesChapter}, at \magicref{GeminalSelfInitializingSection}.

\fileend