\filestart

\section{Introduction}
The aim of these notes is to identify and draw attention to a certain simple and categorically natural kind of mathematical structure which both serves as an abstract environment for the reasoning used in establishing \Loeb/'s theorem in its traditional instances, and furthermore allows this and the associated theorems and fixed-point results of the \Goedel/-\Loeb/\ modal logic to be vastly generalized.

Some such \Loeb/-style fixed point phenomena have been explored in the literature before, but these notes aim to highlight a particularly minimal, simple, general abstraction that covers several threads of work in the literature, abstracting both the work on the \Goedel/-\Loeb/\ incompleteness theorems via arithmetic universes a la Joyal, and the work on \Loeb/'s theorem as a guarded fixed point combinator and on guarded (co)inductive types by Birkedal et al.

The core idea of our work is the identification of those essentially algebraic theories satisfying the property that every model of these theories contains also, as part of its structure, a homomorphism into an internal model of the same theory. We call these theories \quote{introspective theories} and their models \quote{geminal categories}.

The structure of an introspective theory turns out to be viewable as a categorification of the Hilbert-Bernays derivability conditions of provability logic. In particular, it induces a bifunctor on each geminal category which is formally similar to the $\Box(A \implies B)$ operator of the \Goedel/-\Loeb/\ modal logic. In our demonstration of \Loeb/'s theorem, the relationship between \Loeb/'s theorem and presheaves is also highlighted, including the applicability of \Loeb/'s theorem to non-representable presheaves, which has previously gone unremarked upon.

We stress that our interest is in the \emph{minimal} categorical structure which naturally reflects the abstract structure of the G\"odelian argument. Hence, we do not in our general theory make even such common presumptions as regularity, cartesian closure, or the existence of coproducts/disjunction, as none of this turns out to be necessary for the derivation of \Loeb/'s theorem. Our original interest in these structures was driven by questions of logic (\quote{What is the minimal categorical structure naturally capturing the abstract structure of \Godel/ codes and their use in \Godel/'s incompleteness theorem? What are the implications for the foundations of mathematics or foundations of logic?}), rather than questions of programming or type theory (\quote{What design of programming language gives good support for writing programs using guarded recursion? How can we semantically reason about such a programming language?}), though the connections are whatever they are.

It is worth noting explicitly the similarity and differences between our structures and the structures explored in existing work on guarded recursion with universes. Roughly speaking, our structure + cartesian closure is equivalent to the typical structure of guarded recursion with a universe $u : U \vdash \El(U) : \mathrm{Type}$ + the novel presumption that the endofunction on $U$ corresponding to the $\later$ operator factors through the global sections map of a $U$-small lexcategory. This $U$-small intermediary may even be taken to be $\later U$ itself. We do not make this correspondence formal but leave its formalization for future work. Thus, relatively speaking, we are both interested in an additional presumption ($U$-smallness) and in the removal of one (cartesian closure).

Adding this $U$-smallness presumption to our theory allows us to \emph{derive} guarded recursion (at both the term and type levels), instead of having to presume guarded recursion by fiat. (Thus, we have also removed a second presumption, while retaining it as a consequence). In this way, our work is notably differentiated from much other literature on guarded recursion. We consider this derivation the primary or most important theorem of this work. (Of course, importance is for others to judge. At any rate, it is the novel result motivating our interest in these structures in the first place.)

This derivation does not depend upon cartesian closure and our dropping of the presumption of cartesian closure from our theory lets it encompass models such as the initial arithmetic universe. Thus we discover a form of guarded recursion within the traditional logical incarnations of \Goedel/'s incompleteness theorem, such as within the initial arithmetic universe (which exhibits \Goedel/'s incompleteness theorem a la Joyal). Similarly we have models within other contexts of a traditional logical flavor, such as connected to the initial topos with a natural numbers object, or within categorical structures corresponding directly to the definable functional relations modulo provable equality of Peano Arithmetic or of ZF. I believe this is the first demonstration of the fact that these logical theories support guarded recursion not just at the level of propositions (where this amounts to \Loeb/'s theorem in its traditional sense), but also for general terms and types.

In addition to such traditional finitary logical theories, we give a similar demonstration of the initial topos with countable coproducts as another model of our theory. As this incarnation of our theory contains both uncomputable and uncountable structure, yet is constructed in a very similar way to the traditional logical incarnations, this should vividly dispel the oft-repeated canard that the \Godel/-\Lob/ phenomenon in logic is fundamentally about or constrained to computability. (As amounts to the same thing, this illustrates the phenomenon is not constrained to structures internalizable in the initial arithmetic universe).

Our axiomatic framework also admits presheaf models, such as those already studied for general recursion, but admitting a wider variety of presheaf-based models than those as well. In keeping with provability logic but as differentiated from much existing work on guarded recursion, our models are not forced to generally validate or have an inhabitant of the type $\neg \neg \Box A$ for arbitrary $A$. In this way, our models support interpretation of Boolean provability logic, not requiring us to restrict ourselves to intuitionistic logic for nontriviality.

Our abstract framework also admits free instances, and we establish the nature of these free instances in some detail. In so doing, the distinction and yet also close relationship between categorical models of provability logic which do and don't validate $X \vdash \Box X$ in general is formally clarified (this amounting to the relationship between introspective theories and geminal categories). In particular, this allows us to show that every theorem applying in general to the internal categories of introspective theories applies just as well to introspective theories themselves.

Though introspective theories are our fundamental objects of interest, along the way, we consider also relaxations of the definition of introspective theories to encompass more general structures (such as we call \quote{pre-introspective theories} or \quote{locally introspective finite product theories}) which, while not supporting the derivation of the \Godel/-\Lob/ phenomena, allow us to state other theorems and constructions in their natural generality and note broader connections with other mathematics.

Finally, as a fun application of our theory, we consider the implications for a form of material set theory whose comprehension schema resembles unrestricted comprehension except guarded by modal operators; we call this \quote{modally naive material set theory}. Our theory shows us how to model this as a fixed point of $X = \Box \mathcal{P}(X)$ and thus proves the consistency of this theory (even within Boolean logic, using our aforementioned models of Boolean provability logic, taking us beyond what could be demonstrated with existing models of guarded recursion). In such a modally naive set theory, attempts to establish inconsistency via Russell's paradox instead only establish the sentence of modal logic corresponding to \Godel/'s second incompleteness theorem. We demonstrate also how this theory simultaneously satisfies modalized versions of both the Axiom of Foundations and its ostensible opposite, Aczel's Axiom of Anti-Foundation.

It should be noted that the core of this work was originally developed during my\footnote{It seems suddenly silly to use the otherwise conventional \quote{our}, \quote{we}, etc, for this one particular sentence.} time in graduate school from 2006 to 2013, as the research which was to become my doctoral dissertation, but I did not write it up suitably at that time. Though an explosion of other work has been published on guarded recursion and on arithmetic universes since then, it is hoped that this unification of those two strands remains a strong contribution of novel results and insight. Where we do touch upon ground that has already been covered by others, it is hoped that at least the clean and simple exposition our framework allows is valuable.

\fileend