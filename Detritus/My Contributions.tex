Alas, in the time since I originally had these ideas, a lot of other people have come to also think about Loeb as a modalized fixed point combinator along the lines of Lawvere's theorem. Still, the following remain my original contributions beyond what is already out there as of now, I think:

But I think what I have that other people don't have is realizing fully the relationship between Loeb's theorem and FUNCTORIAL fixed points (including the ability to generate the latter from the former), realizing the value in this context of internal categories and not merely enriched categories, realizing the clear distinction and yet relationships between introspective theories and GL-categories, and realizing the full simplicity to which the context for all this Goedelian phenomena can be reduced (that is, the simplicity of the latest definition of an introspective theory).

Fret fret fret. It's only a matter of time. I better get it done.

Oh, and I also have the Loeb argument for the uniqueness of fixed points and the initiality/terminality properties, that no one else has right now, right. I list these so I can keep track for myself, when I need to pitch the thing, write an intro or an abstract or whatever, of what the original contributions now are beyond what other people have already done.

And perhaps the naive set theory thing I can do is the clearest illustration of something I can illustrate in my framework that no one else seems to have thought about, because it makes such use of functorial fixed points.