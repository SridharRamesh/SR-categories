\section{Modal logic}
\subsection{The box operator}
The following notation will be very convenient for us going forward. It is also suggestive of connections with modal logic we will eventually explore.

Let $\langle T, C \rangle$ be a locally introspective theory. \TODOinline{Allow this notation for pre-introspective theories too?}

We say a presheaf on $C$ is locally $T$-small if the map from its category of elements to $C$ has $T$-small fibers. In other words, $P(c)$ is represented by an object of $T/t$ for each $t$-definable object $c$ of $C$.

Note that the category of locally $T$-small presheaves on $C$ is itself a $T$-indexed lexcategory. We will refer to this as $\Psh{C}$.

Thus, we have three $T$-indexed lexcategories of note: $T$ itself (considered as a $T$-indexed category through the self-indexing $T/-$), $C$, and $\Psh{C}$.

Between these, we also have a cycle of $T$-indexed lexfunctors, like so:

\[\begin{tikzcd}
	&& {T/-} \\
	\\
	C &&&& {\Psh{C}}
	\arrow["\introF"', from=1-3, to=3-1]
	\arrow["{c \; \mapsto \Hom_{C}(-, c)}"', from=3-1, to=3-5]
	\arrow["{P \mapsto P(1)}"', from=3-5, to=1-3]
\end{tikzcd}\]

Here, the bottom arrow is the Yoneda embedding, sending each object of $C$ to the corresponding representable presheaf. The right arrow takes a presheaf to its global elements. The left arrow is the $\introF$ which is part of the structure of an introspective theory.

In general, we will write $\Box$ for a roundtrip around this diagram, starting from any of its three nodes.

Thus, we will write $\Box$ for the $T$-indexed lexfunctor from $T$ to itself given by $t \mapsto \Hom_C(1, \introF(t))$.

We will ALSO write $\Box$ for the $T$-indexed lexfunctor from $C$ to itself given by $c \mapsto \introF(\Hom_C(1, c))$.

And we will ALSO write $\Box$ for the $T$-indexed lexfunctor from $\Psh{C}$ to $\Psh{C}$, which sends the presheaf $P$ to the presheaf represented by $\introF(P(1))$.

When we want to clarify precisely the domain we are operating on, we may write $\Box_T$, $\Box_C$, or $\Box_{\Psh{C}}$, as appropriate.

As the Yoneda embedding is naturally thought of as the inclusion of a full subcategory, identifying $C$ with the corresponding representable presheaves within $\Psh{C}$, we may also think of this last instance of $\Box$ as a $T$-indexed lexfunctor from $\Psh{C}$ to $C$.

Note that, having set up these various notions of $\Box$, we find that $\Box$ and each of the maps in the diagram \quote{commutes} in the appropriate sense; that is, they can be seen as preserving each other.

For example, $\introF$ preserves $\Box$, in that both $\introF(\Box(-))$ and $\Box(\introF(-))$ yield the same $T$-indexed functor from the self-indexing of $T$ to $C$. This is readily seen by unwinding their definitions: These are both $\introF(\Hom_C(1, \introF(-)))$.

We also have that taking $\Hom_C(1, -)$ preserves $\Box$ in the same way. $\Hom_C(1, \Box(-)) = \Box(\Hom_C(1, -)) = \Hom_C(1, \introF(\Hom_C(1, -)))$. More generally, consider the map $G$ which assigns to every locally $T$-small presheaf $P$ upon $C$ its object of global elements $P(1)$. Then $G$ preserves $\Box$ in the same way: $G(\Box P) = \Box(G(P)) = \Hom_C(1, \introF(P(1)))$, where we make sense of $G(\Box P)$ by identifying $\Box P$ within $c$ with the presheaf it represents.

Given object $c$ of $C$, and locally $T$-small presheaf $P$ on $C$, we will write $c \implies P$ to indicate the exponential presheaf $P^c$. That is, the presheaf $P(- \times c)$. We may also write $c \implies d$ where $d$ is an object of $C$ as well, to mean $c \implies P$ for the presheaf $P = \Hom_C(-, d)$ represented by $d$. Thus, $c \implies d$ is the presheaf $\Hom_C(- \times c, d)$.

Note that for any locally $T$-small presheaf $P$ on $C$ and object $c$ of $C$, we have that $P(c)$ can be identified with the presheaf $c \implies P$ evaluated at $1$, and thus $\introF(P(c))$ can be identified with $\Box(c \implies P)$. In particular, by considering the case when $P$ is represented by object $d$, we find that $\Box(c \implies d)$ is naturally identifiable with $\introF(\Hom_C(c, d))$.

The above was all discussed for $T$, $C$, and $\Psh{C}$ considered as $T$-indexed lexcategories, but this all descends to corresponding structure on their global aspects as well. Keep in mind, in the global aspect context, $\introF$ is the same as $\introS$, so wherever in the above we discussed $\introF$, a corresponding statement holds as well for $\introS$, when considering just the global aspect.

The choice of this $\Box$ notation for these purposes is meant to convey an analogy with the $\Box$ operator of modal logic, and in particular, with the provability operator of provability logic. We will explore this more in later remarks.

The key point here is that the rules of the $\Box$ operator in a Kripke normal modal logic are essentially the rules of a lex endofunctor on a category, and any of our $\Box$ operators is certainly lex as a composite of lex functors.

Furthermore, each of our $\Box$ operators comes with a natural transformation from $\Box$ to $\Box \Box$ corresponding to the so-called 4 axiom $\Box A \vdash \Box \Box A$.

For the $\Box$ operator on $T$ this is clear, as the natural transformation $\introN$ from identity to $\Box$ encodes the even stronger property $t \vdash \Box t$. The 4 axiom is the special case where $t$ here is itself of the form $\Box A$.

We do not have such a strong natural transformation from identity to $\Box$ as acting on the other corners of the triangle ($C$ or $\Psh{C}$). However, by taking the natural transformation from $t$ to $\Box_T t$, whiskering it on both sides along the trips from any other corner of the triangle into and out of $T$, and then applying the commutativity of $\Box$ with each leg of the triangle, we get a natural transformation from $\Box$ to $\Box \Box$ at each other corner of the triangle as well.

Thus, our $\Box$ follows all the rules of the modal logic K4, in each of these contexts. Later, we shall see that conversely, the general logic followed by $\Box$ in a locally introspective theory is no stronger than K4, while in an introspective theory, it is the modal logic GL. Indeed, in the very next chapter we will see how in an introspective theory we get the last ingredient for the modal logic GL, \Loeb's theorem.