What to do next:

1. Write out T-indexed definition of introspective theories
DONE

2. Write out proof that slices categories of introspective theories are introspective theories, using above definition.
DONE (on one account; another account should still be written up)

3. Write out proof that this is equivalent to S, N definition (formalized not lexly).
DONE

4. Write out proof that six axiom definition of GL categories is the initial introspective theory. [And more generally for GL-Xes]

5. Write out a particular lex formalization of S, N definition and discuss the issues involved in doing so; i.e., "strict" introspective theories.

6. Write out proof that every strict introspective theory is a GL-category.

Definitely won't get that all done in a week, but whatever.

Then:
7. Back to categorical Löb's theorem. Things that are needed here: Deriving P(P(1)) |- P(1) from X nearly (but not necessarily exactly) a fixed point match for P(X) [P here is some presheaf such as [](- -> A)], constructing the near-fixed point needed to bootstrap this all and get the conclusion unconditionally, demonstrating that fixed points for functions that respect isos are unique, demonstrating that our fixed points for functors are initial algebras and terminal coalgebras.

8. Writing out all the examples of models in Models.tex.

9. Writing out relationship to GL modal logic (K4 is easy, and then 6 gives Löb's theorem; the fact that the logic we get doesn't go beyond GL modal logic I don't know off the top of my head a good clean way to see directly. It would be nice. We can derive it from the PA model and known relationship between the PA model and GL logic). The GLS modal logic can also be discussed, and the relationship between GL logic, GLS logic, GL categories, and introspective theories [there is also a logic GLT, with A |- []A as proper internalizable axiom |- [](A -> []A), but where unmodalized negation and implication are not around as operators; this should correspond to introspective theories].

...

100. Write out naive set theory thing.
----
Speculative:

Write out expository "Related work" thing about how in some sense, what we are focusing is the orthogonal complement to all the work that demonstrates the details of how arithmetic universes contain internal initial models of finitely specified lex theories: That gives us Gödel codes in a specific context, but we are focused on what can be done in any abstract context where we have Gödel codes. The two of these facts combine to tell us how arithmetic universes have the Gödel phenomenonon, but they are complementary legs. The part we are working on, people understand non-abstractly, but we wish to show the abstraction, to emphasize the bare minimum of machinery involved (lex categories, not even regular categories, much less coproducts, lists, etc) and also to bring out the categorical aspects of Löb's theorem (fixed points for presheaves, these needn't be proposition-valued, we get initial algebras and terminal coalgebras and they coincide, etc). Our free models demonstrate that the initiality is not core to GL phenomena, just one instance where it arises. Our free models allow us to consider working with this axiomatization taken as given, abstractly.

Speculatively, we should be able to make an introspective theory or GL-category like the Sigma-1 or full PA models that has instead some uncomputable flavor, like doing the same thing over some oracle, and this should give the kind of model that people do not ordinarily consider, showing some value to the abstraction. [Not that people aren't aware of how Tarski's indefinability theorem applies over all these things too, but, yeah]

Perhaps we can turn any semilattice model of GL modal logic into a corresponding GL-category (one where this semilattice appears as the subobjects of 1 or as the preorder reflection of the whole category), and similarly for GLT and introspective theories? That'd be great. Instead of just focusing on the subobjects of 1, we should also automatically get some corresponding fact for the full space by then using reg/lex completion or ex/lex completion. It will be much easier to at least show that any semilattice model of GL embeds into some introspective theory's subobjects, which suffices to show that GL is the appropriate modal logic for introspective theories.

The right result of the above sort should be that given any GLS lattice T' and GL lattice C', with a meet and [] preserving homomorphism from the former to the latter, we can create a corresponding <T, C, ...> introspective theory.

In T-indexed world, it's like there are large sets (actual sets, arbitrary presheaves), and small sets (objects of T, representable presheaves). All arguments are just like that. An enriched category is like a large but locally small category, an internal category is like a small category. Many of our arguments do not depend on the smallness of C, just its local smallness.

***

Talk about free monad extending [].

Talk about naive set theory.

Talk about how models needn't be computable, or arithmetically definable, or countable, or any such thing. We can add an arbitrary function from N to N or such things to a standard theory and still get another introspective theory.

Sigma-1 construction can be seen as simple thing followed by Ex/Lex construction.

Needn't have an NNO in models, I think. At any rate, the arithmetic content is clearly low, since arithmetic is not visible in the theory of GL-categories in any direct way.

Write out non-lex 2-theory of introspective theories.