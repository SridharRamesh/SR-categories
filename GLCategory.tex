\filestart

\section{Geminal categories: The free introspective theory}\label{GeminalChapter}
\subsection{Preview}
In this chapter, we build the machinery to give an explicit yet tractably compact description of the initial introspective theory (which we call the theory of \quote{geminal categories}). This is the key result of this chapter.

We also show the remarkable result that any strict introspective theory can itself be equipped in a natural way as a model of this initial introspective theory; that is, any strict introspective theory can be seen as a geminal category.

(This last statement is easy to misinterpret, so let me be a bit more clear as to what I mean by this. I do not mean the trivial statement that every introspective theory extends the initial introspective theory. Rather, I mean that the theory of strict introspective theories extends the initial introspective theory (even though the theory of strict introspective theories is not itself an introspective theory).)

We will also discuss a partial converse of sorts, a way to extract an introspective theory from a geminal category, with the extracted introspective theory having a certain terminality property (that is, we construct a sort of co-free introspective theory induced by the given geminal category).

This chapter requires some preliminary concepts to be established in \magicref{MultiplyInternal} and \magicref{StrictIntrospSection}. The basic definitions concerning geminal categories are then given in \magicref{GeminalFirstDefnSection} through \magicref{GeminalSecondDefnSection}. After all this machinery has been built, the key result that the theory of geminal categories is in fact the initial introspective theory is ultimately demonstrated in \magicref{InitialIntrospectiveTheorySection}. We then discuss co-free constructions in \magicref{CofreeGeminalSection}.

\subsection{Multiply internal structures}\label{MultiplyInternal}
Before we get to the main material of this chapter, it will be helpful to introduce the concept of \quote{multiply internal} structures, which are used heavily throughout this chapter.

\sTODOinline{Relate the following to the $\cartwith{\theoryT}$ operation. An (n+1)-tuply internal model of $\theoryT$ is an internal model of $\cartwith{}^n T$.}

Let $C_0$ be a lexcategory, and let $C_1$ be the global aspect of a lexcategory internal to $C_0$. Now suppose given some structure $S$ internal to $C_1$. We may say that this structure $S$ is \defined{doubly internal} to $C_0$.

We may iterate this process. Suppose now that $C_2$ is the global aspect of some lexcategory internal to $C_1$, which in turn remains the global aspect of some lexcategory internal to $C_0$. We can now speak of structures internal to $C_2$ as being triply internal to $C_0$.

And in general, given a sequence $C_0, C_1, C_2, \ldots, C_n$ where each $C_{i + 1}$ is the global aspect of a lexcategory internal to $C_i$, we may speak of structures internal to $C_n$ as being $(n + 1)$-tuply internal to $C_0$ (and in the same way $n$-tuply internal to $C_1$, $(n - 1)$-tuply internal to $C_2$, and so on). That is, we recursively define an $(n + 1)$-tuply internal structure as a structure internal to the global aspect of an $n$-tuply internal lexcategory, with the base case being that the only $0$-tuply internal lexcategory of some $C$ is $C$ itself.

Next, observe that whenever $C$ is a lexcategory and $D$ is a $C$-indexed locally \repsmall/ lexcategory, the global sections functor $\Hom_D(1, -)$ can be seen as an indexed lexfunctor from $D$ to the self-indexing $C/-$; in particular, the global aspect of this lets us see $\Hom_D(1, -)$ as a lexfunctor from the global aspect of $D$ to $C$ itself. Let us write $\Gamma_D : \Glob{D} \to C$ to refer to this last lexfunctor, or drop the subscript and write simply $\Gamma$ where there is no need to disambiguate which $D$ we are referencing.

(In particular, when writing $\Gamma(S)$ with no subscript on the $\Gamma$, we always mean $\Gamma_X(S)$ where $S$ is singly internal to $X$, though $X$ may in turn be internal or multiply internal to some other category.)

Thus, if $S$ is some structure internal to the global aspect of $D$, we find that $\Gamma_D(S)$ is a structure of the same sort internal to $C$. In this way, any doubly-internal structure $S$ yields a singly-internal structure $\Gamma(S)$, and more generally, any $(n + 1)$-tuply internal structure $S$ yields an $n$-tuply internal structure $\Gamma(S)$.

By applying this process $m$ times in a row, we can turn any $(n + m)$-tuply internal structure $S$ into a corresponding $n$-tuply internal structure $\Gamma^m(S)$, where $\Gamma^m$ is slight abuse of notation for applying $m$ different global sections functors in a row. \sTODOinline{Clarify that n here can't be 0. Or rather, for $n = 0$, we have to understand $0$-tuply internal structures as structures internal to $\Set$.}

\begin{observation}\label{GlobOfGlob}
If $C$ is a lexcategory, $D$ is a $C$-indexed locally \repsmall/ lexcategory, and $S$ is a $\Glob{D}$-internal structure, then the global aspect of $S$ is the same as the global aspect of $\Gamma(S)$. \sTODOinline{Write out why this is so}
\end{observation}

Indeed, if $S$ is internal to some lexcategory $C$, we can think of $C$ as itself internal to $\Set$, and thus $S$ as doubly internal to $\Set$, and then $\Gamma(S)$ is itself the global aspect of $S$. More generally, if $S$ is $n$-tuply internal to some lexcategory $C$, we can in the same way think of $S$ as $(n + 1)$-tuply internal to $\Set$, and then the global aspect of $S$ is given by $\Gamma^n(S)$.

Next, observe that if $F : C \to D$ is a lexfunctor between lexcategories, and $S$ is some structure internal to $C$, then not only do we have $F(S)$ as a similar structure internal to $D$, but furthermore, the action of $F$ induces a homomorphism from any $c$-indexed aspect of $S$ to the $F(c)$-indexed aspect of $F(S)$. In particular, there is an induced homomorphism from the global aspect of $S$ to the global aspect of $F(S)$. We may use the same name $F$ to refer to this induced homomorphism $F : \Glob{S} \to \Glob{F(S)}$ as well, or we may refer to it as $\InducedHomo{F}{S} : \Glob{S} \to \Glob{F(S)}$ to be explicit.

In particular, if $S$ is an internal lexcategory, then our original lexfunctor $F : C \to D$ induces also a lexfunctor $F : \Glob{S} \to \Glob{F(S)}$. If in turn, we also have some lexcategory $S_2$ internal to $\Glob{S}$, then again in the same way, we have an induced lexfunctor from $\Glob{S_2}$ to $\Glob{F(S_2)}$. And again, in convenient abuse of notation, we may refer to this induced lexfunctor as $F : \Glob{S_2} \to \Glob{F(S_2)}$ as well, or we may refer to it as $\InducedHomo{F}{S}$ to be explicit.

The upshot of all this is that a lexfunctor $F : C \to D$ induces also for each $n$-tuply internal structure $S$ in $C$, also an $n$-tuply internal structure $F(S)$ in $D$, as well as a homomorphism $\InducedHomo{F}{S} : \Glob{S} \to \Glob{F(S)}$, all in a suitably coherent way.

This process can be carried out in the internal logic of a lexcategory as well. That is, if $F : C \to D$ is an internal lexfunctor between internal lexcategories, and $S$ is some multiply internal structure in $C$, we get an internal homomorphism $\InducedHomo{F}{S} : \Gamma(S) \to \Gamma(F(S))$ in the same way.

\begin{lemma}\label{InducedGlobalCommute}
If $F : C \to D$ is a strict lexfunctor, and $Q$ is a $C$-internal lexcategory, then $F \circ \Gamma_Q = \Gamma_{F(Q)} \circ \InducedHomo{F}{Q}$. That is to say, the following outer diagram commutes, as evidenced by the inner chase of an arbitrary datum $m$ in $\Glob{Q}$:

% https://q.uiver.app/#q=WzAsOCxbMCwwLCJcXEdsb2J7UX0iXSxbMywwLCJcXEdsb2J7RihRKX0iXSxbMCwzLCJDIl0sWzMsMywiRCJdLFsxLDEsIm0iXSxbMiwxLCJGKG0pIl0sWzEsMiwiXFxIb21fUSgxLCBtKSJdLFsyLDIsIkYoXFxIb21fUSgxLCBtKSkgPSBcXEhvbV97RihRKX0oMSwgRihtKSkiXSxbMCwxLCJcXEluZHVjZWRIb21ve0Z9e1F9Il0sWzAsMiwiXFxHYW1tYV9RIiwyXSxbMSwzLCJcXEdhbW1hX3tGKFEpfSJdLFs0LDYsIiIsMix7InN0eWxlIjp7InRhaWwiOnsibmFtZSI6Im1hcHMgdG8ifX19XSxbNCw1LCIiLDAseyJzdHlsZSI6eyJ0YWlsIjp7Im5hbWUiOiJtYXBzIHRvIn19fV0sWzUsNywiIiwwLHsic3R5bGUiOnsidGFpbCI6eyJuYW1lIjoibWFwcyB0byJ9fX1dLFs2LDcsIiIsMix7InN0eWxlIjp7InRhaWwiOnsibmFtZSI6Im1hcHMgdG8ifX19XSxbMiwzLCJGIiwyXV0=
\[\begin{tikzcd}
	{\Glob{Q}} &&& {\Glob{F(Q)}} \\
	& m & {F(m)} \\
	& {\Hom_Q(1, m)} & {F(\Hom_Q(1, m)) = \Hom_{F(Q)}(1, F(m))} \\
	C &&& D
	\arrow["{\InducedHomo{F}{Q}}", from=1-1, to=1-4]
	\arrow["{\Gamma_Q}"', from=1-1, to=4-1]
	\arrow["{\Gamma_{F(Q)}}", from=1-4, to=4-4]
	\arrow[maps to, from=2-2, to=3-2]
	\arrow[maps to, from=2-2, to=2-3]
	\arrow[maps to, from=2-3, to=3-3]
	\arrow[maps to, from=3-2, to=3-3]
	\arrow["F"', from=4-1, to=4-4]
\end{tikzcd}\]
\end{lemma}

Sometimes, we write $F[S]$ instead of $F(S)$ to make clear that we are not simply using parentheses to group compositions of functors, but rather, specifically mean the application of $F$ to an internal or multiply internal structure $S$.

Multiply internal structures can equivalently be thought of as multiply indexed structures (in the sense of \magicref{PreliminariesMultipleIndexing}) satisfying suitable \repsmallness/ conditions, but they are probably more easily understood in the presentation just given.

\subsection{Strict introspective theories}\label{StrictIntrospSection}
It will be technically convenient for us to work in this chapter with a slightly less \quote{presentation-free} variant of our notion of introspective theories.

\begin{definition}\label{StrictIntrospDefn}
A \defined{strict introspective theory} is a strict lexcategory $T$, a lexcategory $C$ internal to $T$, a strict lexfunctor $\introS$ from $T$ to the global aspect of $C$, and a natural transformation $\introN$ from $\id_T$ to $\Hom_C(1, \introS(-))$.
\end{definition}

As usual, to name a strict introspective theory, we can enumerate the entire ordered tuple $\langle T, C, \introS, \introN \rangle$, or sometimes we just note $\langle T, C \rangle$ or $T$ explicitly and leave the rest implicit.

The definition of a strict introspective theory differs from the definition of an ordinary introspective theory (\magicref{DefnIntrospSN}) in the following ways: $T$ is made strict (thus, its internal structures can be considered up to equality instead of mere isomorphism), we demand the selection of $C$ as a particular $T$-internal lexcategory up to equality (instead of simply up to presenting equivalent indexed categories), and we take $\introS$ as a strict lexfunctor (thus, $\introS$ preserves chosen basis limits on-the-nose).

Clearly, any strict introspective theory presents some introspective theory. Conversely, we have the following:

\begin{theorem}\label{StrictifyingIntrosp}
Any introspective theory $\langle T, C, \introS, \introN \rangle$ is presented by some strict introspective theory.
\end{theorem}
\begin{proof}
Suppose given an introspective theory $\langle T, C, \introS, \introN \rangle$. By definition of the \repsmallness/ of $C$, we can choose some lexcategory $C_{int}$ internal to $T$ which presents the $T$-indexed category $C$. (That is, even though $C$ itself is only specified up to equivalence of indexed categories, we can choose a specific presentation of it by a \repsmall/ indexed strict category $C_{int}$ which is specified  more fine-grainedly up to isomorphism of indexed strict categories.)

Now using \magicref{FreeStrictifyLexcategory}, let $T_{strict}$ be some strict lexcategory which presents $T$ and which has the freeness property that any lexfunctor from $T$ to a strict lexcategory $L$ is presented by some strict lexfunctor from $T_{strict}$ to $L$. Because $T_{strict}$ presents $T$, we can choose some specific internal lexcategory $C_{strict}$ in $T_{strict}$ (this $C_{strict}$ being specified up to equality!) which presents $C_{int}$. Because $C_{strict}$ presents $C_{int}$ which in turn presents $C$, $\introS$ can be viewed as a (non-strict) lexfunctor from $T$ to the global aspect of $C_{strict}$. Now using the freeness property of $T_{strict}$, we obtain a strict lexfunctor $\introS_{strict}$ from $T_{strict}$ to the global aspect of $C_{strict}$, such that $\introS_{strict}$ presents $\introS$.

Finally, we deal with $\introN$. Natural transformations are essentially unaffected by strictness considerations. That is, given parallel strict functors $A_{strict}$ and $B_{strict}$, natural transformations between these are in bijection with natural transformations between the non-strict functors these present. So our original $\introN$ corresponds to a unique natural transformation between the identity on $T_{strict}$ and $\Hom_{C_{strict}}(1, \introS_{strict}(-))$.

Thus, we have obtained a strict introspective theory $\langle T_{strict}, C_{strict}, \introS_{strict}, \introN \rangle$ presenting the introspective theory $\langle T, C, \introS, \introN \rangle$.
\end{proof}

Strict introspective theories are slightly more convenient than introspective theories for phrasing the results of this chapter, because strict introspective theories are themselves an essentially algebraic notion. That is, there is an essentially algebraic theory such that the models of this theory are the strict introspective theories. (This is in precisely the same way that the theory of strict categories is essentially algebraic, while the theory of categories construed up to equivalence is not quite essentially algebraic.)

As with any essentially algebraic theory, we get automatically a corresponding notion of homomorphism.

\begin{definition}\label{StrictIntrospHomoDefn}
A \defined{homomorphism} between strict introspective theories $\langle T_1, C_1, \introS, \introN \rangle$ and $\langle T_2, C_2, \introS, \introN \rangle$ is a strict lexfunctor $H : T_1 \to T_2$ such that $H[C_1] = C_2$, $H[\introN_t] = \introN_{H(t)}$ for each object $t$ of $T_1$ (i.e., the whiskerings of $H$ along the two $\introN$s match), and the following diagram commutes:

% https://q.uiver.app/?q=WzAsNCxbMCwwLCJUXzEiXSxbMiwwLCJUXzIiXSxbMCwxLCJcXEdsb2J7Q18xfSJdLFsyLDEsIlxcR2xvYntDXzJ9Il0sWzAsMSwiSCJdLFswLDIsIlxcaW50cm9TIiwyXSxbMiwzLCJcXEluZHVjZWRIb21ve0h9e0NfMX0iLDJdLFsxLDMsIlxcaW50cm9TIl1d
\[\begin{tikzcd}
	{T_1} && {T_2} \\
	{\Glob{C_1}} && {\Glob{C_2}}
	\arrow["H", from=1-1, to=1-3]
	\arrow["\introS"', from=1-1, to=2-1]
	\arrow["{\InducedHomo{H}{C_1}}"', from=2-1, to=2-3]
	\arrow["\introS", from=1-3, to=2-3]
\end{tikzcd}\]

(In the above diagram, the bottom arrow is using the notation from \magicref{MultiplyInternal} to indicate the action of $H$ as a homomorphism from the global aspect of the $T_1$-internal structure $C_1$, to the global aspect of the corresponding $T_2$-internal structure $H[C_1] = C_2$.)
\end{definition}

Such homomorphisms are closed under composition and thus we obtain the category of strict introspective theories.

As the category of models of an essentially algebraic theory, this category must have an initial object. That is, there is a strict introspective theory with a unique homomorphism into any other strict introspective theory. In this chapter, we will find a tractable explicit description of this initial strict introspective theory.

\subsection{Defining geminal categories}\label{GeminalFirstDefnSection}
\sTODOinline{Note that throughout the following, essentially nothing about the actual lex structure of lexcategories specifically matters. All that matters is the concepts of multiply internal structures. This would all work just as well for other \quote{tree-categories}.}

\begin{definition}[Geminal category]\label{VerboseGeminalCatDefn}
A \defined{geminal category}\footnote{Another evocative name for this concept might be \quote{nesting doll category}.} internal to lexcategory $C_0$ consists of several ingredients:

\begin{itemize}
    \item 
    The first ingredient is an infinite sequence $C_1, C_2, C_3, \ldots$, in which each $C_i$ (for $i \geq 1$) is the global aspect of a lexcategory $C'_i$ internal to $C_{i - 1}$.
\end{itemize}

Thus, each $C'_{i + n}$ is $n$-tuply internal to $C_i$.

(Throughout the following, it will be useful to keep in mind that we are using these general naming habits: Primed names are used for internal structures, while the corresponding unprimed names indicate the corresponding global aspects. Furthermore, names subscripted with index $i$ arise from structure internal to $C_{i - 1}$.)

\begin{itemize}
    \item
    The second ingredient comprising a geminal category is an infinite sequence of internal lexfunctors $F'_1, F'_2, F'_3, \ldots$, where each $F'_i : C'_i \to \Gamma_{C'_i}[C'_{i + 1}]$ is internal to $C_{i - 1}$ (for $i \geq 1$).
\end{itemize}

Pictorially, this can be envisioned like so: 

% https://q.uiver.app/#q=WzAsMTEsWzEsMCwiQydfMSJdLFsyLDAsIlxcR2FtbWFfMVtDJ18yXSJdLFsyLDEsIkMnXzIiXSxbMywxLCJcXEdhbW1hXzJbQydfM10iXSxbMCwwLCJDXzA6Il0sWzAsMSwiQ18xOiJdLFswLDIsIkNfMjoiXSxbMywyLCJDJ18zIl0sWzQsMiwiXFxHYW1tYV8zW0MnXzRdIl0sWzAsMywiXFxsZG90cyJdLFs0LDMsIlxcbGRvdHMiXSxbMCwxLCJGJ18xIl0sWzIsMywiRidfMiJdLFs3LDgsIkYnXzMiXSxbNSw0LCJcXEdhbW1hXzEiXSxbNiw1LCJcXEdhbW1hXzIiXSxbOSw2LCJcXEdhbW1hXzMiXV0=
\[\begin{tikzcd}
	{C_0:} & {C'_1} & {\Gamma_1[C'_2]} \\
	{C_1:} && {C'_2} & {\Gamma_2[C'_3]} \\
	{C_2:} &&& {C'_3} & {\Gamma_3[C'_4]} \\
	\ldots &&&& \ldots
	\arrow["{F'_1}", from=1-2, to=1-3]
	\arrow["{F'_2}", from=2-3, to=2-4]
	\arrow["{F'_3}", from=3-4, to=3-5]
	\arrow["{\Gamma_1}", from=2-1, to=1-1]
	\arrow["{\Gamma_2}", from=3-1, to=2-1]
	\arrow["{\Gamma_3}", from=4-1, to=3-1]
\end{tikzcd}\]

Here, the first row is structure internal to $C_0$, the second row is structure internal to $C_1$ (thus, doubly internal to the ambient $C_0$), the third row is structure internal to $C_2$ (thus, triply internal to the ambient $C_0$), and so on. We also for convenience use the abbreviation $\Gamma_i$ for $\Gamma_{C'_i} : C_i \to C_{i - 1}$ for $i \geq 1$, illustrating these in the vertical line on the left of the picture.

In keeping with our naming convention, we shall also use $F_i : C_i \to C_{i + 1}$ to refer to the global aspect of $F'_i$. Note that the codomain of $F_i$ is simply $C_{i + 1}$, as \magicref{GlobOfGlob} tells us that $\Glob{\Gamma[C'_{i + 1}]} = \Glob{C'_{i + 1}} = C_{i + 1}$. Thus, these $F_i$ line up more straightforwardly than the $F'_i$:

% https://q.uiver.app/?q=WzAsNCxbMCwwLCJDXzEiXSxbMSwwLCJDXzIiXSxbMiwwLCJDXzMiXSxbMywwLCJcXGxkb3RzIl0sWzAsMSwiRl8xIl0sWzEsMiwiRl8yIl0sWzIsMywiRl8zIl1d
\[\begin{tikzcd}
	{C_1} & {C_2} & {C_3} & \ldots
	\arrow["{F_1}", from=1-1, to=1-2]
	\arrow["{F_2}", from=1-2, to=1-3]
	\arrow["{F_3}", from=1-3, to=1-4]
\end{tikzcd}\]

Finally, the last ingredients we require are some equations:

\begin{itemize}
    \item 
     We require that $F_i[C'_j] = C'_{j + 1}$ and $F_i[F'_j] = F'_{j + 1}$ for $j > i \geq 1$.
     
    \item
    Furthermore, we require that the following diagram of lexfunctors internal to $C_{i - 1}$ commutes, for each $i \geq 1$. We call this equation $E_i$.
    
% https://q.uiver.app/#q=WzAsNCxbMCwwLCJDJ19pIl0sWzAsMiwiXFxHYW1tYV9pIFtDJ197aSArIDF9XSJdLFsyLDIsIlxcR2FtbWFfaSBbXFxHYW1tYV97aSArIDF9IFtDJ197aSArIDJ9XV0iXSxbMiwwLCJcXEdhbW1hX2kgW0MnX3tpICsgMX1dIl0sWzAsMSwiRidfaSIsMl0sWzAsMywiRidfaSJdLFszLDIsIlxcR2FtbWFfaSBbRidfe2kgKyAxfV0iXSxbMSwyLCJcXEluZHVjZWRIb21ve0YnX2l9e0MnX3tpICsgMX19IiwyXV0=
\[\begin{tikzcd}
	{C'_i} && {\Gamma_i [C'_{i + 1}]} \\
	\\
	{\Gamma_i [C'_{i + 1}]} && {\Gamma_i [\Gamma_{i + 1} [C'_{i + 2}]]}
	\arrow["{F'_i}"', from=1-1, to=3-1]
	\arrow["{F'_i}", from=1-1, to=1-3]
	\arrow["{\Gamma_i [F'_{i + 1}]}", from=1-3, to=3-3]
	\arrow["{\InducedHomo{F'_i}{C'_{i + 1}}}"', from=3-1, to=3-3]
\end{tikzcd}\]

That is, we require that $\InducedHomo{F'_i}{C'_{i + 1}} \circ F'_i = \Gamma_i[F'_{i + 1}] \circ F'_i$. This could be glossed as \quote{$F'_i \circ F'_i = F'_{i + 1} \circ F'_i$}, in severe abuse of notation.
\end{itemize}

This concludes the definition of a geminal category internal to $C_0$.
\end{definition}

By a \defined{geminal category} simpliciter, we mean of course the case where $C_0 = \Set$. (Note that in this case, $C'_1$ can be identified with its global aspect $C_1$, in the same way that any structure internal to $\Set$ can be identified with its global aspect, as the global elements functor from $\Set$ to $\Set$ is the identity.). We wrote out here the definition for general $C_0$, instead of specifically for $C_0 = \Set$, in order to emphasize certain symmetries in this definition.

When being fully explicit, we reference a geminal category by enumerating $\langle C'_1, C'_2, C'_3, \ldots; F'_1, F'_2, F'_3, \ldots \rangle$. Given such a geminal category $K$, we may write $\underlying{K}$ to refer to its underlying lexcategory $C'_1$.

All aforementioned structure apart from $C_0$ itself has been given as $i$-tuply internal to $C_0$ for some $i > 0$. Thus, all of this structure is indeed given by diagrams within $C_0$.

Indeed, this definition of geminal category is manifestly essentially algebraic. That is, there is an essentially algebraic theory such that models of that theory internal to $C_0$ are the same thing as geminal categories internal to $C_0$.

Our ultimate goal will be to show that this theory of geminal categories is the initial introspective theory. This is the whole motivation for our study of geminal categories. But to show this result, we must develop some other machinery first.

\subsection{Geminal category homomorphisms}
As geminal categories are defined by an essentially algebraic theory, we automatically get a notion of homomorphism between geminal categories. It amounts to the following:

\begin{definition}\label{VerboseGeminalCatHomoDefn}
Given two geminal categories $\langle C'_1, C'_2, C'_3, \ldots; F'_1, F'_2, F'_3, \ldots \rangle$ and $\langle D'_1, D'_2, $ $D'_3, \ldots; \phi'_1, \phi'_2, \phi'_3, \ldots \rangle$, a \defined{homomorphism} from the former to the latter consists of a strict lexfunctor $H : C'_1 \to D'_1$ such that $H[C'_i] = D'_i$ and $H[F'_i] = \phi'_i$ for each $i > 1$, while also the following diagram commutes:

% https://q.uiver.app/?q=WzAsNCxbMCwwLCJDJ18xIl0sWzIsMCwiRCdfMSJdLFsyLDIsIlxcR2xvYntIW0MnXzJdfSA9IFxcR2xvYntEXzInfSJdLFswLDIsIlxcR2xvYntDXzInfSJdLFswLDEsIkgiXSxbMCwzLCJGJ18xIiwyXSxbMywyLCJcXEluZHVjZWRIb21ve0h9e0MnXzJ9IiwyXSxbMSwyLCJcXHBoaSdfMSJdXQ==
\[\begin{tikzcd}
	{C'_1} && {D'_1} \\
	\\
	{\Glob{C_2'}} && {\Glob{H[C'_2]} = \Glob{D_2'}}
	\arrow["H", from=1-1, to=1-3]
	\arrow["{F'_1}"', from=1-1, to=3-1]
	\arrow["{\InducedHomo{H}{C'_2}}"', from=3-1, to=3-3]
	\arrow["{\phi'_1}", from=1-3, to=3-3]
\end{tikzcd}\]
\end{definition}

\begin{theorem}\label{GeminalContainsGeminal}
Given any geminal category $K = \langle C'_1, C'_2, C'_3, \ldots; F'_1, F'_2, F'_3, \ldots \rangle$, we have also that $\langle C'_2, C'_3, C'_4, \ldots; F'_2, F'_3, F'_4 \ldots \rangle$ comprises a geminal category internal to $\underlying{K} = C'_1$. We refer to this internal geminal category as $\InteriorGeminal{K}$.

We furthermore have that $F'_1$ acts as a geminal category homomorphism from $K$ to the global aspect of $\InteriorGeminal{K}$. We refer to this homomorphism as $\IntoSelf{K} : K \to \Gamma[\InteriorGeminal{K}]$.
\end{theorem}
\begin{proof}
This is all direct by definition.

For the first part, each condition imposed upon each $C'_{i}$ or $F'_{i}$ in the definition of a geminal category comes with an analogous condition imposed upon $C'_{i + 1}$ or $F'_{i + 1}$. Thus, it is immediate that the given $\InteriorGeminal{K}$ satisfies the conditions to be a geminal category internal to $\underlying{K}$.

For the second part, the definition of a geminal category directly imposes upon $F'_1$ precisely the conditions which are necessary for $F'_1$ to comprise a geminal category homomorphism from $K$ to the global aspect of $\InteriorGeminal{K}$. In particular, equation $E_1$ from \magicref{VerboseGeminalCatDefn} is identical to the commutative diagram from \magicref{VerboseGeminalCatHomoDefn}, in this context.
\end{proof}

Via the yoga of functorial semantics, \magicref{GeminalContainsGeminal} states how the theory of geminal categories can be equipped as an introspective theory. In detail, this is given like so:

\begin{construction}\label{GLCatTheoryIsIntrosp}
Let $\GLCatTheory$ be the free strict lexcategory with an internal geminal category (that is, in the terminology of \magicref{QuasiTheoryTheory}, we take $\GLCatTheory$ to be the classifying strict lexcategory $\classifying{\theoryT}$, where $\theoryT$ is the theory of geminal categories). 

Thus, strict lexfunctors from $\GLCatTheory$ to any other strict lexcategory $D$ correspond to geminal categories internal to $D$, while natural transformations between such lexfunctors correspond to homomorphisms between these $D$-internal geminal categories.

Let $K$ denote the $\GLCatTheory$-internal geminal category corresponding to the identity functor on $\GLCatTheory$.

By \magicref{GeminalContainsGeminal} in the internal logic of $\GLCatTheory$, we obtain also a geminal category $\InteriorGeminal{K}$ internal to $\underlying{K}$, as well as a homomorphism $\IntoSelf{K} : K \to \Gamma[\InteriorGeminal{K}]$.

Thus, there is some strict lexfunctor $\introS$ from $\GLCatTheory$ to the global aspect of $\underlying{K}$, corresponding to $\InteriorGeminal{K}$. Furthermore, there is some natural transformation $\introN$ from the identity functor on $\GLCatTheory$ to $\Hom_{\underlying{K}}(1, \introS(-))$, corresponding to $\IntoSelf{K}$.

Putting this together, we have a strict introspective theory $\langle \GLCatTheory, \underlying{K}, \introS, \introN \rangle$.
\end{construction}

\sTODOinline{Note that there is a more limited analogue of the above, where we observe that the free X with an internal geminal Y is itself a geminal X, whenever Y extends X and X extends the notion of a lexcategory. The difficulty with turning this into an introspective theory is that the property we really depend on from the free lexcategory $L$ with an internal gadget of some sort is not just the 1-categorical property that it has a unique homomorphism to every other gadget, but the 2-categorical property that the category of homomorphisms from it to another lexcategory and natural transformations between those is equivalent to the 1-category of internal gadgets within that codomain lexcategory. This is true for lexcategories, but not necessarily for other doctrines, and I believe this is related to how $\Hom(1, -)$ is always a lexfunctor (thus turning internal models into genuine models,) but not always an X-functor. The subtle role of this in the above proof should be highlighted.}

\subsection{Compactly defined geminal categories}\label{GeminalSecondDefnSection}
The above all amounts to an infinitary presentation of the theory of geminal categories. For this reason, we call it the \quote{verbose presentation} of geminal categories. However, it turns out this same theory can be finitely axiomatized as well.

\begin{definition}[Geminal category, compact presentation]\label{CompactGeminalCatDefn}
A \defined{compactly presented geminal category} internal to lexcategory $C_0$ consists of the structure $C'_i$, $F'_i$, and equations $E_i$ of the verbose presentation, but only for $i \in \{1, 2\}$.

(Here, in interpreting the codomain of $F'_2$, we take $C'_3$ to be $F_1[C'_2]$, and in interpreting the equation $E_2$, we take $F'_3$ to be $F_1[F'_2]$)

That is, a compactly presented geminal category internal to $C_0$ consists of the following six pieces of data:

\begin{itemize}
    \item A lexcategory $C'_1$ internal to $C_0$, whose global aspect we call $C_1$. We refer to $\Gamma_{C'_1} : C_1 \to C_0$ as $\Gamma_1$.
    
    \item A lexcategory $C'_2$ internal to $C_1$, whose global aspect we call $C_2$. We refer to $\Gamma_{C'_2} : C_2 \to C_1$ as $\Gamma_2$.
    
    \item A lexfunctor $F'_1 : C'_1 \to \Gamma_1[C'_2]$, internal to $C_0$. We call the global aspect of this $F_1 : C_1 \to C_2$.
    
    \item A lexfunctor $F'_2 : C'_2 \to \Gamma_2[C'_3]$, internal to $C_1$.
    
    (Here, $C'_3$ is defined as $F_1[C'_2]$.)
    
    \item The equation $\InducedHomo{F'_1}{C'_{2}} \circ F'_1 = \Gamma[F'_2] \circ F'_1$, internal to $C_0$. We call this equation $E_1$.
    
    \item The equation $\InducedHomo{F'_2}{C'_{3}} \circ F'_2 = \Gamma[F'_3] \circ F'_2$, internal to $C_1$. We call this equation $E_2$.
    
    (Here, $F'_3$ is defined as $F_1[F'_2]$.)
\end{itemize}
\end{definition}

As usual, we reference a compactly presented geminal category by enumerating the ordered tuple $\langle C'_1, C'_2; F'_1, F'_2 \rangle$.

Clearly, the structure defining a compactly presented geminal category is part of the structure in our verbose definition of a geminal category. But in fact, these are equivalent definitions.

\begin{theorem}\label{GeminalCompactIsVerbose}
The structure of a compactly presented geminal category uniquely determines the further structure of a geminal category (as originally defined in \magicref{VerboseGeminalCatDefn})).
\end{theorem}
\begin{proof}
By definition, in a geminal category, we must have that $F_1[C'_j] = C'_{j + 1}$ and $F_1[F'_j] = F'_{j + 1}$ for each $j > 2$.

Accordingly, if we are given the structure in \magicref{CompactGeminalCatDefn}, and we are to extend it to all the further structure in \magicref{VerboseGeminalCatDefn}, it inductively must be the case that $C'_j = F_1^{j - 2}[C'_2]$ and $F'_j = F_1^{j - 2}[F'_2]$ for each $j > 2$. Adopt these definitions throughout the following accordingly.

The equations given to us directly in the compact presentation are the equations $E_1$ and $E_2$ of the verbose presentation. Furthermore, we obtain the equation $E_i$ for $i > 2$ by applying $F_1^{n - 1}$ to $E_2$.

What remains is only to see that each $F_i$ for $i > 1$ also takes $C'_j$ to $C'_{j + 1}$ and takes $F'_j$ to $F'_{j + 1}$, for $j > i \geq 1$.

We prove this by induction on $i$. For the base case of $i = 1$, we have ensured this by construction. As for the inductive step, suppose we know this already holds for $i$. Then $F_{i + 1}[C'_j] = F_{i + 1} [F_i [C'_{j - 1}]] = F_i [F_i [C'_{j - 1}]] = F_i [C'_j] = C'_{j + 1}$, where the second step is by $E_i$ and the other steps are by our induction hypothesis. And similarly with $F'$ in place of $C'$ throughout as well.
\end{proof}

\begin{corollary}\label{CompactGeminalCatHomoDefn}
In \magicref{VerboseGeminalCatHomoDefn}, the conditions $H[C'_i] = D'_i$ and $H[F'_i] = \phi'_i$ automatically follow for all $i > 2$ once they hold for $i = 2$.
\end{corollary}

Thus, we can go back and forth between thinking of geminal categories in either the verbose or compact presentation as we please, whichever is most convenient at any moment.

\subsection{Geminal categories from introspective theories}

\begin{construction}\label{IntrospAsGeminal}
From a strict introspective theory $\langle T, C, \introS, \introN \rangle$, we obtain a geminal category $\langle T, C; \introS, \introN_{C} \rangle$, whose underlying lexcategory is $T$. This is the canonical way to view an introspective theory as a geminal category.
\end{construction}
\begin{proof}
It is immediate in the definition of a strict introspective theory that $C$ is a lexcategory internal to $T$, and $\introS$ is a lexfunctor from $T$ to $\Glob{C}$. This gives us the first three out of the six ingredients of \magicref{CompactGeminalCatDefn}.

As for $\introN_{C}$, meaning the components of the natural transformation $\introN$ at the objects of $C$, this gives us a $T$-internal lexfunctor from $C$ to $\Hom_C(1, \introS[C]) = \Gamma[\introS[C]]$. This is the fourth ingredient of \magicref{CompactGeminalCatDefn}.

What remains are to verify equations $E_1$ and $E_2$. In this context, $E_1$ is a special case of \magicref{SMatchesN}, while $E_2$ is given by the naturality of $\introN$ with respect to the components of $\introN_C$ themselves.

This completes the construction. We observe furthermore that strict introspective theory homomorphisms are automatically geminal category homomorphisms between the geminal categories obtained by this construction.
\end{proof}

There is another closely related construction which is of even more importance:

\begin{construction}\label{IntrospContainsGeminal}
From a strict introspective theory $\langle T, C, \introS, \introN \rangle$, we obtain a $T$-internal geminal category $\langle C, \introS[C]; \introN_{C}, \introS[\introN_{C}] \rangle$, whose underlying lexcategory is $C$.
\end{construction}
\begin{proof}
This is the result of first obtaining the geminal category $\gamma = \langle T, C, \introS, \introN_C \rangle$ from \magicref{IntrospAsGeminal}, and then forming $\InteriorGeminal{\gamma}$.
\end{proof}

We now are ready to prove our main result about geminal categories.

\subsection{The free introspective theory}\label{InitialIntrospectiveTheorySection}
\begin{theorem}\label{InitialIntrospectiveTheory}
The strict introspective theory given in \magicref{GLCatTheoryIsIntrosp} is the initial strict introspective theory.
\end{theorem}
\begin{proof}
We must show there is a unique homomorphism from the strict introspective theory $\langle \GLCatTheory, K \rangle$ of \magicref{GLCatTheoryIsIntrosp} to any other strict introspective theory $\langle T, D \rangle$.

Such a homomorphism is comprised of a strict lexfunctor $H : \GLCatTheory \to T$ satisfying certain conditions. By the nature of $\GLCatTheory$, this amounts to a geminal category $\langle D'_1, D'_2, D'_3, \ldots; F'_1, F'_2, F'_3, \ldots \rangle$ internal to $T$ satisfying certain conditions.

One particular geminal category internal to $T$ is the one that is given by $\gamma = \langle D, \introS[D]; \introN_D, \introS[\introN_D] \rangle$, as noted at \magicref{IntrospContainsGeminal}. In verbose terms, this geminal category is $\langle D, \introS[D], \introS[\introS[D]], \ldots;$ $ \introN_D, \introS[\introN_D], \introS[\introS[\introN_D]], \ldots \rangle$, with each successive component being $\introS$ applied to the previous component.

What remains is to show that the lexfunctor $H : \GLCatTheory \to T$ corresponding to this $\gamma$ uniquely satisfies the conditions of \magicref{StrictIntrospHomoDefn}.

The condition \quote{$H[C_1] = C_2$} in \magicref{StrictIntrospHomoDefn} says in this context that we must use a geminal category whose underlying lexcategory is $D$.

The condition concerning $\introN$ in \magicref{StrictIntrospHomoDefn}, along with the definition of $\introN$ in \magicref{GLCatTheoryIsIntrosp}, says that we must use a geminal category whose first lexfunctor component is $\introN_{D}$.

Finally, the commutative diagram concerning $\introS$ in \magicref{StrictIntrospHomoDefn}, along with the definition of $\introS$ in \magicref{GLCatTheoryIsIntrosp}, says we must use a geminal category such that each successive component of this geminal category is $\introS$ applied to the previous component.

The conjunction of these conditions clearly is uniquely satisfied by $\gamma$. This completes the proof.
\end{proof}

\begin{observation}\label{EveryIntrospModelsInitialIntrospRemark}
Given the result of \magicref{InitialIntrospectiveTheory}, we can rephrase \magicref{IntrospAsGeminal} as telling us that every strict introspective theory is a model of the initial introspective theory, so to speak. In other words, there is a lexfunctor interpreting the initial introspective theory into the theory of strict introspective theories. This is quite remarkable!
\end{observation}

\subsection{Geminal gadgets}
We have now successfully described the initial introspective theory. But we can also take our free construction results a bit further than this.

Specifically, every introspective theory is, among other things, an essentially algebraic theory extending the theory of strict lexcategories. That is, we have a functor from the category of introspective theories to the category of extensions of the essentially algebraic theory of strict lexcategories (essentially, this functor takes $\langle T, C, \introS, \introN \rangle$ to $\langle T, C \rangle$). This functor has a left adjoint.

Put in other words, for any essentially algebraic theory $Th$ such that models of $Th$ come with an underlying strict lexcategory, there is a free strict introspective theory $\langle T, C, \introS, \introN \rangle$ with a designated $T$-internal model of $Th$ with underlying lexcategory $C$.

For simplicity as a first introduction, everything done previously was the special case where $Th$ was simply the theory of strict lexcategories itself. But now we describe the more general results, which follow by almost exactly the same reasoning as used before:

Specifically, let models of $Th$ be called \quote{gadgets}, and maps between them called \quote{gadget homomorphisms}. Then the free introspective theory extending $Th$ is the theory of \quote{geminal gadgets}, with the definition of a \quote{geminal gadget} being exactly as in either definition of a \quote{geminal category}, but with all instances of lexcategories and lexfunctors replaced by gadgets and gadget homomorphisms.

This is by exactly the same arguments as we have just given. All the results and arguments given earlier in this chapter apply just as well mutatis mutandis when lexcategories and lexfunctors are replaced by gadgets and gadget homomorphisms, except for \magicref{IntrospAsGeminal} (it will not be the case that an arbitrary strict introspective theory can be viewed as a geminal gadget). However, the analogue of the construction \magicref{IntrospContainsGeminal} still holds (i.e., given an introspective theory $\langle T, C \rangle$ such that $C$ is the underlying lexcategory of a gadget, then the geminal category structure which $C$ is equipped with by \magicref{IntrospContainsGeminal} furthermore underlies geminal gadget structure).

\sTODOinline{Perhaps also discuss the straightforward notion of a non-strict geminal category or gadget: One for which $C_1$ is a non-strict lexcategory or gadget, and $F_1$ needn't be strict either (preserves finite limits but not necessarily on the nose), but everything else remains strict. Every non-strict geminal gadget straightforwardly admits a presentation by a strict one, by using a presentation of $C_1$ with no nontrivial equations on objects in a suitable sense.}

\sTODOinline{
\subsection{Modal logic in geminal categories}
\begin{observation}
In every introspective theory $\langle T, C \rangle$, we have a $T$-internal endofunctor $\Box_C$ on $C$ modeling the modal logic GL. Thus, this is in particular true of the initial introspective theory, which by \magicref{InitialIntrospectiveTheory} is the theory of geminal categories. Thus, any geminal category $K$ comes with such an operation $\Box_{|K|}$ on $|K|$, and this operation is furthermore preserved by geminal category homomorphisms.
\end{observation}

\sTODOinline{Guide readers by showing how we have the axioms of GL modal logic in a geminal category, but do NOT have A |- []A. Weave our archetypal examples into here.}
}

\subsection{Co-free introspective theories and geminal categories}\label{CofreeGeminalSection}

We have above discussed how to create free introspective theories, which can be thought of as produced by a certain left adjoint functor\footnote{Specifically, left adjoint to the forgetful functor from strict introspective theories to strict lexcategories with a designated internal lexcategory.}. In this section, we discuss some right adjoint constructions, which can be thought of as \quote{co-free}.

\begin{construction}
Construing strict introspective theories as geminal categories via \magicref{IntrospAsGeminal} gives us a functor from the category of strict introspective theories to the category of geminal categories (or more generally, a functor from the category of $V$-internal strict introspective theories to the category of $V$-internal geminal categories, for any fixed lexcategory $V$). This functor has a right adjoint.

As this works for arbitrary lexcategories $V$, this right adjoint admits an explicit description, as a purely lex construction.
\end{construction}
\openDetails
For linguistic convenience, we shall in the following take $V$ as $\Set$, but it will be clear that the same explicit construction works for any ambient lexcategory $V$.

We are tasked with showing that, for any geminal category $C_1$, there is a suitably terminal introspective theory with a geminal category homomorphism to $C_1$.

Let $C_1 = \langle C_1, C'_2; F_1, F'_2 \rangle$ be a geminal category, let $T = \langle T, C, \introS, \introN \rangle$ be a strict introspective theory (which we can also construe as a geminal category via \magicref{IntrospAsGeminal}), and let $H : T \to C_1$ be a geminal category homomorphism.

Let us use the suggestive name $\Box_{C_1}$ for the endolexfunctor $\Gamma_{C'_2} \circ F_1 : C_1 \to C_1$.

By virtue of $H$ being a geminal category homomorphism, we have that $H \circ \Box_T = \Box_{C_1} \circ H$. In detail, this is seen via the following commutative diagram:

% https://q.uiver.app/?q=WzAsNixbMywwLCJDXzEiXSxbMywxLCJcXEdsb2J7QydfMn0iXSxbMywyLCJDXzEiXSxbMCwwLCJUIl0sWzAsMSwiXFxHbG9ie0N9Il0sWzAsMiwiVCJdLFswLDEsIkZfMSIsMl0sWzEsMiwiXFxHYW1tYV97QydfMn0iLDJdLFswLDIsIlxcQm94X3tDXzF9IiwwLHsiY3VydmUiOi01fV0sWzMsMCwiSCJdLFszLDQsIlxcaW50cm9TIl0sWzQsMSwiXFxJbmR1Y2VkSG9tb3tIfXtDfSIsMV0sWzQsNSwiXFxHYW1tYV9DIl0sWzUsMiwiSCIsMl0sWzMsNSwiXFxCb3hfVCIsMix7ImN1cnZlIjo1fV1d
\[\begin{tikzcd}
	T &&& {C_1} \\
	{\Glob{C}} &&& {\Glob{C'_2}} \\
	T &&& {C_1}
	\arrow["{F_1}"', from=1-4, to=2-4]
	\arrow["{\Gamma_{C'_2}}"', from=2-4, to=3-4]
	\arrow["{\Box_{C_1}}", curve={height=-30pt}, from=1-4, to=3-4]
	\arrow["H", from=1-1, to=1-4]
	\arrow["\introS", from=1-1, to=2-1]
	\arrow["{\InducedHomo{H}{C}}"{description}, from=2-1, to=2-4]
	\arrow["{\Gamma_C}", from=2-1, to=3-1]
	\arrow["H"', from=3-1, to=3-4]
	\arrow["{\Box_T}"', curve={height=30pt}, from=1-1, to=3-1]
\end{tikzcd}\]

In the above diagram, the left side is the definition of $\Box_T$ and the right side is the definition of $\Box_{C_1}$. The top rectangle is one of the conditions in \magicref{VerboseGeminalCatHomoDefn} and the bottom rectangle is by \magicref{InducedGlobalCommute}.

Thus, the whiskering of $\introN : \id_T \to \Box_T$ along $H$ yields a natural transformation from $H$ to $H \circ \Box_T = \Box_{C_1} \circ H$. Illustrated like so:

% https://q.uiver.app/?q=WzAsNixbNCwwLCJDXzEiXSxbNCwxLCJcXEdsb2J7QydfMn0iXSxbNCwyLCJDXzEiXSxbMCwwLCJUIl0sWzEsMSwiXFxHbG9ie0N9Il0sWzAsMiwiVCJdLFswLDEsIkZfMSIsMl0sWzEsMiwiXFxHYW1tYV97QydfMn0iLDJdLFswLDIsIlxcQm94X3tDXzF9IiwwLHsiY3VydmUiOi01fV0sWzMsMCwiSCJdLFszLDQsIlxcaW50cm9TIl0sWzQsMSwiXFxJbmR1Y2VkSG9tb3tIfXtDfSIsMV0sWzQsNSwiXFxHYW1tYV9DIl0sWzUsMiwiSCIsMl0sWzMsNSwiXFxpZCIsMix7ImxldmVsIjoyLCJzdHlsZSI6eyJoZWFkIjp7Im5hbWUiOiJub25lIn19fV0sWzE0LDQsIlxcaW50cm9OIiwyLHsic2hvcnRlbiI6eyJzb3VyY2UiOjIwfX1dXQ==
\[\begin{tikzcd}
	T &&&& {C_1} \\
	& {\Glob{C}} &&& {\Glob{C'_2}} \\
	T &&&& {C_1}
	\arrow["{F_1}"', from=1-5, to=2-5]
	\arrow["{\Gamma_{C'_2}}"', from=2-5, to=3-5]
	\arrow["{\Box_{C_1}}", curve={height=-30pt}, from=1-5, to=3-5]
	\arrow["H", from=1-1, to=1-5]
	\arrow["\introS", from=1-1, to=2-2]
	\arrow["{\InducedHomo{H}{C}}"{description}, from=2-2, to=2-5]
	\arrow["{\Gamma_C}", from=2-2, to=3-1]
	\arrow["H"', from=3-1, to=3-5]
	\arrow[""{name=0, anchor=center, inner sep=0}, "\id"', Rightarrow, no head, from=1-1, to=3-1]
	\arrow["\introN"', shorten <=4pt, Rightarrow, from=0, to=2-2]
\end{tikzcd}\]

This natural transformation from $H$ to $\Box_{C_1} \circ H$ acts as a functor $\beta$ from $T$ to the category of $\Box_{C_1}$-coalgebras, such that $\beta$ followed by the forgetful functor to $C_1$ yields $H$. As $H$ is a strict lexfunctor, this $\beta$ is also a strict lexfunctor, when the category of $\Box_{C_1}$-coalgebras is construed as a strict lexcategory in the manner of \magicref{BoxCoalgebrasInGeminal}.

Not only that, but every coalgebra $m : c \to \Box_{C_1} c$ in the range of $\beta$ has the property that the following diagram commutes \TODO:

% https://q.uiver.app/?q=WzAsNCxbMCwwLCJjIl0sWzAsMiwiXFxCb3ggYyJdLFsyLDAsIlxcQm94IGMiXSxbMiwyLCJcXEJveF4yIGMiXSxbMCwxLCJtIiwyXSxbMCwyLCJtIl0sWzIsMywiXFxCb3ggbSJdLFsxLDMsIjRfYyIsMl1d
\[\begin{tikzcd}
	c && {\Box c} \\
	\\
	{\Box c} && {\Box^2 c}
	\arrow["m"', from=1-1, to=3-1]
	\arrow["m", from=1-1, to=1-3]
	\arrow["{\Box m}", from=1-3, to=3-3]
	\arrow["{4_c}"', from=3-1, to=3-3]
\end{tikzcd}\]

\sTODOinline{Explain above diagram}

\TODOinline{Write out details of how, by restricting to coalgebras for $\Box_{C'_1}$ with the suitable property, we get an introspective theory, which is terminal among all introspective theories which map into this geminal category}

Let $K$ be the full subcategory of $\Box_{C_1}$ coalgebras with the specified property. It is readily verified that $K$ is closed under the chosen basic limits of $\Box_{C_1}$ \TODOinline{As an equalizer between strict lexfunctors}, and thus $K$ can be seen as a full sub-strict-lexcategory of the $\Box_{C_1}$ coalgebras.

Note that the diagram defining $C'_2$ as an internal lexcategory in $C_1$ is in fact a diagram within the subcategory $K$, as \TODOinline{Cite relevant part of definition of a geminal category}. Thus, we may consider $C'_2$ as a $K$-\repsmall/ lexcategory as well.

Define the strict lexfunctor $\introS : K \to \Glob{C'_2}$ by composing the forgetful functor from $K$ to $C_1$ with $F_1 : C_1 \to \Glob{C'_2}$.

Note that $\Hom_{C_1}(1, \introS(-)) : K \to K$ takes each $m : c \to \Box_{C_1} c$ in $K$ to $4_c : \Box_{C_1} c \to \Box_{C_1} \Box_{C_1} c$, by \TODO.

Finally, define the natural transformation $\introN$ as sending each $m : c \to \Box_{C_1} c$ in $K$ to the following coalgebra map from $m$ to $\Box_{C_1} m$ (the coalgebra map whose underlying map in $C_1$ is $m$ itself). This square commutes because of the defining property of $K$: \TODO.

Thus, we obtain a strict introspective theory $K = \langle K, C'_2, \introS, \introN \rangle$.

Let the strict lexfunctor $k$ be the inclusion of $K$ into the $\Box_{C_1}$ coalgebras, followed by the forgetful functor from the $\Box_{C_1}$ coalgebras to $C_1$. Observe that $k$ acts as a geminal category homomorphism $: K \to C_1$, by \TODO.

Note that the construction of $K$ and $k$ depends only upon the geminal category $C_1$. Furthermore, observe for any strict introspective theory $T$ and geminal category homomorphism $H: T \to C_1$ as above, the $\beta$ as described above becomes the unique strict introspective theory homomorphism from $T$ to $K$ such that $\beta H = k$, by \TODO. 

Thus, the operation turning $C_1$ into $K$ is the right adjoint to the functor turning strict introspective theories into geminal categories.
\closeDetails

\begin{corollary}
As left adjoints preserve initial objects, the above tells us that the initial strict introspective theory (which we described in \magicref{InitialIntrospectiveTheory}) is also the initial geminal category.
\end{corollary}

\begin{construction}
Consider the functor from the category of geminal categories to the category of strict lexcategories with a designated internal geminal category, given by sending any geminal category $G$ to the strict lexcategory $|G|$ with internal geminal category $\InteriorGeminal{G}$. This functor has a right adjoint.

In other words, for any strict lexcategory $C_0$ with an internal geminal category $\gamma$, there is a geminal category $G$ equipped with a strict lexfunctor $H : \underlying{G} \to C_0$ satisfying $H[\InteriorGeminal{G}] = \gamma$, which is terminal among all so equipped geminal categories (in the sense that for any other such geminal category $K$ with strict lexfunctor $J : \underlying{K} \to C_0$ satisfying $J[\InteriorGeminal{K}] = \gamma$, there is a unique geminal category homomorphism $M : K \to G$ such that $H \circ M = J$).

This co-free $G$ admits an explicit description as $C_0 \times \Glob{\underlying{\gamma}}$ suitably equipped. (Indeed, just as before, this whole construction can be carried out internal to an arbitrary ambient lexcategory.)
\end{construction}
\openDetails
Let $\gamma = \langle C'_1, C'_2, C'_3, \ldots; F'_1, F'_2, F'_3, \ldots \rangle$. Throughout the following, let unprimed names denote global aspects of primed names, in our usual fashion. For convenience, we will also say simply \quote{lexfunctor} in the following to mean \quote{strict lexfunctor}.

Let $G_0 = C_0 \times C_1$. We have that $\gamma \times \InteriorGeminal{\gamma}$ is a geminal category $\langle G'_1, G'_2, G'_3, \ldots; \phi'_1, \phi'_2, \phi'_3, \ldots \rangle$ internal to $G_0$, with each $G'_n = C'_n \times C'_{n + 1}$ and $\phi'_n = F'_n \times F'_{n + 1}$, for $n \geq 1$.

Let lexfunctor $\phi_0 : G_0 \to G_1$ be defined by $\phi_0(c_0, c_1) = (c_1, F_1(c_1))$. It's straightforward to then verify that $G = \langle G_0, G'_1, G'_2, \ldots; \phi_0, \phi'_1, \phi'_2, \ldots \rangle$ is a geminal category (with the only nontrivial aspect being the equation $E_0$, as it were). \sTODOinline{Verify the trivial bits to be sure they're trivial, and the nontrivial bit to be sure it's also obvious enough to not need further comment}

We also clearly have a projection lexfunctor $H$ from $\underlying{G} = C_0 \times C_1$ to $C_0$, and by unfolding definitions, this does indeed satisfy $H[\InteriorGeminal{G}] = \gamma$.
\closeDetails
Having described the construction's details, we now prove that this construction has the stated terminality property:

\begin{proof}
Suppose given any geminal category $K = \langle K_0, K'_1, K'_2, \ldots; P_0, P'_1, P'_2, \ldots \rangle$ and lexfunctor $J : K_0 \to C_0$ such that $J[\InteriorGeminal{K}] = \gamma$. As ever, we will use unprimed names for the global aspects of primed names.

A lexfunctor $M$ from $\underlying{K} = K_0$ to $\underlying{G} = C_0 \times C_1$ is given by a pair of lexfunctors $J_0 : K_0 \to C_0$ and $J_1 : K_0 \to C_1$. Since $H$ is simply projection of the $C_0$ component, we will have that $H \circ M = J$ precisely when $J_0 = J$. Thus, specifying such $M$ is given by specifying $J_1$ alone. We must show that there is a unique $J_1$ making this $M$ into a geminal category homomorphism from $K$ to $G$.

Keeping in mind \magicref{VerboseGeminalCatHomoDefn}, adapted to this context, we see the conditions for such $M$ to be a geminal category homomorphism. First of all, we must have that $M[\InteriorGeminal{K}] = \InteriorGeminal{G}$, which is to say, $J[\InteriorGeminal{K}] = \gamma$ (which has already been presumed) and $J_1[\InteriorGeminal{K}] = \InteriorGeminal{\gamma}$. On top of this, the final condition for $M$ to be a geminal category homomorphism is that the following diagram commutes:

% https://q.uiver.app/?q=WzAsNCxbMCwwLCJLXzAiXSxbMiwwLCJDXzAgXFx0aW1lcyBDXzEiXSxbMiwyLCJDXzEgXFx0aW1lcyBDXzIiXSxbMCwyLCJLXzEiXSxbMCwxLCJNID0gXFxsYW5nbGUgSiwgSl8xIFxccmFuZ2xlIl0sWzAsMywiUF8wIiwyXSxbMywyLCJcXEluZHVjZWRIb21ve019e0snXzF9IiwyXSxbMSwyLCJcXHBoaV8wIFxcOyAgW1xcdGV4dHt3aGljaCBpc30gXFw7IChjXzAsIGNfMSkgXFxtYXBzdG8gKGNfMSwgRl8xKGNfMSkpXSJdXQ==
\[\begin{tikzcd}
	{K_0} && {C_0 \times C_1} \\
	\\
	{K_1} && {C_1 \times C_2}
	\arrow["{M = \langle J, J_1 \rangle}", from=1-1, to=1-3]
	\arrow["{P_0}"', from=1-1, to=3-1]
	\arrow["{\InducedHomo{M}{K'_1}}"', from=3-1, to=3-3]
	\arrow["{\phi_0 \;  [\text{which is} \; (c_0, c_1) \mapsto (c_1, F_1(c_1))]}", from=1-3, to=3-3]
\end{tikzcd}\]

This diagram commutes just in case both of the following diagrams commute, which separately consider its $C_1$ and $C_2$ components:

% https://q.uiver.app/?q=WzAsNCxbMCwwLCJLXzAiXSxbMiwwLCJDXzAgXFx0aW1lcyBDXzEiXSxbMiwyLCJDXzEiXSxbMCwyLCJLXzEiXSxbMCwxLCJNID0gXFxsYW5nbGUgSiwgSl8xIFxccmFuZ2xlIl0sWzAsMywiUF8wIiwyXSxbMywyLCJcXEluZHVjZWRIb21ve0p9e0snXzF9IiwyXSxbMSwyLCIoY18wLCBjXzEpIFxcbWFwc3RvIGNfMSJdLFswLDIsIkpfMSIsMV1d
\[\begin{tikzcd}
	{K_0} && {C_0 \times C_1} \\
	\\
	{K_1} && {C_1}
	\arrow["{M = \langle J, J_1 \rangle}", from=1-1, to=1-3]
	\arrow["{P_0}"', from=1-1, to=3-1]
	\arrow["{\InducedHomo{J}{K'_1}}"', from=3-1, to=3-3]
	\arrow["{(c_0, c_1) \mapsto c_1}", from=1-3, to=3-3]
	\arrow["{J_1}"{description}, from=1-1, to=3-3]
\end{tikzcd}\]

% https://q.uiver.app/?q=WzAsNSxbMCwwLCJLXzAiXSxbMiwwLCJDXzAgXFx0aW1lcyBDXzEiXSxbMiwyLCJDXzIiXSxbMCwyLCJLXzEiXSxbMSwxLCJDXzEiXSxbMCwxLCJNID0gXFxsYW5nbGUgSiwgSl8xIFxccmFuZ2xlIl0sWzAsMywiUF8wIiwyXSxbMywyLCJcXEluZHVjZWRIb21ve0pfMX17SydfMX0iLDJdLFsxLDIsIihjXzAsIGNfMSkgXFxtYXBzdG8gRl8xKGNfMSkiXSxbMCw0LCJKXzEiLDFdLFs0LDIsIkZfMSIsMV1d
\[\begin{tikzcd}
	{K_0} && {C_0 \times C_1} \\
	& {C_1} \\
	{K_1} && {C_2}
	\arrow["{M = \langle J, J_1 \rangle}", from=1-1, to=1-3]
	\arrow["{P_0}"', from=1-1, to=3-1]
	\arrow["{\InducedHomo{J_1}{K'_1}}"', from=3-1, to=3-3]
	\arrow["{(c_0, c_1) \mapsto F_1(c_1)}", from=1-3, to=3-3]
	\arrow["{J_1}"{description}, from=1-1, to=2-2]
	\arrow["{F_1}"{description}, from=2-2, to=3-3]
\end{tikzcd}\]

In each of the above diagrams, the top-right triangle trivially commutes, so the commutativity condition for the overall square is equivalent to the commutativity of the bottom-left triangle.

From the diagram for the $C_1$ component, we see that $J_1$ is uniquely determined as $\InducedHomo{J}{K'_1} \circ P_0$. All that remains is to verify that this choice of $J_1$ does indeed satisfy the condition $J_1[\InteriorGeminal{K}] = \InteriorGeminal{\gamma}$, as well as the condition of the commutative diagram for the $C_2$ component.

\bigskip
For the former condition, we have the chain of equations $J_1[\InteriorGeminal{K}]$ 

$ = \InducedHomo{J}{K'_1} [P_0[\InteriorGeminal{K}]] $ 

$= \InducedHomo{J}{K'_1} [\InteriorGeminal{\InteriorGeminal{K}}] $ 

$= \InteriorGeminal{J[\InteriorGeminal{K}]} $ 

$= \InteriorGeminal{\gamma}$. \sTODOinline{Are we sure all these equations are correct?}

\bigskip
And as for the final commutativity condition, this follows like so:

% https://q.uiver.app/?q=WzAsNixbMCwwLCJLXzAiXSxbMCwyLCJLXzEiXSxbNSwwLCJDXzEiXSxbNSwyLCJDXzIiXSxbMiwwLCJLXzEiXSxbMiwyLCJLXzIiXSxbMCwxLCJQXzAiLDJdLFsyLDMsIkZfMSJdLFswLDQsIlBfMCIsMl0sWzQsMiwiXFxJbmR1Y2VkSG9tb3tKfXtLJ18xfSIsMl0sWzAsMiwiSl8xIiwwLHsiY3VydmUiOi01fV0sWzEsNSwiXFxJbmR1Y2VkSG9tb3tQXzB9e0snXzF9Il0sWzUsMywiXFxJbmR1Y2VkSG9tb3tKfXtLJ18yfSJdLFs0LDUsIlBfMSJdLFsxLDMsIlxcSW5kdWNlZEhvbW97Sl8xfXtLJ18xfSIsMix7ImN1cnZlIjo1fV0sWzEwLDQsIiIsMCx7InNob3J0ZW4iOnsic291cmNlIjoyMCwidGFyZ2V0IjoyMH0sInN0eWxlIjp7ImhlYWQiOnsibmFtZSI6Im5vbmUifX19XSxbNSwxNCwiIiwyLHsic2hvcnRlbiI6eyJ0YXJnZXQiOjIwfSwic3R5bGUiOnsiaGVhZCI6eyJuYW1lIjoibm9uZSJ9fX1dXQ==
\[\begin{tikzcd}
	{K_0} && {K_1} &&& {C_1} \\
	\\
	{K_1} && {K_2} &&& {C_2}
	\arrow["{P_0}"', from=1-1, to=3-1]
	\arrow["{F_1}", from=1-6, to=3-6]
	\arrow["{P_0}"', from=1-1, to=1-3]
	\arrow["{\InducedHomo{J}{K'_1}}"', from=1-3, to=1-6]
	\arrow[""{name=0, anchor=center, inner sep=0}, "{J_1}", curve={height=-30pt}, from=1-1, to=1-6]
	\arrow["{\InducedHomo{P_0}{K'_1}}", from=3-1, to=3-3]
	\arrow["{\InducedHomo{J}{K'_2}}", from=3-3, to=3-6]
	\arrow["{P_1}", from=1-3, to=3-3]
	\arrow[""{name=1, anchor=center, inner sep=0}, "{\InducedHomo{J_1}{K'_1}}"', curve={height=30pt}, from=3-1, to=3-6]
	\arrow[shorten <=4pt, shorten >=4pt, Rightarrow, no head, from=0, to=1-3]
	\arrow[shorten >=4pt, Rightarrow, no head, from=3-3, to=1]
\end{tikzcd}\]

In the above diagram, the top equation is our definition $J_1 = \InducedHomo{J}{K'_1} \circ P_0$, and the bottom equation follows from this definition as well \sTODOinline{we may wish to clarify the reader's understanding of $\InducedHomo{J}{K'_2}$, which involves a doubly internal structure}. The left square commutes as part of the definition of $K$ being a geminal category, and the right square commutes because $J[P'_1] = F'_1$ (which was part of our presumption that $J[\InteriorGeminal{K}] = \gamma$). \sTODOinline{Make sure this all makes sense.}

This completes the proof of the terminality property of $G$.
\end{proof}

\sTODOinline{Observe that the above two co-free constructions can be combined, to find the terminal introspective theory with a suitable lexfunctor into a given lexcategory yielding a given internal geminal category}

\sTODOinline{Do a recap section}

\sTODOinline{Close this chapter with a discussion of how, in many contexts in mathematics, it is the geminal categories (the $\Glob{C}$) which seem to play a more primary role than the introspective theories (the $T$). For example, we are typically more interested in the full syntactic category of PA than the $\Sigma_1$-restricted syntactic category, or in presheaves on the discrete set of worlds (including arbitrary subsets of worlds as propositions) rather than specifically those presheaves which respect the accessibility relation (restricting us to only the \quote{open} propositions). Why, then, has our development of ideas in this document focused on introspective theories rather than geminal categories? This is partly a matter of taste. We could have taken geminal categories as the fundamental notion, and not considered any introspective theory other than the theory of geminal categories. We could still derive \Loeb/'s theorem for geminal categories and so on. But the definition of introspective theories is much cleaner than that of geminal categories, and permits this extraction of the definition of geminal categories as its initial model, which is a beautiful result, helping to motivate the definition of geminal categories. Having the three- or four-axiom concept of an introspective theory also sometimes helps us more easily recognize that an example is a geminal category, without having to verify the six axioms of a geminal category directly. Also, having the concept of introspective theories clarifies which presheaves we can apply Löb's theorem (or \magicref{IntrospTyConFixedPoints}) to. To emphasize this last point, it would be very useful to have an explicit example of a presheaf on a geminal category which we CANNOT apply Löb's theorem to (or perhaps cannot even make sense of the box operator applied to).

We should have a discussion section on comparing and contrasting introspective theories and geminal categories, and the motivation to study both and their relation.}

\sTODOinline{Demonstrate that we do NOT get Loeb's theorem internal to a geminal category G for arbitrary presheaves P on \underlying{G'}, thus demonstrating the necessity of the presheaf existing within an introspective theory. The presheaf needs to be parametrized by a parameter from an object of an enclosing introspective theory. So P(S(X)) |- []P(S(X)) is available.}

\fileend